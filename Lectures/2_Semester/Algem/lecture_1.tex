% !TEX root = ../../../main.tex

\section{Алгебра многочленов}

\begin{definition}
    Многочленом называется функция \(f: \R \ra \R, f = a_nx^n + a_{n-1}x^{n-1} + \dots + a_1x + a_0\)
\end{definition}

\begin{definition}
    \(\F[x]\) --- множество всех многочленов над \(\F\) (с коэффициентами в \(\F\))
\end{definition}

\subsection{Операции над многочленами}
\begin{enumerate}
    \item \(+\) --- сложение
    \item \(\cdot\) --- умножение
    \item \(\cdot\lambda\) --- домножение на константу
\end{enumerate}

\begin{note}
    Многочлены над \(\R\) образуют коммутативное кольцо
\end{note}

\begin{definition}
    Алгебра над полем \(\F\) называется называется множество \(A\), с определенными на нем операциями \(+, \cdot, \cdot \lambda\), которое удовлетворяет следующим условиям:
    \begin{enumerate}
        \item \((A, +, \cdot\lambda)\) --- линейное пространство над $\F$
        \item \((A, +, \cdot)\) --- кольцо (необязательно коммутативное)
        \item \(\lambda(xy) = x(\lambda y) = (\lambda x)y, \lambda \in \F, x, y \in A\)
    \end{enumerate}
\end{definition}

\begin{example}\indent
    \begin{enumerate}
        \item \(\R[x]\)
        \item \(M_n(\F)\)
        \item \(\Z_p[x]\)
    \end{enumerate}
\end{example}

\begin{note}
    Возникает проблема: в \(\Z_p[x]\) сущесвтует многочлен \(x^p - x \equiv 0 \forall x \in \Z_p\). Но тогда у нас будет конечный базис в \(\Z_p[x]\), чего не хотелось бы. Определим многочлен по-другому:
\end{note}

\begin{definition}
    Многочленом над коммутативным кольцом с 1 \(R\) называется бесконечная пооследовательность \(a_0, a_1 \dots \), в которой лишь конечное число коэффициентов отличны от 0. Такие пооследовательности называются финитными.
\end{definition}

\subsection{Операции над новыми многочленами}
Пусть \(A = (a_i), B = (b_i)\)
\begin{enumerate}
    \item \(A + B = C \Lra c_i = a_i + b_i\)
    \item \(A \cdot B = C \Lra c_k = \sum_{i = 0}^ka_ib_{k-i}\)
    \item \(A \cdot \lambda = C \Lra c_k = \lambda\cdot a_i\)
\end{enumerate}

\begin{proposition}
    \(R[x]\) --- коммутативное кольцо относительно ''\(+\)'', ''\(\cdot\)''
\end{proposition}
\begin{proof}\indent
    \begin{enumerate}
        \item \((R[x], +)\) --- абелева группа (очев)
        \item \(A \cdot B = B \cdot A\) --- тут мы пользуемся тем, что \(R\) --- коммутативное кольцо. Поэтому в сумме \(\sum_{i = 0}^ka_ib_{k-i}\) если переставить множители местами, ничего не поменяется
        \item \(A(BC) = (AB)C\) 
        \[\sum_{i = 0}^na_i\left(\sum_{j = 0}^{n-i}b_jc_{n-i-j}\right) = \sum_{i = 0}^k\sum_{j = 0}^{n-i}a_ib_jc_{n-i-j} = \sum_{i + j + k = n}a_ib_jc_k = \sum_{k = 0}^nc_k\left(\sum_{i = 0}^{n-k}a_ib_{n-k-i}\right)\]
        \item \(A(B + C) = AB + AC\) --- Достаточно раскрыть скобки, чтобы проверить, мне лень техать.
    \end{enumerate}
\end{proof}

\begin{corollary}
    \(R[x]\) --- бесконечномерное линейное пространство с базисом \(1, x, x^2 \dots\).
\end{corollary}
\begin{corollary}
    Нетрудно проверить, что в \(R[x]\), \(1 = (1, 0, 0, \dots)\). Аналогично, \(x^n = (\underbrace{0, 0, \dots 0}_{n} , 1, 0, 0, \dots)\)
\end{corollary}

\begin{definition}
    Старший коэффициент --- последний ненулевой элемент последовательности.
\end{definition}
\begin{definition}
    Индекс старшего коэффициента называется степенью многочлена \(\deg P\). У многочлена \((0, 0, \dots)\) степень зависит от контекста. Мы будем считать, что его степень \(-\infty\).
\end{definition}
\begin{definition}
    Кольцо с \(1 \ne 0\) называется областью целостности, если в нем нет делителей нуля.
\end{definition}
\begin{proposition}
    Пусть  \(R\) --- область целостности. Тогда \(ab = ac, a\ne 0 \Ra b = c\)
\end{proposition}
\begin{proof}
    \[a(b - c) = 0\]
    \[b - c = 0\]
    \[b = c\]
\end{proof}
\begin{proposition}
    \(A, B \in R[x], 1 \in R\). Тогда:
    \begin{enumerate}
        \item \(\deg A + \deg B \le \max(\deg A, \deg B)\)
        \item \(\deg AB \le \deg A + \deg B\)
    \end{enumerate}
    Причем, если \(R\) --- область целотности, то во втором пункте будет равенство.
\end{proposition}
\begin{proof}
    Все понятно
\end{proof}
\begin{corollary}
    Если \(R\) --- область целостности, то \(R[x]\) --- тоже.
\end{corollary}
\begin{definition}
    Многочлен от \(n\) переменных определяется рекурсивно: многочлен от одной переменной --- как мы определяли выше, далее \(R[x_1, x_2, \dots x_{n}] = R[x_1, x_2, \dots x_{n-1}][x_{n}]\).
\end{definition}
\subsection{Деление многочленов с остатком}
\begin{theorem}
    Пусть \(F\) --- поле, \(A, B \in F[x], B \ne 0\). тогда
    \begin{enumerate}
        \item \(\exists! Q, R: A = BQ + R, \deg R < \deg B\)
    \end{enumerate}
\end{theorem}
\begin{proof}
    Существование доказывается алгоритмом деления в столбик. Проверим единственность:
    \[BQ + R = BS + T\]
    \[BQ - BS = T - R\]
    \[\deg (B(Q - S)) > \deg(T - R)\]
    Противоречие
\end{proof}

\begin{theorem}[Безу]
    Пусть \(P \in F[x]\). Тогда \(P(x) - P(c) \vdots (x - c)\)
\end{theorem}
\begin{proof}
    Разделим многочлен \(P\) на \(x - c\) с остатком. Получится \(P = Q(x - c) + R\), причем \(R\) --- константа. Тогда подставим \(x = c\), получим, что \(R = P(c)\).
\end{proof}

\subsubsection{Схема Горнера}
Задан многочлен:
\[P(x) = a_0 + a_1 x + a_2 x^2 + a_3 x^3 + \ldots + a_n x^n, \quad a_i \in \R\]

Пусть требуется вычислить значение данного многочлена при фиксированном значении \(x = x_0\). Представим многочлен \(P(x)\) в следующем виде:
\[P(x) = a_0 + x(a_1 + x(a_2 + \cdots x(a_{n-1} + a_n x) \dots))\]

Определим следующую последовательность:
\[b_n = a_n,\]
\[b_{n-1} = a_{n-1} + b_n x_0,\]
\[\vdots\]
\[b_i = a_i + b_{i+1} x_0,\]
\[\vdots\]
\[b_0 = a_0 + b_1 x_0.\]

Искомое значение \(P(x_0)\) есть \(b_0\). Покажем, что это так.

В полученную форму записи \(P(x)\) подставим \(x = x_0\) и будем вычислять значение выражения, начиная с внутренних скобок. Для этого будем заменять подвыражения через \(b_i\)
\[
\begin{array}{ll}
 P(x_0) & = a_0 + x_0(a_1 + x_0(a_2 + \cdots x_0(a_{n-1} + a_n x_0)\dots)) = \\
  & = a_0 + x_0(a_1 + x_0(a_2 + \cdots x_0 b_{n-1}\dots)) = \\
  & ~~ \vdots \\
  & = a_0 + x_0 b_1 = \\
  & = b_0.
\end{array}
\]
\subsection{НОД двух многочленов. Алгоритм Евклида}
\begin{definition}
    Многочлен \(f\) делится на \(g\), еcли \(f = gh\) для некототорого \(h\)
\end{definition}
\begin{definition}
    Многочлены \(f, g\) называются ассоциированными, если \(f \vdots g, g \vdots f\).
\end{definition}
\begin{definition}
    Многочлен  \(d\) называется Наибольшим общим делителем двух многочленов \(f, g\), если:
    \begin{enumerate}
        \item \(f \vdots d, g \vdots d\)
        \item \(f \vdots d', g \vdots d' \Ra d \vdots d', d \vdots d'\)
    \end{enumerate}
\end{definition}
\begin{theorem}[О представлении НОДа]\indent
    \begin{enumerate}
        \item НОД любых двух многочленов существует
        \item НОД любых двух многочленов представим в виде их линейной комбинации
    \end{enumerate}
\end{theorem}
