% !TEX root = ../../../main.tex

% \section{Преобразование Абеля}
Пусть $a_n, b_n$ --- последовательности комплексных чисел $m \in \N$ и $A_k = \sum_{i = 1}^{k} a_i$. Тогда $a_k = A_k - A_{k-1}$ (если считать, что $A_0 = 0$) и $\sum_{k = m}^{n}a_kb_k = \sum_{k = m}^{n}(A_k - A_{k-1})b_k = \sum_{k = m}^{n}A_kb_k - \sum_{k = m}^{n}A_{k-1}b_k$.
Следовательно, справедливо тождество (Преобразование Абеля) $\sum_{k = m}^{n}a_kb_k = A_nb_n - A_{m-1}b_m - \sum_{k = m}^{n - 1}A_k(b_{k+1} - b_k)$
\begin{lemma}[Абеля]
    Пусть $a_n, b_n$ --- последовательности, причем $\{b_n\}$ монотонна. Если $\left\lvert \sum_{i = m}^k a_i\right\rvert \le M \forall k$, то $\left\lvert \sum_{k = m}^{n}a_kb_k\right\rvert \le 2M(|b_n| + |b_m|)$
\end{lemma}
\begin{proof}
    Считаем, что $a_k = 0$ при $k < m$. Тогда $\left\lvert \sum_{k = m}^{n}a_kb_k\right\rvert = \left\lvert A_nb_n - \sum_{k = m}^{n}A_k(b_{k+1} - b_k)\right\rvert \le |A_n||b_n| - \sum_{k = m}^{n}|A_k||b_{k+1} - b_k| \le M(|b_n| + \left\lvert \sum_{k = m}^{n-1}(b_{k+1} - b_k)\right\rvert )$. Т.к. $\{b_n\}$ монотонна, то $b_{k+1} - b_k$ одного знака $\forall k$, тогда 
    $$\left\lvert \sum_{k = m}^{n}a_kb_k\right\rvert \le M(|b_n| + |b_n - b_{m}|)$$
\end{proof}

\begin{note}
    Если $m = 1, \{b_n\}$ нестрого убывает и неотрицательна, $c \le A_k \le C$, то 
    $$cb_1 \le \sum a_kb_k \le Cb_1$$
\end{note}

\begin{lemma}[Абель]
    Пусть $f \in R[a, b], g$ --- монотонна на $[a, b]$. Если  $\left\lvert \int_{a}^{x} f(t) \,dt\right\rvert \le M \forall x \in [a, b]$, то 
    $$\left\lvert \int_{a}^{b} f(x)g(x) \,dx\right\rvert  \le 2M(|g(a)| + |g(b)|)$$
\end{lemma}
\begin{proof}
    Пусть $T_n = \{x_k\}_{k = 0}^n$ --- разбиение отрезка $[a, b]$ на $n$ равных частей. Положим $\Delta_kg = g(x_k) - g(x_{k-1}), \sigma_n = \sum_{k = 1}^{n} g(\xi_k)\int_{x_{k-1}}^{x_k}f(t)dt, \xi_k \in [x_{k-1}, x_k]$. 
    Тогда $\alpha_n := \left\lvert \int_{a}^{b}f(x)g(x)dx - \sigma_n\right\rvert = \left\lvert \sum_{k = 1}^{n}\int_{x_{k-1}}^{x_k}f(t)g(t)dt - \sum_{k = 1}^{n} g(\xi_k)\int_{x_{k-1}}^{x_k}f(t)dt\right\rvert = $ \newline $ = \left\lvert \sum_{k = 1}^{n}\int_{x_{k-1}}^{x_k}f(t)g(t)dt - \sum_{k = 1}^{n} g(\xi_k)\int_{x_{k-1}}^{x_k}f(t)dt\right\rvert \le \sum_{k = 1}^{n}\int_{x_{k-1}}^{x_k}|f(t)||g(t) - g(\xi_k)|dt$. Т.к. $g$ --- монотонна, $\Delta_kg$ все одного знака и $|g(x) - g(\xi_k)| \le |\Delta_kg|$. Тогда $\alpha_n \le \sum_{x_{k - 1}}^{x_k}|f(x)||\Delta_kg|dx$. Т.к. $f \in R[a, b] \Ra \exists c (|f| \le c)$
    $$\sum_{k = 1}^nc|\Delta_kg|(\underbrace{x_k - x_{k-1}}_{\frac{b - a}{n}}) = c\frac{b - a}{n}\left\lvert \sum_{k  =1}^{n} \Delta_kg\right\rvert = c\frac{b - a}{n}|g(x_n) - g(x_0)| = c\frac{b - a}{n}|g(b) - g(a)|$$
    Таким образом, $0 \le \alpha_n \le c\frac{b - a}{n}|g(b) - g(a)|$, но правая часть $\ra 0$, поэтому $\alpha_n \ra 0$. Тогда достаточно оценить $\sigma_n$. Применим лемму 1, где $b_k = g(\xi_k), a_k = \int_{x_{k-1}}^{x_k}f(t)dt$. Тогда $b_n$ --- монотонная последовательность.
    $$\left\lvert \sum_{i = 1}^{k} a_i\right\rvert = \left\lvert \int_{a}^{x_k}f(t)dt\right\rvert 
    \le M$$
    Откуда получаем, что $|\sigma_n| \le 2M(|b_1 + |b_n|) = 2M(|g(b)| + |g(a)|)$. Выбрав $\xi_1 = a, \xi_n = b$
    $$\left\lvert \int_{a}^{b}f(x)g(x)dx \right\rvert = \left\lvert \int_{a}^{b}f(x)g(x)dx + \sigma_n - \sigma_n\right\rvert \le \alpha_n + \sigma_n \le 2M(|g(a)| + |g(b)|) + \underbrace{\alpha_n}_{\ra 0}$$
\end{proof}

\section{Несобственный интеграл}
\begin{definition}
    Функция $f$ назывется локально интегрируемой по Риману, на промежутке $I$, если $\forall [a, c] \subset I (f \in R[a, c])$
\end{definition}

\begin{definition}
    Пусть $-\infty < a < b \le +\infty$ и $f$ локально интегрируема на $[a, b]$. Предел $\int_a^b f(x)dx := \lim_{c \ra b - 0}\int_{a}^{c}f(x)dx$ называется несобственным интегралом $f$ на $[a, b]$. Если предел существует и конечен, то $\int_{a}^{b}f(x)dx$ называют сходящимся, иначе --- расходящимся.
\end{definition}

Пусть $b \in \R$, $f$ локально интегрируема на $[a, b)$ и ограничена, тогда $f \in R[a, b]$ (при любом доопределении в точке $b$) и по свойству непрерывности определенного интеграла с переменным пределом, несобственный интеграл сопадает с определенным

$\int_{a}^{b}f(x)dx = \lim_{x \ra b - 0} \int_{a}^{x}f(t)dt$, т.е. новая ситуация имеет место в случае $b = +\infty$ или $b \in \R$ и $f$ неограничена на $[a, b)$. Ряд свойств определенного интеграла перносится на несобственный, т.к. можно применить предельныйм переход.

\begin{proposition}[Принцип локализации]
    Пусть $f$ локально интегрируема на $[a, b)$. Тогда для любого $a^* \in (a, b)$ несобственный интеграл $\int_{a^*}^{b}f(x)dx$ и $\int_{a}^{b}f(x)dx$ сходятся или расходятся одновременно, причем, в случае сходимости:
    $$\int_{a}^{a^*}f(x)dx + \int_{a^*}^{b}f(x)dx = \int_{a}^{b}f(x)dx$$
\end{proposition}
\begin{proof}
    Заметим, что по аддитивности (нормального интеграла)
    $$\int_{a}^{a^*}f(x)dx + \int_{a^*}^{c}f(x)dx = \int_{a}^{c}f(x)dx$$
    Но т.к. 
    $$\lim_{c \ra b} \int_{a}^{c}f(x)dx = \int_{a}^{b}f(x)dx$$
    В предельном переходе получаем требуемое.
\end{proof}

\begin{proposition}[Линейность]
    Если несобственные интегралы $\int_a^b f(x)dx, \int_a^b g(x)dx$ сходятся и $\lambda, \mu \in \R$, то сходятся и несобственный интеграл 
    $$\int_a^b (\lambda f(x) + \mu g(x))dx = \lambda\int_a^b f(x)dx + \mu\int_a^b g(x)dx$$
\end{proposition}
\begin{proof}
    Фигачим предельный переход
\end{proof}

\begin{proposition}[Формула Ньютона-Лейбница]
    Пусть $f$ локально интегрируема на $[a, b)$, $F$ --- первообразная на $[a, b)$, тогда
    $$\int_a^bf(x)dx = F(b) - F(a)$$
\end{proposition}
\begin{proof}
    Фигачим предельный переход
\end{proof}

\begin{example}
    Хотим узнать сходимость 
    $$\int_1^{+\infty} \frac{dx}{x^\alpha}$$
    В зависимости от $\alpha$
    \begin{enumerate}
        \item[$\alpha \ne 1$:]
        $$\int_1^{+\infty} \frac{dx}{x^\alpha} = \left. \frac{x^{-\alpha + 1}}{-\alpha + 1}\right|_1^{+\infty} = \left\{\begin{array}{l}
            \frac{1}{\alpha - 1}, \alpha > 1 \\
            +\infty, \alpha < 1
        \end{array}\right.$$
        \item[$\alpha = 1$:] 
        $$\int_1^{+\infty} \frac{1}{x}dx = \left.\ln x\right|_1^{+\infty} = +\infty$$
    \end{enumerate}
\end{example}

\begin{example}
    Аналогично проверяется, что 
    $$\exists \int_{0}^{1}\frac{dx}{x^\alpha} \Lra \alpha < 1$$
    Причем сходится к \(\frac{1}{1 - \alpha}\)
\end{example}