% !TEX root = ../../../main.tex

Короче, как-то будем сдавать какой-то экзамен. Очень сложно, ничего не понятно

\section{Вступление}
Вот у нас были натуральные числа:
\[
\begin{array}{c}
    0 = \emptyset \\
    1 = \{\emptyset\} \\
    2 = \{\emptyset, \{\emptyset\}\} = \{0, 1\} \\
    \vdots\\
    n + 1 = \{0, 1, 2, \dots n\} \\
\end{array}
\]
Вопрос: что будет в бесконечности?
\[
\begin{array}{c}
    \omega = \{0, 1, 2, \dots\} \\ 
    \omega + 1 = \{0, 1, 2, \dots, \omega\} \\ 
    \omega + 2 = \{0, 1, 2, \dots, \omega, \omega + 1\} \\
    \vdots \\
    2\omega = \dots\\
    2\omega + 1 = \dots\\
    \vdots\\
    3\omega = \dots\\
    \vdots\\
    \omega\cdot\omega = \dots\\
\end{array}
\]
Таким образом, получаем различные многочлены от \(\omega\), если продолжать этот абсурд, то получится \(\omega^\omega\), потом получится \(\underbrace{\omega^{\omega^{\omega^{\dots^\omega}}}}_{\omega} \) и короче всякое такое.

\section{Фундированные множества}
\begin{definition}
    Пусть \(S\) --- ЧУМ. Тогда \(S\) называется Фундированным, если \(\forall A \subset S \exists \min A\)
\end{definition}
\begin{example}[Фундированные]\indent
    \begin{enumerate}
        \item \(\N, \le\)
        \item \(\N, |\)
        \item \(\{a, b\}^*, \sqsubset \)
    \end{enumerate}
\end{example}
\begin{example}[Не фундированные]\indent
    \begin{enumerate}
        \item \(\Z, \le\)
        \item \(\N, \ge\)
        \item \([0, 1], \le \)
        \item \(\{a, b\}^*, \le_{lex} \)
    \end{enumerate}
\end{example}

\subsection{Свойства, эквивалентные фундированности}
\begin{enumerate}
    \item (БС) Невозможность бесконечного спуска
    \[\nexists a_1 > a_2 > a_3 \dots\]
    \item (Ст) Стабилизация
    \[\forall a_1 \ge a_2 \ge a_3 \dots \Ra \exists k: \forall n > k (a_k = a_n)\]
    \item (ТИ) Трансфинитная индукция
    \[\forall x (\forall y < x\;\; \phi(y) \ra \phi(x)) \Ra \forall z \phi(z)\]
\end{enumerate}
\begin{theorem}
    Свойства Фундированность, БС, Ст, ТИ эквивалентны.
\end{theorem}
\begin{proof}\indent
    \begin{enumerate}
        \item \(\neg \text{Ф} \Ra \neg \text{БС}\). Пусть \(A \neq \emptyset, \nexists \min A\). Тогда \(\forall a_1 \in A \exists a_2 \in A: a_2 < a_1\). Используя аксиому выбора (выбирая по одному элементу из оставшихся), получается бесконечную убывающую последовательность.
        \item \(\neg \text{Ф} \La \neg \text{БС}\). Тогда существует \(a_1 > a_2 > a_3 \dots\). Рассмотрим это множество, в нем не будет минимального элемента.
        \item \(\neg \text{БС} \Ra \neg \text{Ст}\). Тогда существует \(a_1 > a_2 > a_3 \dots\). Заметим, что для это последовательности неверна стабилизация.
        \item \(\neg \text{БС} \La \neg \text{Ст}\). Рассмотрим последовательность, которая не стабилизируется. Тогда \(\forall n \exists k: a_n > a_k\). Тогда \(\exists\) бесконечная убывающая цепочка.
        \item \(\neg \text{Ф} \Ra \neg \text{ТИ}\). \(A \ne \emptyset\) --- множество без минимального элемента, \(\phi(x) \Lra x \notin A \Ra \phi(x) \not\equiv 1\).
        \[\forall y < x\;y\notin A \Ra x \notin A\]
        Утверждение вверу верно, т.к. \(\forall y < x (y \notin A, x \in A) \Ra x = \min A\).
        \item \(\neg \text{Ф} \La \neg \text{ТИ}\). Тогда для некоторго \(\phi\) верно, что  
        \[\forall x (\forall y < x \; \phi(y) \ra \phi(x))\]
        Но
        \[\neg\forall z \phi(z) (1)\]
        Пусть \(A = \{z | \phi(z) = 0\}\). Причем \(A\) непусто, т.к. \((1)\). Тогда рассмотрим минимальный элемент в \(A\) и получим противоречие с определением ТИ.
    \end{enumerate}
\end{proof}

\begin{definition}
    Вполне упорядоченное множество --- Линейная упорядоченность + Фундированность
\end{definition}
\begin{example}
    \[
    \begin{array}{lc}
        \N, \le & \omega \\\\
        \left\{1 - \frac{1}{n}| n \in \N_+\right\} & \omega \\\\
        \left\{1 - \frac{1}{n}| n \in \N_+\right\}\cup\left\{2 - \frac{1}{n}| n \in \N_+\right\} & \omega\cdot2 \\\\
        \left\{k - \frac{1}{n}| k, n \in \N_+\right\} & \omega^2 \\\\
        \left\{1 - \frac{1}{n} - \frac{1}{m}| m, n \in \N_+\right\} & \omega^2 \\\\
        \left\{1 - \frac{1}{n} - \frac{1}{m} - \frac{1}{k}| m, n, k \in \N_+\right\} & \omega^3 \\\\
        \left\{1 - \frac{1}{n_1} - \frac{1}{n_2} - \dots - \frac{1}{n_k}| k \text{ --- произвольное}\right\} & \text{не фундированное}\\\\
        \left\{k - \frac{1}{n_1} - \frac{1}{n_2} - \dots - \frac{1}{n_k}| k \text{ --- произвольное}\right\} & \omega^\omega \\\\
    \end{array}
    \]
\end{example}

\begin{definition}
    Пусть \(S\) --- ВУМ. Тогда \(K \subset S\) называется начальным отрезком, если \(\forall x, y ((x \in K \wedge y < x) \ra y \in K)\)
\end{definition}
Эквивалентные свойства:
\[\forall x \in K \forall y \notin K x < y\]
\[\forall x, y ((x \notin K \wedge y > x) \ra y \notin K)\]
\subsection{Непосредственно следующие элементы}
\begin{proposition}
    \(S\) --- ВУМ, \(x \in S, x\) --- не наибольший в \(S \Ra \exists!y(y > x \wedge \neg\exists z y > z > x)\).
\end{proposition}
\begin{proof}
    \(\exists\) --- из Фундированности, \(y = \min\{t \in S | t > x\}\)
\end{proof}
\begin{definition}
    \(y\) из предыдущего утверждения называется непосредственно следующим элементом после \(x\) и обозначается \(x + 1\).
\end{definition}

\begin{note}
    \[[0, a] = [0, a + 1)\]
\end{note}

\begin{theorem}
    \(K\) --- начальный отрезок \(S \Ra K = S \vee K = [0, a)\)
\end{theorem}
\begin{proof}
    Если \(K = S\), то победили, иначе рассматриваем \(a = \min(S \setminus K)\). Докажем, что \(K = [0, a)\).
    \begin{enumerate}
        \item \(K \subset [0, a)\): Если \(x \in K, x > a\), то \(a \in K\), но \(a \in S \setminus K\)
        \item \(K \supset [0, a)\): Если \(x < a, x \notin K\), то \(a \ne \min(S \setminus K)\) --- противоречие.
    \end{enumerate}
\end{proof}

\begin{example}\indent
    \begin{enumerate}
        \item \(S\)
        \item \([0, \alpha] = \{x | x \le \alpha\}\)
        \item \([0, \alpha) = \{x | x < \alpha\}\)
    \end{enumerate}
\end{example}

\subsection{Предельные элементы}
\begin{definition}
    \(z\) назывется предельным элементом, если \(\nexists y (z = y + 1)\).
\end{definition}
или 
\begin{definition}
    \(z\) назывется предельным элементом, если
    \[\forall y < z \exists t \in (y, z)\]
\end{definition}

\begin{theorem}
    \(S\) --- ВУМ, \(x \in S \Ra \exists l \in S, k \in \N: x = l + k = l + \underbrace{1 + 1 + \dots + 1}_{k} \)
\end{theorem}

\subsection{Сложение и умножение Фундированных множеств и ВУМов}
Сложение и умножение определены так же, как и для ЧУМов.
\begin{theorem}
    \begin{enumerate}
        \item \(A, B\) --- фундированные, тогда и \(A + B\) --- тоже.
        \item \(A, B\) --- ВУМ, тогда и \(A + B\) --- тоже.
        \item \(A, B\) --- фундированные, тогда и \(A \cdot B\) --- тоже.
        \item \(A, B\) --- ВУМ, тогда и \(A \cdot B\) --- тоже.
    \end{enumerate}
\end{theorem}
\begin{proof}
    \(C \subset A \sqcup B\):
    \begin{enumerate}
        \item \begin{enumerate}
            \item \(C \cap A \ne \emptyset \Ra \min(C \cap A) \text{ --- существует, т.к. } A \text{ --- фундированное}\)
            \item \(C \cap B \ne \emptyset \Ra \min(C \cap B) \text{ --- существует, т.к. } B \text{ --- фундированное}\)
        \end{enumerate}
        \item  Подмножество ЛУМа --- ЛУМ, поэтому победили по (1).
    \end{enumerate}
\end{proof}

\begin{note}
    Любое подмножество ВУМ --- тоже ВУМ
\end{note}
\begin{note}
    Множество предельных элемнтов ВУМа --- ВУМ
\end{note}
\begin{note}
    Между любыми двумя предельными элементами бесконечно много других
\end{note}
\begin{note}
    Элементы между соседними предельными элементами образуют множество, \(\approxeq \omega\)
\end{note}
 
\begin{theorem}[О структуре ВУМ]
    \(S\) --- ВУМ, тогда \(\exists L\) --- тоже ВУМ, конечное множество \(K\), такие, что \(S \cong \omega \cdot L + K\)
\end{theorem}

\begin{theorem}[О сравнимости ВУМов]
    Любые два ВУМа либо изоморфны, либо один из них изоморфен начальному отрезку другого.
\end{theorem}
\begin{proof}
    Рекурсивно построим \(f: A \ra B\). Пусть \(f\) определена на \([0, x) \subset A\). Тогда определим \(f(x) = \min(B \setminus f([0, x)))\) (рекурсивно). 
    \begin{theorem}[О трансфинитной рекурсии]
        Если задано правило \(f(x) = F(f|_{[0, x)})\), то есть ровно одна \(f\), которая соответствует правилу.
    \end{theorem}
    \begin{proof}\indent
        \begin{enumerate}
            \item[] \textbf{Единственность}. Пусть \(f, g\) --- 2 подходящие функции.
            \[\{x | f(x) \ne g(x)\} \ne \emptyset \Ra \exists m = \min\{x| f(x) \ne g(x)\} \Ra f|_{[0, m)} = g|_{[0, m)}\]
            Но тогда \(f(m) = F(f|_{[0, m)}) = F(g|_{[0, m)}) = g(m)\), противоречие.

            \item[] \textbf{Cуществование}. По трансфинитной индукции докажем сущесвование \(f|_{[0, x)}\), соответствующее \(F\).
            \[\forall y < x \exists f|_{[0, y)} \Ra \exists f|_{[0, x)}\]
            \begin{enumerate}
                \item \(x = w + 1 \Ra \exists f|_{[0, w)}, f(w) = F(f|_{[0, w)})\)
                \item \(x\) --- предельное --- в следующий раз
            \end{enumerate}
        \end{enumerate}
    \end{proof}
    \begin{theorem}[Обобщенная теорема о трансфинитной рекурсии]
        \(F\) может быть частично определена, тогда \(f\) определена на начальном отрезке.
    \end{theorem}
\end{proof}