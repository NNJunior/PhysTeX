% !TEX root = ../../../main.tex

\section{Вступление}
\subsection{Литература}
\begin{enumerate}
    \item Верещагин Н.К., Шень А.''Лекции по мат.логике'':
    \begin{enumerate}
        \item[ч.1] Начало теории множеств
        \item[ч.2] Языки и исчисления
        \item[ч.3] Вычислимые функции
    \end{enumerate}
\end{enumerate}

\textbf{Синтаксис} (Правила составления ормул) $\longleftrightarrow$ \textbf{Семантика} (сопоставление формального выражения некоторому смыслу).
Мы начнем с \textbf{Семантики}. Из-за лингвистической истории этого вопроса, в каждом формальном языке ест \textbf{алфавит}.

\section{Алфавит и языки}
\subsection{Определения}
\textbf{\underline{Определение:}} алфавит --- множество символов. Мы будем считать, что алфавит непуст и конечен.

\textbf{\underline{Определение:}} слово --- конечная последовательность символов алфавита (может быть пустым). Оно состоит из букв (элементов алфавита). Пустое слово обозначается $\varepsilon$

\textbf{\underline{Определение:}} язык --- множество слов. Пустой язык $\emptyset$

\textbf{\underline{Определение:}} синглтон --- язык, состоящий только из пустого слова. $\{\varepsilon\}$

\subsection{Операции}
\begin{enumerate}
    \item Конкатенация $u \cdot v$ --- дописываение слова
    \item Длина $|u|$ --- длина слова
    \item Возведение в степень $u^n = \underbrace{uu\dots u}_n$
    \item Обращение $u^R = u_nu_{n-1}\dots u_1$, если $u = u_1u_2\dots u_n$
    $(u\cdot v)^R = v^Ru^R$
\end{enumerate}
\subsection{Отношения}
\begin{enumerate}
    \item Префикс $u \sqsubset v \Leftrightarrow \exists w: uw = v$
    \item Суффикс $u \sqsupset v \Leftrightarrow \exists w: wu = v$
    \item Подслово $u \subset v$ --- вычеркнуть часть букв (сохраняя порядок), получив $v$ из $u$
\end{enumerate}
\subsection{Операции над языками}
\begin{enumerate}
    \item[0.] Теоретико-множественные
    \item Конкатенация $L\cdot M = \{uv | u \in L, v \in M\}$, причем
    \begin{enumerate}
        \item $L\cdot \emptyset = \emptyset\cdot L = \emptyset$
        \item $L\cdot \{\varepsilon\} = \{\varepsilon\}\cdot L = L$
    \end{enumerate}
    \item Возведение в степень $L^n = \underbrace{LLL\dots L}_n$, причем $L^0 = \{\varepsilon\}$
    \item Итерация (Звезда Клини) $L^*=L^0\cup L^1\cup L^2\dots$
    \item Плюс Клини $L^+=L^1\cup L^2\dots$. Тогда $L^+ = L^* \cdot L$
\end{enumerate}
\textbf{Мы будем считать, что натуральные числа начинаются с 0, но во всех местах это на всякий случай будут писать.}
\\\\

\textbf{\underline{Определение 1:}} Правильная скобочная последовательность (ПСП) --- последовательность скобок, разбитых на пары, где в каждой паре ''('' идет раньше '')''.
$$(_1\ (_2\ )_1\ )_2 - тоже\ правильная$$

\textbf{\underline{Определение 2:}} Правильная скобочная последовательность (ПСП) --- последовательность скобок, индуктивно построенная из правил: 
\begin{enumerate}
    \item $\varepsilon$ --- ПСП
    \item $S$ --- ПСП $\Leftrightarrow (S)$ --- ПСП
    \item $S, T$ --- ПСП $\Rightarrow S\cdot T$ --- ПСП
\end{enumerate}

\textbf{\underline{Определение 3:}} Баланс СП --- число ''('' - число '')''

\textbf{\underline{Определение 4:}} ПСП --- такая СП, что ее баланс равен 0, а баланс любого префикса $\ge 0$.
\begin{proof}
Равносильность определений:
\indent
    \begin{enumerate}
        \item $\textbf{\underline{Опр. 1}} \Rightarrow \textbf{\underline{Опр. 3}}$. Все скобки разбиты на пары $\Rightarrow$ баланс = 0. Более того, в паре ''('' идет раньше '')'', поэтому в каждом префиксе не может быть только '')'' из одной пары. В каждой паре тогда сумма $\ge 0 \Rightarrow$ в префиксе сумма $\ge 0$. 
        \item $\textbf{\underline{Опр. 2}} \Rightarrow \textbf{\underline{Опр. 1}}$. Такой алгоритм позволяет явным образом разбить скобки на пары.
        \item $\textbf{\underline{Опр. 3}} \Rightarrow \textbf{\underline{Опр. 2}}$. Доказательство: индукция по длине СП.
        \begin{enumerate}
            \item[] \textbf{База:} $S = \varepsilon$ --- очевидно 
            \item[] \textbf{Переход:} $|S| > 0$. Тогда из \textbf{\underline{Опр. 3}} следует, что первый символ --- ''(''. Рассмотрим кратчайший непустой префикс с балансом = 0. Если это все слово, то тогда $S = (S')$, т.к. это минимальный префикс, где баланс обратился в 0 (первую скобку надо было чем-то ''убить'', значит, последняя скобка идет с ней в паре). Тогда утверждение работает и для $S'$ по предположению индукции. Иначе, это какой-то префикс $T$. Тогда $S = TL$. По предположению индукции, $T, L$ --- ПСП, т.к. баланс $T$ равен 0. Значит Ч.Т.Д.
        \end{enumerate}
    \end{enumerate}
\end{proof}
Получается, что все три определения эквивалентны.