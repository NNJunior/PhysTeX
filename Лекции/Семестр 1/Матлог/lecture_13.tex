% !TEX root = ../../../main.tex

\subsection{Элиминация кванторов}
Рассмотрим следующее множество:
$$\langle \N, S, 0, =\rangle$$

\begin{theorem}
    Любая формула, записанная в этой сигнатуре $S, 0, =$ эквивалентна в интерпретации некоторой бескванторной формуле.
\end{theorem}
\begin{proof}
    Ведем индукцию по построению формулы. 
    \begin{enumerate}
        \item[] \textbf{База:} Атомарные формулы --- бескванторные
        \item[] \textbf{Переход:}
        \begin{enumerate}
            \item[$\neg$:] $\phi \eqcirc \neg\psi \Rightarrow$ по предположению индукции, $\psi \sim \psi'$, $\psi'$ --- бескванторная, тогда $\phi \sim \neg\psi'$ --- тоже бескванторная
            \item[$\wedge, \vee, \rightarrow$:] $\phi \eqcirc (\psi * \eta) \Rightarrow$ по предположению индукции, $\psi \sim \psi', \eta \sim \eta'$, причем $\psi', \eta'$ --- бескванторные, $\Rightarrow \phi \sim (\psi' * \eta')$  --- тоже бескванторная
            \item[$\forall$:] $\forall x \phi \leftrightarrow \neg \exists x \neg \phi$
            \item[$\exists$:] $\exists x \phi$. Как быть? Одна из идей: заменить бесконечную конъюнкцию на конечную. Но сначала приведем саму формулу $\phi$ к бескванторному виду $\phi'$.

            Заметим, что атомарные формулы у нас могут быть только вида $$S(S(\dots(S(u))\dots)) = S(S(\dots(S(v))\dots))$$
            Где $u, v$ --- либо переменные, либо $0$. Но тогда:
            $$\begin{array}{cc}
                u \eqcirc v \eqcirc x & \text{ формула $\perp$ или $\top$ } \\
                S(S(\dots(S(x))\dots)) = 0 & \perp \\
                S(S(\dots(S(y))\dots)) = x & x = y + c\\
                S(S(\dots(S(x))\dots)) = y & x = y - c \\
            \end{array}$$
            Итого: $\exists x \phi$, причем $\phi$ --- булевая комбинация $\perp, \top$ и равенств вида $x = d, x = y + c, x = y - c$. Рассмотрим $t_1, t_2, \dots t_k$ --- все ''правые'' части этих равенств. В таком случае, при $x \not\in \{t_1, t_2, \dots t_k\}$, формула не будет зависеть от $x$, т.к. все равенства вида $x = t_i$ будут заведомо ложны. Тогда 
            $$\exists x \phi \sim \phi|_{\text{все $x_i = t$ ложны}} \vee \bigvee \phi[^{t_i}/_x]$$
            (Все выражения с вычитанием преобразуются в сложение в другой части)
        \end{enumerate}
    \end{enumerate}
\end{proof}

\begin{definition}
    2 интерпретации одной сигнатуры называются элементарно эквивалентными, если в них верны одни и те же формулы первого порядка.
\end{definition}

\begin{theorem}
    $\langle \R, \le \rangle \sim \langle \Q, \le \rangle$
\end{theorem}
\begin{proof}
    В обеих интерпретациях верна теорема об элиминации кванторов, причем элиминация происходит посимвольно одинаково. Отличие от предыдущего --- в формуле $\exists x \phi$ заменим $\psi$ на эквивалентную ей ДНФ.
    $$\begin{array}{cc}
        x = y & (x \le y \wedge y \le x)  \\
        x < y & (x \le y \wedge \neg (y \le x))
    \end{array}$$
    $$\phi = C_1 \vee C_2 \vee \dots \vee C_n$$
    Причем $C_i$ --- конъюнкции $x_j \le y_j \leftrightarrow (x_j < y_j \vee x_j = y_j)$ или $\neg (x_j \le y_j) \leftrightarrow x_j > y_j$. Тогда раскроем по дистрибутивности скобки в $C_i$, получится $\phi = C_1' \vee C_2' \vee \dots \vee C_n'$, где $C_i'$ --- конъюнкция формул вида $x_j = y_j$ или $x_j < y_j$.
    $$\exists x \phi \sim \exists x (C_1' \vee \dots \vee C_n') \sim \exists x C_1' \vee \exists x C_2' \vee \dots \vee \exists x C_n'$$
    Причем,
    $$\exists x C_i' \eqcirc \exists x ((x > a_1) \wedge \dots \wedge (x > a_p) \wedge (x < b_1) \wedge \dots \wedge (x < b_q) \wedge (x = c_1) \wedge \dots \wedge (x = c_r) \dots$$
    $$\dots \wedge (\text{формулы, значение которых не зависит от $x$}))$$
    Но тогда, в силу того, что оба наши порядка плотны, такое число $x$ либо одновременно существует и там и там, либо одновременно не существует, следовательно, изначальные порядки эквивалентны.
\end{proof}

\subsection{Игра Эренфойхта}

\begin{definition}
    Игра Эренфойхта. Пусть заданы две интерпретации $A, B$, сигнатуры, состоящие только из предикатных символов ($p_1, \dots p_n$). Играют два игрока (их обычно называют Новатор и Консерватор или Спойлер и Дубликатор). Новатор фиксирует число ходов $m$. На $i$-ом ходу выбраны $a_1, a_2, \dots a_{i-1} \in A, b_1, b_2, \dots b_{i-1} \in B$ и Новатор выбирает либо $a_i \in A$, а Консерватор выбирает $b_i \in B$, или наоборот. Цель новатора --- чтобы на каком-то наборе для какого-то предиката, стало выполнено $p_j(a_{i_1}, a_{i_2}, \dots a_{i_l}) \ne p_j(b_{i_1}, b_{i_2}, \dots b_{i_l})$. Цель консерватора --- ему помешать.
\end{definition}

\subsubsection{Примеры}

\begin{example}
    $$\langle \N, \le\rangle, \langle \Z, \le\rangle$$
    Хотим понять, что $\exists x \forall y\;x \le y$ --- верно в $\N$, но не в $\Z$.
    2 хода:
    $$\begin{array}{c|cc}
        \text{№ хода} & \text{Новатор} & \text{Консерватор} \\
        \hline
        1 & 0 \in \N & b \in \Z \\
        2 & b-1 \in \Z & a \in \N \\
    \end{array}$$
    Получилось, что $a \ge 0$ верно, но $b-1 \ge b$ ложно.
\end{example}

\begin{example}
    $$\langle \Z, \le\rangle, \langle \Q, \le\rangle$$
    $$\forall y \forall z (y < z \rightarrow \exists v (y < v < z))$$
    3 хода:
    $$\begin{array}{c|cc}
        \text{№ хода} & \text{Новатор} & \text{Консерватор} \\
        \hline
        1 & 0 \in \Z & b_0 \in \Q \\
        2 & 1 \in \Z & b_1 \in \Q \\
        3 & \text{Победили, если $b_0 > b_1$, иначе называем $\frac{b_0 + b_1}{2}$} & a \in \Z
    \end{array}$$
    Получилось, что $0 < a < 1$ должно быть верно в $\Z$, а это неправда, тогда Новатор победил.
\end{example}


\begin{example}
    $$\langle \Q, \le\rangle, \langle \R, \le\rangle$$
    $$\forall y \forall z (y < z \rightarrow \exists v (y < v < z))$$
    Выигрывает консерватор, даже если не фиксировать количество ходов. Новатор ставит точку либо совпадающую с уже выбранными, либо больше всех, либо меньше всех, либо внутри интервала. Ввиду плотности $\Q$, консерватор всегда сможет повторить выбор.
\end{example}


\begin{example}
    $$\langle\Z, \le\rangle,  \langle\{0, 1\}\times\Z, \text{лексикографическое сравнение}\rangle$$
    Выигрывает консерватор, но только для фиксированного числа ходов. Разобьем интервалы между точками, которые могут быть, на две группы: малые и большие (бесконечные или конечные, но длина больше, чем $2^l$, где $l$ --- количество ходов до конца игры). Поддерживаем инвариант: между соответствующим точками либо малые и одинаковые, либо большие интервалы (может быть, разные). Новатор не может поделить один интервал на 2 малых. В любом случае мы сможем повторить ход.
\end{example}

\begin{theorem}
    Если в игре Эренфойхта победит Новатор, то интерпретации не являются эквивалентными, и являются таковыми иначе.
\end{theorem}