% !TEX root = ../../../main.tex

% \begin{theorem}[О полноте]
%     $\vdash \phi \Rightarrow \phi$ --- выводима
% \end{theorem}

\subsection{Дополнительные правила вывода}
\begin{definition}
    Вывод из множества посылок $\Gamma$ --- это последовательноть $\phi_1, \phi_2 \dots, \phi_n$, где $\phi_i$ --- либо аксиома, либо $\in \Gamma$, либо получается по m.p.
\end{definition}


\subsubsection{Лемма о Дедукции}
\begin{lemma}[О дедукции]
    $\Gamma \vdash (A \rightarrow B) \Leftrightarrow \Gamma \cup \{A\} \vdash B$
\end{lemma}
\begin{proof}\indent
    \begin{enumerate}
        \item[$\Rightarrow$] $$\left.\begin{array}{l}
           \left.\begin{array}{c}
            \dots  \\
            \dots  \\
            \dots  \\
            A \rightarrow B
        \end{array}\right\}\text{вывод  $A\rightarrow B$ из $\Gamma$}  \\
            A\text{ --- элемент $\Gamma \cup \{A\}$ (посылка)} \\
            B\text{ --- m.p.} \\
        \end{array}\right\}\text{вывод $B$ из $\Gamma \cup \{A\}$}$$
        \item[$\Leftarrow$] Пусть $ \Gamma \cup \{A\} \vdash B$. Тогда существует вывод $\phi_1, \phi_2, \dots \phi_n \eqcirc B$. Каждое $\phi_i$ --- либо аксиома, либо $\in \Gamma$, либо $\eqcirc A$, либо выводится по m.p. Мы докажем по индукции, что $\Gamma \vdash A \rightarrow \phi_i$.
        \begin{enumerate}
            \item $\phi_i$ --- аксиома. Вывод:
                \begin{enumerate}
                    \item $\phi_i$
                    \item $\phi_i \rightarrow (A \rightarrow \phi_i)$ --- аксиома 1.
                    \item $A \rightarrow \phi_i$ --- m.p.
                \end{enumerate}
            \item $\phi_i \in \Gamma$ --- аналогично
            \item $\phi_i = A$. На прошлой лекции мы выводили $A \rightarrow A$.
            \item $\phi_i$ по m.p.: $\exists j, k < i: \phi_k \eqcirc (\phi_j \rightarrow \phi_i)$. По предположению индукции, 
            $$\left.\begin{array}{l}
           \left.\begin{array}{c}
            \dots  \\
            \dots  \\
            \dots  \\
            A \rightarrow \phi_j
        \end{array}\right\}\text{вывод из $\Gamma$}\\
        \begin{array}{c}
             \dots \\
            \dots \\
            A\rightarrow(\phi_j \rightarrow \phi_i)\text{ --- m.p.} \\
        \end{array}  \\
        \end{array}\right\}\text{вывод из $\Gamma$}$$
        $$\Rightarrow (A \rightarrow (\phi_j \rightarrow \phi_i)) \rightarrow ((A \rightarrow \phi_j) \rightarrow (A\rightarrow \phi_i))\text{ --- аксиома 2}$$
        $$\Rightarrow (A \rightarrow \phi_j) \rightarrow (A\rightarrow \phi_i)\text{ --- m.p.}$$
        $$\Rightarrow A\rightarrow \phi_i\text{ --- m.p.}$$
        \end{enumerate}
    \end{enumerate}
\end{proof}

\begin{example}[Силлогизм]
    $$\begin{array}{c}
        (A \rightarrow B) \rightarrow ((B \rightarrow C) \rightarrow (A \rightarrow C))  \\
        \{A \rightarrow B\} \vdash  (B \rightarrow C) \rightarrow (A \rightarrow C) \\
        \{A \rightarrow B, B \rightarrow C\} \vdash (A \rightarrow C) \\
        \{A, A \rightarrow B, B \rightarrow C\} \vdash C
    \end{array}$$
\end{example}
Тогда вывод последней формулы можно провести следующим образом:
\begin{enumerate}
    \item $A$ --- посылка
    \item $A\rightarrow B$ --- посылка
    \item $B$ --- m.p. 1, 2
    \item $B\rightarrow C$ --- посылка
    \item $C$ --- m.p 3, 4
\end{enumerate}


\begin{example}
    $\vdash (A \wedge B) \rightarrow (B \wedge A) \Leftrightarrow (A \wedge B) \vdash (B \wedge A)$`
    \begin{enumerate}
        \item $(A \wedge B)$ --- посылка
        \item $(A \wedge B) \rightarrow B$ --- аксиома 4
        \item $B$ --- m.p. 1, 2
        \item $(A \wedge B) \rightarrow A$ --- аксиома 3
        \item $A$ --- m.p. 1, 4
        \item $B \rightarrow (A \rightarrow (B \wedge A))$ --- аксиома 5
        \item $A \rightarrow (B \wedge A)$ --- m.p. 3, 6
        \item $(B \wedge A)$ --- m.p. 5, 7
    \end{enumerate}
\end{example}

\begin{example}
    $\vdash (A \rightarrow \neg A) \rightarrow \neg A$
    \begin{enumerate}
        \item[1..5.] $A \rightarrow A$
        \item[6.] $(A \rightarrow A) \rightarrow ((A \rightarrow \neg A) \rightarrow \neg A)$ --- аксиома 10
        \item[7.] $(A \rightarrow \neg A) \rightarrow \neg A$ --- m.p. 5, 6
    \end{enumerate}
\end{example}

\subsubsection{Рассуждение от противного}
$$\frac{\Gamma, A \vdash B\ \ \ \ \Gamma, A \vdash \neg B}{\Gamma \vdash \neg A}$$
\begin{proof}
    $$\left.\begin{array}{l}
        \Gamma, A \vdash B \Leftrightarrow \Gamma \vdash A \rightarrow B  \\
        \Gamma, A \vdash \neg B \Leftrightarrow \Gamma \vdash A \rightarrow \neg B
    \end{array}\right\}\Rightarrow\Gamma \vdash \neg A (\text{аксиома 10})$$
\end{proof}

\subsubsection{Законы де Моргана}
я снова умер(

Короче, мы вывели дофига разных законов и порассуждали, зачем они нужны, как их использовать и тд. 

\subsubsection{Правило сечения}
$$\frac{\Gamma\vdash  A\ \ \ \ \ \Delta, A \vdash B}{\Gamma, \Delta \vdash B}$$

\subsubsection{Введение/разбиение конъюнкции}
$$\frac{\Gamma, A\wedge B \vdash  C}{\Gamma, A, B \vdash C}$$
\\
$$\frac{\Gamma, A, B \vdash C}{\Gamma, A\wedge B \vdash  C}$$
\\
$$\frac{\Gamma, \vdash A \wedge B}{\Gamma\vdash  A\ \ \ \ \ \ \Gamma\vdash B}$$
\\
$$\frac{\Gamma\vdash  A\ \ \ \ \ \ \Gamma\vdash B}{\Gamma, \vdash A \wedge B}$$

\subsubsection{Разбор случаев}
$$\frac{\Gamma, A\vdash C\ \ \ \ \ \ \Gamma, B\vdash C}{\Gamma, A\vee B\vdash C}$$

\subsubsection{Правила без названия}
$$\frac{\Gamma \vdash A}{\Gamma\vdash A\vee B}$$
$$\frac{\Gamma \vdash B}{\Gamma\vdash A\vee B}$$

\subsubsection{Правило контрпозиции}
$$\frac{\Gamma, A \vdash B}{\Gamma, \neg B \vdash \neg A}$$