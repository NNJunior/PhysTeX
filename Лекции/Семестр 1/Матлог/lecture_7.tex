% !TEX root = ../../../main.tex

\section{Теорема о полноте}

\begin{theorem}[О полноте]
    Если $\phi$ --- тавтология, то тогда она выводима.
\end{theorem}

\subsection{Первое доказательство}

% $$\frac{\Gamma, A \vdash B\ \ \ \ \ \ \Gamma, \neg A \vdash B}{\Gamma \vdash B}$$

\begin{definition}
    Обозначим $p^\varepsilon = \left\{\begin{array}{l}
        p, \varepsilon = 1  \\
        \neg p, \varepsilon = 0
    \end{array}\right.$
\end{definition}

\begin{lemma}[Базовая Лемма]\indent
\begin{enumerate}
    \item $A, B \vdash A \wedge B$
    \item $\neg A, B \vdash \neg(A \wedge B)$
    \item $A, \neg B \vdash \neg(A \wedge B)$
    \item $\neg A, \neg B \vdash \neg(A \wedge B)$
    \item $A, B \vdash A \vee B$
    \item $\neg A, B \vdash A \vee B$
    \item $A, \neg B \vdash A \vee B$
    \item $\neg A, \neg B \vdash \neg(A \vee B)$
    \item $A, B \vdash A \rightarrow B$
    \item $\neg A, B \vdash A \rightarrow B$
    \item $A, \neg B \vdash \neg(A \rightarrow B)$
    \item $\neg A, \neg B \vdash A \rightarrow B$
    \item $\neg A \vdash \neg A$
    \item $A \vdash \neg (\neg A)$
\end{enumerate}
\end{lemma}

\begin{lemma}[Основная Лемма]
    Пусть $\phi$ --- формула от $n$ переменных $p_1, p_2, \dots p_n$, $(a_1, a_2, \dots a_n) \in \{0, 1\}^n, \phi(a_1, a_2, \dots a_n) = a \in \{0, 1\}$. Тогда $p_1^{a_1}, p_2^{a_2}, \dots p_n^{a_n} \vdash \phi^a$.
\end{lemma}
\begin{proof}
    Индукция по построению формулы.
    \begin{enumerate}
        \item[] \textbf{База:} переменная. $p_i^{a_i} \vdash p_i^{a_i}$.
        \item[] \textbf{Переход:} пусть, например, $\phi = (\xi \wedge \nu)$. $\xi(a_1, a_2, \dots a_n) = a, \nu(a_1, a_2, \dots a_n) = b \Rightarrow \phi(a_1, a_2, \dots a_n) = ab$. По базовой лемме, $p_1^{a_1}, p_2^{a_2}, \dots p_n^{a_n} \vdash \xi^a, p_1^{a_1}, p_2^{a_2}, \dots p_n^{a_n} \vdash \nu^b$. По базовой лемме, $\xi^a, \nu^b \vdash \phi^{ab}$. Запишем три вывода подряд, получим желаемое
    \end{enumerate}
\end{proof}

Тогда если $\phi$ --- тавтология, то при всех $(a_1, \dots a_n): \phi(a_1, a_2, \dots a_n) = 1$, тогда по правилу исчерпвающего разбора случаев, формула $\phi$ будет выводима, т.к. вне зависимости от любой переменной, $\phi$ будет истинна.

\subsection{Второе доказательство}
Пусть $\Gamma$ --- множество пропозициональных формул. Тогда:

\begin{definition}
    $\Gamma$ --- совместно, если при некоторых значениях переменных, все формулы из $\Gamma$ истинны.
\end{definition}

\begin{definition}
    $\Gamma$ --- противоречиво, если из нее можно вывести $\psi, \neg \psi$ одновременно для некоторого $\psi$.
\end{definition}

\begin{definition}
    $\Gamma$ --- полное, если для любого $\phi$ верно $\Gamma \vdash \phi$ или $\Gamma \vdash \neg \phi$.
\end{definition}

\begin{theorem}
    $\Gamma$ совместна $\Leftrightarrow \Gamma$ непротиворечива.
\end{theorem}
\begin{proof}\indent
    \begin{enumerate}
        \item $\Gamma$ --- противоречива, тогда она совместна. Если $\Gamma$ совместна, то все формулы из $\Gamma$ верны на некотором наборе. Тогда если $\Gamma \vdash \phi$, то и $\phi$ --- тоже верна на этом наборе. Аналогично и для $\neg \phi$. Но формулы $\phi, \neg \phi$ не могут быть одновременно истинны, тогда они не могут одновременно выводиться из $\Gamma$.
        \item $\Gamma$ --- несовместна, тогда она противоречива. Пусть $\Delta$ непротиворечива. 
        \begin{lemma}
            $\Gamma$ непротиворечива $\Rightarrow \Gamma \subset \Delta$ для некоторого полного непротиворечивого $\Delta$.
        \end{lemma}
        \begin{proof}[для счетного множества переменных]
            Если переменных счетное число, то и формул тоже счетное число. Пусть $\phi_1, \phi_2, \dots$ --- все формулы. Определим $\Gamma_i$ по индукции:
            $$\Gamma_0 = \Gamma, \Gamma_i = \left\{\begin{array}{l}
                \Gamma_{i-1} \cup \{\phi_i\}\text{, если это непротиворечиво}  \\
                \Gamma_{i-1} \cup \{\neg\phi_i\} \text{, иначе}
            \end{array}\right.$$
            Заметим, что все $\Gamma_i$ --- непротиворечивы, т.к.
            $$\left.\begin{array}{l}
                \Gamma_{i-1} \cup \{\phi_i\}\text{ --- противоречиво } \Rightarrow \Gamma_{i-1}\vdash\neg\phi_i \\
                \Gamma_{i-1} \cup \{\neg\phi_i\} \text{ --- противоречиво }\Rightarrow \Gamma_{i-1}\vdash\phi_i
            \end{array}\right\} \Rightarrow \Gamma_{i-1}\text{ --- противоречиво }$$
            Тогда 
            $$\Delta = \bigcup_{i = 0}^{\infty}\Gamma_i$$
            --- тоже непротиворечиво, т.к. в противном случае, вывод, при помощи котрого мы получили противоречие использует конечное число формул, но тогда он вывелся из какого-то конечного подмножества, но такого не может быть, т.к. все конечные подмножества непротиворечивы.
        \end{proof}
        \begin{lemma}
            $\Delta$ --- полное и непротиворечивое, тогда оно совместное
        \end{lemma}
        \begin{proof}
            $\Delta$ полное, тогда для переменной $p_i$ верно $\Delta\vdash p_i$ или $\Delta\vdash \neg p_i$. Рассмотрим следующий набор значений:
            $$a_i = \left\{\begin{array}{l}
                1, \Delta \vdash p_i  \\
                0, \Delta \vdash\neg p_i
            \end{array}\right.$$
        \end{proof}
    \end{enumerate}
\end{proof}

\begin{proof}[Теоремы о полноте]\indent
    \begin{enumerate}
        \item[] \textbf{Корректность} $\vdash \phi \Rightarrow \{\neg \phi\}$ --- противоречиво $\Rightarrow$ несовместно $\Rightarrow \forall a \neg \phi(a) = 0 \Rightarrow \forall \phi(a) = 1 \Rightarrow \phi$ --- тавтология
        \item[] \textbf{Полнота} $\phi$ --- тавтология $\Rightarrow \{\neg \phi\}$ несовместна $\Rightarrow \{\neg \phi\}$ --- противоречиво, тогда, т.к. $\frac{\neg \phi \vdash B\ \ \ \neg \phi \vdash B}{\vdash \neg(\neg\phi)} \Rightarrow \vdash\phi$
    \end{enumerate}    
\end{proof}