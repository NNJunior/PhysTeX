% !TEX root = ../../../main.tex

$$\text{Формулы }\left\{\begin{array}{l}
    \text{выполнимые}  \\
    \text{опровержимые}
\end{array}\right.$$
Мы хотим выразить различные задачи в терминах выполнимости формул.

\section{Решение задач сведением к выполнимости формулы}
\subsection{Задача про раскраску графа}
Дан граф $G = (V, E)$. Необходимо найти функцию $f: V \rightarrow \{1, 2, 3\}: (u, v) \in E \Rightarrow col(u) \ne col(v)$.

$$\begin{array}{ccc}
    \text{Цвет вершины $u$} & \mapsto & (p_u, q_u) \\
    \hline
    \text{Не существует} & & 00 \\
    1 & & 01 \\
    2 & & 10 \\
    3 & & 11 \\
\end{array}$$
Тогда $\forall p_u, q_u (p_u \vee q_u)$ и ребро может быть проведено между вершинами$(u, v)$, если $(p_u \ne p_v) \vee (q_u \ne q_v)$. Итоговая формула --- конъюнкция условий, и, если она выполнима, то задача имеет решение.

\subsection{Задача про расстановку ферзей}
Доска $n\times n, p_{ij}$ --- истинна, если на $(i, j)$-ой клетке стоит ферзь. Тогда можно записать в терминах $p_{ij}$ утверждение ''ни один ферзь не бьет никакого другого''. Это делается так:
\begin{enumerate}
    \item $(p_{i1} \vee p_{i2} \vee p_{i3} \vee \dots \vee p_{in})$ --- на $k$-ой горизонтали стоит хотя бы один ферзь.
    \item $(\neg p_{ij} \vee \neg p_{ik})$ --- на $i$-ой горизонтали стоит хотя не более одного ферзя.
    \item $(\neg p_{ik} \vee \neg p_{jk})$ --- на $i$-ой вертикали стоит хотя не более одного ферзя.
    \item $(\neg p_{ij} \vee \neg p_{i+k, j+k})$ --- ферзи не бьют друг друга по направлению главной диагонали.
    \item $(\neg p_{ij} \vee \neg p_{i-k, j+k})$ --- ферзи не бьют друг друга по направлению побочной диагонали.
\end{enumerate}

\subsection{Задача о клике}
Дан граф $G, q_{uv} = 1 \Leftrightarrow (u, v) \in E$. Требуется понять, существует ли клика из $k$ вершин?

$$\bigvee_{(v_1, v_2, \dots v_k)}\bigwedge_{i\ne j} q_{v_iv_j} \text{ --- длина} \sim C_n^k $$

Тогда в общей формуле будет порядка 
$$\frac{n!}{k!(n-k)!} > \frac{(n-k)^k}{k!} > \left(\frac{n-k}{k}\right)^k$$
множителей, что очень много. Попробуем по-другому записать условие задачи: введем переменные $p_{iu}$ --- ''вершина $u$ является $i$-ой в клике'', $i \in \{1, \dots k\}$. Тогда накладываются следующие условия:
\begin{enumerate}
    \item $(p_{i1} \vee p_{i2} \vee \dots \vee p_{in})$.
    \item $i \ne j \Rightarrow(\neg p_{iv} \vee \neg p_{jv})$ --- у одной вершины не может быть двух номеров.
    \item $(u, v) \in E \Rightarrow (\neg p_{iu} \vee \neg p_{jv})$ --- внутри клики все вершины соединены.
\end{enumerate}

\subsection{Правило резолюции}
$$\frac{A \vee x\;\;\;B\vee \neg x}{A \vee B}$$

$A \vee B$ называется резольвентой

\subsubsection{Пустой дизъюнкт $\perp$}
$$\frac{x \;\;\;\;\neg x}{\perp}$$
$$\frac{x \vee y \;\;\;\;\neg x \vee \neg y}{y \vee \neg y}$$

Пусть дана КНФ, будем рассматривать ее как набор дизъюнктов.

\begin{proposition}
    Если на данном наборе выполняется $A\vee x, B \vee \neg x$, то и выполняется $A \vee B$.
\end{proposition}
\textbf{Следствие} если исходная формула выполнима, то и все ее резольветны тоже.

\subsection{Метод резолюций}
Строим все новые резольвенты, пока не выведем $\perp$ или не прекратится появление новых дизъюнктов.

\begin{theorem}[О корректности метода резолюций]
    Если исходная формула выполнима, то нельзя вывести $\perp$.
\end{theorem}
\begin{proof}
    Если можно вывести $\perp$, то он будет истинный но он $\equiv 0$.
\end{proof}

\begin{theorem}[О полноте]
    Если нельзя вывести $\perp$, то формула выполнима
\end{theorem}
\begin{proof}
    Разобьем все дизъюнкты на классы. $C_i$ --- дизъюнкты, зависящие только от переменных $p_1, p_2, \dots p_i$. $C_0 = \varnothing$, т.к. $C_0 \subset \{\perp\}$. Будем доказывать по индукции, что выполнены все дизъюнкты из $C_i$.
    \begin{enumerate}
        \item[] \textbf{База:} $C_0$, все дизъюнкты выполнены.
        \item[] \textbf{Переход:} Пусть все формулы из $C_{i-1}$ выполнены на значениях $a_1, a_2 \dots a_{i-1}$. Рассмотрим формулы из $C_i$, которые не будут выполнены на этом наборе. Предположим, что среди них есть и формула с $p_i$ и формула с $\neg p_i$: $p_i \vee D_0, \neg p_i \vee D_1$. Тогда $D_0(p_1, p_2, \dots p_i) = 0 = D_1(p_1, p_2, \dots p_i)$. Но $D_0 \vee D_1$ получается как резольвента этих двух формул, тогда она выполнима, т.к. лежит в $C_{i-1}$, противоречие. Тогда все формулы либо с $p_i$, либо с $\neg p_i$, тогда положим $p_i$ так, чтобы этот множитель выполнялся.
    \end{enumerate}
\end{proof}