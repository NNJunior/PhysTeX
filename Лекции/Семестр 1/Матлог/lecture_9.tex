% !TEX root = ../../../main.tex

\subsubsection{Использование резолюций для проверки тавтологий}
$\phi$ --- тавтология $\Leftrightarrow \neg \phi$ --- противоречие $\Leftrightarrow \neg \phi$ --- не выполнима.
$\phi$ --- тавтология $\Leftrightarrow$ из некоторой задачи о выполнимости КНФ, построенной по $\neg \phi$ можно вывести $\perp$. 

\subsubsection{Преобразование Цейтина}
\begin{example}
    $(p \wedge q) \vee (r \rightarrow \neg s)$. Запишем:
    $$\begin{array}{ccc}
        u = p \wedge q & \mapsto & (\neg u \vee p) \wedge (\neg u \vee q) \wedge (u \vee \neg p \vee \neg q)\\
        t = \neg s & \mapsto & (\neg t \vee \neg s) \wedge (t \vee s)\\
        v = r \rightarrow t & \mapsto & (r \vee v) \wedge (\neg t \vee v) \wedge (\neg v \vee \neg r \vee t)\\
        w = u \vee v & \mapsto & (\neg u \vee w) \wedge (\neg v \vee w) \wedge (\neg w \vee u \vee v)\\
    \end{array}$$
\end{example}

Все формулы могут быть одновременно верны $\Rightarrow \phi$ --- не тавтология. Это равносильно получению 3-КНФ из всех этих формул (в каждой скобке $\le 3$ литералов). На 2-КНФ метод резолюций работает за $O(n)$, но на 3-КНФ уже за экспоненциально долгое время. Хотелось бы придумать алгоритм, который работает быстрее. Эта проблема, также известная как проблема $P$ vs $NP$ на данный момент на имеет решения

\section{Языки первого порядка}
\subsection{Алфавит}
\begin{enumerate}
    \item Индивидные переменные $x, y, z, \dots$
    \item Функциональные символы (с указанием числа аргументов (например, ''$-$'' может быть и бинарным, и унарным)) $f^{(1)}, g^{(2)}, \dots$, в том числе константные символы (функциональные символы валентности 0) $0, e, \pi, \dots$
    \item Предикатные символы (с указанием валентности) $P^{(1)}, Q^{(3)}, \dots$
    \item $\neg, \wedge, \vee, \rightarrow$
    \item Кванторы $\forall, \exists$
    \item Служебные символы ''$($'', ''$,$'', ''$)$''
\end{enumerate}

Символы из пунктов 2, 3 называются сигнатурами

\subsection{Термы}
\begin{enumerate}
    \item $x$ --- переменная, тогда $x$ --- терм
    \item $c$ --- константный символ, то $c$ --- терм
    \item $t_1, t_2, \dots t_k$ --- термы, $f$ --- функциональный символ валентности $k$, то $f(t_1, t_2, \dots t_k)$ --- терм.
\end{enumerate}
\subsection{Формулы}
\begin{center}
    \textsc{Формулы --- это не термы, нумерация сквозная, чтобы у каждого правила был свой номер!!!}
\end{center}
\begin{enumerate}
    \setcounter{enumi}{5}
    \item $t_1, t_2, \dots t_k$ --- термы, $P$ --- предикатный символ валентности $k$, то $P(t_1, t_2, \dots t_k)$ --- формула
    \item $\phi$ --- формула, тогда $\neg \phi$ --- формула
    \item $\phi, \psi$ --- формулы тогда $(\phi * \psi)$, где ''$*$'' --- это одна из бинарных логических связок, тоже формула.
    \item $\phi$ --- формулы тогда $\exists x\phi, \forall x\phi$ --- тоже формула.
\end{enumerate}
Тогда нам не запрещены записи вида $\exists x \forall xP(x)$ или $\exists x P(y)$. Часто добавляют отдельный вид атомарных формул с равенством $t_1 = t_2$. Но при таком определении появляются вопросы, например:
$$\exists x x < y$$
Что такое $y$? Что такое ''$<$''? Откуда берем $x$?

\subsection{Интерпретация}
$M$ --- непустое множество --- носитель интерпретации. 
\begin{enumerate}
    \item $f$ --- функциональный символ валентности $k > 0$. $[f]: M^k \rightarrow M$. 
    \item $c$ --- константный символ, тогда $[c] \in M$.
    \item $P$ --- предикатный символ валентности $k$, тогда $[P]: M^k \rightarrow \{0, 1\}$.
\end{enumerate}

Также, пусть $Var$ --- множество переменных. Оценкой называется произвольная функция $\pi: Var \rightarrow M$. Тогда, если заданы носитель интерпретации и оценка, то определены значения всех термов и формул.
\begin{proof}\indent
    \begin{enumerate}
        \item $t \eqcirc x \Rightarrow [t](\pi) = \pi(x)$
        \item $t \eqcirc C \Rightarrow [t](\pi) = [C]$
        \item $t \eqcirc f(t_1, t_2, \dots t_k) \Rightarrow [t](\pi) = [f]([t_1](\pi), [t_2](\pi), \dots [t_k](\pi))$
        \item $\phi \eqcirc P(t_1, t_2, \dots t_k) \Rightarrow [t](\pi) = [P]([t_1](\pi), [t_2](\pi), \dots [t_k](\pi))$
        \item $\phi \eqcirc \neg \Rightarrow \psi [\phi](\pi) = neg([\psi](\pi))$
        \item Для бинарных связок все как и в ПФах.
        \item Обсудим потом
    \end{enumerate}
\end{proof}