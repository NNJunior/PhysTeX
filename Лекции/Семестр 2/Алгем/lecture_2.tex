% !TEX root = ../../../main.tex

\section{Неприводимые многочлены}
\begin{definition}
    Пусть \(F\) --- поле, \(F[x]\) --- кольцо многочленов над \(F\). Многочлен \(P \in F[x], \deg P > 0\) называется наприводимым, если \(AB = P \Ra \deg A = 0 \vee \deg B = 0\). Иначе говоря, многочлен неприводим над полем \(F\), если он не раскладывается в произведение многолченов более низких степеней.
\end{definition}

\begin{example}
    \(x^2 + 1 \in \R[x]\) --- неприводим. Очев, т.к. не имеет корней.
\end{example}
\begin{example}
    \(x^2 + 1 = (x - i)(x + i) \in \Cm[x]\)
\end{example}
\begin{proposition}
    \(P\) --- неприводимый, тогда \(\forall A: (A, P) = \left[\begin{array}{l}
        \sim 1 \\
        \sim P
    \end{array}\right.\)
\end{proposition}
\begin{proof}
    По-другому быть не может, т.к. у \(P\) нет делителей, кроме 1 и самого себя.
\end{proof}
\begin{lemma}[Евклида]
    Пусть \(P\) --- неприводиный многочлен, \(AB \vdots P \Ra A \vdots P \vee B \vdots P\).
\end{lemma}
\begin{proof}
    От противного, тогда \(A \not\vdots P, B \not\vdots P\). Тогда по теореме о представлении НОДа в виде линейной комбинации:
    \[(A, P) = u_1A + u_2P = 1\]
    \[u_1AB + u_2PB = B \vdots P\]
    Противоречие
\end{proof}

\begin{theorem}[Основная Теорема Арифметики]
    Пусть \(A\) --- ненулевой многочлен из \(F[x], F\) --- поле. Тогда \(\exists! P_1, P_2 \dots P_n\) c точностью до перестановки множителей и домножения на константу, где \(P_i\) --- неприводим и \(A = \Pi_{i = 1}^n P_i\).
\end{theorem}
\begin{proof}\indent
    \begin{enumerate}
        \item[] \textbf{Существование.} По индукции: либо он неприводим и очев, либо нет, тогда разложим и для каждого множителя разложим его.
        \item[] \textbf{Единственность.} Пусть не единственно, будем тогда сокращать на \(P_1, P_2, \dots\). Получим, что в разложении должны содержаться многочлены, пропорциональные \(P_1, \dots P_n\) соответственно
    \end{enumerate}
\end{proof}

\begin{corollary}
    Если \(A \vdots P\), то разложение \(A\) является подмножеством разложения \(P\)
\end{corollary}

\subsection{Корни многочленов}

\begin{definition}
    Многолчен \(P\) имеет корень \(c\) кратности \(k\), если \(P\vdots (x-c)^k, P \not\vdots(x-c)^{k+1}\)
\end{definition}

\begin{definition}
    Если многочлен раскладывается в произведение линейных множителей над полем \(F\), то он называется линейно факторизуемым над ним.
\end{definition}

\begin{note}
    Основная Теорема Арифметики неверна (разложение может быть не единственным) для случаев, когда \(F\) --- коммутативное кольцо.
\end{note}

\subsection{Основная Теорема Алгебры}

На лекции было миллион лемм и вспомогательных утвеждений, но я просто вставлю доказательство из лекции по матану, потому что я так могу (и потому что доказательство идейно не отличается от него).

\begin{theorem}[Больцано-вейерштрасса]
    Пусть $\{z_n\}$ ограничена, то есть $\exists C > 0: \forall n (|z_n| \le C)$. Тогда у нее существует сходящаяся подпоследовательность
\end{theorem}
\begin{proof}
    По обычной теореме Больцано-Вейерштрасса, ищем подпоследовательность, действительная часть которой имеет предел. В ней выбираем последовательность, мнимая часть которой имеет предел. Получили.
\end{proof}

\begin{definition}
    Функция $f: E \subset \Cm \rightarrow \Cm$ непрерывна в точке $z_0$, если $\forall \{z_n\} \subset E (z_n \rightarrow z_0 \Rightarrow f(z_n) \rightarrow f(z_0))$.
\end{definition}

\begin{proposition}
    Пусть $f: \{|z| \le R\} \rightarrow \R$ непрерывна на $\Cm$. Тогда $\exists z_0, |z_0| \le R, \inf_{|z| \le R} f(z) = f(z_0)$.
\end{proposition}
\begin{proof}
    $$m = \inf_{|z| \le R} f(z)$$
    Рассмотрим $r_n \rightarrow m, r_n > m$. $\exists z_n, |z_n| \le R, m \le f(z_n) \le r_n$
    В частности, $f(z_n) \rightarrow m$. При этом, $\{z_n\}$ --- ограничена, $\Rightarrow \exists z_{n_k} \rightarrow  z_0 \Rightarrow |z_0| \le R$. В частности $||z| - |z_0|| \le |z - z_0|$. В силу непрерывности $f$ в $z_0$: $f(z_{n_k}) \rightarrow f(z_0), f(z_{n_k}) \rightarrow m \Rightarrow m = f(z_0)$.
\end{proof}

\begin{theorem}
    Пусть \(f \in \Cm[z], \deg f > 0\). Тогда \(f\) имеет корень.
\end{theorem}
\begin{proof}
    \begin{enumerate}
        \item Покажем, что $\exists z_0 \in \Cm \inf_{z \in \Cm} |P(z)| = |P(z_0)|$. Для начала возьмем $R \ge 1$. 
        $$\left|\sum_{k = 0}^{n-1} a_kz^k\right| \le \sum_{k = 0}^{n-1} |a_k||z|^k \le |z|^n\sum_{k = 0}^{n-1}|a_k| = A$$
        Теперь рассмотрим $|z| \ge \frac{2A}{|a_n|} \Rightarrow A|z|^{n-1} \le \frac{1}{2}|a_n||z|^n$. Тогда $|P(z)| \ge |a_nz^n| - \left|\sum_{k = 0}^{n-1}a_kz^k\right| = \frac{1}{2}|a_n|z^n$. Возьем радиус $R = \max\left\{1, \frac{2A}{|a_n|}, \sqrt[n]{\frac{2|a_0|}{|a_n|}}\right\}$. Тогда при $|z| \ge R$ выполнено $|P(z)| \ge |P(0)|$, поэтому $\inf_{\Cm}|P(z)| = \inf_{|z| \le R}|P(z)|$. Но тогда найдется такое $|z_0| \le R$, что у нас $\inf_{\Cm}|P(z)| = |P(z_0)|$
        \item Докажем, что если $P(z_0) \ne 0$, то $\exists z_* \in \Cm |P(z_*)| < |P(z_0)|$. Рассмотрим многочлен $Q(z) = \frac{P(z + z_0)}{P(z_0)}$. Тогда $Q(0) = 1$. Обозначим через $\alpha_k$ --- наименьший коэффициент $Q$, отличный от 0 и $k \ge 1$. $Q(z) = 1 + \alpha_kz^k + \dots$. Возьмем $z_1 \in \Cm$, $\alpha_kz_1^k = -1$, пусть $t \in (0, 1)$. $Q(tz_1) = 1 - t^k + t^{k+1}\phi(t)$, $\phi(t)$ --- многочлен степени $n - k - 1$. $C$ --- наибольший из модулей коэффициентов $\phi(t)$, тогда $|\phi(t)| \le C(n - k)$. Тогда
        $$Q(tz_1) < 1 - t^k|\phi(t)| \le 1 - t^k(1 - tC(n - k))$$
        Рассмотрим произвольное $t \in \left(0, \frac{1}{C(n - k)}\right)$. Тогда $|Q(tz_1)| < 1$. Но тогда при $z_* = tz_1$ верно, что $|P(z_*)| < |P(z_0)|$
        
        Но тогда, точка $z_0$ (из первого пункта) такова, что $P(z_0) = 0$.
    \end{enumerate}
\end{proof}

\subsection{Следствия из основной теоремы алгебры}
\begin{definition}
    Поле \(F\) называется алгебраически замкнутым, если \(\forall f \in F[x], \deg f > 0\) он имеет хотя бы один корень.
\end{definition}
\begin{corollary}
    Поле \(\Cm\) --- алгебраически замкнуто
\end{corollary}
\begin{corollary}
    Любой многолчен из \(\Cm\) --- линейно факторизуем.
\end{corollary}
\begin{corollary}
    Любой многолчен из \(\R\) раскладывается в произведение многочленов 1 и 2 степени
\end{corollary}
\begin{proof}
    Пусть \(c \notin \R\) --- корень \(f\). Тогда \(\overline{c}\) --- тоже. Но тогда \(f \vdots (x - c)(x - \overline{c})\)
\end{proof}

\subsection{Формальная производная}
\begin{definition}
    \[(a_nx^n + \dots + a_1x + a_0)' = na_nx^{n-1} + \dots + 2a_2x + a_1\]
\end{definition}
\begin{note}
    Все свойства обычной производной верны
    \begin{enumerate}
        \item \((\alpha P + \beta Q)' = \alpha P' + \beta Q'\)
        \item \((PQ)' = P'Q + PQ'\)
        \item \(\left(P_1 P_2 P_3 \dots P_{n-1} P_n\right)' = P_1' P_2 P_3 \dots P_{n-1} P_n + P_1 P_2' P_3 \dots P_{n-1} P_n + \dots P_1 P_2 P_3 \dots P_{n-1} P_n' \) 
        \item \((P^n)' = nP^{n-1}P'\)
    \end{enumerate}
    \begin{proof}
        \begin{enumerate}
            \item Доказательство по определению: \\ \(\left(\sum_{i = 1}^k \alpha_i x^i\right)' + \left(\sum_{i = 1}^i \beta_k x^{i - 1}\right)' = \left(\sum_{i = 1}^k i \alpha_i x^{i - 1}\right) + \left(\sum_{i = 1}^i i \beta_i x^{i - 1}\right)' = \\ = \sum_{i = 0}^k (\alpha_i + \beta_i) \cdot i x^{i - 1} = \left(\sum_{i = 0}^k (\alpha_i + \beta_i) x^i\right)'\)
            \item Раскроем скобки:
            \(\left(\sum_{i = 0}^k \alpha_i x^i \cdot Q(x)\right)' = \left(\sum_{i = 0}^k \sum_{j = 0}^l \alpha_i \beta_j x^{i + j}\right)' = \\ = \sum_{i = 0}^k \sum_{j = 0}^l (i + j)\alpha_i \beta_j x^{i + j - 1} = \sum_{i = 0}^k \sum_{j = 0}^l i \alpha_i \beta_j x^{i + j - 1} + \sum_{i = 0}^k \sum_{j = 0}^l j \alpha_i \beta_j x^{i + j - 1} = \\ = P'Q + PQ'\)
            \item Индукция: \textit{База} $n = 2 \to \text{см п. 2}$. \textit{Переход:} Объединим $P_{n-1}P_n$ в $Q \to$ 2 раза п. 2.
            \item Применяем п. 4 для $P_1 = P_2 = \dots = P_{n-1} = P_n = P$
        \end{enumerate}
    \end{proof}
\end{note}