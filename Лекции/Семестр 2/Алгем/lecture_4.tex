% !TEX root = ../../../main.tex

\subsection{Структура линейного оператора}
\begin{definition}
    \(\chi_A(\lambda) = \det(A - \lambda E)\) --- характеристический многочлен линейного оператора \(A\).
\end{definition}

\begin{proposition}[О свойствах характеристического многочлена]\indent
    \begin{enumerate}
        \item \(\chi_A(\lambda) = 0, \lambda \in F\), тогда \(\lambda\) --- собственное значение оператора \(A\).
        \item \(\chi_A(\lambda)\) не зависит от выбора базиса, в котором записывается \(A\).
    \end{enumerate}
\end{proposition}
\begin{proof}\indent
    \begin{enumerate}
        \item \(Ax = \lambda x \Lra (A - \lambda E)x = 0\) имеет нетривиальное решение \(\Lra \det(A - \lambda E) = 0\)
        \item
        \(\det(SAS^{-1} - \lambda E) = \det(SAS^{-1} - \lambda SES^{-1}) = \det S \det S^{-1} \det (A - \lambda E) = \det(A - \lambda E)\)
    \end{enumerate}
\end{proof}

\begin{definition}
    Линейный оператор называется диагонализируемым, если существует базис, в котором его матрица является диагональной
\end{definition}

\begin{theorem}[Критерий диагонализируемости]
    Пусть \(\phi: V \ra V\) --- линейный оператор, \(\lambda_1, \lambda_2 \dots \lambda_k\) --- все попарно различные корни характеристического многочлена. Тогда следующие утверждения эквивалентны:
    \begin{enumerate}
        \item \(\phi\) --- диагонализируема
        \item В \(V\) суещствует бащис, состоящий из собственных векторов для \(\phi\)
        \item \(V = V_{\lambda_1} \oplus V_{\lambda_2} \oplus \dots \oplus V_{\lambda_k}\)
    \end{enumerate}
\end{theorem}
\begin{proof}\indent
    \begin{enumerate}
        \item[\(1 \Ra 2\)] Рассмотрим базис, в котором матрица имеет диагоанльный вид \(\mathfrak{E}\). Но тогда \(\forall e \in \mathfrak{E}: Ae = \lambda_i e\) для некоторого \(i\). Но тогда этот базис состоит из собственных векторов.
        \item[\(2 \Ra 3\)] Рассмотрим отдельно базисы, отвечающие собственным значениям \(\lambda_1, \lambda_2 \dots \lambda_k\). Они образуют базисы пространств \(V_{\lambda_1}, V_{\lambda_2}, \dots V_{\lambda_k}\). Но тогда \(V = V_{\lambda_1} \oplus V_{\lambda_2} \oplus \dots \oplus V_{\lambda_k}\)
        \item[\(3 \Ra 1\)] Рассмотрим объединение базисов этих пространств. Т.к. каждый вектор полученного базиса будет собственным, в каждой строке и каждом столбце матрицы будет записано ровно одно число. Но тогда можно поменять местами векторы базиса так, чтобы матрица была диагональной
    \end{enumerate}   
\end{proof}

\subsection{Алгебраическая и Геометрическая кратности}
\begin{definition}
    Алгебраической кратностью собственного значения \(\lambda_0\) многочлена \(\chi_A(\lambda)\) называется кратность его как корня данного многочлена.
\end{definition}
\begin{definition}
    Геометрической кратностью собственного значения \(\lambda_0\) называется \(\dim V_{\lambda_0}\)
\end{definition}

\begin{proposition}
    Пусть \(\phi: V \ra V\) --- линейный оператор, \(U \le V\), \(U\) инвариантно относительно \(phi, \psi = \phi|_U \in \mathcal{L}(U)\). Тогда \(\chi_\phi \vdots \chi_\psi\)
\end{proposition}
\begin{proof}
    \[\mathfrak{E} = (\underbrace{\underbrace{e_1, e_2, \dots e_k}_{\text{Базис } U}, e_{k + 1}, \dots e_n}_{\text{Базис }V})\]
    Но тогда:
    \[A_\phi = \left(\begin{array}{c|c}
        A_\psi & B \\ 
        \hline
        0 & C \\
    \end{array}\right) \Ra \chi_\phi(\lambda) = |A_\phi - \lambda E| = |A_\psi - \lambda E||C - \lambda E| = \chi_\psi(\lambda)\chi_C(\lambda)\]
\end{proof}

\begin{corollary}
    Пусть \(\lambda\) --- собственное значение \(\phi: V \ra V\). Тогда \(geom(\lambda) \le alg(\lambda)\)
\end{corollary}
\begin{proof}
    Рассмотрим \(V_\lambda, \psi = \phi|_{V_\lambda}\). Тогда
    \[\psi = \left(\begin{array}{cccc}
        \lambda & 0 & \dots & 0\\
        0 & \lambda & \dots & 0\\ 
        \vdots & \vdots & \ddots & \vdots\\
        0 & 0 & \dots & \lambda \\
    \end{array}\right)\]
    И тогда \(\chi_\psi(t) = (\lambda - t)^k\), где \(k = \dim V_{\lambda}\)
    По вышедоказанному утверждению, \(\chi_\phi \vdots \chi_\psi \Ra \chi_\phi(t) \vdots (\lambda - t)^k \Ra alg(\lambda) \ge geom(\lambda)\)
\end{proof}

\begin{corollary}
    Если \(\chi_\phi\) --- не линейно факторизуем, то \(\phi\) --- не диагонализируем
\end{corollary}

\begin{theorem}
    Пусть \(\phi: V \ra V\) --- линейные оператор. Тогда он диагонализируем тогда и только тогда, когда 
    \begin{enumerate}
        \item \(\phi\) --- линейно факторизуем над \(F\)
        \item \(\forall i\;\; alg(\lambda_i) = geom(\lambda_i)\)
    \end{enumerate}
\end{theorem}
\begin{proof}\indent
    \begin{enumerate}
        \item[\(\Ra\)] Т.к. \(V = V_{\lambda_1} \oplus V_{\lambda_2} \oplus \dots \oplus V_{\lambda_k}\), то \(\dim V = \dim V_{\lambda_1} + \dim V_{\lambda_2} + \dots + \dim V_{\lambda_k}\). Но тогда 1 и 2 верны, т.к. \(alg(\lambda_i) \ge geom(\lambda_i)\), и, при этом \(\sum alg(\lambda_i) = \sum geom(\lambda_i)\).
        \item[\(\La\)] В таком случае \(\sum alg(\lambda_i) = \sum geom(\lambda_i)\). Но тогда \(\dim V = \sum geom(\lambda_i)\). Но тогда \(V = V_{\lambda_1} \oplus V_{\lambda_2} \oplus \dots \oplus V_{\lambda_k}\)
    \end{enumerate}
\end{proof}

\begin{definition}
    Жорданова клетка порядка \(n\), отвечающая собственному значению \(\lambda\) --- это матрица:
    \[J_n(\lambda) = \left(\begin{array}{cccccc}
        \lambda & 1 & 0 & \dots & 0 & 0 \\
        0 & \lambda & 1 & \dots & 0 & 0 \\
        0 & 0 & \lambda & \dots & 0 & 0 \\
        \vdots & \vdots & \vdots & \ddots & \vdots & \vdots \\
        0 & 0 & 0 & \dots & \lambda & 1 \\
        0 & 0 & 0 & \dots & 0 & \lambda \\
    \end{array}\right)\]
\end{definition}

\begin{note}
    Линейной факторизуемости \(\chi_\phi\) недостаточно, чтобы утверждать диагонализируемость \(\phi\). Например, \(J_n(\lambda)\) не диагонализируема, т.к. \(J_n(\lambda) - \lambda E\) имеет размерность решений \(2\), но \(\chi_{J_n(\lambda)}(t) = (\lambda - t)^n\).
\end{note}

\begin{proposition}
    \(\phi: V \ra V, \phi_\lambda = \phi - \lambda E\). Тогда следующие утверждения эквивалентны:
    \begin{enumerate}
        \item Подпространство \(U \le V\) инвариантно относительно \(\phi\).
        \item \(\exists \lambda \in F: U\) --- инвариантно относительно \(\phi_\lambda\).
        \item \(\forall \lambda \in F: U\) --- инвариантно относительно \(\phi_\lambda\).
    \end{enumerate}
\end{proposition}
\begin{proof}\indent
    \begin{enumerate}
        \item[\(1 \Ra 2\)] очевидно, \(\lambda = 0\)
        \item[\(2 \Ra 3\)] \((\phi - \lambda E)x = \mu x \Ra (\phi - \lambda E - \lambda_1 E)x = (\mu - \lambda_1) x\)
        \item[\(3 \Ra 1\)] Тоже очевидно
    \end{enumerate}
\end{proof}

\begin{proposition}
    Пусть \(\phi: V \ra V\) --- линейный оператор и \(\chi_\phi(t)\) линейно факторизуем. Тогда у \(\phi\) найдется \(n - 1\) мерное подпространство.
\end{proposition}

Я СДАЮСЬ
он победил