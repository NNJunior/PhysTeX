% !TEX root = ../../../main.tex

\subsection{Алгоритм Эдмондса}

\textit{Алгоритм сжатия соцветий - для поиска максимального паросочетания в двудольном графе}

\textbf{\underline{Напоминание:}}
С прошлой лекции знаем, что по теореме Бержа паросочетание является максимальным тогда и только тогда, когда в графе нет увеличающих путей. Будем пользоваться этим утверждением.

\underline{Мотивация:} Для случай, когда в графе нет нечетных циклов, находить паросочетание мы научились (см алгоритм Куна в двудольном графе), теперь постараемся это сделать для других случаев.

\begin{definition}
  Пусть $M$ - паросочетание. Тогда $d(M)$ - количество \underline{ненасыщенных} вершин относительно $M$.
\end{definition}
\begin{note}
 Понятно, что в максимальном паросочетании $d(M)$ минимальное.
\end{note}
\begin{lemma}{(Татт, Берж)}
  $\forall$ паросочетания $M$:
  $$d(M) \ge max_{R \subset V} (C_{odd}(G \setminus R) - |R|)$$
  \textit{$C_{odd}$ - количество компонент связности с нечетным числом вершин}
\end{lemma}

\begin{proof}
Чтобы насытить все вершины из какой-то нечетной компоненты, надо взять хотя бы одно ребро и хотя бы 1 веошину из $R$. Если $C_{odd}(G - R) > |R|$, то по крайней мере $C_{odd}(G - R) - |R|$ не могут быть покрыты целиком.
\end{proof}

\begin{definition}
  Чередующееся дерево (aternating tree):
  \begin{enumerate}
    \item Корень - ненасыщенная вершина
    \item Четная глубина вершина $\to$ синий цвет
    \item Нечетная глубина $\to$ красный цвет
    \item Из красной вершины "вниз" уходит только 1 ребро из паросочетания
  \end{enumerate}
\end{definition}

\textbf{Алгоритм построения дерева}:
\begin{enumerate}
  \item Берем ненасыщенную вершину, назваем ее корнем и красим в синий
  \item Берем ребра из текущей вершины (всегда вершина на этом этапе синяя). Если есть ребро в насыщенную вершину, "подгружаем" ребро из паросочетания и запускаемся от появившейся новой синей вершины
  \item Если нет ребра из синей вершины в насыщенную, то алгоритм для этой вершины закончен, продолжаем строить от других синих вершин
  \item Когда алгоритм заканчивается, берем новую ненасыщенную вершину и строим от нее дерево
\end{enumerate}
\begin{proposition}
  Если в процессе построения дерева найдена ненасыщенная вершина, то найден увеличивающий путь.
\end{proposition}

\begin{proposition}
  Пусть ребра, исходящие из синих вершин, ведут только в красные. Тогда увеличивающего пути нет $\then$ return
\end{proposition}

\begin{proof}
  Пусть всего построили $k$ чередующихся деревьев. Тогда синих вершин = количество красных + $k$ (поскольку в каждом дереве количество синих равно количеству красных + 1).

  Удали все красные вершины из графа, тогда по лемме Татта-Бержа получаем, что $d(M) \ge k$. А значит, улучшить $M$ нельзя.
\end{proof}

\begin{proposition}
  Если между синими есть хотя бы одно ребро между синими вершинами. Тогда можно перейти к графу с меньшим числом вершин.
\end{proposition}

\begin{proof}
Понятно, что ребро должно быть внутри одного дерева (иначе во время построения одного из деревьев мы бы рассматривали синюю вершину как неиспользованную, поэтому мы бы просто увеличили первое дерево). Несложно увидеть, что в графе будет нечетный цикл, начинающийся в синей вершине (общем предке) и заканчивающийся на двух связных синих вершинах (далее, соцветие). Сжимаем соцветие в одну вершину и запускаем алгоритм заново.
\end{proof}

\begin{proposition}
  Пусть $G'$ отличается от $G$ сжатием одного соцветия. Тогда $\exists$ увеличивающий путь в $G$ $\Longleftrightarrow$ $\exists$ увеличивающий путь в $G'$. 
\end{proposition}

\begin{proof}
  Ну вроде понятно...
\end{proof}

\textit{\underline{Асимптотика:}} $O(n^2 m)$

\begin{note}
  Можно свести к $O(nm)$, если после сжатия соцветия продолжаем строить чередующиеся деревья без явного построения нового графа. 
\end{note}

\textit{P.S. Проанализиров алгоритм, можно понять, что в теореме Татта-Бержа знак равенство, ведь алгоритм заканчивается именно в момент равенства} 