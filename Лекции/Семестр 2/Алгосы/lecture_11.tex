% !TEX root = ../../../main.tex


\section{Потоки}

\begin{definition}
  Сеть (транспортная сеть) - кортеж $(G, s, t, c)$, где 
  \begin{enumerate}
    \item $G$ - ориентированный граф
    \item $s, t$ - вершины графа, $s$ называется истоком, $t$ - стоком.
    \item $c$ : $E \to \Z \ge 0$ - ограничение на ребро ($capacity$) 
  \end{enumerate}
\end{definition}

\begin{definition}
  Поток в сети $f: V \times V \to \Z$ удовлетворяет свойствам:
  \begin{enumerate}
    \item $\forall(u, v) f(u, v) \le c(u, v)$
    \item $\forall (u, v): f(u, v) = -f(v, u)$ - \textit{антисимметричность потока}
    \item $\forall v \in V \setminus \{s, t\}: \sum_{u \in V} f(v, u) = 0$ 
  \end{enumerate} 
\end{definition}

\textit{Последнее условие можно понимать, как правило "сумма втекающего потока равна сумме вытекающего"}

\begin{note}
По умолчанию считаем, что для всех пар вершин без ребер пропускная способность $c$ равна 0.
\end{note}

\begin{definition}
  Величина потока $f$: $$ |f| = \sum_{v \in V} f(s, v)$$
\end{definition}

\begin{definition}
  Пусть $G$ - сеть, $f$ - поток в ней. Остаточная сеть $$G_f: \forall(u, v) c_f(u, v) = c(u, v) - f(u, v)$$
  
  \textit{$c_f$ - остаточная пропускная способность ребра.}

  \textit{Если появляется ребро с $c = 0$, то его в граф удобно не добавлять.}
\end{definition}

\begin{lemma}
  $|f| = max \Longleftrightarrow$ в $G_f$ нет пути из $s$ в $t$. 
\end{lemma}

\begin{definition}
Пусть $G$ - сеть, $(S, T)$ - разрез, если $s \in S, t \in T$; $S \sqcup = V(G)$.  

\textbf{Величина разреза:} $$c(S, T) = \sum_{u \in S, v \in T} c(u, v)$$

\textbf{Величина потока через разрез:}$$f(S, T) = \sum_{u \in S, v\in T}$$
\end{definition}

\begin{proposition}
  Если $(S, T) - $разрез, а $f$ - поток, то $f(S, T) = |f|$
\end{proposition}

\begin{proof}
\text{}


  $f(S, T) = f(S, V) - f(S, S) = f(S, V) = f(\{s\}, V) + f(S\setminus \{s\}, V) = |f| + 0 = |f|$
\end{proof}

\begin{theorem}{(Форда-Фалкерсона)} Следующие условия эквивалентны:
  \begin{enumerate}
    \item $|f| - max$
    \item в $ G_f$ нет пути из s в $t$
    \item $\exists (S, T): c(S, T) = |f|$
  \end{enumerate}
\end{theorem}

\begin{proof}
  \text{}


  \begin{enumerate}
    \item[$1 \then 2$] Если поток максимален, но в $G_f$ есть путь, то пустим по нему еще поток, увеличивая размер $|f|$
    \item[$2 \then 3$] $S$ - все вершины, достижимые из $s$ в $G_f$ $T = V \setminus S$, то есть недостижимые 
    $$c(S, T) = \sum_{u \in S, v \in T} c(u, v) = \sum_{u \in S, v \in T} f(u, v) = |f|$$
    \item[$3 \then 1$] Раз любой поток меньше любого разреза $|f|_{max} = c_{min}(S, T) $
  \end{enumerate}
\end{proof}

\subsection{Алгоритм Форда-Фалкерсона}

Пока в $G_f$ есть путь из $s$ в $t$. Вдоль этого пути пускает $min_e c_f(e)$ потока, насыщая какое-то ребро.

\textbf{Асимптотика}: $O(F \cdot E)$

Однако такой алгоритм может работать неоптимально, например, находить много раз путь размера 3, а не 1000. 

\subsection{Алгоритм Эдмондса-Карпа}

\textbf{Основная идея:} $dfs \to bfs$

Мы будем искать всегда кратчайший по количеству ребер путь из в $s$ в $t$

\textbf{Асимптотика:} O$(VE^2)$

\begin{lemma}
  Пусть $G_f$ и $G_{f'}$ - две последовательных состояния остаточной сети в последнем алгоритме. Обозначим за $d(v) = dist(s, v)$ в $G_f$, аналогично определим $d'(v)$.

  Тогда $\forall v: d(v') \ge d(v)$
\end{lemma}

\begin{proof}
  Пусть это не так. Среди всех вершин с $d(v') < d(v)$ возьмем ту, в которой значение $d(v')$ минимально. Тогда

  $d'(u) + 1 = d'(v) \then d'(u) < d'(v) \then d'(u) \ge d(u)$ из минимальности $d'(v)$

  \begin{enumerate}
    \item[Случай 1] Ребро $(u, v)$ было в $G_f$. 
    \begin{proof}
      $d(v) \le d(u) + 1 \le d'(u) + 1 = d'(v)$. Противоречие
    \end{proof}

    \item[Случай 2] Ребра $(u, v)$ не было в $G_f$. 
    \begin{proof}
      Это значит, что при переходе от $G_f$ до $G_{f'}$ пускается поток вдоль обратного ребра.

      $d(v) = d(u) - 1 \le d'(u) - 1 = d'(v) - 2$. Противоречие
    \end{proof}
  \end{enumerate}
\end{proof}

\begin{lemma}
  Каждое ребро в алгоритме ЭК насыщается не больше $O(V)$. 
\end{lemma}

\subsection{Техника масштабирования}

\textbf{Основная идея:} ищем поток величины $2^k$

Пусть $C$ - максимальная $capacity$ в исходном графе.

Далее, пишем цикл:
\begin{center}
  \begin{verbatim}
    for k = log C, ..., 0:
      все c_f -> [c_f / 2^k]
      находим поток в этом графе
      разжимаем обратно все ребра, умножая на 2^k
  \end{verbatim}
\end{center}

Таким образом, асимптотика можно преобразовать до $O(E^2 \log{C})$

\begin{lemma}
  Пусть $F - max$ поток в исходной сети. $F_k$ - суммарный найденный поток после итераций $\log{C} ... k$. Тогда $F \le F_k + 2^k \cdot E$
\end{lemma}
\begin{lemma}
  $\forall k$ алгоритм при поиске потока находит не больше $2E$ путей.
\end{lemma}
