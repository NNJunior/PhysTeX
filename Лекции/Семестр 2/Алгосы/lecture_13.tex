% !TEX root = ../../../main.tex

\subsection{Глобальный минимальный разрез}\subsubsection{Алгоритм Штор-Вагнера}

\begin{note}
  Как мы знаем по прошлым теоремам, минимальный разрез между вершинами $s, t$ в графе равен максимальному потоку, протекающему между ними. Поэтому можно запустить Эдмондса-Карпа $n^2$ раз и выбрать минимальный среди них. Но постараемся улучшить асимптотику.
\end{note}

Пусть мы нашли maxFlow для некоторых $s, t$. Как они могут быть размещены относительно искомого глобального разреза? 

\begin{enumerate}
  \item Либо $s, t$ содержатся в одной доли относительно разреза.
  \item Либо они лежат в разных долях.
\end{enumerate}

Заметим, что во втором случае максимальный поток между $s, t$ будет равен минимальному разрезу в графе, который будет равен минимального глобальному разрезу.

Во втором случае мы можем слить $s, t$  в новую вершину $r$, пересчитав ребра:
\begin{center}
$\textbf{Было: }\xymatrix{
  s \ar @{.>}[r]^{c} & A\\
  t \ar @{.>}[r]^{d} & A
} \hspace{1cm}\ \ \textbf{Стало:}\xymatrix{\\
  r \ar @{->}[r]^{c + d} & A
}$
\end{center}

Таким образом, ответ для графа $G$:
$$MinGlobalCut(G) = min(MaxFlow(G, s, t), MinGlobalCut(G_{(s,t) \to r}))$$

 \textit{Последнее, что нам осталось сделать - это правильным образом выбрать $s, t$}

 \begin{enumerate}
  \item Пусть $a_1$ - произвольная точка графа.
  \item На $i$-ом шаге определим $A_i = \{a_1, a_2, \dots, a_{i - 1}\}$. Далее, за $a_i$ обозначим вершину в $G \setminus A_i$, суммарное капасити до которой от вершин $A_i$ будет минимальным:
  $$a_i = arg \ max_{u \in G \setminus A_i} \left( \sum_{v \in A_i} c(u, v)\right)$$
 \end{enumerate}

 Оказывается, что минимальный разрез между вершинами $s = a_{n-1}, t = a_n$, где $n$ - количество вершин в графе, равен минимальному потоку между $S = G \setminus t, T = \{t\}$. 

 \textbf{Если Вы все еще не понимаете, как работает алгоритм, то}

\begin{enumerate}
  \item Строим последовательность $a_1, a_2, \dots, a_n$ по принципу выше.
  \item Выбираем за $s$ предпоследнюю вершину, а за $t$ - последнюю.
  \item Находим минмимальный разрез, как сумма капасити, ведущих из $t$. 
  \item Сливаем $s$ и $t$ в одну вершину и запускаем алгоритм для нового графа. 
  \item Считаем минимум из п. 3 и п. 4
\end{enumerate}

\textbf{Асимптотика:} $O(n^3)$. 

\textit{$n$ раз запускаем алгоритм, $n^2$ - строим последовательность}

Строить последовательность можно так:
\begin{enumerate}
  \item Считаем суммарные капасити от каждой вершины до всех в графе
  \item Находим максимум, добавляем рассматриваемую вершину в множество $A_i$, пересчитываем все значения из п 1 (добавляем капасити, связанные с добавляемой вершиной) и повторяем
\end{enumerate}

\textbf{Корректность:}

Нам осталось доказать тождество:
$$MinCut(a_{n-1}, a_n) = \sum_{v \in A_{n-1}}c(v, a_n)$$

Неравенство "$\le$" очевидно, поэтому докажем знак "$\ge$". 

Фиксируем некоторый разрез $(S, T), a_{n-1} \in S, a_n \in T$

\begin{proposition}
  Назовем вершину $a_i$ активной, если $a_i$ и $a_{i - 1}$ не содержатся в одной доли.
\end{proposition}

\begin{lemma}
  $a_i$ - активная $\then$ $c(\{a_i\}, A_{i-1}) \le c(S \cap A_i, T \cap A_i)$
 \end{lemma}

 \begin{proof}
  Доказательство индукцией по номеру активной вершину.

  \textbf{База:} $a_j$ - первая активная вершина. Тогда Б.О.О все $a_i$ с $i < j$ лежат в $S$, а $a_j$ - в $T$. Тогда в условии леммы стоит тождественное равенство.

  \textbf{Переход:}
    Пусть $a_u, a_v$ - две последовательные активные вершины. Тогда все вершины между $a_u$ и $a_v$ лежат в одной доли.

    $c(\{a_v\}, A_{v - 1}) = c(\{a_v\}, A_{u - 1}) + c(\{a_v\}, A_{v - 1} \setminus A_{u - 1})$. 

    По определению выбора $a_i$ и предположению индукции справедливо:
    $$c(\{a_v\}, A_{u - 1}) \le c(\{a_u\}, A_{u - 1}) \le c(S \cap A_u, T \cap A_{u})$$

    Для завершения перехода остается заметить неравенство:
    $$c(\{a_v\}, A_{v - 1} \setminus A_{u - 1})  + c(S \cap A_u, T \cap A_{u})\le c(S \cap A_v, T \cap A_{v})$$
 \end{proof}

 Теперь докажем корректность алгоритма:
 \begin{proof}
  $a_n$ по определению $(S, T)$ активна, поэтому для нее можно применить лемму:

  $$c(\{a_n\}, A_{n-1}) \le c(S \cap A_n, T \cap A_n) = c(S, T)$$
 \end{proof}

\subsection{Потоки минимальной стоимости}

Теперь считаем, что на каждом ребре написаны пропускная способность и стоимость 1 единицы потока.

\begin{definition}
  $Min-cost \ k-flow$ - задача поиска $s-t$ потока величины $k$
\end{definition}

\textit{Поскольку в нашей теории периодически приходится вводить обратные ребра, то будем считать, что на них написаны противоположные стоимости}

\begin{definition}
  $$cost(f) = \frac{1}{2} \sum_e{f(e) \cdot cost(e)}$$ 
\end{definition}

\subsubsection{Алгоритм $Min-cost \ k-flow$}
Будем использовать наивный жадный алгоритм - найдем путь минимальный стоимости из $s$ в $t$ и пустим по нему поток величины 1. Повторим рассуждение $k$ раз.

\textbf{Важно!} Считаем, что нет циклов отрицательной стоимости.

\begin{lemma}{(Критерий минимальности)}
  $f$ - поток величины $k$. Тогда он минимален $\Longleftrightarrow$ в $G_f$ нет циклов отрицательной стоимости.
\end{lemma}

\begin{proof}
\text{}

  \begin{enumerate}
    \item[$\then$] Очев
    \item[$\Longleftarrow$] Предположим, что нашелся $f^*$ - $Min-cost \ k-flow$ такой, что $cost(f^*) < cost(f)$. Введем функцию $g: g(e) = f^*(e) - f(e)$.
    
    Докажем, что $g$ - поток величины 0 в \underline{$G_f$}. Проверим по определению:

    \begin{enumerate}
      \item $g(e) = f^*(e) - f(e) \le c(e) - f(e) = c_f(e)$
      \item $g(u, v) = f*(u, v) - f(u, v) = -f*(v, u) + f(v, u) = -g(v, u)$
      \item $v \cancel{\in} \{s, t\}$
      
      $$\sum_{u}{g(v, u)} = \underbrace{\sum_{u}f^*(v, u)}_{0} - \underbrace{\sum_{u}f(u, v)}_{0} = 0 $$

      \textbf{Почему величина потока 0?}

    \item Распишем по определению:
   $$\sum_{u}{g(s, u)} = \underbrace{\sum_{u}f^*(s, u)}_{k} - \underbrace{\sum_{u}f(s, u)}_{k} = 0 $$

   \item $cost(g) =  cost(f^*) - cost(f) \le 0$
  \end{enumerate}

  $g$ - поток величины 0 $\then$ по лемме о декомпозиции потоков его можно представить объединением циклов в $G_f$. Мы предполагали, что циклов нет, поэтому получаем противоречие. 


  \end{enumerate}
\end{proof}

\begin{proposition}
  Пусть в $G$ нет циклов отрицательного веса, пусть $P$ - самый дешевый путь от $s$ до $t$. $f$ - поток величины 1 вдоль $P$. Тогда в $G_f$ нет отрицательных циклов. 
\end{proposition}
\begin{proof}
  Пусть в $G_f$ появился отрицательный цикл $C$. Рассмотрим $g = P + C$ - поток в $G$ величины 1. Причем стоимость $g$ < стоимости $f$. Тогда $g$ - (по лемме о декомпозиции потоков) объединение нескольких циклов, которые имеют неотрицательный вес, и одного пути $p$. Но тогда $cost(g) \ge cost(p) \ge cost(f)$.  
\end{proof}

\textbf{Асимптотика:} $O(k\underbrace{VE}_{\text{Форд-Беллман}})$

Можно ввести \textbf{Потенциал Джонсона}:
$$\phi: V \to \Z. \ \ cost_\phi(u, v) = cost(u,v) + \phi(u) - \phi(v) $$
Тогда  $cost_\phi(p) = cost(p) + \phi(s) - \phi(t)$.

Чтобы ввести $\phi$, запустим 1 раз алгоритм Форда-Беллмана и положим $\phi(v) = dist(s, v)$. 

\begin{proposition}
  Все стоимости $cost_\phi$ неотрицательны
\end{proposition}

После такого введения можно использовать алгоритм Дейкстры.


