% !TEX root = ../../../main.tex

\section{Продолжение ДП}

\subsection{Рюкзак с оптимизацией}

\begin{enumerate}
  \item[$\longrightarrow$] \(w\) предметов
  \item[$\longrightarrow$] \(w_i\) - вес
  \item[$\longrightarrow$] \(c_i\) - стоимость  
\end{enumerate}

Решение мы помним с прошлой лекции, а сейчас займемся оптимизацией памяти:

\(dp[i][\circ]\) зависят только от $dp[i - 1][\circ]$, поэтому нам достаточно хранить не всю таблицу целиком, а всего 2 слоя, с которыми мы работаем.

Итог - память \(O(W)\)

\subsubsection{Альтернативный вариант}

Будем действовать от стоимости предметов:

\begin{enumerate}
  \item Заводим массив $dp[0\dots n-1][0\dots C-1]$, где $C = \sum_{i = 0} ^ {n - 1} c_i$
  \item $dp'[i][b]$ - min суммарный вес предметов, имеющих номера \(\le i\), и общую стоимость b
  \item $dp'[i][b] = min(dp[i-1][b], dp'[i-1][b-c_i] + w_i$)
\end{enumerate}
\textit{Это используется, если суммарная стоимость значительно меньше суммарного веса.}

\subsection{Динамическое программирование с помощью матриц}
Попробуем найти $F_n = F_{n-1} + F_{n-2}$ с методом матриц.
$$\left(\begin{array}{c}
  F_n \\
  F_{n-1}
\end{array}\right) = \left(\begin{array}{c c}
 1 & 1 \\
  1 & 0
\end{array}\right) \cdot \left(\begin{array}{c}
  F_{n-1} \\
  F_{n-2}
\end{array}\right) = \left(\begin{array}{c c}
  1 & 1 \\
   1 & 0
 \end{array}\right) \cdot \left(\begin{array}{c c}
  1 & 1 \\
   1 & 0
 \end{array}\right) \cdot \left(\begin{array}{c}
   F_{n-2} \\
   F_{n-3}
 \end{array}\right) = \left(\begin{array}{c c}
  1 & 1 \\
   1 & 0
 \end{array}\right)^{n-1}\cdot \left(\begin{array}{c}
  F_{1} \\
  F_{0}
\end{array}\right)$$

\textsc{\textbf{Бинарное возведение матрицы в степень}}

Проводится так же, как и для натуральных чисел:

\begin{equation}
  a^n =
  \begin{cases}
    1,& \text{если $n$ = 0} \\
    \left(a^{\frac{n}{2}}\right)^2,& \text{если $n$ четно}\\
    a \cdot a ^ {n-1},& \text{если $n$ нечетно}
  \end{cases}
\end{equation}

Если две матрицы имеют размеры $k \times k$, то их произведение можно найти за $O(k^3)$

Тогда $A^n$ описанным алгоритмом находится за $O(k^3 \log{n})$

 \underline{Задача} $a_n = \lambda a_{n - 1} + \mu a_{n-2} + 1$
$$\left(\begin{array}{c}
  a_n \\
  a_{n-1} \\
  1
\end{array}\right) = \left(\begin{array}{c c c}
  \lambda & \mu & 1 \\
  1 & 0 & 0\\
  0 & 0 & 1
\end{array}\right) \cdot \left(\begin{array}{c c c}
  a_{n-1} \\
  a_{n-2} \\
  1
\end{array}\right)$$

Взяв произведение этих матриц, получим ответ за $O(3^3 \log{n})$

 \underline{Задача} Пусть $G$ -- невзвешенный ориентированный граф. Найти количество путей длины ровно \(k\) из вершины $x$ в вершину $y$ 
 
 \textit{На ввод дается матрица смежности $M$, где $m_{ij}$ = 1 $\Leftrightarrow$ есть ребро $i \to j$, а 0 иначе.}

 Пусть $dp[v][l]$ - количество путей длины $l$ от $x$ до $v$.

 Тогда $dp[v][l] = \sum_{u \in v: M_{uv} = 1} dp[u][l - 1]$
 $$\left(\begin{array}{c}
  dp[1][l]\\
  dp[2][l] \\
  \ddots \\
  dp[v][l]\\
  \ddots\\
  dp[n][l]
 \end{array}\right) = M^T \cdot \left(\begin{array}{c}
  dp[1][l - 1]\\
  dp[2][l - 1] \\
  \ddots \\
  dp[v][l - 1]\\
  \ddots\\
  dp[n][l - 1]
 \end{array}\right)$$

\textbf{Комментарий от эксперта:}

\texttt{Пересчет динамики получается домножением столбца $dp[v][i-1]$ на транспонированную матрицу смежности слева}
\begin{proposition}
  $M^k$ - количество путей из $u$ в $v$ длины ровно $k$.
\end{proposition}

\subsection{Задача} Найти количество путей длины \(\leq k\) из $x$ в $y$.

Можно найти ответ из суммы $(M^0 + M^1 + \dots + M^k)_{xy}$, но как ее посчитать быстро?

Введем $f(M, k) = (M^k, M^0 + M^1 + \dots + M^{k - 1}).$
\begin{enumerate}
  \item $k = 0 \to f(M, k) = (E, E)$
  \item  $k \cancel{\vdots} 2 \to f(M, k - 1) = (M^{k - 1}, M^0 + M^1 + \dots + M^{k - 2})$, откуда $f(M, k) = f(M, k - 1)$, в котором умножили первый элемент на $M$, предварительно прибавив его ко второму.
  \item  $k \vdots 2$, $f(M, k)$ получается из $f(M, \frac{k}{2})$ умножением первой части, увеличенной на 1, на вторую и возведением первой части в квадрат.
\end{enumerate}

\textsc{\textbf{Второй комментарий от эксперта:}}

\texttt{По формуле геометрической прогрессии $\sum_{i = 0} ^ {k} M^{i} = (M^{k + 1} - E) \cdot {(M - E) ^ {-1}}.$ Если $M - E$ необратима, подкрутим её коэффициент на 0.00001.}

\subsection{Задача} 
Пусть $G$ - граф. Надо проверить, есть ли хотя бы 1 путь из \(x\) в \(y\) длины ровно $k$?

$$d[v][l] = \bigvee_{u}(dp[u][l - 1] \wedge M_{uv})$$

Обозначим $A * B = C$, где $*$ - булевское умножение, такое выражение:

$$c_{ij} = \bigvee_{k}(A_{ik} \wedge B_{kj})$$
\begin{proposition}
  $M^{*k}_{uv} = 1, \text{если есть пусть $u \to v$, а иначе 0}$

  Такое тоже работает за $O(n^3 \log{k})$
\end{proposition}
\subsection{Задача} $G$ - взвешенный граф. Хотим найти $min$ стоимость пути длины ровно $k$ из $x$ в $y$.

Пусть $dp[v][l]$ - минимальная стоимость пути $x \to v$ за $l$ ребер.
Тогда его можно найти по формуле $min(dp[u][l-1] + cost(u, v))$

Обозначим: $A \circ B = C$, где $$c_{ij} = min_{k}(a_{ik} + b_{kj})$$

$(A \circ B) \circ C = A \circ (B \ circ C)$

\begin{proposition}
  
  $M^{\circ k}$ - минимальная стоимость пути из $u$ в $v$, используя ровно $k$ ребер.
\end{proposition}

