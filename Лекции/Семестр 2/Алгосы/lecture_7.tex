% !TEX root = ../../../main.tex

\subsection{Двусторонний алгоритм Дейкстры}
\textit{\underline{Задача:}} хотим найти минимальный путь при условии, что все веса неотрицательные.

\textbf{Идея решения:} такая же, что и в двустороннем bfs - наращивать слои до их пересечения. 

$$dist(s, t) = min_m(d_s[m] + d_t[m])$$

Пусть запущены два алгоритма Дейкстры. Будем каждый раз раскрывать вершины с $min\ d$. Завершаем алгоритм, когда какая-то вершина раскрыта с обеих сторон.

\section{Алгоритм $A^*$}

\begin{enumerate}
  \item Заведем функцию $h(v)$ - эвристика - оценка на $dist(v, t)$

Например, $h(v)$ - евклидово расстояние между $v$ и $t$.

\textit{Естественно, стоит брать как можно более точную оценку для эвристики, чтобы алгоритм работал эффективнее}.

\item  Пусть $g[v]$ - текущая $min$ длина пути от $s$ до $v$.
$$f(v) = g[v] + h(v)$$

\item Применяем алгоритм Дейкстры на $f$:
\begin{verbatim}
  g = {+\infty, +\infty,..., +\infty}; g[s] = 0
  // Всегда поддерживаем f(v) = g[v] + h(v)
  Заводим кучу, кладем в нее s;
  while (куча не пуста) {
    Достаем $v$ - вершину из кучи с минимальным значением $f$
    Удаляем ее из кучи; if(v == t) break;
    Раскрываем v; обновляем соседей;
  }
\end{verbatim}
\end{enumerate}

\begin{definition}
  Эвристика $h$ называется допустимой, если $\forall v: \ h(v) \le dist(v, t)$
\end{definition}

\begin{definition}
  Эвристика $h$ называется монотонной, если $h(t) = 0, \ h $удовлетворяет неравенству треугольника:

  $$h(u) \le h(v) + x$$
\end{definition}


\begin{theorem}
  \text{}

  \begin{enumerate}
    \item Если $h$ - монотонная эвристика, то в $A^*$ каждая вершина раскрывается не больше 1 раза, причем все $g$ находятся корректно.
  
  Это в частности означает, что $A^*$ ведет себя не хуже, чем алгоритм Дейкстры

  \item Если $h$ - допустимая эвристика, то алго $A^*$ может раскрывать вершины по несколько раз (exp(n)), но все $g$ найдутся корректно.
  \item Если $h$ не является допустимой, то ничего не гарантируется. Но обычно приближается не очень плохо \textbf{(???)}
  \end{enumerate}
\end{theorem}

\begin{lemma}
  Пусть $k$ - монотонная. Тогда вершины в $A^*$ раскрываются в порядке неубывания $f$
\end{lemma}

\begin{proof}
  $f[v] - min$ в куче. 
  \begin{enumerate}
    \item $f[u]$ не изменяется, тогда $f[u] \ge f[v]$, так как $v$ - минимальная в куче
    \item $f[u]$ становится равным $\cancel{g[v]} + x + h(u) \ge^{?} f(v) = \cancel{g[v]} + h(v)$, что верно по неравенству треугольника.
  \end{enumerate}
\end{proof}

\textbf{Теперь докажем утверждения теоремы:}

\begin{proof}
  По лемме $f$ не убывает $\then$ вершина $v$ не может извлечься из кучи больше 1 раза.

  \textbf{Почему $g$ находится корректно?}

  Пусть $u$ - первая из $s$ нераскрытая вершина.

  $f[v] = g[v] + h(v) > f[u]?$

  $$f[u] = g[u] + h[u] = dist(s, u) + h(u)$$
  \textit{Пусть $l$ - размер пути от $u$ до $v$}
  $$h(u) \le h(v) + l$$
  $$g[v] > dist(s, u) + l$$
  $f[v] = g[v] + h(v) \ge dist(s, u) + l + h(v) \ge dist(s, u) + l + h(u) - l = f(u)$. Противоречие
\end{proof}

\begin{proposition}
  $dist(s, t) = -\infty$ тогда и только тогда, когда существует цикл отрицательного веса.
\end{proposition}
\begin{proof}
  \begin{enumerate}
    \item[$\Longleftarrow$] Очевидно
    \item[$\then$] Пусть $M$ - ограничение сверху на абсолютное значение всех весов, то есть $|w(e)| \le M$. Рассмотрим путь из $s$ до $t$ веса меньше, чем $-Mn$, где $n$ - количество ребер. Тогда, очевидно, он зацикливается.
  Если цикл имеет неотрицательный вес, то отбросим его, получив более короткий путь. Действуя таким образом, получим либо искомый цикл, лиюо противоречие.
  \end{enumerate}
\end{proof}

\section{Алгоритм Форда-Беллмана}
$w: E \to R$. Хотим найти $dist(s, v) \forall v$ 

\begin{enumerate}
  \item Заведем $dp[k][v]$ - минимальная стоимость пути от $s$ до $v$, который испольует не более $k$ ребер.
  \item 
 \begin{equation}
    dp[0][k] = 
    \begin{cases}
      0 & v == s\\
      +\infty & v \ne s
    \end{cases}
  \end{equation}

  \item  \underline{Переход:}
  \begin{equation}
    dp[0][k] = min
    \begin{cases}
      min(dp[k - 1][v])\\
      min_{(u, v) \in E} dp[k - 1][u] + cost(u, v)\\
    \end{cases}
  \end{equation}
\end{enumerate}

\begin{note}
Если в графе нет отрицательных циклов, то $dp[n - 1][v] = dist(s, v) \forall v$

\textit{Асимптотика:} $O(nm)$
\end{note}

\textbf{Что делать в случае отрицательных циклов?}

\begin{proposition}
  \begin{enumerate}
    \item Если $C$ - отрицательный цикл, достижимый из $S$, то $\exists v \in C: dp[n][v] < dp[n - 1][v]$
    \item Если для некоторого $t: d[n][t] < dp[n - 1][t]$. то $\exists$ отрицательный цикл такой, что $S \to C \to t$.
  \end{enumerate}
\end{proposition}

\underline{\textsc{Вывод:}} Чтобы найти все вершины с $dist(s, t) = -\infty$, достаточно запустить $dfs$ от всех вершин $v$, для которых $dp[u][v] < dp[n - 1][v]$.
