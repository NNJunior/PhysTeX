% !TEX root = ../../../main.tex

\section{Паросочетания (в двудольных графах)}

\begin{definition}
$G = (V, E)$ - граф, $M \subset E$ называется паросочетанием, если никакие два ребра из $M$ не имеют общих концов. 
\end{definition}

\textit{Подобные темы будут часто встречаться в задачах о назначениях, где нужно находить максимальное паросочетание}

\begin{definition}
  \text{}

  \begin{enumerate}
    \item  Назовем ребро $e$ \underline{насыщенным}, если $e \in M$
    \item Назовем вершину $v$ \underline{насыщенной}, если есть насыщенное ребро с этой вершиной.
  \end{enumerate}
 
\end{definition}

\begin{definition}
  Путь $P$ в графе называется увеличивающим относительно паросочетания $M$, если:

  \begin{enumerate}
    \item $P$ - простой
    \item $|P| \ge 1$, то есть содержит хотя бы 1 ребро
    \item Концы $P$ - ненасыщенные
    \item Типы ребер вдоль $P$ чередуются
  \end{enumerate}
\end{definition}
$$\xymatrix{
  *={\circ} \ar@{.}^{\cancel \in M}[r] &  *={\bullet} \ar@{-}^{\in M}[r] & *={\bullet} \ar@{.}^{\cancel \in M}[r] & ... \ar@{-}^{\in M}[r]  & *={\bullet} \ar@{.}^{\cancel \in M}[r] & *={\bullet} \ar@{-}^{\in M}[r] & *={\bullet} \ar@{.}^{\cancel \in M}[r] &*={\circ}
}$$

\begin{theorem}[Берж]
  Паросочентание $M$ - максимальное (то есть самое большое по размеру среди всех паросочетаний) $\Longleftrightarrow$ относительно $M$ нет увеличивающих путей.
\end{theorem}

\begin{proof}
  \text{}

  \begin{enumerate}
    \item[$\then$] От противного. Пусть $M$ - максимальное, но существует увеличивающий путь $P$. Выполнив чередование вдоль $P$, заменяя ребра не из $M$ на ребра из $M$, мы его увеличим.
    \item[$\Longleftarrow$] От противного. Пусть $M'$ - максимальное паросочетание, $H = (V, M \Delta M')$. Тогда в $H$ степень каждой вершины не превосходит 2, так как каждое ребро берется из паросочетания, в котором степень каждой вершины равна 1.
    \begin{lemma}
      Если $\Delta(H) \le 2$, то любая компонента связности в $H$ - либо простой путь, либо простой цикл.
    \end{lemma}

    \textit{$\Delta(H)$ - максимальная степень вершины}

    \begin{proof}
      Ходим, ходим, ходим - либо зациклились (тогда предпериода нет), либо дошли до конца пути.
    \end{proof}
  \end{enumerate}

  Теперь заметим, что $H$ не содержит нечетных циклов, то есть циклов нечетной длины. Ведь цикл получен из ребер $M$ и $M'$, а значит, в нем ребра вида $M$ и вида $M'$ чередуются $\then$ длина цикла четна.

  \newpage
  \textbf{Итог:} В $H$ могут быть только
  
  \begin{enumerate}
    \item[$\Rightarrow$] четные циклы
    \item[$\Rightarrow$] четные пути
    \item[$\Rightarrow$] нечетные пути
  \end{enumerate}

  Первые два типа содержат одинаковое число ребер из $M$ и $M'$, тогда по нашему предположению существует хотя бы 1 нечетный путь, причем в нем количество ребер из $M'$ больше. Тогда это будет увеличивающим для $M$. Противоречие.
\end{proof}

\begin{note}
Для поиска наибольшего паросочетания можно находить увеличивающий путь, делать из него паросочетание и т. д.
\end{note}

\subsection{Алгоритм Куна:} $G$ - двудольный граф. Положим $M$ максимальное паросочетание, равное изначально $\varnothing$. Пока в $G$ есть увеличивающий путь относительно $M$, выполняем вдоль него чередование, увеличивая $|M|$.

\textbf{Как находить увеличивающий путь?}

Заметим, что увеличивающий путь в двудольном графе - это просто путь в ориентированном графе из ненасыщенной вершины левой доли в ненасыщенную вершину правой доли. А значит, нам нужно просто запустить $dfs$.

\begin{center}
\begin{verbatim}
  // L - левая доля 
  // R - правая доля
  vector<vector<int>> g; // g[v] - список соседей вершин L
  vector<int> match; // match[u] = -1, если u (из R) не насыщенна, и сосед слева иначе
  vector<bool> used = {false, false, ..., false};

  // функция проверки "есть ли увеличивающий путь с начальной вершиной v"
  bool Augment(int v) {
    if (used[v]) return false;
    used[v] = true;
    for (int to : g[v]) {
      if ((match[to] == -1) || Augment(match[to])) {
        match[to] = v; return true;
      }
    }
    return false;
  }

  for (int v = 0; v < n; ++v) {
    if (Augment(v)) {
      used = {false, false, ..., false};
    }
  }
\end{verbatim}
\end{center}

\textit{Асимптотика $O(ans \cdot m)$}

\begin{note}
  Кажется, что после $Augment(v) = true$ надо начинать проверку с самой первой вершины, но это не так. 
\end{note}
\begin{proposition}
  Пусть $M$ получено из $M$ чередованием вдоль увелчивающего пути. Пусть из $v   \not \exists$ увеличающего пути относительно $M$. Тогда из $v  \not \exists$ увеличивающего пути относительно $M'$.
\end{proposition}
\begin{proof}
  Пусть $x, y$ - ненасыщенные концы увеличивающего путя L, из которого $M$ была получена. Пусть существует увеличивающий путь K из $v$ относительно $M'$. Понятно, что $v$ лежит вне L. Пусть пути $K$ и $L$ пересекаются в вершине $z$ (Почему пересекаются?), тогда какой-то из путей $v \to z \to x$ или $v \to z \to y$ будет увеличивающим относительно $M$, то есть будет рассмотрен нами.
\end{proof}

\begin{definition}
  Независимое множество графа $G$ - подмножество вершин, где никакие 2 вершины не соединены ребром. 
\end{definition}

\begin{definition}
  Вершинное покрытие графа $G$ - подмножество вершин $C$ такое, что любое ребро графа содержит хотя бы один элемент $C$. 
\end{definition}

\begin{theorem}[Кёниг]
  В двудольном графе размер минимального вершинного покрытия равен размеру максимального паросочетания. 
\end{theorem}

\begin{proof}
  Предъявим алгоритм:

  \begin{enumerate}
    \item Находим максимальное паросочетание. 
    \item Ориентируем все ребра так, чтобы ребра не из $M$ вели из левой доли в правую, а ребра из $M$ - из правой в левую.
    \item Запустим обход графа из всех ненасыщенных вершин $L$. 
  
  Обозначим за $L^+, R^+$ - посещенные вершины из соответсвующих долей графа, а $L^-, R^-$ - непосещенные
    \item $L^+ \cup R^-$ - максимальное независимое множество, $(L^- \cup R^+)$ - минимальное вершинное покрытие. 
  
  Это так, так как не существует ребер из 
  \begin{enumerate}
    \item $L^+ \to R^-$
    \item $R^+ \to L^-$
    \item $R^- \to L^+$
  \end{enumerate}
  \end{enumerate}


  \item Почему $(L^- \cup R^+) -$ минимальное вершинное покрытие?
  
  Заметим, что все вершины в этом множестве насыщенные. 
  
  $L^-:$ все ненасыщенные из левой доли лежат в $L^+$

  $R^+:$ если есть ненасыщенная, то мы нашли увеличивающий путь.


  При этом каждое ребро $M$ пересекается с $L^- \cup R^+$ не более чем 1 концом. А тогда размер этого множества не больше размера паросочетания. Оценка в другую сторону очевидна. 
\end{proof}