% !TEX root = ../../../main.tex

\begin{corollary}
    Если \(f\) бесконечно дифферецируема на интервале, содержащем точку \(x_0\) и \((x_0 - r, x_0 + r)\) и  \(\exists M > 0 \forall x \in (x_0 - r, x_0 + r) \forall k |f^{(k)}(x)| \le M\), то \(\forall x \in (x_0 - r, x_0 + r)\;\;f(x) = \sum_{n = 0}^\infty \frac{f^{(k)}(x)}{k!}(x - x_0)^k\)
\end{corollary}

\begin{corollary}
    Ряды Маклорена \(e^x, \sin x, \cos x\) сходятся к этим функциям \(\forall x \in \R\), т.е. 
    \[e^x = \sum_{n = 0}^\infty \frac{x^n}{n!},\;\;\;\sin x = \sum_{n = 0}^\infty (-1)^n\frac{x^{2n + 1}}{(2n+1)!},\;\;\;\cos x = \sum_{n = 0}^\infty (-1)^n\frac{x^{2n}}{(2n)!}\]
\end{corollary}
\begin{proof}
    \((e^x)^{(k)} = e^x, (\sin x)^{(k)} = \sin\left(x + \frac{\pi}{2}k\right), (\cos x)^{(k)} = \cos\left(x + \frac{\pi}{2}k\right)\). Поэтому при \(|x| \le \delta: (e^x)^{(k)} \le e^\delta, (\sin x)^{(k)} \le 1, (\cos x)^{(k)} \le 1\)
\end{proof}


\begin{theorem}
    Пусть \(\alpha \ne \N_0, C_\alpha^n = \frac{\alpha(\alpha - 1)\dots(\alpha - n + 1)}{n!}, C_\alpha^0 = 1\). Тогда \((1 + x)^\alpha = \sum_{n = 0}^\infty C_\alpha^nx^n, |x| < 1\)
\end{theorem}
\begin{proof}
    \(f(x) = (1 + x)^\alpha \Ra f^{(n)}(x) = \alpha(\alpha - 1)\dots(\alpha - n + 1)(1 + x)^{\alpha - n} \Ra \frac{f^{(n)}(0)}{n!} = C_\alpha^n\). Имеем при \(x \ne 0\)
    \[\lim_{n \ra \infty}\frac{|C_\alpha^{n + 1}x^{n + 1}|}{|C_\alpha^nx^n|} = \lim_{n \ra \infty} \frac{n - \alpha}{n + 1}|x| = |x|\]
    По признаку Даламбера при \(|x| < 1\) ряд абсолютно сходится, при \(|x| > 1\) --- абсолютно расходится. Тогда \(R = 1\). Обозначим \(g(x) = \sum_{n = 0}^\infty C_\alpha^nx^n\) и покажем, что \(g \equiv f\) на \((-1, 1)\), т.е. \(g(x)(1 + x)^{-\alpha} = 1 \forall x \in (-1, 1)\). Имеем 
    \[g(x)(1 + x)^{-\alpha} = (1 + x)^{-\alpha}\sum_{n = 1}^\infty nC_\alpha^nx^{n - 1} - \alpha(1 + x)^{-\alpha - 1}\sum_{n = 0}^\infty C_\alpha^nx^n =\]
    \[ (1 + x)^{-\alpha - 1}\left(\sum_{n = 1}^\infty nC_\alpha^nx^{n - 1} + \sum_{n = 1}^\infty nC_\alpha^nx^n - \alpha\sum_{n = 0}^\infty C_\alpha^nx^n\right) = \]
    \[(1 + x)^{-\alpha - 1}\left(\sum_{n = 0}^\infty (n + 1)C_\alpha^{n + 1} - \sum_{n = 0}^\infty (\alpha - n)C_\alpha^n \right) = 0\]
    Следовательно, \(g(x)(1 + x)^{-\alpha}\) постоянна на \((-1, 1)\). \(g(0) = 1 \Ra g(x)(1 + x)^{-\alpha} = 1\)
\end{proof}

\begin{note}
    Покажем, что биномиальный ряд при \(\alpha > 0\) сходится равномерно на \([-1, 1]\).
\end{note}
\begin{proof}
    Рвссмотрим числовой ряд \(\sum_{n = 0}^\infty\left|C_\alpha^n\right|\). Для него \(\left|\frac{C_\alpha^{n + 1}}{C_\alpha^n}\right| = \frac{n - \alpha}{n + 1} = 1 - \frac{\alpha + 1}{n} + O\left(\frac{1}{n^2}\right)\). Следовательно, по признаку Гаусса при \(\alpha > 0\), ряд схоодтся на \([-1, 1]\). Но тогда \(\forall x \in [-1, 1] |C_\alpha^nx^n| \le |C_\alpha^n|\)
\end{proof}

\begin{example}
    Рассмотрим \(\frac{1}{1 + x} = \sum_{n = 0}^\infty (-1)^nx^n\) на \((-1, 1)\). Тогда по следствию из теоремы \(\ln(1 + x) = \sum_{n = 0}^\infty (-1)^{n - 1}\frac{x^n}{n!}\). Т.к. ряд сходится при \(x = 1 \Ra \) равномерно сходится на \([0, 1] \sum_{n = 1}^\infty \frac{(-1)^n}{n!} = \ln 2\).
\end{example}

\begin{problem}
    Разложить \(\arctg\). Получив разложение, найти сумму \(\sum_{n = 0}^\infty \frac{(-1)^n}{2n + 1}\)
\end{problem}

\section{Метрические пространства}
\subsection{Метрики и нормы}
\begin{definition}
    Пусть \(X \ne \emptyset\) --- произвольное множество. Функция \(\rho: X \times X \ra \R\) называется метрикой на \(X\), если \(\forall x, y, z \in X\) выполнено
    \begin{enumerate}
        \item \(\rho(x, y) \ge 0, \rho(x, y) = 0 \Lra x = y\)
        \item \(\rho(x, y) = \rho(y, x)\)
        \item \(\rho(x, y) + \rho(y, z) \ge \rho(x, z)\)
    \end{enumerate}
\end{definition}
\begin{definition}
    \((X, \rho)\) --- метрическое пространство.
\end{definition}

\begin{example}
    Пусть \(X\) --- произвольное непустое множество, \(\rho(x, y) = \left\{\begin{array}{l}
        0, x = y \\
        1, x \ne y
    \end{array}\right.\). Тогда \((X, \rho)\) --- метрическое пространство.
\end{example}
\begin{proof}
    Предоставляется читателю в качестве нетрудного упражнения.
\end{proof}

\begin{definition}
    \(\rho\) из прошлого примера называется называется дискретной метрикой
\end{definition}

\begin{definition}
    Пусть \(V\) --- линейное пространство над \(\R, \Cm\). Функция \(\|x\|: V \ra \R\) называется нормой на \(V\), если 
    \begin{enumerate}
        \item \(\|x\| \ge 0, \|x\| = 0 \Lra x = 0\)
        \item \(\|\alpha x\| = |\alpha|\cdot\|x\|\)
        \item \(\|x + y\| \le \|x\| + \|y\|\)
    \end{enumerate}
\end{definition}

\begin{definition}
    Пара \((V, \|x\|)\) называется нормированным линейным пространством
\end{definition}

\begin{lemma}
    Всякое нормированное пространство является метрическим, для \(\rho(x, y) = \|x - y\|\)
\end{lemma}
\begin{proof}\indent
    \begin{enumerate}
        \item \(\|x - y\| \ge 0, \|x - y\| = 0 \Lra x - y = 0 \Lra x = y\)
        \item \(\|x - y\| = |-1|\|y - x\| = \|y - x\|\)
        \item \(\|x - y\| + \|y - z\| \ge \|x - z\|\)
    \end{enumerate}
\end{proof}
Рассмотрим \(X = \R^n\), \(x = (x_1 \dots x_n), y = (y_1, y_2 \dots y_n)\).
\begin{example}
    \(\|x\| = \sqrt{\sum_{k = 1}^n |x_k|^2}\) --- норма, \(\rho(x, y) = \sqrt{\sum_{k = 1}^n |x_k - y_k|^2}\) --- метрика.
\end{example}
\begin{example}
    \(\|x\| = \left(\sum_{k = 1}^n |x_k|^p\right)^\frac{1}{p}\) --- норма, \(\rho(x, y) = \left(\sum_{k = 1}^n |x_k - y_k|^p\right)^\frac{1}{p}\) --- метрика.
\end{example}
\begin{proof}\indent
    \begin{enumerate}
        \item \(\|x - y\| \ge 0, \|x\| = 0 \Lra x = 0\) --- очев
        \item \(\|x - y\| = \|y - x\|\) --- очев
        \item Буквально неравество Минковского (см 1 семестр)
    \end{enumerate}
\end{proof}
\begin{example}
    \(\|x\| = \max\{x_i\}\) --- метрика, \(\rho(x, y) = \max\{x_i - y_i\}\)
\end{example}

\begin{definition}
    Пусть \((X, \rho)\) --- метрическое пространство, \(a \in X, r > 0\). \(B_r(a) = \{x \in X | \rho(x, a) < r\}\) называется открытым шаром в центре \(a\) и радиуса \(r\)
\end{definition}
\begin{definition}
    Пусть \((X, \rho)\) --- метрическое пространство, \(a \in X, r > 0\). \(\overline{B_r}(a) = \{x \in X | \rho(x, a) \le r\}\) называется замкнутым шаром в центре \(a\) и радиуса \(r\)
\end{definition}

\begin{definition}
    Пусть \((X, \rho)\) --- метрическое пространство. Множество \(E\) называется ограниченным, если \(\exists a \in X, r \in \R: E \subset B_r(a)\)
\end{definition}