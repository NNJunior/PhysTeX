% !TEX root = ../../../main.tex

\begin{definition}
    Пусть \(\{x_n\} \subset X, a \in X\). Говорят, что \(x_n\) сходится к \(a\), если \(\rho(x_n, a) \ra 0\). Пишут \(\lim_{n \ra \infty}x_n = a\) или \(x_n \ra a\).
\end{definition}

\begin{note}
    \[\forall \epsilon > 0 \exists N \in \N: \forall n > N (x_n \in B_\epsilon(a))\]
\end{note}

\begin{corollary}
    \(x_n \ra a, x_n \ra b \Lra a = b\)
\end{corollary}
\begin{proof}
    \(0 \le \rho(a, b) \le \underbrace{\rho(a, x_n)}_{\ra 0} + \underbrace{\rho(x_n, b)}_{\ra 0}\)
\end{proof}

\begin{corollary}
    \(x_n \ra a \Ra \{x_n\}\) --- ограничена (то есть \(\{x_n\}\) ограничено как множество).
\end{corollary}
\begin{proof}
    \(\rho(x_n, a) \ra 0 \Ra \{\rho(x_n, a)\}\) ограничена \(\Ra \exists R \in \R: R > \sup\{\rho(x_n, a)\} \Ra x_n \in B_R(a)\).
\end{proof}

\begin{corollary}
    Пусть \(\{x_n\}, \{y_n\}: x_n \ra a, y_n \ra b\) --- последовательности в нормированном линейном пространестве, \(\{\alpha_n\} \subset \R: \alpha_n \ra \alpha\). Тогда
    \begin{enumerate}
        \item \(x_n + y_n \ra a + b\)
        \item \(\alpha_nx_n \ra \alpha a\)
    \end{enumerate}
\end{corollary}
\begin{proof}\indent
    \begin{enumerate}
        \item \(\|x_n + y_n - (a + b)\| \le \underbrace{\|x_n - a\|}_{\ra 0} + \underbrace{\|y_n - b\|}_{\ra 0}\)
        \item \(\|\alpha_nx_n - \alpha x\| = \|\alpha_n x_n - \alpha x_n + \alpha x_n - \alpha a\| \le \underbrace{|\alpha_n - \alpha|}_{\ra 0}\|x_n\| + |\alpha|\underbrace{\|x_n - a\|}_{\ra 0}\)
    \end{enumerate}    
\end{proof}
\subsection{Топология метрических пространств}
\begin{definition}
    Пусть \((X, \rho)\) --- метрическое пространство, \(E \subset X\).
    \begin{enumerate}
        \item \(x \in int\;E \Lra \exists \epsilon > 0: B_\epsilon(x) \subset E\). Множество \(int\;E\) называются множеством внутренних точек
        \item \(x \in ext\;E \Lra \exists \epsilon > 0: B_\epsilon(x) \subset X \setminus E\). Множество \(ext\;E\) называются множеством внешних точек
        \item \(x \in \delta E \Lra \forall \epsilon > 0: B_\epsilon(x) \cap E \ne \emptyset, B_\epsilon(x) \cap (X \setminus E) \ne \emptyset\). Множество \(\delta E\) называются множеством граничных точек
    \end{enumerate}
\end{definition}

\begin{definition}\indent
    \begin{enumerate}
        \item \(X = int\;E \sqcup ext\;E \sqcup \delta E\)
        \item \(ext\;E = int\;(X \setminus E)\)
    \end{enumerate}
\end{definition}

\begin{definition}
    Множество \(G\subset X\) называется открытым, если все его точки являются внутренними (\(G = int\;G\))
\end{definition}
\begin{definition}
    Множество \(G\subset X\) называется открытым, если \(X\setminus G\) открыто
\end{definition}
\begin{proposition}\indent
    \begin{enumerate}
        \item Открытый шар \(B_r(a)\) открыт
        \item Замкнутый шар \(\overline{B_r}(a)\) замкнут
        \item \(int\;E\) открыто
    \end{enumerate}
\end{proposition}
\begin{proof}\indent
    \begin{enumerate}
        \item \(x \in B_r(a)\). Положим \(\epsilon = r - \rho(x, a)\). Тогда если \(y \in B_\epsilon(x) \Ra \rho(y, a) \le \rho(y, x) + \rho(x, a) \le \epsilon + \rho(x, a) \le r \Ra B_\epsilon(x) \subset B_r(a)\).
        \item \(x \in X \setminus \overline{B_r}(a)\). \(\epsilon = \rho(x, a) - r\Ra\) аналогично пункту 1), \(X \setminus \overline{B_r}(a)\) --- открыто, т.е. \(\overline{B_r}(a)\) --- замкнуто
        \item \(x \in int\;E \Ra \exists B_\epsilon(x) \subset E \Ra B_\epsilon(x) \subset int\;E\), т.к. \(B_\epsilon(x)\) --- открыто.
    \end{enumerate}
\end{proof}

\begin{lemma}
    Объединение любого количества открытых множеств и пересечение конечного количества открытых множеств является открытым множеством
\end{lemma}
\begin{proof}
    Аналогично случаю для \(\R\)
\end{proof}

\begin{definition}
    \(\stackrel{\circ}{B}_r(a) = B_r(a) \setminus \{a\}\)
\end{definition}

\begin{definition}
    Точка \(x \in X\) называется предельной множества \(E\), если \(\forall \epsilon > 0 \stackrel{\circ}{B}_\epsilon(x) \cap E \ne \emptyset\)
\end{definition}

Множество всех предельных точек принято обозначать через \(E'\)

\begin{theorem}[Критерий замкнутости]
    Следующие утверждения равносильны:
    \begin{enumerate}
        \item \(E\) --- замкнуто
        \item \(E \supset \delta E\)
        \item \(E \supset ext\;E\)
        \item \(\forall \{x_n\} \subset E (x_n \ra x \Ra x \in E)\)
    \end{enumerate}
\end{theorem}
\begin{proof}\indent
    \begin{enumerate}
        \item[\(1 \Rightarrow 2\):] Пусть \(x \in X\setminus E \Rightarrow \exists B_\varepsilon(x) \subset X\setminus E\), т.е. \(x\) --- внешняя точка \(E\). Тогда \(\delta E \subset E\)
        \item[\(2 \Rightarrow 3\):] Пусть \(x\) --- предельная точка тогда она либо внутренняя, и тогда \(x \in E\), либо граничная, но \(\delta E \subset E \Rightarrow x \in E\)
        \item[\(3 \Rightarrow 4\):] Пусть \(\{x_n\} \subset E, x_n \rightarrow x\). Тогда либо \(\exists x_n = x\) и тогда \(x \in E\), либо \(x\) --- предельная точка, и она \(\in E\).
        \item[\(4 \Rightarrow 1\):] Рассмотрим \(x \in X\setminus E\). Пусть она не является внутренней для \(X\setminus E\). Тогда \(\forall \varepsilon > 0 \exists B_{\varepsilon}(x) \cap E \ne \varnothing \Rightarrow\) рассмотрим последовательность точек \(x_n\in \exists B_{\varepsilon}(x) \cap E: x_n \rightarrow x\). Такая последовательность существует по Аксиоме Выбора (\(\exists \phi: 2^X \ra X: \phi(x) \subset X \Ra x_n = \phi\left(B_\frac{1}{n}(x)\right)\)). Но тогда \(x \in E\). Противоречие 
    \end{enumerate}    
\end{proof}

\begin{definition}
    \(\overline{E} = E \cup \delta E\) ---  замыкание множества \(E\)
\end{definition}

\begin{note}\indent
    \begin{enumerate}
        \item \(\overline{E} = X \setminus ext\;E\)
        \item \(F \supset E\), причем \(F\) --- замкнутое. Тогда \(F \supset \overline{E}\)
    \end{enumerate}
\end{note}
\begin{proof}
    \indent
    \begin{enumerate}
        \item \(X = int\;E \cup ext\;E \cup \delta E\).
        \item \(X \setminus F \subset X \setminus E \Ra X \setminus F \subset int\;(X \setminus E) \Ra F \supset \overline{E}\).
    \end{enumerate}
\end{proof}

\begin{note}
    \(x \in \overline{E} \Lra \forall \epsilon > 0 B_\epsilon(x) \cap E \ne \emptyset \Lra \exists \{x_n\} \subset E(x_n \ra x)\)
\end{note}

\begin{definition}
    \(x \in X\) называется точкой прикосновения \(E\), если \(\forall \epsilon > 0 B_\epsilon(x) \cap E \ne \emptyset\)
\end{definition}

\subsection{Подпространство метрического пространства}
\begin{definition}
    Пусть \((X, \rho)\) --- метрическое пространство, \(\emptyset \ne E \subset X\). Тогда \(\rho|_{E\times E}\) --- метрика на \(E\). Пара \((E, \rho|_{E\times E})\) называется подпространством \((X, \rho)\), \(\rho|_{E\times E}\) называется индуцированной метрикой на \(E\)
\end{definition}

\begin{definition}
    \(B_r^E(x) = \{y \in E | \rho(x, y) < \epsilon\}\)
\end{definition}

\begin{note}
    \(B_r^E(x) = B_r^X(x)\cap E\)
\end{note}

\begin{lemma}
    \(U\) открыто в \(E \Lra \exists V \subset X: U = V \cap E\), причем \(V\) открыто
\end{lemma}
\begin{proof}\indent
    \begin{enumerate}
        \item[\(\Ra\)] \(x \in U \Ra \exists B_{\epsilon_x}^E(x) \subset U\), т.е. \(U = \bigcup_{x \in U}B_{\epsilon_x}^E(x)\). Положим \(V = \bigcup_{x \in U}B_{\epsilon_x}^X(x)\) --- открытое в \(X\). Тогда \(V \cap E = \bigcup_{x \in U}(B_{\epsilon_x}^X(x)\cap E) = \bigcup_{x \in U}B_{\epsilon_x}^E(x) = U\)
        \item[\(\La\)] \(x \in U = V \cap E\), где \(V\) открыто в \(X \Ra \forall x \in V \exists B_\epsilon^X(x) \subset V \Ra B_\epsilon^E(x) = B_\epsilon^E(x) \cap E \subset V \cap E\).
    \end{enumerate}    
\end{proof}

\begin{example}
    \(X = \R, E = (-1, 3]\).
    \begin{enumerate}
        \item \(A = (1, 3] = (1, 4) \cap E\) --- открыто в \(E\) (но не в \(X\))
        \item \(B = (-1, 0)\) замкнута в \(E\) (но не в \(X\))
        \item \(C = (0, 1]\) --- не замкнуто и не открыто
    \end{enumerate}
\end{example}