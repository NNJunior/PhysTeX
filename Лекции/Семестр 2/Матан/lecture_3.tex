% !TEX root = ../../../main.tex

\subsection{Несобственные интегралы от знакопеременных функций}
Изучм вопросы сходимости несобственных интегралов от функций ни в какой функции точки \(b\).

\begin{lemma}
    Пусть \(f, g\) --- локально интегрируемы на \([a, b)\) и \(\int_a^b g(x)dx\) --- абсолютно сходится. Тогда несобственные интегралы 
    \[\int_a^b (f(x) + g(x))dx, \int_a^b f(x)dx\]
    Либо одинаково расходятся, либо одновременно сходятся условно, либо одновременно сходятся абсолютно.
\end{lemma}
\begin{proof}
    Абсолютная сходимость влечет сходимость, поэтому \(\int_a^bg(x)dx\) сходится. Тогда по линейности 
    \[\int_a^b f(x) dx = \int_a^b (f(x) + g(x)) dx - \int_a^b g(x) dx\]
    И заключаем, что интегралы \(\int_a^b (f(x) + g(x))dx, \int_a^b f(x)dx\) сходятся одновременно.
    При этом, 
    \[|f + g| \le |f| + |g|, |f| \le |f + g| + |g|\]
    Тогда по критерию сравнения, получаем, что \(\int_a^b |f(x) + g(x)|dx, \int_a^b |f(x)|dx\) сходятся одновременно, т.е. \(\int_a^b (f(x) + g(x))dx, \int_a^b f(x)dx\) абсолютно сходятся одновременно.
\end{proof}

\begin{theorem}[Признак Дирихле]
    Пусть \(f, g\) локально интегрируемы на \([a, b)\), причем
    \begin{enumerate}
        \item \(F(x) = \int_a^x f(t)dt\) ограничена на \([a, b)\)
        \item \(g(x)\) --- монотонна
        \item \(g \ra 0\) при \(x \ra b - 0\)
    \end{enumerate}
    Тогда \(\int_a^b f(x)g(x)dx\) сходится.
\end{theorem}
\begin{proof}
    Существует такая константа \(M: |F| \le M\). Тогда \(\forall \xi \in [a, b)\) имеем \(\left\lvert \int_\xi^x f(t)g(t)dt\right\rvert = |F(x) - F(\xi)| < 2M\). Пусть \(\epsilon > 0\). Тогда \(\exists b' \in [a, b) \forall x \in (b', b) \left(|g(x)| \le \frac{\epsilon}{2M}\right)\). По лемме Абеля, для интервалов \(\forall [\xi, \eta] \subset (b', b)\) выполнено \(\left\lvert \int_\xi^\eta f(x)g(x)dx\right\rvert < 2\cdot2M (|g(\xi)| + |g(\eta)|) < \epsilon\). Далее применяем свойство Коши.
\end{proof}

\begin{note}
    Условия 1, 2 выполнены если \(f\) непрерывна и имеет ограниченную первообразную на \([a, b)\), а \(g\) дифференцируема и \(g'\) сохраняет знак на \([a, b)\).
\end{note}

\begin{example}
    Исследуем сходимость и абсолютную сходимость интеграла
    \[I(\alpha) = \int_1^{+\infty}\frac{\sin kx}{x^\alpha}dx, \alpha \in \R (k > 0)\]
    Делаем замену \(t = kx\) и получаем следующее:
    \[I(\alpha) = \int_1^{+\infty}\frac{\sin t}{t^\alpha}dt\]
    \begin{enumerate}
        \item \(\alpha > 1\).
        \[\left\lvert \frac{\sin t}{t^\alpha}\right\rvert \le \frac{1}{t^\alpha} \Ra \int_1^{+\infty} \frac{|\sin t|}{t^\alpha}dt \text{ --- сходится}\]
        То есть \(I(\alpha)\) сходится абсолютно
        \item \(\alpha \le 0\). Проверим расходимость при помощи Коши. 
        \[\exists \epsilon_0 = \forall \Delta > 1 \exists \xi = 2\pi n > \Delta, \eta = 2\pi n + \pi > \Delta\]
        \[\left\lvert \int_\xi^\eta \frac{\sin t}{t^\alpha}dt\right\rvert = \int_\xi^\eta t^{-\alpha}\sin t dt \ge (2 \pi n)^{-\alpha} \int_{2\pi n}^{2\pi n + \pi} \sin t dt = (2 \pi n)^{-\alpha} \cdot 2 \ge 2\]
        Тогда по критерию Коши, \(I(\alpha)\) расходится.
        \item \(\alpha \in (0, 1]\).
        \[f(x) = \sin t, g(t) = \frac{1}{t^\alpha}, F(t) = \int_1^t \sin s \;\;ds \text{ --- ограничена на 
        \(1, +\infty\)}\]
        Тогда \(I(\alpha)\) сходится по признаку Дирихле. Теперь проверим абсолютную сходимость:
        \[\left\lvert \frac{\sin x }{x^\alpha}\right\rvert \ge \frac{\sin^2 x}{x^\alpha} = \frac{1}{2}\left(\frac{1}{x^\alpha} - \frac{\cos 2x}{x^\alpha}\right) \ge 0\]
        При этом \(\int \frac{1}{x^\alpha}\) --- расходится, а \(\int \frac{\cos 2x}{x^\alpha}\) --- сходится. Тогда их разность расходится.
    \end{enumerate}
    Тогда \(I(\alpha)\) сходится при \(\alpha > 0\) и абсолютно сходится при \(\alpha > 1\)
\end{example}

\begin{theorem}[Признак Абеля]
    Пусть \(f, g\) локально интегрируемы на \([a, b)\), причем
    \begin{enumerate}
        \item \(\int_a^b f(x)dx\) сходится
        \item \(g\) монотонна на \([a, b)\)
        \item \(g\) ограничена на \([a, b)\)
    \end{enumerate}
    Тогда 
    \[\int_a^b f(x)g(x) dx\]
    сходится.
\end{theorem}
\begin{proof}
    Из монотонности и ограниченности следует, что \(\exists \lim_{x \ra b - 0} g(x) = c 
    \in \R\). Поэтому \(\int_a^b f(x)(g(x) - c)dx\) сходится, но тогда \(\int_a^b f(x)g(x)dx = \int_a^bf(x)(g(x) - c)dx + c\int_a^b f(x)dx\) --- сходится
\end{proof}

\begin{example} 
\[\int_1^{+\infty}\frac{\sin x}{\sqrt{x} - \sin x}dx\]
\[\frac{\sin x}{\sqrt{x} - \sin x} \sim_{x \ra +\infty} \frac{\sin x}{\sqrt{x}}\]
Так делать нельзя, т.к. свойство, котоыре мы использовали выше, рабоатет только для неотрициательных функций. Как правильно:

\[g(x) = \frac{\sin x }{\sqrt{x}} = \frac{\sin x(\sqrt{x} - (\sqrt{x} - \sin x))}{\sqrt{x}(\sqrt{x} - \sin x)}\]
\[g(x) = \frac{\sin^2 x}{x(1 - \frac{\sin x}{\sqrt{x}})} \sim \frac{\sin^2 x}{x} = \frac{1}{2} \left(\frac{1}{x} - \frac{\cos x}{x}\right) \ge 0 \Ra \int_1^{+\infty} g(x)dx \text{ --- расходится}\]
\end{example}

\begin{center}
    \foo{Короче говоря, принцип сравнения для знакопеременных функций не применим}
\end{center}

\begin{corollary}[Из теоремы 4]
    Пусть \(f, g\) локально интегрируемы на \([a, b)\) и \(g\) монотонна на \([a, b)\), \(\lim_{x \ra b - 0} g(x) = c \in \R \setminus \{0\}\). Тогда \(\int_a^b f(x)g(x)dx, \int_a^b f(x)dx\) либо одновременно расходятся, либо одновременно сходятся условно, либо одновременно сходятся абсолютно.
\end{corollary}
\begin{proof}
    Из сходимости \(\int_a^b f(x)dx\) следует сходимость \(int_a^b f(x)g(x)dx\) по теореме 4. Т.к. \(c 
    \ne 0\), то \(\exists a^* \in [a, b) \forall x \in [a^*, b) (g(x) \ne 0)\). Следовательно, \(f = fg \cdot \frac{1}{g}\) на \([a, b)\). По теореме 4, сходимость \(\int_{a^*}^bf(x)g(x)dx\) влечет \(\int_{a^*}^b f(x)dx\), а значит, \(\int_a^b f(x)dx\) сходится
\end{proof}

\subsection{Несобственные интегралы с несколькими особенностями}
\begin{definition}
    Пусть \(a < b \in \overline{\R}\), функция \(f\) определена на \(a, b\) за исключением, быть может, конечного числа точек. 
    \begin{enumerate}
        \item Точка \(c \in (a, b)\) называется особенностью \(f\), если \(\forall [\alpha, \beta]: c \in [\alpha, \beta] \subset (a, b)\) функция \(f \notin R[\alpha, \beta]\). 
        
        \item Точка \(b\) называется особенностью \(f\), если либо $b = +\infty$, либо $b \in \R$ и $f \cancel{\in} R[\alpha, b] \forall a < \alpha < b$

        \textit{Заметим, что такое определение работает для любого доопределения $f$ в точке $b$.}
    \end{enumerate}
\end{definition}
    \begin{note}
        $f$ не имеет особенностей на $(c, d) \rightarrow$ $f$ локально интегрируема на $(с, d)$.
    \end{note}
    \begin{proof}
        Пусть $[u,v] \subset (a,b)$
        Докажем, что $f \in R [u,v]$
        По условию $\forall x \in [u,v] \exists [\alpha_x,\beta_x]$
        $$\bigcup_{x \in [u, v]}(\alpha_x, \beta_x) \supset [u, v]$$
        Тогда по лемме Гейне-Бореля есть конечное покрытие этого отрезка. 
        Рассмотрим его.
        По аддитивности $f$ интегрируема на некотором отрезке, содержащем $[u, v] \Rightarrow$ и на $[u, v]$
    \end{proof}
    \begin{definition}
        Пусть $c_1 < c_2 < \dots < c_{N-1} $ - все особенности функции $f$ на $(a, b)$, причем определим \(c_0 = a, c_N = b\).

    \angsq{\text{$\xi_k \in (c_{k-1}, c_k), \text{ где } k \in \{1, 2, \dots, N\}$}}
    \end{definition}

    Несобственным интегралом $\int_{a}^b f(x) dx$ называется совокупность интегралов $\int_{c_{k-1}}^{\xi_k}f(x)dx$ и $\int_{\xi_k}^{c_k} f(x)dx$

    Причем если все интегралы и их суммы имеют смысл в $\overline{\R}$, то несобственным интегралом называют именно сумму.
        
    
        
