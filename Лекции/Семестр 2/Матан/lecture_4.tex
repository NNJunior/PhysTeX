% !TEX root = ../../../main.tex

\section{Числовые Ряды}
\begin{definition}
    Пусть \(\{a_n\}\) --- последовательность чисел. Выражение \(\sum_{i = 1}^{\infty} a_i = a_1 + a_2 + \dots \) Называется числовым рядом с \(n\)-ым членом \(a_n\). При этом сумма \(\sum_{i = 1}^N a_i = S_N\) называется \(N\)-ой частичной суммой, а \(\lim_{N\ra\infty}S_N\) называется суммой ряда. Тогда кратко пишут: \(S = \sum_{i = 1}^\infty a_i\). Причем, если указанный предел конечен, то ряд называется сходящимся, иначе --- расходящимся.
\end{definition}

\begin{example}[Геометрический ряд]
    Рассмотрим ряд 
    \[\sum_{i = 1}^{\infty}z^n\]
    \begin{enumerate}
        \item \(|z| < 1\). Заметим, что \(1 + z + \dots + z^N = \frac{1 - z^{N+1}}{1 - z}\). Тогда \(\lim_{N\ra\infty}z^{N+1} = 0\) и \(\exists\lim_{N\ra\infty}S_N = \frac{1}{1 - z}\)
        \item \(|z| \ge 1\). Заметим, что \(z^N = S_{N} - S_{N - 1} \ra 0\). Но \(z^N \not\ra 0\), тогда этот ряд расходится. \(\nexists\lim_{N\ra\infty}S_N\). 
    \end{enumerate}
\end{example}
\begin{note}
    Геометрический ряд сходится \(\Lra |z| < 1\).
\end{note}

\begin{lemma}[Телескопические ряды]
    Для числовой последовательности \(\{s_n\}\) рассмотрим последовательность \(a_n = s_n - s_{n + 1}\). Тогда
    \[\sum_{i = 1}^\infty a_i \text{ сходится} \Lra \{s_i\} \text{ сходится}\]
    В этом случае, 
    \[\sum_{i = 1}^\infty a_i = s_1 - \lim_{N\ra\infty}s_N\]
\end{lemma}
\begin{proof}
    \[S_n = (s_1 - s_2) + (s_2 - s_3) + \dots + (s_n - s_{n+1}) \Ra \lim_{n\ra\infty}S_n = s_1 - \lim_{n\ra\infty}s_n\]
\end{proof}
\begin{example}
    \[\sum_{i = 1}^\infty \frac{1}{i(i+1)}\]
    \[a_i = \frac{1}{i} - \frac{1}{i + 1}\]
    Т.к. \(\frac{1}{i}\) --- 
\end{example}
\begin{proposition}[Локализация]
    Для любого \(m \in \N\) ряды 
    \[\sum_{i = 1}^\infty a_n, \sum_{i = m + 1}^\infty a_n\]
    Сходятся или не сходятся одновременно, причем, если сходятся, то 
    \[\sum_{i = 1}^\infty a_n = \sum_{i = 1}^m a_n + \sum_{i = m + 1}^\infty a_n\]
\end{proposition}
\begin{proof}
    При \(N > m\) имеем 
    \[\sum_{i = 1}^N = \sum_{i = 1}^m a_n + \sum_{i = m + 1}^N a_n\]
    Поэтому 
    \[\sum_{i = 1}^\infty a_n, \sum_{i = m + 1}^\infty a_n\]
    Сходятся или не сходятся одновременно. В случае сходимости достаточно применить предельный переход для доказательства равенства.
\end{proof}
\begin{definition}
    Ряд \(r_N = \sum_{i = N + 1}^\infty a_n\) называется \(N\)-ым остатком ряда
\end{definition}
\begin{proposition}[Линейность]
    Пусть ряды \(\sum_{i = 1}^\infty a_n, \sum_{i = 1}^\infty b_n\) сходятся и \(\lambda, \mu \in \Cm\). Тогда сходится и ряд \(\sum_{i = 1}^\infty \lambda a_n + \mu b_n\), причем
    \[\lambda\sum_{i = 1}^\infty a_n + \mu\sum_{i = 1}^\infty b_n = \sum_{i = 1}^\infty \lambda a_n + \mu b_n\]
\end{proposition}
\begin{proof}
    Очевидно при предельном переходе
\end{proof}
\begin{proposition}[Необходимое условие сходимости ряда]
    Если \(\sum_{i = 1}^\infty a_n\) сходится, то \(\lim_{n\ra\infty} = 0\)
\end{proposition}
\begin{proof}
    \(a_n = S_n - S_{n-1}\). Тогда \(S_n - S_{n-1} \ra \sum_{i = 1}^\infty a_n - \sum_{i = 1}^\infty a_n = 0 = \lim_{n\ra\infty}a_n\)
\end{proof}
Обратное неверно:
\begin{example}[Гармонический ряд]
    \[\sum_{k = 1}^\infty \frac{1}{k}\]
    \(H_n = \sum_{k = 1}^n \frac{1}{k}\). Но тогда: 
    \[H_{2n} - H_n \ge n\frac{1}{2n} = \frac{1}{2}\]
    Противоречие, т.к. если ряд сходится, то начиная с какого то момента все \(H_n\) удалены друг от друга не более чем на \(\epsilon \forall \epsilon > 0\).
\end{example}
\begin{theorem}[Коши]
    \(\sum_{i = 1}^\infty a_n\) сходится \(\Ra \forall \epsilon > 0 \exists N \forall n > N \forall m, n \ge N \left(\left|\sum_{k = m + 1}^n a_k\right| < \epsilon\right)\)
\end{theorem}
\begin{proof}
    Утверждение является переформулировкой критерия Коши для последовательности \(s_n\).
\end{proof}
\begin{definition}
    Если ряд \(\sum_{i = 1}^\infty |a_n|\) сходится, то его называют абсолютно сходящимся, иначе --- условно сходящимся.
\end{definition}
\begin{corollary}
    Абсолютная сходимость влечет условную сходимоть
\end{corollary}
\begin{proof}
    \[\forall n > m \left|\sum_{k = m + 1}^n a_k\right| \le \sum_{k = m + 1}^n |a_n|\]
\end{proof}
\begin{corollary}
    \(a_n \in \Cm, a_n = u_n + iv_n\). Тогда ряд сходится \(\Lra\) сходятся ряды вещественной и мнимой частей. При этом,
    \[\sum_{n = 1}^\infty a_n = \sum_{n = 1}^\infty u_n + i\sum_{n = 1}^\infty v_n\]
\end{corollary}
Пусть \(a_n \in \R\) --- последовательность вещественных чисел. Рассмотрим \(f_a: [1, +\infty) \ra \R, f_a(x) = a_n, n \le x < n + 1\)
\begin{lemma}[О равносходимости]
    Пусть \(a_n \in \R\). Тогда \[\sum_{i = 1}^n\text{ сходится } \Lra \int_1^{+\infty}f_a(x)dx\text{ сходится}\]
    Причем в случае сходимости, левая и правая части равны
\end{lemma}
\begin{proof}
    Пусть интеграл сходится:
    \[s_n = \sum_{k = 1}^\infty a_k \Ra s_n = \int_1^{n + 1}f_a(x)dx\]
    Пусть ряд сходится, \(s_n\) --- его частичная сумма (\(n = [x]\)). Тогда:
    \[\left|\int_1^xf_a(t)dt - S\right| \le \left|\int_1^xf_a(t)dt - S_n\right| + |S_n - S| \le  \left|\int_1^xf_a(t)dt - \int_1^{n+1}f_a(t)dt\right| + |S_n - S| \le\]
    \[ \le |a_n| + |S_n - S|\]
\end{proof}
\subsection{Ряды с неотрицательными членами}
\begin{lemma}
    Пусть \(a_n \ge 0 \forall n \in \N\). Тогда ряд \(a_n\) сходится тогда и только тогда, когда \(\{S_n\}\) --- ограничена.
\end{lemma}
\begin{proof}
    Т.к. \(S_{n+1} = S_n + a_n \Ra S_n\) нестрого возрастает. По теореме о монотонной последовательности, она сходится \(\Lra\) она ограничена.
\end{proof}
\begin{theorem}[Признак сравнения]
    Пусть  \(0 \le a_n \le b_n \forall n \in \N\)
    \begin{enumerate}
        \item Если \(\sum_{n = 1}^\infty b_n\) сходится, то и \(\sum_{n = 1}^\infty a_n\) --- тоже
        \item Если \(\sum_{n = 1}^\infty a_n\) расходится, то и \(\sum_{n = 1}^\infty b_n\) --- тоже
    \end{enumerate}
\end{theorem}
\begin{proof}
    По лемме о равносходимости данная теорема равносильна признаку сходимости для интегралов
\end{proof}
\begin{corollary}
    Пусть \(a_n, b_n \ge 0 \forall n \in \N\), \(a_n = O(b_n), n\ra \infty\). Тогда справедливо заключение предыдущей теоремы
\end{corollary}
\begin{corollary}
    Пусть \(a_n, b_n \ge 0 \forall n \in \N\), \(\exists \lim_{n\ra\infty}\frac{a_n}{b_n} \in (0, +\infty)\). Тогда \(\sum_{i = 1}^\infty a_n, \sum_{i = 1}^\infty b_n\) сходятся или расходятся одновременно
\end{corollary}

\begin{theorem}[Интегральный признак]
    Пусть \(f\) нестрого убывает и неотрицательна на \([1, +\infty]\). Тогда последовательность \(u_n = f(1) + \dots + f(n) - \int_{1}^{n+1}f(t)dt\) сходится, в частности, \(\sum_{i = 1}^\infty a_n, \int_{1}^{+\infty}f(t)dt\) сходятся или расходятся одновременно
\end{theorem}