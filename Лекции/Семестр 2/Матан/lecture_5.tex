% !TEX root = ../../../main.tex

\begin{theorem}[Интегральный признак]
    Пусть \(f\) нестрого убывает и неотрицательна на \([1, +\infty)\). Тогда последовательность \(u_n = f(1) + \dots + f(n) - \int_{1}^{n+1}f(t)dt\) сходится, в частности, \(\sum_{i = 1}^\infty a_n, \int_{1}^{+\infty}f(t)dt\) сходятся или расходятся одновременно
\end{theorem}
\begin{proof}
    Докажем, что последовательность \(u_n\) нестрого убывает и ограничена снизу. В силу монотонности \(f\) имеем:
    \[f(k+1) \le \int_k^{k+1}f(t)dt \le f(k)\]
    Тогда 
    \[u_n = \sum_{k = 1}^n f(k) - \sum_{k = 1}^{n-1}\int_k^{k+1}f(t)dt = \sum_{k = 1}^{n - 1}\left(f(x) - \int_k^{k + 1}f(t)dt\right) + f(n) \ge f(n) \ge 0 \Ra u_n \ge 0\]
    \[u_{n+1} - u_n = f(n+1) - \int_n^{n + 1}f(x)dx \le 0\]
    Следовательно, по теореме о пределе монотонной последовательности \(u_n\) --- сходится.
\end{proof}

\begin{example}
    Исследуем на сходимость \(\sum_{n = 1}^{\infty} \frac{1}{n^\alpha}\).
    \begin{enumerate}
        \item \(\alpha \le 0\) --- расходится, т.к. его члены не бесконечно маленькие.
        \item \(\alpha > 0\), тогда \(f(x) = \frac{1}{x^\alpha}\) --- неотрицательная монотонная функция. Но тогда по интегральному признаку, этот ряд сходится \(\Lra\) сходится \(\int_1^{+\infty} \frac{1}{x^\alpha}dx \Lra \alpha > 1\).
    \end{enumerate}
    \(\Ra\) сходится при \(\alpha > 1\).
\end{example}

\begin{example}
    Рассмотрим гармонический ряд, обозначим \(H_n = \sum_{i = 1}^n \frac{1}{n}, u_n = H_n - \int_1^n \frac{1}{x}dx = H_n - \ln n\). По интегральному признаку \(\exists \lim_{n\ra \infty}u_n = \gamma\)  --- константа Эйлера-Маскерони.
\end{example}

\begin{theorem}[признак Коши]
    Пусть \(a_n \ge 0, q = \limsup_{n \ra \infty}\sqrt[n]{a_n}\).
    \begin{enumerate}
        \item \(q < 1 \Ra \sum_{n = 1}^\infty a_n\) --- сходится
        \item \(q > 1 \Ra \sum_{n = 1}^\infty a_n\) --- расходится, причем \(a_n \not\ra 0\)
    \end{enumerate}
\end{theorem}
\begin{proof}\indent
    \begin{enumerate}
        \item Пусть \(q_0 \in (q, 1)\). Выберем \(N\) так, что \(\sup_{n \ge N}\sqrt[n]{a_n} \le q_0\). Но тогда \(\sqrt[n]{a_n} \le q_0 \Ra a_n \le q_0^n \Ra\) по принципу локализации, т.к. \(\sum_{n = 1}^\infty q_0^n\) сходится, то и \(\sum_{n = 1}^\infty a_n\) сходится.
        \item Т.к. \(q\) --- частичный предел, то \(\exists \sqrt[n]{a_{n_k}} \ra q \Ra \exists k_0: \forall k > k_0\;\;a_{n_k} > 1\).
    \end{enumerate}
\end{proof}

\begin{theorem}[признак Даламбера]
    Пусть \(a_n \ge 0, \overline{r} = \limsup_{n \ra \infty}\frac{a_{n + 1}}{a_n}, \underline{r} = \liminf_{n \ra \infty}\frac{a_{n + 1}}{a_n}\).
    \begin{enumerate}
        \item \(\overline{r} < 1 \Ra \sum_{n = 1}^\infty a_n\) --- сходится
        \item \(\underline{r} > 1 \Ra \sum_{n = 1}^\infty a_n\) --- расходится, причем \(a_n \not\ra 0\)
    \end{enumerate}
\end{theorem}
\begin{proof}\indent
    \begin{enumerate}
        \item Пусть \(r \in (\overline{r}, 1)\). Выберем \(N\) так, что \(\sup_{n \ge N}\frac{a_{n + 1}}{a_n} \le r\), а значит \(\frac{a_{n + 1}}{a_n} \le r \forall n > N\). Но тогда \(a_{n + m} \le a_nr^m \forall n > N\). Но тогда, т.к. \(\sum_{m = 1}^\infty a_nr^m\) --- сходится, то, по локализации, сходится и \(\sum_{n = 1}^\infty a_n\)
        \item Т.к. \(\underline{r}\) --- частичный предел, то \(\exists \frac{a_{n + 1}}{a_n} \ra \underline{r} \Ra \exists k_0: \forall k > k_0\;\;a_{n_k} \ge a_{n_{k_0}}\). Но тогда \(a_n \ne\ra 0\).
    \end{enumerate}
\end{proof}

\begin{note}
    Если в признаке Коши \(q = 1\) или в признаке Даламбера \(\underline{r} \le 1, \overline{r} \ge 1\), то в общем случае нельзя определить тип сходимости \(a_n\).
\end{note}
\begin{example}
    \[\sum_{n = 1}^\infty \frac{1}{n} \Ra q = 1, \overline{r} = 1, \underline{r} = 1 \text{ --- расходится}\]
    \[\sum_{n = 1}^\infty \frac{1}{n^2} \Ra q = 1, \overline{r} = 1, \underline{r} = 1 \text{ --- сходится}\]
\end{example}

\begin{example}
    \[\sum_{i = 1}^\infty 2^{(-1)^n - n}\]
    \begin{enumerate}
        \item \[\frac{a_{2k + 1}}{a_{2k}} = \frac{2^{-1 - (2k + 1)}}{2^{-1 - 2k}} = \frac{1}{8}\]
        \item \[\frac{a_{2k}}{a_{2k - 1}} = \frac{2^{-1 - 2k}}{2^{-1 - (2k - 1)}} = \frac{1}{2}\]
    \end{enumerate}
    Итого, признак Даламбера ничего нам не дал. Посмотрим на признак Коши:
    \[\sqrt[n]{a_n} = 2^{-1 + \frac{(-1)^n}{n}} \ra \frac{1}{2} \Ra \sum_{i = 1}^\infty 2^{(-1)^n - n} \text{ --- сходится}\]
\end{example}

\begin{problem}
    Доказать, что 
    \[\liminf_{n \ra \infty}\frac{a_{n + 1}}{a_n} \le \liminf_{n \ra \infty}\sqrt[n]{a_n} \le \limsup_{n \ra \infty}\sqrt[n]{a_n} \le \limsup_{n \ra \infty}\frac{a_{n + 1}}{a_n}\]
\end{problem}

\begin{theorem}[Признак Гаусса]
    Пусть \(a_n \ge 0, \exists S > 1, A \in \R\), ограниченная последовательность \(\alpha_n\), такие, что \(\frac{a_{n + 1}}{a_n} = 1 - \frac{A}{n} + \frac{\alpha_n}{n^s}\). Тогда \(\sum_{i = 1}^\infty a_n\) --- сходится только при \(A > 1\).
\end{theorem}
\begin{proof}
    Покажем, что \(\{n^Aa_n\}\) сходится к положительному числу. Рассмотрим \(v_n = \ln(n^Aa_n)\) и \(\sum_{n = 1}^\infty w_n\), где \(w_n = v_{n + 1} - v_n\). Тогда \(w_n = \ln\left(\frac{n + 1}{n}\right)^A + \ln\left(\frac{a_{n + 1}}{a_n}\right) = A\ln\left(1 + \frac{1}{n}\right) + \ln\left(1 - \frac{A}{n} + O\left(\frac{1}{n^2}\right)\right) = \left(\frac{A}{n} + O\left(\frac{1}{n^2}\right)\right) + \left(-\frac{A}{n} + O\left(\frac{1}{n^2}\right)\right) = O\left(\frac{1}{n^2}\right)\)
\end{proof}
% \begin{proof}[Неверное]
%     При \(n > 1\) имеем
%     \[a_n = a_1\prod_{k = 1}^{n - 1}\frac{a{k + 1}}{a_k} = a_1\exp\left(\sum_{k = 1}^{n - 1}\ln \frac{a_{k +1}}{a_k}\right) = \]
%     \[= a_1\exp\left(\sum_{k = 1}^{n - 1}\ln\left(1 - \frac{A}{k} - \frac{\alpha_k}{k^s}\right)\right)\]. По формуле Тейлора, \(\ln(1 + x) = x + O(x^2), x \ra 0\). Поэтому
%     \[a_n = a_1\exp\left(\sum_{k = 1}^{n - 1}\left(-\frac{A}{k} + \frac{\alpha_k}{k^s} + O\left(\frac{1}{k^2}\right)\right)\right)\]
%     \[\sum_{k = 1}^n\frac{1}{k} = \ln n + O(1), \sum_{k = 1}^n \frac{1}{k^\alpha} \text{ сходится при }\alpha > 1 \Ra a_n = a_1\exp\left(-A\ln n + O(1)\right) = a_1 \frac{e^{O(1)}}{n^A}\]
%     Но тогда \(\sum_{n = 1}^\infty a_n\) сходится только при \(A > 1\)
% \end{proof}

\subsection{Числовые ряды с произвольными членами}
\begin{lemma}
    Пусть \(b_n\) абсолютно сходится. Тогда \(\sum_{n = 1}^\infty(a_n + b_n), \sum_{n = 1}^\infty a_n\) либо одновременно расходятся, либо сходятся условно, либо сходятся абсолютно.
\end{lemma}
\begin{proof}
    Аналогично доказательству для несобственных интегралов
\end{proof}

\begin{theorem}[Признак Лейбница]
    Если \(\alpha_n\) монотонна и \(\alpha_n \ra 0\), то \(\sum_{n = 1}^\infty (-1)^n\alpha_n\) сходится и \(|S_n - S| \le |a_{n + 1}|\)
\end{theorem}
\begin{proof}
    Пусть \(\alpha_n\) нестрого убывает, в частности, \(\alpha_n \ge 0 \forall n\)
    \[S_{2n + 2} - S_{2n} = \alpha_{2n + 1} - \alpha_{2n + 2} \ge 0 \Ra \{S_{2n}\} \text{ --- нестрого возрастает}\]
    \[S_{2n + 1} - S_{2n - 1} = -\alpha_{2n} + \alpha_{2n + 1} \le 0 \Ra \{S_{2n}\} \text{ --- нестрого убывает}\]
    \[S_{2n + 1} - S_{2n} = \alpha_{2n + 1} \ge 0 \Ra \forall n, m\]
    \[S_{2n} \le S_{2N} \le S_{2N + 1} \le S_{2m + 1}, N = \max\{n, m\}\]
    Причем \(\{S_{2n}\}, \{S_{2n+1}\}\) --- ограничены \(\Ra S_{2n} \ra S', S_{2n+1} \ra S'', S_{2n + 1} - S_{2n} = \alpha_n \ra 0 \Ra S' = S''\). Кроме того, \(|S_n - S| \le |S_n - S_{n + 1}| = |a_{n + 1}|\)
\end{proof}