% !TEX root = ../../../main.tex

Рассмотрим \(f(t) = \ln t\).
\[\int_1^{n + 1} \ln tdt = t\ln t|_{1}^{n + 1} - \int_1^{n + 1}dt = (n + 1)\ln(n + 1) - n\]
Следовательно, сходится последовательность 
\[(n + 1)\ln(n  + 1)- n - \sum_{k = 1}^n\ln k - \frac{1}{2}\ln n = (n + 1)\left(\ln n + \ln \left(1 + \frac{1}{n}\right)\right) - \ln n! - \frac{1}{2}\ln n - n = \]
\[ = \left(n + \frac{1}{2}\right)\ln n + (n + 1)\left(\frac{1}{n} + o\left(\frac{1}{n}\right)\right) - \ln n! - \ln e^n\]
Следовательно, сходится \(\underbrace{\left\{\left(n + \frac{1}{2}\right) \ln n + \ln n! + n\right\}}_{\ln\frac{n!e^n}{n^{n + \frac{1}{2}}}}\)

Поэтому, \(\ln\frac{n!e^n}{n^{n + \frac{1}{2}}} \ra C > 0\) и \(n! \sim C\frac{n^{n  + \frac{1}{2}}}{e^n}\). Найдем \(C\), пользуясь формулой Валлиса
\[\pi = \lim_{n \ra \infty} \frac{1}{n}\left(\frac{(2n)!!}{(2n - 1)!!}\right)^2\]
Имеем 
\[\frac{1}{n}\left(\frac{(2n)(2n - 2)\dots2}{(2n - 1)(2n - 3)\dots 1}\right) = \frac{1}{n}\left(\frac{2^{2n}(n!)^2}{(2n)!}\right)^2 \sim \frac{2^{4n}}{n}\frac{C^4n^{4n + 2}e^{4n}}{e^{4n}C^2(2n)^{4n + 1}} = \frac{C^2}{2} \Ra C = \sqrt{2\pi}\]

Но тогда
\[n! \sim \sqrt{2\pi}\left(\frac{n}{e}\right)^n, n \ra \infty\]

\section{Функциональные последовательности и ряды}

Пусть \(f_n, f: E \ra \R\) или \(\Cm\) (все утверждения тоже верны для \(\R\) или \(\Cm\)).

\begin{definition}
  Говорят, что \(f_n\) поточечно сходится к \(f\) на \(E\), если \(\forall x \in E f(x) = \lim_{n \ra \infty}f_n(x)\). Пишут \(f_n \ra f\) на \(E\), и \(f\) называют пределом функциональной последовательности \(f_n\)
\end{definition}
\begin{example}
  \(f_n: [0, 1) \ra \R, f_n(x) = x^n\). Тогда \(f_n \ra f\), при \(f(x) = \left\{\begin{array}{l}
    0, x \in [0, 1) \\
    1, x = 1
  \end{array}\right.\)
  Функция оказалась разрывна!
\end{example}

Распишем определение поточечной сходимост ипо по определению. 
\[f_n \ra f \text{ на } E \Lra \forall x \forall \epsilon > 0 \exists N: 
\forall n > N (|f_n(x) - f(x)| < \epsilon)\]

\begin{definition}
  Говорят, что \(\{f_n\}\) равномерно сходится к \(f\) на множестве \(E\), если 
  \[\forall \epsilon > 0 \exists N: \forall x \forall n > N (|f_n(x) - f(x)| < \epsilon)\]
  Пишут \(f_n \rightrightarrows f\) на \(E\), или \(f_n \rightrightarrows_E f\)
\end{definition}
\begin{note}
  Равномерная сходимость влечет поточечную
\end{note}
\begin{note}
  Если \(f_n \rightrightarrows f\) на \(E\), то \(f\) определена на \(E\) однозначно
\end{note}

\begin{lemma}[Супремум критерий]
  \(f_n \rightrightarrows_E f \Lra \lim_{n \ra \infty}\rho_n = 0\), где \(\rho_n = \sup_{x \in E}|f_n(x) - f(x)|\)  
\end{lemma}
\begin{proof}
  \[\forall x \in E (|f_n(x) - f(x)| < \epsilon), \sup_{x \in E}|f_n(x) - f(x)| \le \epsilon\]
  Эти условия равносильны, поэтому лемма верна
\end{proof}

\begin{problem}
  \(f_n \rightrightarrows f \Lra \forall \{x\} \subset E\;\; \lim_{n \ra \infty} |f_n(x_n) - f(x_n)| = 0\)
\end{problem}

\begin{definition}
  Функциональная последовательность поточечно (равномерно) сходится на множестве \(E\), если найдется такая определенная на \(E\) функция, к которой последовательность поточечно (равномерно) сходится
\end{definition}

Пусть задан функциональный ряд \(\sum_{n = 1}^\infty u_n\), где \(u_n: E \ra \R\)
\begin{definition}
  Говорят, что \(\sum_{n = 1}^\infty u_n\) сходится на \(E\), если \(\forall x \in E \left(\sum_{n = 1}^\infty u_n(x)\right)\) сходится. При этом, функция \(S: E \ra \R, S(x) = \sum_{n = 1}^\infty u_n(x)\) называется суммой ряда \(\sum_{n = 1}^\infty u_n\)
\end{definition}

\begin{definition}
  Функциональный ряд поточечно (равномерно) сходится на \(E\), если последовательность частичных сумм \(S_N = \sum_{n = 1}^N u_n\) поточечно (равномерно) сходится на \(E\)
\end{definition}

\begin{proposition}
  Пусть \(g: E \rightrightarrows \R\) ограничена
  \begin{enumerate}
    \item Если  \(f_n \rightrightarrows f\) на \(E\), то \(gf_n \rightrightarrows gf\) на \(E\)
    \item Если \(\sum_{n = 1}^\infty u_n\) равномерно сходится на \(E\), то \(\sum_{n = 1}^\infty gu_n\) также равномерно сходится на \(E\), причем 
    \[\sum_{n = 1}^\infty gu_n = g\sum_{n = 1}^\infty u_n\]
  \end{enumerate}
\end{proposition}
\begin{proof}\indent
  \begin{enumerate}
    \item Пусть \(|g| \le M\). Для любого \(x \in E\) имеем
    \[|g(x)f_n(x) - g(x)f(x)| \le M|f_n(x) - f(x)|\]
    \[\sup_{x \in E}|g(x)f_n(x) - g(x)f(x)| \le M\sup_{x \in E}|f_n(x) - f(x)|\]
    \item Очевидно
  \end{enumerate}
\end{proof}
\begin{proposition}\indent
  \begin{enumerate}
    \item Если \(f_n \rightrightarrows f\) на \(E\), \(g_n \rightrightarrows g\) на \(E\), то \(\lambda f_n + 
    mu g_n \rightrightarrows \lambda f + \mu g\) на \(E\).
    \item Если \(\sum_{n = 1}^\infty u_n, \sum_{n = 1}^\infty v_n\) равномерно сходится на \(E\), и \(\lambda, \mu \in \R\), то \(\sum_{n = 1}^\infty (\lambda u_n + \mu v_n)\) равномерно сходится на \(E\), причем \(\sum_{n = 1}^\infty (\lambda u_n + \mu v_n) = \lambda \sum_{n = 1}^\infty u_n + \mu \sum_{n = 1}^\infty v_n\)
  \end{enumerate}  
\end{proposition}
\begin{proof}\indent
  \begin{enumerate}
    \item \(f_n + g_n \rightrightarrows f\) на \(E\).
    \[\forall x \in E |(f_n(x) + g_n(x)) - (f(x) + g(x))| \le |f_n(x) - f(x)| + |g_n(x) - g(x)|\]
    \[\sup_{x \in E} |(f_n(x) + g_n(x)) - (f(x) + g(x))| \le \sup_{x \in E}|f_n(x) - f(x)| + \sup_{x \in E}|g_n(x) - g(x)|\]
  \end{enumerate}
\end{proof}

\begin{corollary}
  Если \(\sum_{n = 1}^\infty u_n\) равномерно сходится на \(E\), то \(u_n \rightrightarrows 0\) на \(E\)
\end{corollary}
\begin{proof}
  Если \(S_n\) --- \(n\)-ая частичная сумма \(sum_{n = 1}^\infty u_n\), то \(u_n = S_n - S_{n - 1} \rightrightarrows S - S = 0\)
\end{proof}

\begin{problem}
  Пусть \(f_n \rightrightarrows f\) на \(E\), \(g: D \ra E\), тогда \(f_n \circ g \rightrightarrows f \circ g\) на \(D\)
\end{problem}

\begin{theorem}[Критерий Коши]
  \(\{f_n\}\) равномерно сходится на \(E \Lra \forall \epsilon > 0 \exists N \forall m, n > N \forall x \in E (|f_n(x) - f(x)| \le \epsilon) (1)\)
\end{theorem}
\begin{proof}\indent
  \begin{enumerate}
    \item[\(\Ra\)] Пусть \(f_n \rightrightarrows f\) на \(E\). Зафиксируем \(\epsilon > 0, n \ge N\). Тогда \(\forall x \forall n, m \ge N |f_n(x) -f(x)| \le |f_n(x) - f(x)| + |f_m(x) - f(x)| < \epsilon\)
    \item[\(\La\)] Пусть \(\{f_n\}\) удовлетворяет \((1)\). Тогда \(\forall x \in E \{f_n(x)\}\) фундаментальна. Положим \(f(x) = \lim_{n \ra \infty}f_n(x)\), зафиксируем \(\epsilon > 0\) и выберем \(N\) из условия \((1)\). Тогда 
    \[|f_n(x) - f(x)| \le \epsilon \forall x \in E \forall n \ge N\]
    Это означает, что \(f_n \rightrightarrows f\) на \(E\).
  \end{enumerate}
\end{proof}

\begin{corollary}[Критерий Коши]
  \(\sum_{n = 1}^\infty u_n \text{ равномерно сходится на }E \Lra \forall \epsilon > 0 \exists N \forall m, n \ge N \forall x \in E \left(\left|\sum_{k = m + 1}^n u_k(x)\right| < \epsilon\right)\)
\end{corollary}

\begin{corollary}
  Пусть  \(E \subset \R\), все функции \(f_n\) непрерывны на \(\overline{E}\). Если \(\{f_n\}\) равномерно сходится на \(E\), то \(\{f_n\}\) равномерно сходится на \(\overline{E}\)
\end{corollary}
\begin{proof}
  Зафиксируем \(\epsilon > 0\). Тогда по Критерию Коши, \(\exists N \forall n, m > N \forall x \in E (|f_n(x) - f_m(x)| \le \epsilon)\). Пусть \(y \in \overline{E} \Ra \exists \{x_n\} \subset E: (x_k \ra y)\). В неравенстве \(|f_n(x_k) - f_m(x_k)| \le \epsilon\) переходим к прелельному переходу, получаем, что \(|f_n(y) - f_m(y)| \le \epsilon\). Тогда \(\{f_n\}\) равномерно сходится на \(\overline{E}\)
\end{proof}

\begin{example}
  \(\sum_{n = 1}^\infty \frac{1}{n^x}\) --- сходится  на \((1, \infty)\) неравеномерно.
\end{example}
\begin{proof}
  Предположим противное. Но тогда, по следствию 2, \(\sum_{n = 1}^\infty \frac{1}{n^x}\) равномерно сходтися на \([1, \infty)\), противоречие
\end{proof}