% !TEX root = ../../../main.tex

\section{\(P\) vs \(NP\)}

\subsection{Неформально}

\begin{definition}
    \(P\) --- полиномиальное время. 
\end{definition}

\begin{definition}
    \(NP\) --- недетерминированное полиномиальное время
\end{definition}

\subsection{Задача раскраски}

\begin{problem}[2-Раскраска]
    Возможно ли раскрасить все вершины графа в два цвета так, что никакие две вершины одного цвета не соединены ребром?
\end{problem}
\begin{proof}[Решение]
    Спасибо, Илья Даниилович, за то, что научили такое решать за \(O(n + m)\)
\end{proof}

\begin{problem}[3-Раскраска]
    Возможно ли раскрасить все вершины графа в 3 цвета так, что никакие две вершины одного цвета не соединены ребром?
\end{problem}
\begin{proof}[Решение]
    Пока что Илья Даниилович не научил такое решать. Все известные алгоритмы экспоненциальны. Тем не менее, если раскраска дана, то это можно проверить за \(O(poly)\)
\end{proof}

\subsection{Задача на графах}

\begin{problem}[Эйлеровость графа]
    Является ли данный граф эйлеровым?
\end{problem}
\begin{proof}[Решение]
    Существует критерий эйлеровости, который работает за линию
\end{proof}

\begin{problem}[Гамильтоновость графа]
    Является ли данный граф гамильтоновым?
\end{problem}
\begin{proof}[Решение]
    Нет критерия, который бы легко проверялся. Однако, если дан цикл, то можно проверить, что он гамильтонов
\end{proof}


\subsection{Задача про простоту}


\begin{problem}[Проверка на простоту]
    Является ли данное число простым
\end{problem}
\begin{proof}[Решение]
    Существует алгоритм, работающий за \(O(\log^6 n)\)
\end{proof}

\begin{problem}[Разложение на множители]
    Даны \(n, a, b\). Существует ли \(d: d|n \wedge d \in [a, b]\)
\end{problem}
\begin{proof}[Решение]
    Существует алгоритм для квантового компьютера. Для обычных компьютеров пока что такого нет.
\end{proof}

Проще говоря, \(NP\) --- множество таких задач, которые можно быстро проверить за полиномиальное время.

\subsection{Формально}

\begin{definition}
    \(A \subset \{0, 1\}^*\). Тогда \(A \in P\), если \(\exists M\) --- Машина Тьюринга, что
    \begin{enumerate}
        \item \(\forall x (x \in A \Lra M(x) = 1)\)
        \item \(\exists c, d: \forall x M(x)\) останавливается не больше чем за \(c|x|^d\) шагов
    \end{enumerate}
\end{definition}

\begin{definition}
    \(A \subset \{0, 1\}^*\). Тогда \(A \in NP\), если \(\exists V(x, y)\) --- Машина Тьюринга, что
    \begin{enumerate}
        \item \(\forall x (x \in A \Lra \exists y V(x, y) = 1)\)
        \item \(\exists c, d: \forall x, y V(x, y)\) останавливается не больше чем за \(c|x|^d\) шагов
    \end{enumerate}
\end{definition}

\begin{proposition}
    \(NP \subset EXP\), где \(EXP\) --- множество, разрешимое за \(O(2^{poly(n)})\).
\end{proposition}
\begin{proof}
    В определении выше, количество бит в \(y\) ограничено \(c|x|^d\). Тогда можно перебрать все \(y\).
\end{proof}

\section{Заключение}
Ура! Курс логики закончился. Впереди --- сессия! Желаю всем удачи!