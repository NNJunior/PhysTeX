% !TEX root = ../../../main.tex

\begin{theorem}[О трансфинитной рекурсии]
    Пусть задано рекурсивное правило:
    \[F: f|_{[0, x)} \mapsto f(x) \in R\]
    Тогда \(\exists! f: S \ra R\), т.ч. \(\forall x f(x) = F(f|_{[0, x)})\)
    % Если задано правило \(f(x) = F(f|_{[0, x)})\), то есть ровно одна \(f\), которая соответствует правилу.
\end{theorem}
\begin{proof}\indent
    \begin{enumerate}
        \item[] \textbf{Единственность}. Пусть \(f, g\) --- 2 подходящие функции.
        \[\{x | f(x) \ne g(x)\} \ne \emptyset \Ra \exists m = \min\{x| f(x) \ne g(x)\} \Ra f|_{[0, m)} = g|_{[0, m)}\]
        Но тогда \(f(m) = F(f|_{[0, m)}) = F(g|_{[0, m)}) = g(m)\), противоречие.

        \item[] \textbf{Cуществование}. По трансфинитной индукции докажем сущесвование \(f|_{[0, x)}\), соответствующее \(F\).
        \[\forall y < x \exists f|_{[0, y)} \Ra \exists f|_{[0, x)}\]
        \begin{enumerate}
            \item \(x = w + 1 \Ra \exists f|_{[0, w)}, f(w) = F(f|_{[0, w)})\)
            \item \(x\) --- предельное
            \[y < x \Ra \exists z: y < z < x\]
            \[z < x \Ra \exists f: [0, z) \ra R\]
            Так и доопределяем \(f(y)\) (если разные \(z\) дают разные значения, то противоречие аналогично с доказательством единственности). То есть \(\forall y < x\) задано \(f(y) \Ra f\) задано на \([0, x)\).
        \end{enumerate}
    \end{enumerate}
    По трансфинитной индукции получили, что \(\forall x \phi(x)\). 
    Теперь нужно сделать последний переход ко всему множеству (\textit{Прим. от автора:} мы научились делать ее на начальных отрезках \(\Ra\) для ''самых больших элементов'' потенциально могут быть проблемы, т.к. начальные отрезки --- полуинтервалы. Их мы и будем чинить последним переходом). Если в множестве есть наибольший элемент, то доопределяем так же, как и в случае а) (Важно: наибольший элемент может быть предельным). Если наибольшего элемента нет, то доопределяем значение, как в пункте б).
\end{proof}
\begin{theorem}[Обобщенная теорема о трансфинитной рекурсии]
    \(F\) может быть частично определена, тогда \(f\) определена на начальном отрезке.
\end{theorem}
\begin{proof}
    Добавим значение \(f(x) = \perp\), если функция \(f\) не определена в точке \(x\). Тогда по теореме о Трансфинитной рекурсии, \(\exists! f: S \ra R \cup \{\perp\}\).
\end{proof}
\begin{theorem}[О сравнимости ВУМов]
    Любые два ВУМа либо изоморфны, либо один из них изоморфен начальному отрезку другого.
\end{theorem}
\begin{proof}
    Строим \(f: S \ra T\), заданное правилом \(F(f|_{[0, x)}) = \min (T \setminus f([0, x)))\). По обобщенной теореме о трансфинитной рекурсии, \(\exists! f\), соответствующая \(F\). Есть два случая:
    \begin{enumerate}
        \item \(f\) определена на \(S\). \(Im_f = \left[\begin{array}{l}
            T \\
            \left[0, t\right)
        \end{array}\right.\). Тогда  иначе \(\exists t_1 < t_2: t_1, t_2 \notin Im_f\).
        \item \(f\) определена на \([0, s) \Ra Im_f = T\), иначе доопределим \(f(s)\)
    \end{enumerate}
\end{proof}

\begin{proposition}
    \(S\) --- ВУМ, \(s \in S \Ra s \not\cong [0, s)\)
\end{proposition}
\begin{proof}
    Иначе \(\exists\) монотонная \(g: S \ra [0, s) \Ra\) т.к. \(g(s) \ge s\) (нетрудно доказать) \(\Ra g(s) \notin [0, s)\), противоречие.
\end{proof}
\begin{corollary}
    Из \(S \cong T, S \cong [0, t), T \cong [0, s)\) выполнено ровно 1 утверждение
\end{corollary}

\begin{theorem}[Цермело]
    У любого множества есть равномощный ему ВУМ
\end{theorem}

Из теоремы Цермело и теоремы о сравнимости ВУМов:
\begin{corollary}
    \[\forall A, B\;\;\left[\begin{array}{l}
        \exists B' \subset B: A \cong B' \\
        \exists A' \subset A: B \cong A' \\
    \end{array}\right.\]
\end{corollary}

\section{Ординалы}
\begin{definition}
    \(S\) --- транзитивно, если \(y \in S, x \in y \Ra x \in S\).
\end{definition}
\begin{example}
    \(\emptyset, \{\emptyset\}\) и все элементы \(\N\)
\end{example}
\begin{example}
    \(\{\emptyset, \{\emptyset\}, \{\{\emptyset\}\}\}\)
\end{example}
\begin{definition}
    Ординал --- транзитивное множество, любой элемент которого --- транзитивен.
\end{definition}

Неформально --- порядковый тип (отношение эквивалентности на всех множествах)

\begin{proposition}
    \(\alpha\) --- ординал, тогда \(\beta \subset \alpha\) --- тоже.
\end{proposition}
\begin{proof}
    \(\beta\) --- транзитивно, т.к. \(\beta\) --- элемент ординала. \(\gamma \in \beta \Ra\) по транизитивности \(\alpha \Ra \gamma \in \alpha \Ra \gamma\) --- транзитивно.
\end{proof}

\begin{proposition}
    \(\alpha\) --- ординал \(\Ra \alpha \cup \{\alpha\}\) --- ординал.
\end{proposition}
\begin{proof}
    \[\beta \in \alpha \cup \{\alpha\} \Ra \beta \in \alpha \vee \beta = \alpha\] 
    В обоих случаях, \(\beta\) транзитивно.
    Теперь рассмотрим \(\gamma \in \beta\). 
    \[\begin{array}{l}
        \beta \in \alpha \Ra \gamma \in \alpha \\
        \beta = \alpha \Ra \gamma \in \alpha
    \end{array}\]
    Т.к. \(\alpha\) --- транзитивно, то и \(\gamma\) --- тоже.
\end{proof}
\begin{proposition}
    Объединение любого множества ординалов --- ординал.
\end{proposition}
\begin{proof}
    \[\alpha = \bigcup_{i \in I}\alpha_i\]
    \[\gamma \in \beta, \beta \in \alpha \Ra \gamma \in \beta, \beta \in \alpha_i \Ra \beta, \gamma \in \alpha_i\]
    \(\Ra \beta, \gamma\) транзитивны
\end{proof}

\begin{proposition}
    Ординал --- ВУМ с отношением \(\in\) (как строгого порядка)
\end{proposition}
\begin{proof}\indent
    \begin{enumerate}
        \item Антирефлексивность: По Аксиоме фундированности, \(\neg\exists x_1 \ni x_2 \ni x_3 \dots \Ra x \notin x\)
        \item Антисимметричность: \(\neg\exists x, y (x \in y \wedge y \in x) \Ra\)
        \item Транзитивность: по определению
        \item Линейность: \(x\) --- минимальный элемент, не сравнимый с кем-то, а \(y\) --- минимальный, не сравнимый с \(x\).
        \(z \in x \Ra z \text{ сравнимо с } y\). 
        
        Но \(z \ne y\), поэтому \(y \in z \Ra y \in x\), тогда \(z \in y \Ra x \subset y\).
        
        Теперь, \(w \in y \Ra w\) --- сравним с \(x, w \ne x\). \(x \in w \Ra x \in y\). Поэтому \(w \in x (\Ra y \subset x)\).
        Но тогда \(x = y\), противоречие.
    \end{enumerate}
\end{proof}
\begin{proposition}
    \(\alpha\) --- ординал, \(x \in \alpha \Ra x = [0, x)\)
\end{proposition}
\begin{proof}
    \(y \in x \Ra\) по транизитивности \(y \in \alpha \). \(y \in [0, x)\)  (по определению начального отрезка). \(y \in [0, x) \Ra y < x \Lra y \in x\)
\end{proof}
\begin{theorem}
    Любой ординал --- ВУМ, с отношением порядка \(\in\), при этом отношение ''быть начальным отрезком'' --- тот же порядок. Подмножества являющиеся ординалами --- тоьлко начальные отрезки. То есть \(\in, \subset, \text{''быть начальным отрезком''}\) --- один и тот же порядок, называемый ординальным
\end{theorem}
\begin{proof}
    очев
\end{proof}
\begin{theorem}[О сравнимости ординалов]
    \(\alpha, \beta\) --- ординалы \(\Ra \alpha = \beta, \alpha \in \beta, \beta \in \alpha\)
\end{theorem}
\subsection{Конечные ординалы}
\[\begin{array}{l}
    0 = \emptyset \\
    1 = \{\emptyset\} \\
    2 = \{\emptyset, \{\emptyset\}\} \\ 
    \vdots \\
    n + 1 = \{0, 1, 2, \dots n\} \\ 
\end{array}\]

\subsection{Сложение ординалов}
Неформально: \(A\) --- ВУМ, \(A \cong \alpha\) --- ординал; \(B\) --- ВУМ, \(B \cong \beta\) --- ординал, тогда \(\alpha + \beta\) --- ординал, изоморфный \(A + B\)

Формально:
\begin{enumerate}
    \item \(\alpha + 0 = \alpha\)
    \item \(\alpha + (\beta + 1) = (\alpha + \beta) + 1\)
    \item \(\alpha + \bigcup \gamma_i = \bigcup (\alpha + \gamma_i)\)
\end{enumerate}
\begin{note}
    \[1 + \omega = 1 + \bigcup n = \bigcup (1 + n) = \omega \ne \omega + 1\]
\end{note}

\begin{proposition}
    \[B \ne \emptyset \Ra A + B \not\cong A\]
    \[\beta > 0 \Ra \alpha + \beta > \alpha\]
\end{proposition}
\begin{proof}
    Верно, т.к. \(\alpha\) --- начальный отрезок \(\alpha + \beta\)
\end{proof}

\begin{theorem}[О вычитании]
    \(\alpha \ge \beta \Ra \exists! \gamma: \beta + \gamma = \alpha\)
\end{theorem}
\begin{proof}
    Рассмотрим \(A \cong \alpha, B \cong \beta \Ra C = A \setminus B\). Тогда \(C\) --- ВУМ, \(\gamma\) --- ординал, \(\cong C\).
\end{proof}

\subsection{Умножение ординалов}
\(\alpha \cdot \beta\) --- ординал, изоморфный \(A \cdot B\) (обратный лексикографический порядок)
\begin{enumerate}
    \item \(\alpha \cdot 0 = 0\)
    \item \(\alpha \cdot (\beta + 1) = \alpha\beta + \alpha\)
    \item \(\alpha \cdot (\bigcup \gamma_i) = \bigcup (\alpha + \gamma_i)\)
\end{enumerate}
\begin{note}
    \[2\cdot \omega = \omega, \omega \cdot 2 = \omega + \omega \ne \omega\]
\end{note}
\begin{theorem}[О делении с остатком]
    Пусть \(\alpha \ne 0, \beta\) --- ординалы. Тогда \(\exists! \gamma, \delta: \beta = \alpha\cdot\gamma + \delta, \delta < \alpha\)
\end{theorem}
\begin{proof}
    Пусть \(\mu\) таково, что \(\alpha \cdot \mu > \beta\)
\end{proof}
\subsection{Возведение в степень}
\begin{enumerate}
    \item \(\alpha^0 = 1\)
    \item \(\alpha^{\beta + 1} = \alpha^\beta \cdot \alpha\)
    \item \(\alpha^{\bigcup \gamma_i} = \bigcup \alpha^{\gamma_i}\)
\end{enumerate}
\begin{note}
    \[2^\omega = \omega\]
\end{note}

\subsubsection{ВУМ, изоморфный \(\alpha^\beta\)}
Рассмотрим все функции, которые \(\ne 0\) на конечном числе элементов. Тогда: \(f > g\), если:
\begin{enumerate}
    \item \(\underbrace{\max \{x| f(x) \ne 0\}}_{m_f}  > \underbrace{\max\{x| g(x) \ne 0\}}_{m_g} \)
    \item \(m_f = m_g, f(m_f) > g(m_g)\)
    \item \(m_f = m_g, f(m_f) = g(m_g), \{x < m_f| f(x) \ne 0\} > \max\{x < m_g|g(x) \ne 0\}\)
\end{enumerate}
Или: рассмотрим конечное число точек, где \(f \ne 0 \vee g \ne 0\), сравниваем обратно лексикографически.
\begin{note}
    \(2^\omega\) по этому определению --- обратная двоичная запись натуральных чисел
\end{note}
\begin{theorem}[Об ординарной системе счисления]
    Пусть \(\gamma < \alpha^\beta\). Тогда \(\exists!\) представление \(\gamma = \alpha^{\beta_1}\cdot\alpha_1 + \alpha^{\beta_2}\cdot\alpha_2 + \dots + \alpha^{\beta_n}\cdot\alpha_n\), где \(\beta > \beta_1 > \beta_2 > \dots > \beta_n, \alpha_i < \alpha\)
\end{theorem}