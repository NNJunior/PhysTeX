% !TEX root = ../../../main.tex

\begin{theorem}
    Если \(A, B\) бесконечны, то \(A \cup B \cong A \times B \cong \max \{A, B\}\)
\end{theorem}

\begin{proposition}[Аксиома Выбора]
    Пусть \(A\) --- множество. Тогда существует \(\phi: 2^A \setminus \{A\}\ra A: \forall S \subsetneq A\;\;\phi(S) \notin S\)
\end{proposition}

\begin{definition}
    \((S, \le_S)\) --- корректный фрагмент множества \(A\), если
    \begin{enumerate}
        \item \(S \subset A\)
        \item \((S, \le_S)\) --- ВУМ
        \item \(\forall x \in S\;\; x = \phi([0, x)_S)\), где \(\phi\) --- функция из аксиомы выбора
    \end{enumerate}
\end{definition}
\begin{lemma}
    Если \(S, T\) --- корректные фрагменты, то тогда \(S\) --- начальный отрезок \(T\), или \(T\) --- начальный отрезок \(S\)
\end{lemma}
\begin{proof}
    По теореме о сравнимости, \(T \cong [0, S)\), или \(T \cong S\), или \(S \cong [0, T)\). Докажем, что изоморфизм тождественнен
    Б.О.О. \(T \cong [0, S)\). Пусть \(\mu: [0, S) \ra T\) --- изоморфизм. Пусть \(x\) --- минимальный элемент, такой, что \(\mu(x) \ne x\). Но тогда, т.к. \(x\) --- минимальный такой элемент, тогда \([0, x)_S = [0, x)_T \Ra x = \phi([0, x)_S) = \phi([0, x)_T) \Ra x \in T\). Но тогда \(\Ra \mu(x) <_T x \vee \mu(x) >_T x\). Предположим, что \(\mu(x) > x\), тогда \(z = \mu^{-1}(x) >_S x \Ra \mu(z) = x <_T \mu(x)\). Если \(\mu(x) < x\), тогда \(\mu(\mu(x)) < \mu(x) < x\), противоречие с тем, что \(x\) --- минимальный
\end{proof}

\begin{lemma}
    Объединение \(S = \bigcup S_i\) любого множества корректных фрагментов является корректным фрагментом относительно порядка \(x \le_S y\), если \(\exists i\;\;x \le_{S_i} y\)
\end{lemma}
\begin{proof}
    Докажем все свойства корректного фрагмента:
    \begin{enumerate}
        \item \textbf{Рефлексивность.} \(x \le_S x\)
        \[x \in S \Ra \exists i: x \in S_i \Ra x \le_{S_i} x \Ra x \le_S x\]
        \item \textbf{Антисимметричность.} \(x \le_S y, y \le_S x \Ra x = y\).
        \[\exists i, j: x \le_{S_i} y, x \ge_{S_j} y\]
        При этом, Б.О.О, \(S_i\) --- начальный отрезок \(S_j\). Тогда \(x \le_{S_j} y, x \ge_{S_j} y \Ra x = y\)
        \item \textbf{Транзитивность.} \(x \le_S y, y \le_S z \Ra x \le_S z\)
        \[x \le_{S_i} y, y \le_{S_j} z\]
        При этом, Б.О.О, \(S_i\) --- начальный отрезок \(S_j\). Тогда \(x \le_{S_j} y, y \ge_{S_j} z \Ra x \le_{S_j} z\)
        \item \textbf{Линейность.} Линейность. \(x \in S_i, y \in S_j\), при этом, Б.О.О, \(S_i\) --- начальный отрезок \(S_j\). Тогда \(x, y\) сравимы отношением \(\le_{S_j}\)
        \item \textbf{Фундированность.} Пусть \(x_1 \ge_S x_2 \ge_S x_3 \dots\), где \(x_k \in S_{i_k}\). Докажем, что \(x_k \in S_{i_1}\). Действительно, в проивном случае \(S_{i_k} \not\subset S_{i_1}\), но \(x_k \le x_1\), и, т.к. \(S_{i_1}\) --- начальный отрезок \(S_{i_k}\) и \(x_k \le x_1\), то \(x_k \in S_{i_1}\). Но тогда в эта бесконечная последовательность стабилизируется, из чего следует, что наше множесво --- ВУМ
        \item \textbf{Корректность.}
        \[x \in S \Ra x = \phi([0, x)_S)\]
        \[x \in S_i, y \le_S x \Lra y \le_{S_i} x\]
        \[[0, x)_S = [0, x)_{S_i}\]
        Но тогда \(x = \phi([0, x)_{S_i}) = \phi([0, x)_S)\), т.к. \(S_i\) --- корректный фрагмент.
    \end{enumerate}
\end{proof}

\begin{lemma}
    Объединение всех корректных фрагментов является исходым множеством
\end{lemma}
\begin{proof}
    Пусть объединение всех корректных фрагментов дало \(B \subseteq A\). Тогда \(B \cup \{\phi(B)\}\) --- тоже корректный фрагмент. Проиворечие с тем, что \(B\) содержало все корректные фрагменты.
\end{proof}

\begin{theorem}[Цермело]
    \(\forall A \exists B\) --- ВУМ, такой, что \(B \cong A\)
\end{theorem}
\begin{proof}
    Объединение всех корректных фрагментов будет являться ВУМом и будет равно исходному множеству.
\end{proof}

\begin{corollary}
    Любые два множества сравнимы по мощности
\end{corollary}

\begin{definition}
    Пусть \(A\) --- ЧУМ. Цепь в нем --- его линейно упорядоченное подмножество
\end{definition}
\begin{definition}
    Пусть \(A\) --- ЧУМ, \(S \subset A\). Верхняя грань \(S\) --- такой элемент \(b \in A\), что \(\forall x \in S x \le b\).
\end{definition}
\begin{lemma}[Цорна]
    Пусть у любой цепи есть верхняя грань. Тогда в \(A\) есть максимальный элемент, более того, \(\forall x \in A \exists\) максимальный элемент \(m \ge x\)
\end{lemma}
\begin{proof}
    Пусь \(I\) --- ВУМ, мощнее \(A\). Построим функцию \(f: I \ra A, f(y) =\) элемент, больший всех элементов \(f([0, y))\). По теореме о трансфинитной рекурсии, \(\exists! f\), удовлетворяющая такому условию. \(f\) --- инъективно \(\Ra f\) определена на начальном отрезке \([0, S)\). Образ \(f([0, S))\) --- цепь в \(A\). По условию леммы, у \(f([0, S))\) есть верхняя грань \(m\). \(m \notin f([0, S)) \Ra\) можно доопределить \(m = f(S)\). Если \(m\) --- не максимальный, то можно доопределить \(f(s) = w > m\).
\end{proof}