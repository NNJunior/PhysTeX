% !TEX root = ../../../main.tex

\section{Теория вычислимости}
\subsection{Алгоритм}
Неформально: алгоритм --- это процедура, преображающая данные, закодированные конечными словами, которая тоже имеет конечное описание и выполняется пошагово

\begin{example}[Не алгоритм]
    Метод Ньютона нахождения нуля дифференцируемой. Не является алгоритмом, т.к. нельзя сделать предельный переход, однако, если установить точность с которой мы хотим узнать корень, тогда норм.
\end{example}
\begin{example}[Не алгоритм]
    Метод дележа пирога. Есть пирог, хотим поделить его. Берем нож и несем его над пирогом. Второй человек говорит, когда нам остановиться. Тогда мы и режем пирог. Не является алгоритмом, т.к. время тут не дискретно.
\end{example}

\section{Однолетночная машина Тьюринга}
Одноленточная машина Тьюринга
\[\begin{array}{c|c|c|c|c|c|c|c}
    \hline
    \dots & m & a & t & l & o & g & \dots \\
    \hline
\end{array}\]
Есть бесокнечная в обе стороны лента (см. выше), в которой хранятся некоторые символы некоторого алфавита. Машина тьюринга представляет собой функцию от этой ленты:
\(qa \mapsto rbD\)
\begin{enumerate}
    \item \(q\) --- текущее состояние машины
    \item \(a\) --- символ в ячейке, на которую смотрит машина Тьюринга
    \item \(r\) --- новое состояние машины
    \item \(b\) --- новый символ в ячейке
    \item \(D\) --- направление сдвига \(L, N, R\)
\end{enumerate}

Итак, формально:
\begin{definition}
    Машина Тьюринга --- это кортеж \((\Sigma, \Gamma, Q, q_1, q_0, \delta)\), где \(Sigma, \Gamma, Q\) --- конечные множества
    \begin{enumerate}
        \item \(\Sigma\) --- входной алфавит
        \item \(\Gamma \supset \Sigma\) --- ленточный алфавит. \(\#\) --- пробел, принадлежит \(\Gamma \setminus \Sigma\)
        \item \(Q \cap \Gamma = \emptyset\) --- множество состояний
        \item \(q_1, q_0 \in Q\) --- начальное и кончное состояния соответственно
        \item \(\delta: Q \times \Gamma \ra Q \times \Gamma \times \{L, N, R\}\)
    \end{enumerate}
\end{definition}

Удобно записывать состояние машины Тьюринга так:
\[AqaB\]
\begin{enumerate}
    \item \(a\) --- символ, на который мы сейчас смотрим
    \item \(q\) --- состояние машины в данный момент
    \item \(A, B\) --- лента до и после символа, на который мы смотрим, соответственно
\end{enumerate}

\begin{definition}
    Вычисление --- последовательность конфигураций, где каждая следующая получается из предыдущей по правилу Машины Тьюринга
\end{definition}

\begin{definition}
    Функция \(f: \Sigma^* \ra \Sigma^*\) называется вычислимой, если существует Машина Тьюринга, такая, что
    \[\forall x\left\{\begin{array}{l}
        f(x) \text{ определена }\Ra M(x) = f(x) \\ 
        f(x) \text{ не определена }\Ra M(x) \text{не останавливается} \\
    \end{array}\right.\]
\end{definition}

Заметим, что Машин Тьюринга счетное множество, а функций \(\Sigma^* \ra \Sigma^*\) континуум  \(\Ra \exists\) невычислимые функции.

\begin{proposition}
    У \(f\) конечная область определения \(\Ra f\) вычислима
\end{proposition}
\begin{proposition}
    \(f, g\) --- вычислимы \(\Ra f\circ g\) --- тоже
\end{proposition}

\begin{definition}
    \(S \subset \Sigma^*\) разрешимо, если \(\exists\) Машина Тьюринга с бинарным ответом, такая, что \(M(x) \left\{\begin{array}{l}
        1, x \in S \\
        0, x \notin S
    \end{array}\right.\)
    Иначе говоря, \(S\) разрешимо, если \(\chi_S(x) = \left\{\begin{array}{l}
        1, x \in S \\
        0, x \notin S
    \end{array}\right.\) вычислима
\end{definition}

\begin{proposition}
    \(S\) --- конечное \(\Ra S\) разрешимо
\end{proposition}
\begin{proposition}
    \(S, T\) --- разрешимы \(\Ra S\cup T, S \cap T,\overline{S}\) --- тоже
\end{proposition}

\begin{note}
    Подмножество разрешимого множества может быть неразрешимо.
\end{note}

\begin{theorem}[Критерий Разрешимости]
    \(S\) разрешимо \(\Lra S\) можно перечислить по возрастанию
\end{theorem}
\begin{proof}\indent
    \begin{enumerate}
        \item[\(\Ra\)] 
        \begin{verbatim}
for i in 0, 1, 2...
    if i in S:
        print i
        \end{verbatim}
        \item[\(\La\)] Пусть \(S\) бесконечно, а на вход подается \(x\). Печатаем элементы, пока они \(< x\). Тогда первый элемент, на котором это сломается будет либо равен \(x\), либо будет \(> x\). В первом случае выдаем 1, иначе 0. Для конечных оно и так разрешимо.
    \end{enumerate}
\end{proof}

\begin{definition}
    Множество называется перечислимым, если существует Машина Тьюринга, которая выводит все его элеметы \(S\) и только их
\end{definition}

\begin{theorem}[Поста]
    \(S\) разрешимо \(\Lra S, \overline{S}\) перечислимы
\end{theorem}
\begin{proof}
    \item[\(\Ra\)] пробегаемся по \(\Sigma^*\), и, если элемент \(\in S\), то выводим его. Аналогично для \(\overline{S}\)
    \item[\(\La\)] По очереди перечисляем элементы \(S, \overline{S}\). Рано или поздно мы встретим \(x \Ra\) выведем, где мы его встретили, в \(S\) или \(\overline{S}\).
\end{proof}