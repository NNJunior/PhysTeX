% !TEX root = ../../../main.tex

\subsection{Свойства, эквивалентные перечислимости}
Далее будем считать, что наш алфавит --- \(\{0, 1\}\).
\begin{enumerate}
    \item Можно выводить все элементы, но без повторов
    \item Вычислима полухарактеристическая функция \(\overline{\chi_A}(x) = \left\{\begin{array}{l}
        1, x \in A \\
        \text{не определено, }x \notin A
    \end{array}\right.\)
    \item \(A\) --- область определения некоторой вычислимой функции
    \item \(A\) --- область значений некоторой вычислимой функции
    \item \(A = \emptyset\) или \(A\) --- область значений всюду вычислимой функции.
    \item \(A\) --- проекция разрешимого множества пар \(A = \{x | \exists y (x, y) \in B\}\), где \(B \subset \{0, 1\}^*\times \{0, 1\}^*\)
\end{enumerate}

\begin{proposition}
    \(A\) вычислимо \(\Ra\) 2)
\end{proposition}
\begin{proof}\indent
    \begin{verbatim}
    chi_A(x) {
        fot i in A {
            if i == x { // если встретим x, то вернем 1
                return 1;
            }
        }
    } 
    \end{verbatim}
\end{proof}
\begin{proposition}
    2) \(\Ra\) 3)
\end{proposition}
\begin{proof}
    \(A = Dom\;\overline{\chi_A(x)}\)
\end{proof}
\begin{proposition}
    3) \(\Ra\) 4)
\end{proposition}
\begin{proof}
    Рассмотрим \texttt{f'(x)}:
    \begin{verbatim}
    f'(x) {
        f(x);
        return x;
    }
    \end{verbatim}
    Тогда \(f'(x) = \left\{\begin{array}{l}
        x, x \in Dom\;f \\
        \text{не определено, иначе}
    \end{array}\right.\)
    Заметим, что \(Rad\;f' = Dom\;f\).
\end{proof}
% \begin{proposition}
%     1) \(\Ra\) 5)
% \end{proposition}
% \begin{proof}
    
% \end{proof}
\begin{proposition}
    4) \(\Ra\) 5)
\end{proposition}
\begin{proof}
    Пусть \(A = Ran\;f\). Если \(A \ne \emptyset\), то положим \(a_0\) --- произвольный элемент в \(a_0\). Положим \(f': \{0, 1\}^*\times\N \ra \{0, 1\}^*\), так, что 
    \[f'(x, t) = \left\{\begin{array}{l}
        f(x), \text{ если }f(x)\text{ остановится за }t\text{ шагов} \\
        a_0, \text{ иначе}
    \end{array}\right.\]
    Заметим, что \(Ran\;f = Ran\;f'\), а \(f'\) --- вычислима.
\end{proof}
\begin{proposition}
    5) \(\Ra\) 6)
\end{proposition}
\begin{proof}
    Пусть \(A = Ran\;f\). Положим \(B = \{(y, (x, t)): f(x) = y\text{ за \(t\) шагов}\}\).
\end{proof}
\begin{proposition}
    6) \(\Ra A\) вычислимо
\end{proposition}
\begin{proof}
    Обойдем все пары \((x, y)\), и, если \((x, y) \in B \Ra \) печатаем \(x\).
\end{proof}

\subsection{Универсальная машина Тьюринга}
Гарвардская архитектура машины --- когда есть фиксированная программа и данные, с которыми она работает.

Принстонская архитектура машины --- когда есть некоторый процессор, который может запускать различные программы, которые, в свою очередь, будут взаимодействовать с данными
\begin{definition}
    Универсальная Машина Тьюринга --- такая функция \(U(M, x) = M(x)\) --- по сути, машина, которая запускает машину \(M\) с вводом \(x\).
\end{definition}
\begin{definition}
    Будем считать, что код Машины Тьюринга записан (как-то) последовательностью \(0, 1\). Фунуция \(U: \{0, 1\}^*\times\{0, 1\}^* \ra \{0, 1\}^*\) называется универсальной вычислимой функцией, если
    \begin{enumerate}
        \item \(U\) вычислима как функция от двух аргументов
        \item \(\forall f: \{0, 1\}^* \ra \{0, 1\}^*\), где \(f\) --- вычислима, верно \(\exists p \forall x\;U(p, x) = f(x)\)
    \end{enumerate}
\end{definition}