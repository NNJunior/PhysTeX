% !TEX root = ../../../main.tex

\begin{theorem}
    Универсальная Машина Тьюринга существует.
\end{theorem}

\subsection{Тезис Черча-Тьюринга}
\href{https://ru.wikipedia.org/wiki/Тезис_Чёрча_—_Тьюринга}{\textit{Подробнее...}}

\subsection{Проблема самоприменимости}
Пусть нам дана УМТ \(U\). Рассмотрим \(S = \{p | U(p, p) \text{ остановится}\}\)

\begin{theorem}[Тьюринга]
    \(S\) --- перечислимо, неразрешимо
\end{theorem}
\begin{proof}\indent
    \begin{enumerate}
        \item[] \textbf{Перечислимость:} \(S\) перечислимо, т.к. \(S = Dom\;d(p) = U(p, p)\) --- область определения вычислимой функции
        \item[] \textbf{Нерезрешимость:} от противного. Пусть \(S\) разрешимо \(\Lra \chi_S\) вычислимо. Рассмотрим 
        \[f(x) = \left\{\begin{array}{l}
            0, \chi_S(x) = 0 \\
            \ne U(x, x), \chi_S(x) = 1\\
        \end{array}\right.\]
        Т.к. \(U(p, p)\) --- УВФ, то \(\exists q \forall x U(q, x) = f(x)\). 
        \begin{enumerate}
            \item \(q \in S \Ra U(q, q) = f(q) \ne U(q, q)\)
            \item \(q \notin S \Ra f(q) = 0\), но \(U(q, q)\) не определено
        \end{enumerate}
        Получили противоречие.
    \end{enumerate}
\end{proof}

\subsection{Проблема остановки}
\(H = \{(p, x) | U(p, x) \text{ остановится}\}\)
\begin{theorem}
    \(H\) --- перечислимо, неразрешимо
\end{theorem}
\begin{proof}\indent
    \begin{enumerate}
        \item[] \textbf{Перечислимость:} \(H\) перечислимо, т.к. \(S = Dom\;U(p, x)\) --- область определения вычислимой функции
        \item[] \textbf{Нерезрешимость:} от противного. Пусть \(H\) разрешимо, но тогда и \(S = H \cap D\) тоже разрешимо, где \(D = \{(p, p) | p\text{ проивольное}\}\), противоречие, т.к. \(S\) неразрешимо.
    \end{enumerate}
\end{proof}

Рассмотрим множества:
\begin{enumerate}
    \item \(C = \{p| \forall x, y (U(p, x), U(p, y) \text{ определены }\Ra U(p, x) = U(p, y))\}\)
    \item \(T = \{p | \forall x  (U(p, x)\text{ определено})\}\)
    \item \(FD = \{p | \{x | U(p, x)\text{ определено}\}\text{ конечно}\}\)
\end{enumerate}

\begin{definition}
    Множество \(X\) называется коперечислимым, если \(\overline{X}\) перечислимо
\end{definition}

\begin{proposition}
    \(C\) --- коперечислимо, неразрешимо
\end{proposition}
\begin{proof}
    \(\overline{C} =\{p | \exists (x, y, t, s)\;\;U(p, x) \text{ ост. за \(t\) шагов}, U(p, y) \text{ ост. за \(s\) шагов}, U(p, x) \ne U(p, y)\}\), а оно перечислимо
\end{proof}

\begin{proposition}
    \(T, FD\) --- коперечислимо, некоперечислимо
\end{proposition}

\subsection{\(m\)-сводимость}
\begin{definition}
    \(A \le_m B\), если \(\exists\) вычислимая всюду определенная функция \(f: \{0, 1\}^* \ra \{0, 1\}^*\), такая, что \(\forall x (x \in A \Lra f(x) \in B)\).
\end{definition}
\subsubsection{Свойства \(m\)-сводимости}
\begin{proposition}
    \(A \le_m B, B\) разрешимо \(\Ra A\) --- разрешимо
\end{proposition}
\begin{proof}
    \(\chi_A(x) = \chi_B(f(x))\)
\end{proof}
\begin{proposition}
    \(A \le_m B \Lra \overline{A} \le_m \overline{B}\)
\end{proposition}
\begin{proposition}
    \(A \le_m B, B \le_m C \Lra A \le_m C\)
\end{proposition}

\begin{note}
    Т.к. \(A \le_m A\), получили, что \(\le_m\) --- предпорядок
\end{note}

\begin{proposition}
    \(A \le_m B, B\) перечислимо \(\Ra A\) --- перечислимо
\end{proposition}

\begin{corollary}
    \(A \le_m B, A\) неперечислимо \(\Ra B\) неперечислимо
\end{corollary}

\subsection{Как получать новые неперечислимые множества?}
Заметим, что \(\overline{H}\) --- неперечислимое множество. Если мы покажем, что 
\[\left\{\begin{array}{l}
    \overline{H} \le_m T \\
    \overline{H} \le_m \overline{T} \\
\end{array}\right.\]
То мы получим, что \(T\) --- неперечислимо, некоперечислимо. Однако, удобнее доказывать, что 
\[\left\{\begin{array}{l}
    H \le_m T \\
    H \le_m \overline{T} \\
\end{array}\right.\]
Для этого
\begin{enumerate}
    \item \((p, x)\)
\end{enumerate}
Получили \(T\), неперечислимое и некоперечислимое.

\subsection{Существует ли универасальная тотально вычислимая функция?}
Что мы хотим: \(U_T: \{0, 1\}^* \times \{0, 1\}^* \ra \{0, 1\}^*\), такую, что
\begin{enumerate}
    \item \(\forall p, x\;U_T(p, x)\) определена
    \item \(U_T\) вычислима
    \item \(\forall f: \{0, 1\}^* \ra \{0, 1\}^*\), такая, что \(f\) --- вычислима, всюду определена и \(\exists p \forall x f(x) = U_T(p, x)\)
\end{enumerate}

\begin{proposition}
    \(\nexists U_T\)
\end{proposition}
\begin{proof}
    Иначе рассмотрим \(d'(x) = U_T(x, x) + 1\). Тогда \(\exists p \forall x\;d'(x) = U_T(p, x) = d'(p) \Ra U_T(p, p) + 1 = U_T(p, p)\)
\end{proof}

Рассмотрим множество \(NED = \{p | \exists x U(p, x)\text{ определено}\}\). Оно перечислимо (можно добавить квантор, указывающий, за какое число шагов). 
\begin{proposition}
    \(\overline{NED}\) неперечислимо
\end{proposition}
\begin{proof}
    Покажем, что \(H \le_m NED\). Сопостивим \((p, x) \mapsto q\), так, что \(U(q, y) = \left\{\begin{array}{l}
        \text{не определено}, U(p, x) \text{ не остановится} \\
       1, \text{ иначе} \\
    \end{array}\right.\)
    Пусть \(p_1, p_2, \dots \) --- перечисление \(\overline{NED}\).
\end{proof}