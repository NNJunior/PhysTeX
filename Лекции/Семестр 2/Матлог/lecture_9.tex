% !TEX root = ../../../main.tex

\section{Связь этого бреда с языками первого порядка}
\subsection{Арифметическая иерархия}
В прошлом мы рассматривали свойства \textit{Перечислимость}, \textit{Коперечислимость}, а их пересечение давало нам \textit{Разрешимость}. Также мы рассматривали множества:
\[H = \{(p, x) | U(p, x)\text{ определено}\}\]
\[ED\text{ (Empty Domain) } = \{p|\forall x U(p, x)\text{ не определено}\}\]
\[T = \{p| \forall x U(p, x)\text{ определено}\}\]
\[FD\text{ (Finite Domain) } = \{p| \{x |U(p, x)\text{ определено}\}\text{ конечно}\}\]
Заметим, что эти определения этих множеств можно записать так:
\[(p, x) \in H \Lra \exists t(U(p, x)\text{ остановится за }\le t\text{ шагов})\]
\[p \in ED \Lra \forall (t, x)(U(p, x)\text{ не остановится за }t\text{ шагов})\]
\[p \in T \Lra \forall x \exists t(U(p, x)\text{  остановится за }\le t\text{ шагов})\]
\[p \in FD \Lra \exists N \forall (t, x)(x > N \ra U(p, x)\text{ не остановится за }t\text{ шагов})\]

\begin{definition}
    \(A \in \Sigma_k\), если существует разрешимый предикат \(R\), такой, что
    \[\forall x (x \in A \Lra \exists y_1 \forall y_2 \exists y_3 \dots^\forall_\exists y_k R(x, y_1, y_2, \dots y_k))\]
\end{definition}
\begin{definition}
    \(B \in \Pi_k\), если существует разрешимый предикат \(R\), такой, что
    \[\forall x (x \in B \Lra \forall y_1 \exists y_2 \forall y_3 \dots^\forall_\exists y_k R(x, y_1, y_2, \dots y_k))\]
\end{definition}

Итак, по определениям выше:
\begin{enumerate}
    \item Разрешимые множества \(= \Sigma_0 = \Pi_0\)
    \item Перечислимые множества \(= \Sigma_1\)
    \item Коперечислимые множества \(= \Pi_1\)
\end{enumerate}

\begin{theorem}
    \(\Sigma_i \subset \Sigma_{i + 1}, \Sigma_i \subset \Pi_{i + 1}, \Pi_i \subset \Sigma_{i + 1}, \Pi_i \subset \Pi_{i + 1}\)
\end{theorem}
\begin{proof}
    В предикате \(R\) добавим фиктивный аргумент --- тогда квантор, сооветствующий ему можно поставить в любое место формулы. Если поставить его в начало и в конец, получим желаемое.
\end{proof}

\begin{theorem}[б/д]
    \(\Sigma_i \subsetneq \Sigma_{i + 1}, \Sigma_i \subsetneq \Pi_{i + 1}, \Pi_i \subsetneq \Sigma_{i + 1}, \Pi_i \subsetneq \Pi_{i + 1}\)
\end{theorem}

\begin{proposition}
    \(A \in \Sigma_k \Lra \overline{A} \in \Pi_k\)
\end{proposition}
\begin{proposition}
    \(x \in \overline{A} \Lra \neg \exists y_1 \forall y_2 \dots \exists y_k R(x, y_1, \dots y_k) \Lra \forall y_1 \exists y_2 \dots \forall y_k \neg R(x, y_1, \dots y_k)\)
\end{proposition}

\begin{proposition}
    \(A, B \in \Sigma_k \Ra A \cap B \in \Sigma_k\)
\end{proposition}
\begin{proof}\indent
    \[x \in A \cap B \Lra (x \in A \wedge x \in B) \Lra (\exists y_1 \forall y_2 \dots \exists y_k R(x, y_1, \dots y_k) \wedge \exists z_1 \forall z_2 \dots \exists z_k Q(x, z_1, \dots z_k)) \Lra \]
    \[\Lra \exists y_1\exists z_1 \forall y_2 \forall z_2 \dots \exists y_k \exists z_k (R(x, y_1, \dots y_k) \wedge Q(x, z_1, \dots z_k)) \Lra\]
    \[\exists (y_1, z_1) \forall (y_2, z_2) \dots \exists (y_k, z_k) (R(x, y_1, \dots y_k) \wedge Q(x, z_1, \dots z_k))\]
\end{proof}

Рассмотрим следующий язык:
\begin{definition}
    Язык арифметики: \(\langle0, S, =, +, \cdot\rangle\)
\end{definition}

\begin{definition}
    Предикат \(P: \N^k \ra \{0, 1\}\) называется выразимым в арифметики (или арифметичным), если существует формула \(\phi\) c \(k\) параметрами, такая, что \(P(x_1, \dots x_k) = 1 \Lra \phi(x_1, \dots x_k)\) истинно
\end{definition}

\begin{example}\indent
    \begin{enumerate}
        \item \(x \ge y \Lra \exists z\;x = y + z\)
        
        \item \(x \vdots y \Lra \exists z\;x = y \cdot z\)
        
        \item \(p\text{ --- простое }\Lra(p > 1 \wedge (\forall q (p \vdots q \ra (q = p \vee q = 1))))\)
        \item \(d = \text{НОД}(x, y) \Lra (x \vdots d \wedge y \vdots d \wedge \forall t((x\vdots t \wedge y \vdots t) \ra d \vdots t))\)
        \item \(S\text{ --- степень }2 \Lra \forall d(s \vdots d \ra d = 1 \vee d \vdots 2)\)
        \item \(S\text{ --- степень }4 \Lra \exists q:(q \text{ степень }2 \wedge s = q^2)\)
        \item \(S\text{ --- степень }6 \Lra \exists k \exists (s_0, \dots s_k)(s_0 = 1 \wedge s_k = s \wedge \forall i \in [0, k - 1] s_{i + 1} = 6 \cdot s_i)\), но это не формула первого порядка, проблема
    \end{enumerate}
\end{example}

Чтобы понять, как действовать в последнем пункте, надо научиться как-то кодировать такие кортежи произвольной длины. Традиционно сущесвует два способа:
\begin{enumerate}
    \item \(\beta\)-функция Геделя
    \item Кодирование Смаллиана
\end{enumerate}

\subsection{Построение \(\beta\)-функции}
\begin{lemma}
    \(\forall k \forall B \exists b > B: b + 1, 2b + 1, \dots (k + 1)b + 1\) --- попарно взаимно просты
\end{lemma}
\begin{proof}
    Возьмем \(b = k!\cdot c\), где у \(c\) нет простых делителей \(> k\). Положим \(d = ((i + 1)b + 1, (j + 1)b + 1)\) --- делитель \((j - i)b = (j - i)k!c \Ra\) все простые делители \(d < k \Ra (i + 1)b\) делится на простой делитель \(d\), но и \((i + 1)b\) не них делится, противоречие при \(d > 1\).
\end{proof}
\begin{lemma}
    \(\forall (s_0, s_1, \dots s_k) \exists a \exists b \forall i \in [0, k]\; s_i = a \mod((i + 1)b + 1)\) --- попарно взаимно просты
\end{lemma}
\begin{proof}
    Следует из КТО и предыдущей леммы
\end{proof}
\begin{theorem}[О \(\beta\)-функции]
    Существует \(\beta(a, b, i)\), задаваемая арифметической формулой, т.ч. \(\forall (s_0, s_1 \dots s_k) \exists (a, b): \forall i \in [0, k] s_i = \beta(a, b, i)\)
\end{theorem}
\begin{proof}
    Положим \(\beta(a, b, i) = a \mod (i + 1)b + 1\)
\end{proof}

\begin{proposition}
    Через \(\beta\)-функцию можно выразить предикат \(m = 2^n\)
\end{proposition}
\begin{proof}
    \(\exists (m_0, m_1, \dots m_k): m_0 = 1 \wedge m_n = m \wedge \forall i\;m_i = 2m_{i - 1}\).
\end{proof}


\subsection{Кодирование Смаллиана}
\(n \mapsto \widehat{n} = bin(n + 1)\text{ без ведущей единицы}\).
Тогда \((n, m) \mapsto \widehat{k}\), такое, что \(\widehat{k} = \widehat{n}\cdot\widehat{m}\), где \(\cdot\) --- конкатенация

Второе определение: строим биекцию между строками из 0 и 1 и натуральными числами по возрастанию длины, слова одинаковой длины сравниваем лексигографически
\[\begin{array}{ccccccccccc}
    0 & 1 & 2 & 3 & 4 & 5 & 6 & 7 & 8 & 9 & \dots \\
    \epsilon & 0 & 1 & 00 & 01 & 10 & 11 & 000 & 001 & 010 & \dots
\end{array}\]

\begin{proposition}
    \(\exists\) арифметическая формула \(S(a, b, x)\), такое, что
    \begin{enumerate}
        \item \(\forall a, b \{x | S(a, b, x) = 1\}\) конечно
        \item \(S\) конечно \(\Lra \exists (a, b) \{x | S(a, b, x) = 1\} = S\)
    \end{enumerate}
\end{proposition}
\begin{proof}
    Рассмотрим \(S(a, b, x): axa \sqsupset\sqsubset b \wedge |x| < |a|\) (\(axa\) является подсловом \(b\))
    \begin{enumerate}
        \item очевидно, т.к. \(|x| < |a|\)
        \item рассмотрим a = \(1\underbrace{0 \dots 0}_{L}1, L > \max \{|x_i|\}, b = ax_1ax_2a_3a\dots ax_ka\)
    \end{enumerate}
\end{proof}

