% !TEX root = ../../../main.tex

\section{Квадратичные вычеты и невычеты}

\begin{definition}
    Пусть \(a, m \in \N, (a, m) = 1\). Тогда 
    \begin{enumerate}
        \item[] Если \(\exists x: x^2 \equiv_m a\), то \(a\) называется квадратичным вычетом
        \item[] Если \(\nexists x: x^2 \equiv_m a\), то \(a\) называется квадратичным невычетом
    \end{enumerate}
\end{definition}

Будем рассматривать случай, когда \(m\) --- простое нечетное число

\begin{theorem}[Лагранжа]
    Пусть \(f(x) = a_nx^n + \dots + a_1x + a_0\). Тогда число решений \(f(x) \equiv_p 0\) не превосходит \(n\).
\end{theorem}
\begin{proof}
    От противного: пусть найдутся \(x_1, \dots x_{n+1}\), т.ч. они являются решениями. Заметим, что \(f\) можно представить следующим образом:
    \[
    \begin{array}{rl}
        f(x) & = b_n(x - x_1)\dots(x-x_{n}) \\
        & + b_{n-1}(x - x_1)\dots(x-x_{n-1}) \\
        & \;\;\;\vdots \\
        & + b_1(x - x_1) \\
        & + b_0 \\

    \end{array}
    \]
    Но тогда, подставляя \(x_1 \dots x_{n-1}\) получаем, что все \(b_i = 0 \forall i \le n - 1\). Но тогда \(f(x_{n+1}) \ne 0\). Противоречие.
\end{proof}

\begin{note}
    Если \(m\) --- простое нечетное число, то решений 
    \[x^2 \equiv a^2\]
    Ровно 2 (\(x = \pm a\))
\end{note}

\begin{note}
    Множество всех квадратичных вычетов:
    \[\left\{1^2, 2^2, \dots \frac{p-1}{2}^2\right\}\]
    Итого, квадратичных вычетов \(\frac{p-1}{2}\), ровно как и невычетов.
\end{note}

\begin{definition}
    Символ Лежандра \(\left(\frac{a}{p}\right)\) --- читается ''\(a\) по \(p\)''
    \[
    \left(\frac{a}{p}\right) = \left\{\begin{array}{l}
        0, a = 0 \\
        1, a\text{ --- вычет} \\
        -1, a\text{ --- невычет} \\
    \end{array}\right.
    \]
\end{definition}

\textbf{Анекдот}: посчитать сумму 
\[\frac{4}{p + 1}\sum_{a = 1}^p\left(\frac{a}{p}\right)\]
\begin{solution}[1]
    Если вы знаете, что \(\left(\frac{a}{p}\right)\) --- символ Лежандра, то сумма будет равна 0
\end{solution}
\begin{solution}[2]
    Иначе, вы посчитаете арифметическую прогрессию и получите свою оценку на экзамене
\end{solution}

Рассмотрим уравнение
\[a^{p-1} \equiv_p 1\]
\[\left(a^{\frac{p-1}{2}} - 1\right)\left(a^{\frac{p-1}{2}} + 1\right)  \equiv_p 0\]
Причем, первая скобка имеет не более \(\frac{p-1}{2}\) решений, поэтому, т.к. любой квадратичный вычет ее зануляет, ее решения --- только квадратичные вычеты. Таким обрахзом:
\[\left(\frac{a}{p}\right) \equiv_p = a^{\frac{p-1}{2}}\]
Поэтому можно сказать, что 
\[\left(\frac{a}{p}\right)\left(\frac{b}{p}\right) = \left(\frac{ab}{p}\right)\]
\begin{note}
    \[\left(\frac{-1}{p}\right) = (-1)^\frac{p-1}{2}\]
\end{note}

\begin{proposition}
    Зафиксируем некоторое число \(a\). Пусть \(x\) пробегает числа \(1, 2, \dots \frac{p-1}{2} = p_1\). Рассмотрим числа \(ax = \epsilon_x\cdot r_x\), где \(\epsilon_x \in \{-1, 1\}, r_x \in \{1, 2, \dots, p_1\}\). Тогда \(x \ne y \Ra r_x \ne r_y\).
\end{proposition}
\begin{proof}
    Предположим противное. Тогда \(r_x = r_y, x \ne y\). Но тогда \(\epsilon_x \ne \epsilon_y\), т.к. в противном случае \(ax = ay\), чего быть не может. Но тогда \(r_x \equiv_p -r_y \Ra r_x + r_y \equiv_p 0\), но такого тоже быть не может, т.к. \(r_x, r_y \le \frac{p-1}{2}\).
\end{proof}
\begin{proposition}
    \(\epsilon_x = (-1)^{\left[\frac{2ax}{p}\right] }\)
\end{proposition}
\begin{proof}
    Если \((ax \mod p) \in \{1, 2, \dots p_1\}\), то \((-1)^{\left[\frac{2ax}{p}\right] } = 1\), иначе \((-1)^{\left[\frac{2ax}{p}\right] } = -1\).
\end{proof}
\begin{proposition}
    \[a^{\frac{p-1}{2}} = \Pi_{x = 1}^{p_1}\epsilon_x\]
\end{proposition}
\begin{proof}
    \[a^{\frac{p-1}{2}}\Pi_{x = 1}^{p_1} x = \Pi_{x = 1}^{p_1}\epsilon_x r_x\]
    Причем \(\Pi x = \Pi r_x\), т.к. все \(x\) различны, все \(r_x\) различны и берутся из одного множества. Сократив множители, получим желаемое.
\end{proof}
\begin{proposition}
    \[\left(\frac{a}{p}\right) = a^{\frac{p-1}{2}} = (-1)^{\sum_{x = 1}^{p_1}\left[\frac{2ax}{p}\right] }\]
\end{proposition}
\begin{proof}
    Соединяем предыдущие два утверждения и получаем желаемое.
\end{proof}
\begin{proposition}[Уточнение]
    Пусть \(a\) --- нечетное. Тогда 
    \[\left(\frac{a}{p}\right) = (-1)^{\sum_{x = 1}^{p_1}\left[\frac{ax}{p}\right]}\]
\end{proposition}
\begin{proof}
    Рассмотрим 
    \[\left(\frac{2a}{p}\right) = \left(\frac{2a + 2p}{p}\right) = \left(\frac{4\left(\frac{a + p}{2}\right)}{p}\right) = \left(\frac{\frac{a + p}{2}}{p}\right) = (-1)^{\sum_{x = 1}^{p_1}\left[\frac{2\frac{1}{2}(a + p)x}{p}\right] } = (-1)^{\sum_{x = 1}^{p_1}\left[\frac{ax}{p}\right] + \sum_{x = 1}^{p_1}x} = \]
    \[ = (-1)^{\sum_{x = 1}^{p_1}\left[\frac{ax}{p}\right] + \frac{p_1(p_1 + 1)}{2}} = (-1)^{\sum_{x = 1}^{p_1}\left[\frac{ax}{p}\right] + \frac{p^2 - 1}{8}}\]
    Из этого можно показать, что \(\left(\frac{2}{p}\right) = (-1)^{\frac{p^2 - 1}{8}}\).
    Тогда 
    \[\left(\frac{2a}{p}\right) = \left(\frac{2}{p}\right)\left(\frac{a}{p}\right) = (-1)^\frac{p^2 - 1}{8}\left(\frac{a}{p}\right) = (-1)^{\sum_{x = 1}^{p_1}\left[\frac{ax}{p}\right] + \frac{p^2 - 1}{8}}\]
    Итого получили, что 
    \[\left(\frac{a}{p}\right) = (-1)^{\sum_{x = 1}^{p_1}\left[\frac{ax}{p}\right]}\]
\end{proof}

\begin{theorem}[Квадратичный Закон Взаимности]
    Пусть \(p, q\) --- различные нечентые простые.  Тогда
    \[\left(\frac{p}{q}\right)\left(\frac{q}{p}\right) = (-1)^{p_1q_1}\]
\end{theorem}
\begin{proof}
    \[\left(\frac{p}{q}\right)\left(\frac{q}{p}\right) = (-1)^{\sum_{x = 1}^{q_1}\left[\frac{px}{q}\right] + \sum_{y = 1}^{p_1}\left[\frac{qy}{p}\right]} \]
    Введем множество \(S = \{1, \dots q_1\} \times \{1, \dots p_1\}\). Очевидно, что \(|S| = p_1q_1\). Введем \(S_1 = \{(x, y) \in S | qy < px\}, S_2 = \{(x, y) \in S | qy > px\}\). Тогда \(|S| = |S_1| + |S_2|\), т.к. \(px = qy\) невозможно.

    Причем, \(qy < px \Lra y < \frac{px}{q}, qy > px \Lra \frac{qy}{p} > x\). Заметим, что \(|S_1| = \sum_{x = 1}^{q_1}\left[\frac{px}{q}\right]\), т.к. количество \(y\) для фиксированного \(x\) ровно \(\left[\frac{px}{q}\right]\). Но тогда получаем, что \(|S| = |S_1| + |S_2|\), что и требовалось.
\end{proof}