% !TEX root = ../../../main.tex

\subsection{7-ая проблема Гильберта}
\begin{problem}[7-ая проблема Гильберта]
    Верно ли, что \(\alpha, \beta \in \mathbb{A} \Ra \alpha^\beta \in \mathbb{A}\)?
\end{problem}

\begin{theorem}
    Если \(\alpha \notin \{0, 1\}, \alpha \in \mathbb{A}, \beta \in A \setminus \Q \Ra \alpha^\beta \notin \mathbb{A}\)
\end{theorem}
Доказана А.О. Гельфандом.

\begin{proposition}
    \(e^\pi \notin \mathbb{A}\)
\end{proposition}
\begin{proof}
    Предположим противное. Известно, что \((e^\pi)^i = e^{i\pi} = -1 \Ra -1\) должно быть Трансцендентным
\end{proof}

\section{Геометрия чисел}
\begin{theorem}[Минковского]
    Пусть \(\Omega \subset \R^n\), \(\Omega\) --- выпукло, симметрично относительно \(O\), и \(V(\Omega) > 2^n\) (или \(\ge 2^n\) для случая замкнутого \(\Omega\)). Тогда \(\Omega \cap \Z^n \setminus \{0\} \ne \emptyset\)
\end{theorem}
\begin{proof}
    Рассмотрим \(N_p = \left|\frac{1}{p}\Z^n\cap \Omega\right|\). Интуитивно понятно (дается без доказательства), что 
    \[\frac{N_p}{p^n} \ra V(\Omega)\]
    Т.к. \(\frac{N_p}{p^n}\) --- количество ''кубиков'' размера \(\frac{1}{p}\) внутри \(\Omega\)

    \[\Ra \exists p_0 \forall p > p_0 \frac{N_p}{p} > 2^n \Ra N_p > (2p)^n\]

    Тогда рассомотрим \(a = \left(\frac{a_1}{p}, \frac{a_2}{p}, \dots \frac{a_n}{p}\right), b = \left(\frac{b_1}{p}, \frac{b_2}{p}, \dots \frac{b_n}{p}\right) \in \Omega\), такие, что \(\forall i\;\;a_i \equiv_{2p} b_i \Ra \frac{a - b}{2} \in \Z^n \setminus \{0\}\)
\end{proof}

Рассмотрим \(\R^n\) и базис в нем \(a_1, a_2, \dots a_n\). 
\begin{definition}
    Решетка --- это множество точек \(\Lambda = \{b_1a_1 + \dots + b_na_n | b_i \in \Z\}\)
\end{definition}
\begin{example}
    В \(\R^2\) и базиса \((0, 1), (1, 0)\) решекой является \(\Z^2\)
\end{example}

\begin{theorem}
    \(\Lambda \subset \R^n\) является решеткой \(\Lra \Lambda\) образует группу по сложению, \(\Lambda\) дискретно и ''заполняет все пространство''. То есть, если 
    \begin{enumerate}
        \item \textbf{Дискретность:} Каждая точка \(\Lambda\) изолированная
        \item \textbf{''Заполнение всего \(\R^n\)'':}  \(\exists r > 0 \forall x \in \Lambda \stackrel{\circ}{B}(x) \cap \Lambda \ne \emptyset\)
    \end{enumerate}
\end{theorem}

\begin{definition}
    Определителем \(\Lambda\) (детерминантом \(\Lambda\)) называется величина \(\det \Lambda\), равная модулю определителя матрицы, составленной из векторов произвольного базиса \(\Lambda\).
\end{definition}

\begin{theorem}[Минковского]
    Пусть \(\Omega \subset \R^n\) --- выпуклое и симметричное относительно \(O\) множество. Пусть \(\Lambda\) --- решетка, и \(V(\Omega) > 2^n\det \Lambda\). Тогда \((\Omega \cap \Lambda) \setminus \{0\} \ne \emptyset\)
\end{theorem}
\begin{proof}
    Доказательство аналогично доказательству обычной теоремы Минковского.
\end{proof}

\begin{definition}
    Критический определитель \(\Omega\) --- \(\Delta(\Omega) = \inf\{x | \exists \Lambda \subset \R^n: \det \Lambda = x, (\Omega \cap \Lambda) \setminus \{0\} \ne \emptyset\}\)
\end{definition}

\begin{proposition}
    Из теоремы минковского следует, что \(\forall\Omega\) --- выпуклого и симметрично относительно \(O\)
    \[\frac{V(\Omega)}{\Delta(\Omega)}\le 2^n\]
\end{proposition}
\begin{proof}
    Предположим, что \(\frac{V(\Omega)}{\Delta(\Omega)} > 2^n\). Тогда \(V(\Omega) > 2^n \Delta(\Omega) \Ra \exists \Lambda: V(\Omega) > 2^n \det \Lambda, (\Omega \cap \Lambda) \setminus \{0\} \ne \emptyset\)
\end{proof}

Возникает логичный вопрос: \(\frac{V(\Omega)}{\Delta(\Omega)} \ge ?\)


\begin{theorem}[1945г. Минковского-Главка]
    \(\frac{V(\Omega)}{\Delta(\Omega)} \ge 1\)
\end{theorem}
\begin{theorem}[1950е годы. Роджерс, Шмидт]
    \(\frac{V(\Omega)}{\Delta(\Omega)} \ge cn\)
\end{theorem}

\begin{definition}
    Октаэдр --- это множество точек \(O^n = \{(x_1, x_2, \dots x_n) | |x_1| + |x_2| + \dots + |x_n| \le 1\}\).
\end{definition}

\begin{proposition}
    \(V(O^n) = \frac{2^n}{n!}\)
\end{proposition}