% !TEX root = ../../../main.tex

\section{Вероятноестное детерминирования}

\textit{С прошлой лекции мы знаем про тесты Ферма и Соловея - Штрассена, сейчас рассмотрим что-то новое}

\subsection{Тест Миллера-Рабина}

$$ m - 1 = 2^s \cdot l$$

Пусть $B_{MR} = \{a \in \Z^{*}_{m}\}$, если для $a$ выполнено одно из следующих условий:
\begin{enumerate}
    \item $a^l \equiv_m \pm 1$
    \item $a^{2l} \equiv_m -1$
    \item $a^{4l} \equiv_m -1$
    \item   $a^{\frac{m - 1}{2}} = a^{2^{s-1} l} \equiv_m -1 mod m$
\end{enumerate}

\begin{proof}
    Пусть $m$ - простое. Несложно заметить факт, что для простых чисел существует некоторое дерево ветвлений: $a^{p - 1} = 1$, для половины простых $a^{\frac{p-1}{2}} = 1$, для другой половины $a^{\frac{p-1}{2} = -1}$. Первая половина делится на еще 2 группы простых (для которых $a^{\frac{p-1}{4}} = 1$ и $a^{\frac{p-1}{4}} = -1$). 

    В обратную сторону воспользуемся утвеждением ниже (б/д).
\end{proof}

\begin{proposition}{(б/д)}
    $B_{MR}(m) \subset B_SS(m)$
\end{proposition}

\begin{proposition}
    $\left|B_{MR}(m)\right| \le \left|\Z_m^*(m)\right|$
\end{proposition}
\begin{proof}
    \begin{enumerate}
        \item $m \vdots p, q, r$ - различные простые.
        \item $m = p^\alpha q^\beta \then m - $ не число Кармайкла$\then \left| B_F(m)\right| \le \frac{1}{2}\left|\Z_m^*\right|$
        \item $m = p ^ \alpha$
    \end{enumerate} 

    \textbf{Первый случай:}
    
    Пусть $M_i$ - множество остатков $a$ из первого доказательства, то есть $M_0' = \{a \in \Z_m^* | a^l \equiv_m 1\}, M_j = \{a \in \Z_m^* | a^{2^j \cdot l} \equiv_m -1\}$. Выбирая минимальные $j$, добьемся того, что $B_{MR}(m) = M'0 \sqcup M_0 \sqcup M_1 \sqcup \dots \sqcup M_{s-1}$.

    \begin{proposition}
        $d \in \Z, d > 1$, если $\left|\{a \in \Z_d^* | a^k = -1\}\right| \ne 0 \then \left|\{a \in \Z_d^* | a^k = 1\}\right|$
    \end{proposition}

    \textit{Доказательство простое, достаточно просто взять $a^2$}

    Определим $M_j = \left|\{a \in \Z_m^* | a^{2^j l} \equiv_m -1\}\right|$, что по КТО равносильно системе:
    \begin{enumerate}
        \item $a^{2^j l} \equiv_{p^\alpha} -1$
        \item $a^{2^j l} \equiv_{p^\beta} -1$
         \item $a^{2^j l} \equiv_{p^\gamma} -1$
    \end{enumerate}
    
    \begin{proposition}
        $|N_j|$ = $6 |M_j|$ 
    \end{proposition}

    \begin{proposition}
        $N_j \cup M_{j + k} = \varnothing \then N_j \cup N_{j + k} = \varnothing$
    \end{proposition}

    \begin{proof}
        $a \in N_j \then a^{2^{j}l} \equiv_{m} 1 $
    \end{proof}

    Из описанного следует, что $|M'_0| \sqcup M_0 \sqcup M_1 \sqcup \dots \sqcup M_{s - 1}| \le \frac{1}{3} |N_0 \sqcup N_1 \sqcup \dots \sqcup N_{s - 1}|$
\end{proof}

\subsection{Числа Мерсенна}
\begin{definition}
    Простые числа \(2^p - 1\) называются числами Мерсенна
\end{definition}

\begin{definition}
    Простые числа \(2^{2^n} - 1\) называются числами Ферма
\end{definition}

\begin{note}
    \(\forall n > 5\) числа Ферма составные
\end{note}

\begin{note}
    \(2^n - 1\) --- простое число \(\Ra n\) --- простое
\end{note}
\begin{proof}
    Пусть нет, тогда \(n = mk \Ra 2^{mk} - 1 = (2^m)^k - 1 \equiv_{2^m} 0\)
\end{proof}

\begin{definition}
    \(s_0 = 4, s_{k + 1} = s_k^2 - 2\).
\end{definition}

\begin{lemma}
    \(s_k = (2 + \sqrt{3})^{2^k} + (2 - \sqrt{3})^{2^k}\)
\end{lemma}
\begin{proof}
    По индукции
\end{proof}

\begin{definition}
    \(\Z_m[\sqrt{k}] = \Z_m[x]/_{(x^2 - k)}\), т.е. это все остатки при делении на многочлен \(x^2 - k\)
\end{definition}

\begin{theorem}[Люка-Лемера]
    Тогда \(M_p\) --- простое \(\Lra M_p | s_{p - 2}\)
\end{theorem}
\begin{proof}\indent
    \begin{enumerate}
        \item[\(\La\)] Пусть \(q\) --- наименьший делитель \(M_p\)
        \[(2 + \sqrt{3})^{2^{p - 2}} + (2 - \sqrt{3})^{2^{p - 2}} \equiv_q 0\]
        \[(2 + \sqrt{3})^{2^{p - 2}} \equiv_q - (2 - \sqrt{3})^{2^{p - 2}}\]
        \[(2 + \sqrt{3})^{2^{p - 1}} \equiv_q -1\]
        \[(2 + \sqrt{3})^{2^p} \equiv_q 1\]
        \[ord(2 + \sqrt{3}) | {2^p} \Ra ord(2 + \sqrt{3}) = 2^k, k > p - 1 \Ra ord(2 + \sqrt{3}) = 2^p\]
        \[M_p = 2^p - 1 < ord(2 + \sqrt{3}) = 2^p < || =  q^2 \le M_p\]

        КОРОЧЕ СОРЯН Я НЕ УСПЕЛ
    \end{enumerate}    
    \href{https://ru.wikipedia.org/wiki/Тест_Люка_—_Лемера#Достаточность}{ВОТ ВАМ ВИКИПЕДИЯ}
\end{proof}