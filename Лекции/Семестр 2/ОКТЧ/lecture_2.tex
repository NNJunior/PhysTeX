% !TEX root = ../../../main.tex

\section{Матрицы Адамара}
\begin{definition}
    Матрицей Адамара называется матрица \(A\), если и только если 
    \[[A]_{ij} \in \{1, -1\}\]
    И ее строчки попарно отрогональны (то есть скалярное произведение любых двух строк равно 0)
\end{definition}

\begin{example}
    \begin{enumerate}
        \item \(n = 1\) --- очев
        \item \(n = 2\):
        \[\left(\begin{array}{ccc}
            1 & 1 \\
            1 & -1 \\
        \end{array}\right)\]
        \item \(n = 2\):
        \[\left(\begin{array}{ccc}
            1 & 1 \\
            1 & -1 \\
        \end{array}\right)\]
    \end{enumerate}
\end{example}
\begin{note}
    \(n \ge 2 \Ra n = 2k\)
\end{note}
\begin{proof}
    Очевидно, т.к. если мы перемножим любые две строчки, то тогда в скалярном произведении придется сложить нечетное количество \(\pm 1\), тогда эта сумма точно не будет равна 0.
\end{proof}
\begin{proposition}
    Если у матрицы попарно ортогональны сторчки, то и столбцы --- тоже
\end{proposition}
\begin{definition}
    Нормальная форма матрицы Адамара: когда \(A_1 = A^1 = (1, 1, \dots 1)\)
    \[\left(\begin{array}{cccc}
        1 & 1 & \dots &  1 \\
        1 & \pm 1 & \dots & \pm 1 \\
        \vdots & \vdots & \ddots & \vdots\\
        1 & \pm 1 & \dots & \pm 1 \\
    \end{array}\right)\]
\end{definition}
\begin{note}
    Любую матрицу адамара можно привести к нормальному виду путем домножения строк и столбцов на \(-1\).
\end{note}
\begin{theorem}
    \(n > 2 \Ra n = 4k\)
\end{theorem}
\begin{proof}
    Приведем матрицу Адамара к нормальному виду. Теперь переставим столбцы, чтобы вторая строчка была вида
    \[(\underbrace{1, 1 \dots 1,}_{\frac{n}{2}}, \underbrace{-1, -1, \dots -1}_{\frac{n}{2}} )\]
    А третья строка была вида
    \[(\underbrace{1, 1 \dots 1}_{x}  \underbrace{1, -1, -1, \dots -1}_{\frac{n}{2} - x}, \underbrace{1, 1 \dots 1}_{\frac{n}{2} - x},  \underbrace{-1, -1, \dots -1}_{x} )\]
    Тогда скалярное произведение второй и третьей будет равно 
    \[x - \left(\frac{n}{2} - x\right) - \left(\frac{n}{2} - x\right) + x = 4x - n = 0\]
    Тогда \(x \vdots 4\)
\end{proof}
\begin{theorem}(Гипотеза Адамара)
    Если \(n = 4k\), то матрица Адамара существует.
\end{theorem}
\begin{proof}
    Не доказана
\end{proof}
\begin{definition}
    Кронекеровское произведение матриц \(A * B = C \Ra\)
    \[C = \left(\begin{array}{ccc}
        a_{11}B & \dots & a_{1n}B \\ 
        \vdots & \ddots & \vdots \\ 
        a_{n1}B & \dots & a_{nn}B \\ 
    \end{array}\right) \in M_{mn \times mn}\]
\end{definition}
\begin{proposition}
    Кронекеровское произведение двух матриц Адамара есть матрица Адамара
\end{proposition}
\begin{proof}
    Скалярное произведение двух строк равняется 
    \[\sum_{k = 1}^n\left(\sum_{s = 1}^m a_{ik}a_{jk}b_{i's}b_{j's}\right) = \sum_{k = 1}^na_{ik}a_{jk}\left(\sum_{s = 1}^mb_{i's}b_{j's}\right) = (B_{i'}, B_{j'})\left(\sum_{k = 1}^na_{ik}a_{jk}\right) = (B_{i'}, B_{j'})(A_i, A_j) = 0\]
\end{proof}
\begin{theorem}[Пэли]
    Пусть \(p = 4k + 3\) --- простое число. Тогда \(\exists \) матрица Адамара порядка \(p + 1\).
\end{theorem}
\begin{proof}
    Рассмотрим матрицу порядка \(p\), такую, что \(A_{ab} = \left(\frac{a - b}{p}\right)\) (символ Лежандра). Тогда произведение любых двух строк \(i, j\) равно 
    \[\sum_{b = 1}^p \left(\frac{i - b}{p}\right)\left(\frac{j - b}{p}\right)\]
    \(c = i - b\).
    \[\sum_{c = 1}^p \left(\frac{c}{p}\right)\left(\frac{c - i + j}{p}\right)\]
    Причем, \(c = p \Ra \left(\frac{c}{p}\right) = 0\)
    \[\sum_{c = 1}^{p - 1} \left(\frac{c}{p}\right)\left(\frac{c - i + j}{p}\right) = \sum_{c = 1}^{p - 1} \left(\frac{c}{p}\right)\left(\frac{c(1 + c^{-1}(i - j))}{p}\right) = \sum_{c = 1}^{p - 1}\left(\frac{1 + c^{-1}(i - j)}{p}\right)\]
    При этом, \(i - j, c^{-1} \not\equiv_p 0 \Ra \) выражение \(1 + c^{-1}(i - j)\) пробегает все остатки \(\mod p\), кроме \(1\). Но тогда итоговая сумма равна \(0 - \left(\frac{1}{p}\right) = -1\). Тогда рассмотрим такую матрицу:
    \[C = \left(\begin{array}{c|ccc}
        1 & 1 & \dots & 1 \\
        \hline
        1 &  &  &  \\
        \vdots &  & A &  \\
        1 &  &  &  \\
    \end{array}\right)\]
    Где все нули в \(A\) заменены на \(-1\) (получится матрица \(A'\), причем замены произойдут только на главной диагонали). Докажем, что она подходит. Заметим, что в матрице \(A'\) поровну \(1\) и \(-1\). Тогда скалярное произведение с первой строчкой точно будет \(0\).
    Возьмем строчки \(i, j\) в матрице \(A'\). В их скалярном произведении добавилась \((-1)\left(\frac{i - j}{p}\right) + (-1)\left(\frac{j - i}{p}\right) = 0\). Теперь посчитаем скалярное произведение любых двух строк, к нему просот добавится 1 за счет первого столбца. Тогда это будет матрицей Адамара.
\end{proof}
\begin{theorem}[Пэли]
    Пусть \(p = 4k + 1\) --- простое число. Тогда \(\exists \) матрица Адамара порядка \(2(p + 1)\).
\end{theorem}
\begin{theorem}[б/д]
    \(\forall \epsilon > 0 \exists n_0: \forall n > n_0\) на отрезке \([n, (1 + (1 + \epsilon)n)]\) есть порядок матрицы Адамара
\end{theorem}
\begin{theorem}[переформулировка, тоже б/д]
    \(\exists f: f(n) = o(n)\), такая, что на отрезке \([n, n + f(n)]\) есть порядок матрицы Адамара
\end{theorem}

\subsection{Коды, исправляющие ошибки}
Представим ситуацию: разговариваем с бабушкой. Еще мы с ней общаемся азбукой морзе (отправляем ей 0 или 1) и передаем ей сообщения длины \(n\). Известно, что бабушка неправильно услышит не более чем \(k\) циферок. Как тогда с ней общаться?
\begin{definition}
    Расстояние Хэмминга между словами --- количество несовпадающих координат
\end{definition}
Тогда нам, по сути, надо расположить непересекающиеся ''шары'' радиуса \(k\), состоящие из слов. В таком случае мы сможем определить, какое слово мы передали, т.к. оно будет лежать не более, чем в одном шаре.
\begin{definition}
    \((n, M, d)\)-код --- такой словарь, в котором \(M\) слов, каждое из которых имеет длину \(n\) и минимальное расстояние между любями двумя словами равно \(d\).
\end{definition}
\begin{theorem}[Граница Плоткина]
    Пусть дан \((n, M, d)\)-код, где \(2d > n\). Тогда \(M  \le \frac{2d}{2d - n}\).
\end{theorem}
\begin{proof}[Доказательство неулучшаемости оценки]
    Рассмотрим матрицу Адамара:
    \[\left(\begin{array}{cccc}
        1 & 1 & \dots &  1 \\
        1 & \pm 1 & \dots & \pm 1 \\
        \vdots & \vdots & \ddots & \vdots\\
        1 & \pm 1 & \dots & \pm 1 \\
    \end{array}\right)\]
    И зачеркнем в ней первый столбец. Будем рассматривать строки как слова. Тогда расстояние Хэмминга между ними равно \(\frac{n}{2}\) (т.к. скалярное произведение любых двух равно 0). Тогда получили \(\left(n-1, n, \frac{n}{2}\right)\)-код. Но тогда плоткин дает результат \(\frac{2\frac{n}{2}}{2\frac{n}{2} - (n - 1)} = n\), т.е. мы нашли пример, который точно подходит под оценку.
\end{proof}
\begin{proof}
    Рассмотрим \((n, M, d)\)-код, \(a_{ij} \in \{0, 1\}\)
    \[\sum_{k = 1}^n\sum_{i < j} |a_{ik} - a_{jk}| = \sum_{i < j}\underbrace{\sum_{k = 1}^n|a_{ik} - a_{jk}|}_{\begin{array}{c}
        \text{Хеммингово расстояние} \\
        \text{между \(i\)-ой и \(j\)-ой строками}
    \end{array}}  \ge \sum_{i < j}d = \frac{M(M - 1)}{2}d\]
    Однако заметим, что если в слове \(x\) единиц, то в нем \(M - x\) нулей, и тогда пар \(\{0, 1\}\) в нем будет ровно \(x(M - x) \le \frac{M^2}{4}\). Тогда общая сумма будет \(\le \frac{nM^2}{4}\), т.к. \(\frac{nM^2}{4}\) --- верхняя оценка на количество пар. Но тогда:
    \[\frac{M(M  - 1)}{2}d \le \frac{nM^2}{4}\]
    \[(M  - 1)d \le \frac{nM}{2}\]
    \[2(M - 1)d \le nM\]
    \[M(2d - n) \le 2d\]
    \[M \le \frac{2d}{2d - n}\]
\end{proof}