% !TEX root = ../../../main.tex

\section{Диофантовы приближения, теорема Дирихле}

Рассмотрим число $\pi = 3.1415926...$.

$$\left|\pi - \frac{314}{100}\right| = 0.0015926... \ \ \left|\pi - \frac{22}{7}\right| = 0.0012... $$

\begin{theorem}{(Дирихле)}
  Пусть $\alpha \not \in Q$. Тогда $\exists$ \underline{бесконечно много} дробей $\frac{p}{q}$, что $$\left|\alpha - \frac{p}{q}\right| \le \frac{1}{q^2}$$
\end{theorem}

\begin{proof}
  $Q \in \N$. Рассмотрим деление отрезка $[0, 1]$ на отрезки длины $\frac{1}{Q}$.
  
  Рассмотрим $\{\alpha x \}$, где $x = 0, 1,\dots, Q$. $\exists x_1, x_2: \ x_1 > x_2$ и $\left|\{\alpha x_1 - \alpha x_2 \}\right| \le \frac{1}{Q}$

  $$\left|\alpha x_1 - [\alpha x_1] - \alpha x_2 + [\alpha x_2]\right| \le \frac{1}{Q}$$

  $$\left|\alpha \underbrace{(x_1 - x_2)}_{q} - \underbrace{([\alpha x_1] - [\alpha x_2])}_{p}\right| \le \frac{1}{Q}$$
Если $q \le Q$
  $$\left|\alpha - \frac{p}{q}\right| \leq \frac{1}{qQ} \le \frac{1}{q^2}$$ 
\end{proof}

\begin{note}
Покажем, как получать новые дроби:

Пусть $\alpha$ = $\left|\alpha - \frac{p}{q}\right|, \alpha \le \frac{1}{q^2}, a > 0$

Возьмем $Q_1 \in \N : \frac{1}{Q_1} \le a$. По $Q_1$ найдем соответствующие ей $\frac{p_1}{q_1}$.

\textbf{Почему полученные $p_1, q_1$ не совпадают с $p, q$?}

Как мы доказали, верно следующее:

$$\left|\alpha - \frac{p_1}{q_1}\right| \le \frac{1}{\underbrace{q_1 Q_1}_{< \alpha}} \le \frac{1}{q_1^2}.$$

$$\left|\alpha - \frac{p_1}{q_1}\right| \le \frac{1}{q_1 Q_1} \le \frac{\alpha}{q_1} \leq \left|\alpha - \frac{p}{q}\right| \then \frac{p_1}{q_1} \ne \frac{p}{q}$$
\end{note}

\subsection{Теорема Минковского. Еще одно доказательство теоремы Дирихле}
\begin{theorem}{(Минковского)}
  Пусть $\Omega \subset \R^2: \Omega$ выпукло, симметрично относительно $0, S(\Omega) > 4$. Тогда $(\Omega \cap \Z^2)\setminus{0} \ne \varnothing$ 
  
\end{theorem}

\begin{proof}
  Рассмотрим $N_p$ - все координаты в $\Z^2 \cap \Omega$, имеющие вид $(\frac{a}{p}, \frac{b}{p})$, $a,b,p \in \N$.


$$\frac{N_p}{p^2} \to S(\Omega) > 4, \text{при $p \to \infty$}$$

\textit{Этот факт оставляется без доказательства. Обещали не спрашивать его на экзамене}.

$\exists P: \forall p \ge O \ \frac{N_p}{p^2} > 4$

$$N_p > (2p)^2 \then \exists a = \left(\frac{a_1}{p}, \frac{a_2}{p}\right), b = \left(\frac{b_1}{p}, \frac{b_2}{p}\right) : a \ne b, a_1 \equiv b_1 (2p), a_2 \equiv b_2 (2p)$$

Рассмотрим $\frac{a - b}{2} = \left(\frac{a_1 - b_1}{2p}, \frac{a_2 - b_2}{2p}\right) \in Z^2$

\begin{enumerate}
  \item $-b \in \Omega$, так как $\Omega$ - центрально смметричная.
  \item $\frac{a - b}{2} \in \Omega$, так как $\Omega$ выпукло.
\end{enumerate}
\end{proof}

\begin{note}
  Есть еще усиление теоремы Минковского - в случае замкнутого множества оценка становится нестрогой $(\ge 4)$.
\end{note}

\textbf{Приведем еще одно доказательство теоремы Дирихле}

\begin{proof}
  $\Omega = \{(x, y): \left| y - \alpha x \right| \le \frac{1}{Q}, \left| x\right| \le Q\}$. Если нарисовать на плоскости фигуру, то получится параллелограмм. По формуле площади:

  $$S(\Omega) = 4 \then \text{по теореме} \exists (q, p) \in \Omega, q > 0$$

  $$\left|p - \alpha q\right| \le  \frac{1}{Q} \then \left| \alpha - \frac{p}{q}\right| \le \frac{1}{qQ} \le \frac{1}{q^2}$$
\end{proof}

\section{Цепные дроби}

\subsection{Конечная цепная дробь}

$$a_0 + \frac{1}{a_1 + \frac{1}{a_2 + \frac{1}{a_3 + ...}}}$$

$a_0 \in \Z, a_i \in \N \ \forall i \ge 1$.

Раскрыв скобки, получим $\alpha := [a_0; a_1, a_2, a_3, \dots, a_n] \in Q$

\textbf{Определим теперь цепную дробь индуктивно:}

\begin{enumerate}
  \item $[a_0] = \frac{a_0}{1}$
  \item $[a_0; a_1, \dots, a_n] = a_0 + \frac{1}{[a_1; a_2, \dots, a_n]} = a_0 + \frac{1}{\frac{p}{q}} = a_0 + \frac{q}{p} = \frac{a_0 p + q}{p}$
\end{enumerate}

\begin{definition}
  Подходящая дробь к $\alpha$ - дробь $\frac{p_k}{q_k} = [a_0; a_1, a_2, ..., a_k]$.
\end{definition}

\begin{theorem}
  $$p_{k + 2} = a_{k + 2} p_{k + 1} + p_k$$
  $$q_{k + 2} = a_{k + 2} q_{k 1} + q_k$$
\end{theorem}

\begin{proof}
  Успеем проверить только переход :(

 $$ \frac{p_0}{q_0} = [a_0] = \frac{a_0}{1}$$
$$\frac{p_1}{q_1} = [a_0; a_1] = a_0 + \frac{1}{a_1} = \frac{a_0 a_1 + 1}{a_1}$$
$$\frac{p_2}{q_2} = [a_0; a_1, a_2] = a_0 + \frac{1}{a_1 + \frac{1}{a_2}} = a_0 + \frac{a_2}{a_1 a_2 + 1} = \frac{a_0 a_1 a_2 + a_0 + a_2}{a_1 a_2 + 1}$$

Теперь проверяем утверждение:

$$p_2 =^{?} a_2 p_1 + p_0 = a_2 a_0 a_1 + a_2 + a_0$$
$$q_2 = a_2 q_1 + q_0 = a_2 a_1 + 1$$

\textit{Пытаемся успеть сделать переход:}
$[a_0; a_1, ..., a_m] = a_0 + \frac{1}{[a_1; a_2, ..., a_m]}$


Не успели...
\end{proof}

