% !TEX root = ../../../main.tex

\textit{Проделаем переход индукции с прошлой лекции, напомним, что хотим доказать:}

\begin{theorem}
    $$p_{k + 2} = a_{k + 2} p_{k + 1} + p_k$$
    $$q_{k + 2} = a_{k + 2} q_{k 1} + q_k$$
\end{theorem}

\begin{proof}
    Пусть $[a_0; a_1,\dots, a_n] = \frac{p_n}{q_n}, [a_1; a_2, \dots, a_k] = \frac{p'_k}{q'_k}$

    $$a_0 + \frac{1}{[a_1; a_2, \dots, a_n]} = a_0 + \frac{1}{\frac{p'_n}{q'_n}} = a_o + \frac{q'_n}{p'_n} = \frac{a_0 p'_n + q'_n}{p'_n}$$

    $$p_n = a_0 p'_n + q'_n = a_0 (a_n p'_{n - 1} + p'_{n - 2}) + a_n q'_{n - 1 } + q'_{n - 2} = a_n \underbrace{(a_0 p'_{n - 1} + q'_{n    - 1}) }_{p_{n-1}}+ \underbrace{a_0 p'_{n - 2} + q'_{n - 2}}_{p_{n-2}}$$
\end{proof}

\begin{note}
$p_{n + 2} \cdot q_{n + 1} - p_{n + 1} q_{n + 2} = p_n q_{n + 1} - q_{n}p_{n + 1}$

Так как $p_0 q_1 - q_0 p_1 = a_o a_1 - (a_1 a_0 + 1) = -1$, $p_n q_{n + 1} - q_n p_{n + 1} = (-1)^{n + 1}$ и $\frac{p_n}{q_n}$ нельзя сократить.
\end{note}

\begin{note} $p_{n + 2}q_n - q_{n + 2}p_n = a_{n + 2} (-1)^n$

Из прошлого замечания получаем еще одно тождество:
$$p_{n + 2}q_n - q_{n + 2}p_n = a_{n + 2}\underbrace{(p_{n + 1}q_n - q_{n + 1}p_n)}_{(-1)^n}$$
\end{note}

\begin{proposition}
    Из первого замечания можно понять очень выжный факт:
    \begin{enumerate}
        \item Дроби с нечетным $n$ убывают
        \item Дроби с четным $n$ возрастают
        \item Но все они отличаются друг от друга на небольшое число - $\frac{(-1)^n}{q_n q_{n + 1}}$
    \end{enumerate}
\end{proposition}

\subsection{Бесконечная цепная дробь}

Формально почти все операции над цепными дробями остаются без изменений, но значение дроби определяется как предел подходящего ряда:
$$[a_0;a_1, \dots, a_n,\dots] = \lim_{n \to \infty}\frac{p_n}{q_n}$$
\begin{theorem}{(Докажут на семинаре)}

    Предел всегда существует.
\end{theorem}

\begin{example} $[1; 1, 1, \dots, 1] = ?$

    Пусть $[1; 1, 1, \dots, 1] = \alpha$. $1 + \frac{1}{\alpha} = \alpha \then \alpha = \frac{1 + \sqrt{5}}{2}$
\end{example}

\begin{theorem}
    Если цепная дробь периодична, то ее значение будет являться квадратичной иррациональностью (решением квадратного уравнения c иррациональными коэффицеинтами)
\end{theorem}

\begin{proof}
    $\alpha = [a_0;a_1, \dots, a_k, \overline{b_1, b_2, \dots, b_m}].$


    Аналогично с примером обозначаем дробь $[b_1; b_2, \dots, b_m]$ за $\beta$. Тогда:
    $$b_m + \frac{1}{\beta} = \frac{\beta b_m + 1}{\beta}$$
    $$\frac{\beta}{\beta b_m + 1} + b_{m - 1} = \frac{\beta + \beta b_{m - 1} b_m + b_{m - 1}}{\beta b_m + 1}$$

    Понятно, что если выражать дальше $\beta$, то в числителе и знаменателе будет получаться линейная функция от $\beta$. $\beta = \frac{c_1 \beta + c_2}{c_3 \beta + c_4} \then \beta$ - квадратичная иррациональность.
\end{proof}

\begin{theorem} {(б/д)}
    Верно и обратное.
\end{theorem}

\begin{theorem}
    \(\forall \psi: \psi(q) \ra +\infty \ \exists \alpha > 0:\) неравенство \(\left|\alpha - \frac{p}{q}\right| \le \frac{1}{\psi(q)}\) имеет бесконечно много решений в дробях \(\frac{p}{q}\).
\end{theorem}
\begin{proof}
   Пусть построили дробь \(\alpha = [a_0; a_1, \dots ]\) . Найдем теперь $a_{n + 1}$ из соображений:
   
   $\alpha - \frac{p}{q} \le \frac{1}{q_n q_{n + 1}} <  \frac{1}{\frac{1}{q_n} \psi(q_n) q_n} < \frac{1}{\psi(q_n)}$, тесли выбрать $a_{n + 1}$ $q_{n + 1} = a_{n + 1}q_n + q_{n - 1} > \psi(q_n) \cdot \frac{1}{q_n}$ 
\end{proof}

\begin{definition}
  $\alpha \in A \Longleftrightarrow \alpha$ является корнем какого-то многочлена с целыми коэффициентами.

  \textit{Говорят, $A$ - множество трансцедентных чисел}
\end{definition}

\begin{note}
  Так как $\R$ - континум, а $A$ - не больше множества всех многочленов, которых счетно, то есть числа не в $A$.
\end{note}

\begin{theorem}{(Лиувилля)} Пусть $\alpha \in A, deg \alpha = d$, тогда $\exists c(\alpha): \forall \frac{p}{q} \left|\alpha - \frac{p}{q}\right| \ge \frac{c(\alpha)}{q^d} $
  
\end{theorem}

\begin{definition}
  $\alpha \in A$ называется алгебраическим числом степени $d$, если min степень многочлена, корнем которого является $\alpha$, равна $d$.
\end{definition}

$\alpha \in A, deg \alpha = d$. $f(x) = a_d x^d + \dots + a_1x + a_0$

\begin{enumerate}
  \item $\left|\alpha - \frac{p}{q}\right| \ge 1 \then \left|\alpha - \frac{p}{q}\right|  \ge \frac{1}{q^d} \then c_1(\alpha) = 1$
  \item $\left| \alpha - \frac{p}{q}\right|$
\end{enumerate}
$\frac{p}{q} \ne 0, f(\frac{p}{q}) = \frac{a_d p^d + \dots + a_0 q^d}{q^d}$. $\left|f(\frac{p}{q})\right|\ge \frac{1}{q^d}$

Положим $c(\alpha) = min\{c_1(\alpha), c_2(\alpha)\}$

$$\left|f\left(\frac{p}{q}\right)\right| = \left|\alpha - \frac{p}{q}\right| \cdot \left|\alpha_1 - \frac{p}{q}\right| \cdot \dots \cdot \left|\alpha_{d-1} - \frac{p}{q}\right| \cdot a_d$$ 

$\left|\alpha_i - \frac{p}{q}\right| = \left| \alpha_i - \alpha + \alpha - \frac{p}{q}\right| \le \left|\alpha_i - \alpha \right| + \left|\alpha - \frac{p}{q}\right| \le \underbrace{\left|\alpha_i - \alpha \right|}_{\overline{c_i}(\alpha)}$

$$\left|\alpha - \frac{p}{q}\right| \ge \frac{1}{q^d} \cdot \frac{1}{a_d \prod_{i = 1}^{d - 1} \overline{c_i}(\alpha)}$$