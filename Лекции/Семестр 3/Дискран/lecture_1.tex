% !TEX root = ../../../main.tex

\section{Ну-с, начнем}
\subsection{Определения}
\begin{definition}
    Граф --- \(G = (V, E)\), \(|V| < \infty\). \(V\) --- множество вершин, \(E\) --- множество ререр (подмножество \(V\times V\)). По умолчанию, граф неориентированный, в нем нет петель и ребер. Приставка \textbf{ор} будет означать, что граф ориентированный, приставка \textbf{мульти} будет означать, что разрешены кратные ребра, а приставка \textbf{псевдо} будет означать, что граф разрешены петли. Другими словами, псевдомультиорграф --- граф, в котором разрешено все.
\end{definition}

\begin{definition}
    Маршрут --- последовательность \(v_1e_1v_2e_2\dots v_{n-1}e_{n-1}v_n\), где \(e_i = (v_i, v_{i + 1}) \in E\)
\end{definition}

\begin{definition}
    Путь (цепь) --- незамкнутый маршрут, в котором все ребра разные.
\end{definition}

\begin{definition}
    Цикл --- замкнутый маршрут, в котором все ребра разные.
\end{definition}

\begin{definition}
    Простой путь (цепь) --- путь, в котором все вершины разные.
\end{definition}

\begin{definition}
    Простой цикл --- цикл, в котором все вершины разные (кроме, возможно, начальной и конечной).
\end{definition}

\begin{definition}
    Граф связен, если \(\forall v, u \in V\) существует простая цепь с концами в \(v, u\).
\end{definition}

\begin{note}
    Отношние \(u \sim v\), где \(\sim\) --- ''связны ли две вершины'' является отношением эквивалентности.
\end{note}

\begin{definition}
    Классы эквивалентности по отношению выше называются компонентами связности.
\end{definition}

\begin{definition}
    Пусть \(v \in V\). Степень вершины \(v\) --- \(\deg v\) --- количество ребер, которое исходит из данной вершины (петля добавляет 2 к степени вершины). \(indeg\ v, outdeg\ v\) --- входящие и исходящие степени для орграфов.
\end{definition}

\begin{note}
    \[\sum_{v \in V} \deg v = 2|E|\]
\end{note}

\begin{definition}
    Граф называется \(d\)-регулярным, если \(\forall v \in V \deg v = d\).
\end{definition}

\begin{note}
    Всего существует \(2^{C_n^2}\) графов на \(n\) вершинах (верны пронумерованы числами от \(1\) до \(n\)).
\end{note}

\begin{definition}
    Граф называется деревом, если он связен и в нем нет циклов.
\end{definition}

\begin{note}
    В дереве на \(n\) вершинах \(n - 1\) ребро.
\end{note}

Положим \(\tau_n\) --- количество деревьев на \(n\) вершинах.

\begin{proposition}
    \(\tau_n = n^{n - 2}\)
\end{proposition}
\begin{proof}(Коды Прюфера)
    Построим по дереву следующую последовательность: на каждой итерации будем находить лист с самым маленьким номером, запишем его соседа, а лист удалим. Далее нужно доказать, что по кодам Прюфера однозначно восстанавливается дерево и получить биекцию между кодами длины \(n - 2\) и деревьями на \(n\) вершинах.
\end{proof}

\begin{definition}
    Унициклический граф --- связный граф, в котором \(n\) вершин и \(n\) ребер.
\end{definition}

Положим \(u_n\) --- количество унициклических графов на \(n\) вершинах

\begin{lemma}
    \(F(n, r)\) --- количество лесов с \(r\) деревьями на \(n\) вершинах. \(F(n, r) = r\cdot n^{n - 1 - r}\).
\end{lemma}

Пусть \(r = 3, \dots n\) --- количество вершин в цикле. Тогда
\[u_n = \sum_{r = 3}^n \left(C_n^r \frac{(r - 1)!}{2}F(n, r)\right) = \sum_{r = 3}^n \left(C_n^r \frac{(r - 1)!}{2}r \cdot n^{n - 1 - r}\right)\]

\begin{proposition}
    \(u_n \sim \sqrt{\frac{\pi}{8}}\cdot n^{n - \frac{1}{2}}\)
\end{proposition}

\begin{proof}
    \[C_n^r = \frac{n!}{r!(n - r)!} = \frac{n(n - 1)\dots(n - r + 1)}{r!} = \frac{n^r}{r!}\left(1 - \frac{1}{n}\right)\left(1 - \frac{2}{n}\right)\dots \left(1 - \frac{r-1}{n}\right) = \]
    \[= \frac{n^r}{r!}e^{\ln\left(1 - \frac{1}{n}\right) + \ln\left(1 - \frac{2}{n}\right) + \dots + \ln\left(1 - \frac{r-1}{n}\right)}  = (*) \le \frac{n^r}{r!}e^{-\frac{1}{n}-\frac{2}{n} - \dots - \frac{r - 1}{n}} = \frac{n^r}{r!}e^{-\frac{r(r-1)}{2n}}\]
    С другой стороны
    \[(*) = \frac{n^r}{r!}e^{-\frac{1}{n}-\frac{2}{n} - \dots - \frac{r - 1}{n} + O\left(\frac{1}{n^2} + \frac{4}{n^2} + \dots + \frac{(r - 1)^2}{n^2}\right)} = \frac{n^r}{r!}e^{-\frac{r(r-1)}{2n} + O\left(\frac{r^3}{n^3}\right)}\]

    Итого
    \[u_n = \sum_{r = 3}^n \left(C_n^r \frac{(r - 1)!}{2}r \cdot n^{n - 1 - r}\right) \le \frac{1}{2}n^{n - 1}\sum_{r = 3}^n e^{-\frac{r(r-1)}{2n}}\]
    \[u_n = \sum_{r = 3}^n \left(C_n^r \frac{(r - 1)!}{2}r \cdot n^{n - 1 - r}\right) = \frac{1}{2}n^{n - 1}\sum_{r = 3}^n e^{-\frac{r(r-1)}{2n} + O\left(\frac{r^3}{n^3}\right)}\]
    
    % При этом, \(r^3 = o(n^2) \Lra r = o(n^{\frac{2}{3}})\).
    
    \[u_n = \sum_{r = 3}^n \left(C_n^r \frac{(r - 1)!}{2}r \cdot n^{n - 1 - r}\right) = \frac{1}{2}n^{n - 1}\left(\underbrace{\sum_{r = 3}^{[n^{0.6}]} C_n^r r! n^{-r}}_{S_1} + \underbrace{\sum_{r = [n^{0.6}] + 1}^{n} C_n^r r! n^{-r}}_{S_2}\right)\]

    \[S_2 \le \sum_{r = [n^{0.6} + 1]}^ne^{-\frac{r(r-1)}{2n}} \le n \cdot e^{\frac{n^{0.2}}{2}(1 + o(1))}\]
    
    Помним, что \(r \ge n^{0.6}\).

    \[e^{-\frac{r(r-1)}{2n}} = e^{-\frac{r^2}{2n}(1 + o(1))} \le e^{-\frac{n^{1.2}}{2n}(1 + o(1))} = e^{-\frac{n^{0.2}}{2}(1 + o(1))}\]

    Докажем, что \(\sum_{r = 3}^{[n^{0.6}]}e^{-\frac{r^2}{2n}} \sim \sum_{r = 0}^{\infty}e^{-\frac{r^2}{2n}}\).
    \[\sum_{r = 0}^2e^{\frac{-r^2}{2n}} = O(1)\]
    \[\sum_{r = [n^{0.6}] + 1}^\infty e^{\frac{-r^2}{2n}} = \underbrace{\sum_{r = [n^{0.6}] + 1}^{n^2} e^{\frac{-r^2}{2n}}}_{(1)} + \underbrace{\sum_{r = n^2 + 1}^\infty e^{\frac{-r^2}{2n}}}_{(2)}\]
    Для \((1)\):
    \[r \ge n^{0.6} \Ra e^{\frac{-r^2}{2n}} \le e^{\frac{-n^2}{2n}} = e^{-\frac{n}{2}} \Ra (1) \le n^2e^{-\frac{n}{2}}\]
    Для \((2)\):
    \[\frac{e^{\frac{-(r + 1)^2}{2n}}}{e^{\frac{-r^2}{2n}}}= e^{\frac{-(r+1)^2 + r^2}{2n}} = e^{\frac{-2r - 1}{2n}} = e^{-\frac{r}{n} - \frac{1}{2n}} < e^{-\frac{r}{n}} < e^n\]
    Тогда \((2) \le e^{-\frac{-(n^2 + 1)^2}{2n}}\) (ограничили сверху геометрической прогрессией). Тогда:
    \[S_1 = \sum_{r = 3}^{[n^{0.6}]}e^{-\frac{r(r-1)}{2n} + O\left(\frac{r^3}{n^3}\right)} \sim \sum_{r = 3}^{[n^{0.6}]}e^{-\frac{r(r-1)}{2n}} \sim \sum_{r = 3}^{[n^{0.6}]}e^{-\frac{r^2}{2n}} \sim \sum_{r = 0}^{\infty}e^{-\frac{r^2}{2n}} \sim_{\text{сх.}} \int_0^\infty e^{-\frac{r^2}{2n}}dr = \]
    \[ = \int_0^\infty e^{-\frac{x^2}{2}}\sqrt{n}dx = \sqrt{n}\int_{0}^{+\infty} e^{-\frac{x^2}{2}}dx = \sqrt{n}\frac{\int_{-\infty}^{+\infty} e^{-\frac{x^2}{2}}dx}{2} = \sqrt{n}\frac{\sqrt{2\pi}}{2}\]
\end{proof}