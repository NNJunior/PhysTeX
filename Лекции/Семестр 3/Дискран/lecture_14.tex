% !TEX root = ../../../main.tex

\section{Кнезеровские графы}

\begin{definition}
    \(KG_{n, r} = G(n, r, 0)\) --- Кнезеровский граф
\end{definition}

\begin{note}
    \(|V| = C_n^r, |E| = \frac{1}{2}C_n^rC_{n - r}^r, \alpha(KG_{n, r}) = \left\{\begin{array}{l}
        C_n^r, 2r > n \\
        C_{n - 1}^{r - 1}, 2r \le n
    \end{array}\right.\)

    Также из предыдущих лекцияй, \(\alpha(G(n, r, 0)) = f(n, r, 1)\)
\end{note}

\begin{note}
    \(\omega(G(n, r, 0)) = \left[\frac{n}{r}\right], 2r \le n\).
\end{note}

Получим теперь оценки на хроматическое число кнезеровского графа

\begin{note}
    \(\chi(KG_{n, r}) \ge \omega(KG_{n, r}) = \left[\frac{n}{r}\right]\)
\end{note}

\begin{note}
    \(\chi(KG_{n, r}) \ge \left\lceil \frac{|V|}{\alpha(KG_{n, r})} \right\rceil = \left\lceil\frac{C_n^r}{C_{n - 1}^{r - 1}}\right\rceil = \left\lceil\frac{n}{r}\right\rceil\)
\end{note}

\begin{note}
    \(\chi(KG_{n, r}) \le n\)
\end{note}
\begin{proof}
    Действительно, будем красить все вершины, содержащие \(1\) в первый цвет. Из оставшихся вершин, покрасим во второй цвет все, которые содержат вершину 2. Аналогично будем красить оставшиеся вершины.
\end{proof}

\begin{note}
    \(\chi(KG_{n, r}) \le n - r + 1\)
\end{note}
\begin{proof}
    Будем действовать как в прошый раз. Однако заметим, что на \(n - r + 1\)-ой итерации все вершины исчерпаются. Тогда нам достаточно \(n - r + 1\) цвет
\end{proof}

\begin{note}
    \(\chi(KG_{n, r}) \le n - 2r + 2\)
\end{note}
\begin{proof}
    Будем действовать как в прошый раз. Однако заметим, что на \(n - 2r + 2\)-ой итерации все оставшиеся вершины будут образовывать независимое множество
\end{proof}

Несмотря на то, что верхняя и нижняя оценка расходятся достаточно сильно, есть примеры, для которых данные оценки равны:

\begin{example}
    \(KG_{n, 1} = K_n\) --- клика на \(n\) вершинах. \(\chi(KG_{n, 1}) = n = \left\lceil\frac{n}{1}\right\rceil\)
\end{example}

\begin{example}
    \(KG_{n, \frac{n}{2}}\) --- паросочетание. \(\chi(KG_{n, \frac{n}{2}}) = 2 = \left\lceil\frac{n}{n / 2}\right\rceil\)
\end{example}

\begin{example}
    \(KG_{5, 2}\) --- граф Петерсена. \(\chi(KG_{5, 2}) = 3 = \left\lceil\frac{5}{2}\right\rceil\)
\end{example}

\begin{theorem}
    Пусть \(S^{n - 1}\) --- \(n - 1\)-мерная сфера. Пусть \(S^{n - 1} = A_1 \cup\dots \cup A_n\) и \(\forall i: A_i\) замкнуто, то \(\exists i, \vec{x}: \vec{x} \in A_i, -\vec{x} \in A_i\), т.е. \(A_i\) содержит антиподальные (или, диаметрально противоположные).
\end{theorem}
\begin{proof}
    Тут в следующий раз появится доказательство
\end{proof}

\begin{note}
    Предыдущая теорема равносильна точно такому же утверждению в случае, когда все \(A_i\) открыты.
\end{note}

\begin{note}
    Предыдущая теорема равносильна следующему утверждению: Пусть \(f: S^n \ra \R^n\) --- непрерывное отображение. Тогда \(\exists \vec{x} \in S^n: f(\vec{x}) = f(-\vec{x})\).
\end{note}

Однако, существует усиление данных теорем:
\begin{theorem}
    Пусть \(S^{n - 1}\) --- \(n - 1\)-мерная сфера. Пусть \(S^{n - 1} = A_1 \cup\dots \cup A_n\) и \(\forall i: A_i\) замкнуто или открыто, то \(\exists i, \vec{x}: \vec{x} \in A_i, -\vec{x} \in A_i\), т.е. \(A_i\) содержит антиподальные (или, диаметрально противоположные).
\end{theorem}

\begin{proposition}[Гипотеза Кнезера]
    \(\chi(KG_{n, r}) = n - 2r + 2\)
\end{proposition}
\begin{proof}
    Предположим противное, т.е. \(\chi(KG_{n, r}) \le n - 2r + 1 = d\). Положим \(K_1, \dots K_n\) --- вершины \(KG_{n, r}\). \(K_i \cap K_j = \emptyset \Ra \chi(K_i) \ne \chi(K_j)\). Рассмотрим сферу \(S^d \subset \R^{d + 1}\). Расположим на сфере \(S^d\) некоторые точки \(\vec{x}_1, dots \vec{x}_n\) таким образом, чтобы на каждом ''экваторе'' было не больше \(d\) точек. Будем делать это так: каждый раз будем добавлять точку так, чтобы ничего не сломать. Нетрудно доказать, что такой алгоритм сработает. Сформируем Кнезеровский граф по этим точкам на \(S^d\), т.е. обозначим за \(L_1, \dots L_{C_n^r}\) --- все возможные подмножества \(\{\vec{x_1}, \dots \vec{x_n}\}\) размера \(r\) и будем соединять их ребром, если \(L_i \cap L_j = \emptyset\). Для них аналогично опредеделим раскраску \(\chi\). Положим \(H(\vec{x})\) --- открытая полусфера с центром в \(\vec{x}\). Положим также для \(i = 1, \dots d\), \(A_i = \{x \in S^d: H(\vec{x}) \cap \{\vec{x_1}, \dots \vec{x_n}\} \supset L_j: \chi(L_j) = i\}\). Теперь определим \(A_{d + 1} = \{x \in S^d: |H(\vec{x}) \cap \{\vec{x_1}, \dots \vec{x_n}\}| \le r - 1\}\). Таким образом, \(A_1, \dots A_d\) --- открыты, \(A_{d + 1} = S^d \setminus (A_1 \cup \dots \cup A_d)\) --- замкнуто. Итак, по предыдущей теореме, \(\exists i, \vec{x}: \vec{x} \in A_i, -\vec{x} \in A_i\). 
    \begin{enumerate}
        \item \(i \le d\). Тогда \(H(\vec{x}) \cap \{\vec{x_1}, \dots \vec{x_n}\} \supset L_j: \chi(L_j) = i, H(-\vec{x}) \cap \{\vec{x_1}, \dots \vec{x_n}\} \supset L_k: \chi(L_k) = i\). Но тогда \(L_k \cap L_j = \emptyset \Ra \chi(L_k) \ne \chi(L_j)\). Получили противоречие
        \item \(i = d + 1\). Тогда \(|H(\vec{x}) \cap \{\vec{x_1}, \dots \vec{x_n}\}| \le r - 1, |H(-\vec{x}) \cap \{\vec{x_1}, \dots \vec{x_n}\}| \le r - 1\). Но тогда \(S^d \setminus (|H(\vec{x}) \cup H(-\vec{x})) \cap \{\vec{x_1}, \dots \vec{x_n}\}| \ge d + 1\), то есть, проще говоря, на экваторе, разделяющем \(H(\vec{x}), H(-\vec{x})\) лежит \(\ge d + 1\) точка, что противоречит тому, как мы выбирали их.
    \end{enumerate}
\end{proof}
