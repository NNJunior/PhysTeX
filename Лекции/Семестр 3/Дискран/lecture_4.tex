% !TEX root = ../../../main.tex
\section{Эйлеровость графа}
\subsection{Гамильтоновость графа}
\begin{definition}
  Граф \(G - \)гамильтонов, если \(\exists\) простой цикл, проходящий по всем вершинам графа.
\end{definition}

\begin{theorem}{(Дирака)}

  Пусть \(G = (V, E), n = |V|, \forall v \in V \ deg \  v \le \frac{n}{2}\). Тогда граф \(G\) гамильтонов.
\end{theorem}

\begin{definition}
  \(\varkappa\) - вершинная связность, то есть минимальное количество вершин, которое нужно удалить из графа, чтобы нарушить его связность:

  \[\varkappa(G) = min \{k : \exists W \subseteq V: |W| = k, G |_{V 
  \setminus W} \text{несвязный}\}\]
\end{definition}
\begin{theorem}{(Эрдеш, Хватал)}
  Пусть \(\varkappa(G) \ge \alpha(G)\). Тогда \(G\) гамильтонов. 
  
\end{theorem}

\begin{note}
В 2 предыдущих теоремах связность графа следует из условия. В теорема Дирака 
\end{note}



