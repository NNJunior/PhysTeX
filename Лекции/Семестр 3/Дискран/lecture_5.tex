% !TEX root = ../../../main.tex


\begin{theorem}[Кучера]
    \(\forall \epsilon > 0 \forall \delta > 0 \exists \) последовательность графов \(G_n\) на \(n\) вершинах, \(\exists n_0: \forall n > n_0\) доля тех нумераций, в которых окажется, что \(\frac{\alpha(G)}{\alpha_{\text{ж}, \sigma}} \ge n^{1 - \epsilon}\), не меньше, чем \(1 - \delta\).
\end{theorem}

\begin{definition}
    \(g(G)\) \textit{(от английского girth)} --- обхват графа --- длина кратчайшего простого цикла.
\end{definition}

\begin{definition}
    \(G(n, p)\) --- модель случайного графа Эрдеша и Реньи (также называется биномиальная модель). В данной модели граф выбирается случайно, каждое ребро проводится с вероятностью \(p\)
\end{definition}

\begin{theorem}[Эрдеш]
    \(\forall k, l \in \N: \exists G: \chi(G) > k, g(G) > l\).
\end{theorem}
\begin{proof}
    Положим \(\theta = \frac{1}{2^l}, p = p(n) = n^{1 - \theta}\). Пусть \(X_l(G)\) --- количество циклов длины \(\le l\) в \(G\). \(\E X_l = \sum_{r = 3} \E(\text{число циклов длины \(r\)})\). Представив каждое из слагаемых как индикаторы конкретных циклов, получаем:
    \[\E X_l = \sum_{r = 3}^l \E(\text{число циклов длины \(r\)}) = \sum_{r = 3}^l\underbrace{C_n^r \frac{(r - 1)!}{2}}_{\text{максимальное количество циклов длины \(r\)}}p^r \le \]
    \[\le \sum_{r = 3}^l \frac{n^r}{r!}\cdot\frac{(r - 1)!}{2}p^r \le \sum_{r = 3}^l (np)^r = \sum_{r = 3}^l n^{\theta r} < ln^{\theta l} = l \sqrt{n}\]
    По неравенству Маркова, имеем:
    \[P\left( X_l > \frac{n}{2} \right) \le \frac{l\sqrt{n}}{n/2} \ra 0 \Ra \exists n_1: \forall n \ge n_1 P\left( X_l \le \frac{n}{2} \right) > \frac{1}{2}\]
\end{proof}