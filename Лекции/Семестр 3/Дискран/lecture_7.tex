% !TEX root = ../../../main.tex

\begin{reminder}
Пусть \(p = \frac{c \ln{n}}{n}\)
\begin{enumerate}
  \item Если \(c > 1\), то асимптотически почти наверно \(G(n, p)\) связен
  \item Если \(c < 1\), то асимптотически почти наверно \(G(n, p)\) не связен
\end{enumerate}
\end{reminder}

\[\sum_{k=1}^\frac{n}{2} \underbrace{C_n^k (1-p)^{k(n-k)}}_{a_k(n)} = \underbrace{\sum_{k=1}^{[\frac{n}{\sqrt{\ln{n}}}]}}_{S_1} \dots + \underbrace{\sum_{k = [\frac{n}{\sqrt{\ln{n}}}] + 1}^{\frac{n}{2}}}_{S_2}\]

\(\frac{a_{k+1}(n)}{a_k(n)} \le n (1-p)^{(n-2k-1)} \le n(1-p)^{n(1 + o(1))}; S_2 < n \cdot 2^n (1 - p)^{\frac{n}{\sqrt{\ln{n}}} \cdot \frac{n}{2}} \le n(1 - p)^{n - 2\frac{n}{\sqrt{\ln{n}}} - 1}  \le \)