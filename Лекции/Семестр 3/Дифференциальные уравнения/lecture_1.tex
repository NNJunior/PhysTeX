% !TEX root = ../../../main.tex

\section{Введение}
\subsection{Определения}
\begin{definition}
    Пусть даны \(n, m, k \in \N, \Omega \subset \R \times \underbrace{\R^n \times \R^n}_{k + 1}, F: \Omega \ra \R\). \(F(t, x, \stackrel{.}{x}, \stackrel{..}{x}, \dots x^{(k)}) = 0\) называется Обыкновенным Дифференциальным Уравнением.
\end{definition}

\begin{definition}
    \(x: I \ra \R^n\) --- называется решением, если
    \begin{enumerate}
        \item \(I \subset \R^n\)
        \item \(x \in C^k(I, \R^n)\)
        \item \(F(t, x, \stackrel{.}{x}, \stackrel{..}{x}, \dots x^{(k)}) \equiv 0, t \in I \Lra = 0 \forall t \in I 
        \)
    \end{enumerate}
\end{definition}

\begin{definition}
    Пусть \(\Gamma \subset \R \times \underbrace{\R^n \times \R^n}_{k}\)
    \(x^{(k)} = f(t, x, \stackrel{.}{x}, \stackrel{..}{x} \dots x^{(k - 1)})\) называется Нормальным Обыкновенным Дифференциальным Уравнением.
\end{definition}

\subsection{Примеры}
\begin{example}
    \(t^2 + x^2 + \stackrel{.}{x}^2 = 0\) --- нет решений
\end{example}

\begin{example}
    \(\stackrel{.}{x} = x \Ra x = ce^t, t \in I\).
\end{example}

\begin{definition}[Задача Коши]
    Пусть даны \(x_0, \stackrel{.}{x_0}, \stackrel{..}{x_0}, \dots x_0^{(k)} \in \R^n, t_0 \in \R\).
    \[\left\{\begin{array}{l}
        F(t, x, \stackrel{.}{x}, \stackrel{..}{x}, \dots x^{(k)}) = 0\\
        x(t_0) = x_0,  \stackrel{.}{x}(t_0) = \stackrel{.}{x_0}, \dots x^{(k)}(t_0) = x_0^{(k)}
    \end{array}\right.\]
    Называется Задаей Коши.
\end{definition}

\begin{note}
    \(x(\cdot)\) --- решение задачи Коши, если \(x(\cdot)\) --- решение \((1)\) и \(x(t_0) = x_0,  \stackrel{.}{x_0}(t_0) = \stackrel{.}{x_0}, \dots x_0^{(k)}(t_0) = x_0^{(k)}\)
\end{note}

\begin{theorem}
    Пусть \(n \in \N, k = 1, \Gamma = \subset \R \times \R^n, (t_0, x_0) \in \Gamma \Ra \)
    \[(2)\left\{\begin{array}{l}
        \stackrel{.}{x} = f(t, x) \\
        x(t_0) = x_0
    \end{array}\right.\]
    Рассмотрим \(B((t_0, x_0), r) = \{(t, x) : |t - t_0|^2 + |x - x_0|^2 \le r^2\} \subset \Gamma\). Пололжим \(M = \sup_{(t, x) \in B}(f(t, x)), d = \frac{r}{\sqrt{1 + M^2}}\)

    \(\Gamma\) --- открыто, \(f\) --- непрерывно \(\Ra \exists \) решение \(x: (t_0 - d, t_0 + d) \ra \R^n\)
\end{theorem}

\begin{example}[Условие непрерывности критично]
    Рассмотрим \(f: \R \ra \R: f(x) = \left\{\begin{array}{l}
        1, x \le 0 \\
        -1, x > 0
    \end{array}\right.\) и задачу Коши:
    \[\left\{\begin{array}{l}
        \stackrel{.}{x} = f(x) \\
        x(0) = 0
    \end{array}\right.\]
    Тогда не существует решений.
\end{example}
\begin{proof}
    Пусть существует решение \(x(\cdot)\). Тогда \(\stackrel{.}{x}(0) = f(x(0)) = f(0) = 1 \Ra \exists \epsilon > 0: \stackrel{.}{x}(t) > 0 \forall t \in [0, \epsilon) \Ra \stackrel{.}{x}(0) = 1, x(0) = 0 \Ra x(t) > 0 \forall t \in (0, \epsilon) \Ra \stackrel{.}{x}(t) = f(x(t)) < 0 \forall t \in (0, \epsilon)\).
\end{proof}

\begin{theorem}
    Пусть \(\Gamma\) --- открыто, \(f\) --- непрерывна, \(\frac{\delta f_i}{\delta x_j} \exists\) и непрерывна \(\Ra \exists \) решение \(x\) задачи Коши \((2)\) и \(\forall\) решения \(\tilde{x}: I \ra \R^n\) задачи Коши \((2)\) \(x(t) = \tilde{x}(t), t \in I \cap (t_0 - d, t_0 + d)\).
    % \[f(t, x) = \left(\begin{array}{c}
    %     f_1(t, x) \\
    %     f_2(t, x) \\
    %     \vdots \\
    %     f_n(t, x) \\
    % \end{array}\right), f_i(t, x) = f_i(t, \underbrace{x_1, x_2, \dots x_n}_x)\]
\end{theorem}

\begin{example}[Существенность непрерывность производной (гладкости функции)]
    \[\left\{\begin{array}{l}
        \stackrel{.}{x} = \sqrt[3]{x} \\
        x(0) = 0
    \end{array}\right.\]
    \begin{enumerate}
        \item \(x \equiv 0\) --- решение
        \item \(x(t) \equiv at^b\)
        \item \(abt^{b - 1} \equiv a^{\frac{1}{3}}b^{\frac{1}{3}}\).
    \end{enumerate}
    \[\left\{\begin{array}{l}
        b - 1 = \frac{b}{3} \\
        ab = a^{\frac{1}{3}}
    \end{array}\right. \Ra \left\{\begin{array}{l}
        b = \frac{3}{2} \\
        a^{\frac{2}{3}} = \frac{3}{2}
    \end{array}\right. \Ra \left\{\begin{array}{l}
        b = \frac{3}{2} \\
        a = \left(\frac{2}{3}\right)^{\frac{3}{2}}
    \end{array}\right.\]
    \[\Ra x(t) = \left(\frac{2}{3}\right)^{\frac{3}{2}}t^{\frac{3}{2}}, t > 0\text{ --- решение} \stackrel{.}{x} = \sqrt[3]{x}\]
    Теперь рассмотрим \(\tau \in \R\). По нему можно построить \(a(t - \tau)^b\), удовлетворяющая решению.
\end{example}

\section{Решения различных дифференциальных уравнений}
\subsection{Уравнения с разделяющимися переменными}
\begin{definition}
    Пусть дана \(h \in \mathcal{G} \ra \R, g: I \ra \R, \mathcal{G}, I \subset \R\) --- интервалы. 
\end{definition}

\begin{example}
    \[(3) \stackrel{.}{x} = h(x) \cdot g(t)\]
    \[\frac{dx}{dt} = h(x) \cdot g(t)\]
    \[\frac{dx}{h(x)} = g(t)dt\]
    \[\int \frac{dx}{h(x)} = \int g(t)dt\]
    \[H(x) = G(x) + c\]
    И далее алгебраически находим \(x\).
\end{example}
\begin{proof}[Почему так можно делать]
    Пусть \(h(x) \ne 0 \forall x \in \mathcal{G}, H(x)\) --- первообразная \(x = \frac{1}{h(x)}, G(x)\) --- первообразная \(g(x), x\) --- непрерывная дифференцируемая функция \((t, x(t)) \in I \times \mathcal{G}\). Заметим, что \(x\) --- решение \(\Lra \stackrel{.}{x}(t) \equiv h(x(t)) g(t) \Lra \frac{\stackrel{.}{x}(t)}{h(x(t))} \equiv g(t) \Lra \exists c \in \R: H(x(t)) \equiv G(t) + c\).

\end{proof}

Пусть \(x_0 \in \mathcal{G}, t_0 \in I\) даны, \(h(x_0) = 0, h(x) \ne 0\) при \(x \ne x_0, g(t_0) \ne 0, H(x) = \int_{x_0}^x \frac{d\xi}{h(\xi)}\) сходится при всех \(x \in \mathcal{G}\). Тогда 
\begin{enumerate}
    \item \(x(t) \equiv x_0, t \in I\) является решением \((3)\).
    
    \item \(\tilde{x}(t): H(x(t)) = G(t)\) является решением \((3)\) (\(G(t) = \int_{t_0}^t g(s)ds\)).
    
    \item \(x(t) = \left\{\begin{array}{l}
        x_0, t \le t_0 \\
        \tilde(x)(t), t > t_0
    \end{array}\right.\)
\end{enumerate}