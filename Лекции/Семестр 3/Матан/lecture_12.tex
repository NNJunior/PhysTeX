% !TEX root = ../../../main.tex

\begin{definition}
    Пусть \(k, l \in \N\). Рассмотрим \(S_{k, l} = \{\sigma \in S_{k + l}: \sigma(1) < \dots < \sigma(k), \sigma(k + 1) < \dots < \sigma(k + l)\} \subset S_{k + l}\). Элемент \(\sigma \in S_{k, l}\) называется \((k, l)\)-перетасовкой.
\end{definition}

\begin{definition}
    Пусть \(\omega \in A_k(V), \tau \in A_l(V)\), тогда внешним произведением \(\omega, \tau\) называется функция, определяемая \(\omega \wedge \tau(v_1, \dots v_{k + l}) = \sum_{\sigma \in S_{k, l}}\epsilon(\sigma)\omega(v_{\sigma(1)}, \dots v_{\sigma(k)})\cdot \omega(v_{\sigma(k + 1)}, \dots v_{\sigma(k + l)})\)
\end{definition}

\begin{note}
    Введем обозначение \(h = \omega \otimes \tau, h(v_1, \dots v_{k + l}) = \omega(v_1, \dots v_k)\tau(v_{k. + 1}, \dots v_{k + l}), \sigma h = h(v_{\sigma(1)}, \dots v_{\sigma(k)})\). Тогда \(\omega \wedge \tau = \sum_{\sigma \in S_{k, l}}\epsilon(\sigma) \cdot \sigma(\omega \otimes \tau)\).
\end{note}

\begin{note}
    Покажем, что \(\omega \wedge \tau \in A_{k + l}(V)\).
\end{note}
\begin{proof}
    Полилинейность очевидна. Покажем кососимметричность. Для этого достаточно установить, что \(\omega \wedge \tau(v_1, \dots v_{k + l}) = 0\) если в наборе \(v_1, \dots v_{k + l}\) выполнено \(v_r = v_{r + 1}\). Пусть \(A_1 = \{\sigma \in S_{k, l}: i = \sigma^{-1}(r), j = \sigma^{-1}(r + 1) \le k\} \Ra h_i = 0\), т.к. \(\omega(v_1, \dots v_k) = 0\). Аналогично, \(A_2 = \{\sigma \in S_{k, l}: i = \sigma^{-1}(r), j = \sigma^{-1}(r + 1) > k\} \Ra h_i = 0\), т.к. \(\tau(v_{k + 1}, \dots v_{k + l}) = 0\).

    Теперь рассмотрим \(A_3 = \{\sigma \in S_{k, l}: i = \sigma^{-1}(r) \le k, j = \sigma^{-1}(r + 1) \ge k + 1\}, A_4 = \{\sigma \in S_{k, l}: i = \sigma^{-1}(r) \ge k + 1, j = \sigma^{-1}(r + 1) \le k\}\). Рассмотрим \(t_r = (r\;r + 1)\) --- транспозиция. Имеем \(t_r(A_4) = A_3, t_r(A_3) = A_4\). Поэтому
    \[\epsilon(\sigma)(\sigma h)(v_1, \dots v_{k + l}) + \epsilon(t_r \sigma)(t_r\sigma h)(v_1, \dots v_{k + l}) =\]
    \[= \epsilon(\sigma)(\sigma h)((\sigma h)(v_1, \dots v_{k + l}) - (\sigma h)(v_1, \dots v_{k + l}))= 0\]
\end{proof}

\begin{example}
    Пусть \(\alpha, \beta \in V^*\) рассмотрим \(\alpha \wedge \beta(u, w) = \alpha(v) \beta(w) - \alpha(w)\beta(v)\).
\end{example}

\begin{lemma}
    Внешнее произведение удовлетворяет следующим свойствам
    \begin{enumerate}
        \item \(w_1 \wedge (w_2 \wedge w_3) = (w_1 \wedge w_2) \wedge w_3\)
        \item \(w_1 \wedge (w_2 + w_3) = w_1 \wedge w_2 + w_1 \wedge w_3\).
        \item \(\omega \wedge \tau = (-1)^{\deg \omega \deg \tau}\tau \wedge \omega\).
    \end{enumerate}
\end{lemma}
\begin{proof}\indent
    \begin{enumerate}
        \item Пусть \(k_i = \deg \omega_i, k = k_1 + k_2 + k_3, S_{k_1, k_2, k_3} = \left\{\sigma \in S: \begin{array}{l}
            \sigma(1) < \dots < \sigma(k_1) \\
            \sigma(k_1 + 1) < \dots < \sigma(k_1 + k_2) \\
            \sigma(k_1 + k_2 + 1) < \dots < \sigma(k_1 + k_2 + k_3) \\
        \end{array}\right\}\). Положим \(\omega = \sum_{\sigma \in S_{k_1, k_2, k_3}} \epsilon(\sigma)\sigma(\omega_1 \otimes \omega_2 \otimes \omega_3)\). Т.к. \(\otimes\) ассоциативно, то \(\omega_1 \wedge (\omega_2 \wedge \omega_3) = \omega = (\omega_1 \wedge \omega_2) \wedge \omega_3\).
        \item Непосредственно следует из определения
        \item Рассмотрим биекцию \(S_{k,l} \ra S_{l, k}, \sigma \mapsto \tilde{\sigma}, \tilde{\sigma}(i) = \left\{\begin{array}{l}
            \sigma(k + i), i = 1, \dots l \\
            \sigma(i - l), i = l + 1, \dots k + l \\
        \end{array}\right.\). В таком случае, \(\epsilon(\tilde{\sigma}) = (-1)^{kl}\epsilon{\sigma}\). Тогда:
        \[\omega \wedge \tau(v_1, \dots v_{k + l}) = \sum_{\sigma \in S_{k, l}}\epsilon(\sigma)\omega(v_{\sigma(1)}, \dots v_{\sigma{k}})\tau(v_{\sigma(k + 1)}, \dots v_{\sigma{k + l}}) =\]
        \[(-1)^{kl}\sum_{\tilde{\sigma} \in S_{k, l}}\epsilon(\tilde{\sigma})\omega(v_{\tilde{\sigma}(l + 1)}, \dots v_{\tilde{\sigma}{k + l}})\tau(v_{\tilde{\sigma}(1)}, \dots v_{\tilde{\sigma}{l}})= (-1)^{kl}\tau \wedge \omega\]
    \end{enumerate}
\end{proof}

\begin{definition}
    Пусть \(\Phi: V \ra W\) --- линейное отображение. Для \(\omega \in A_k(W)\) можно рассмотреь \(\Phi^*\omega \in A_k(V)\) по правилу
    \[\Phi^*\omega(v_1, \dots v_k) = \omega(\Phi v_1, \dots \Phi v_k)\]
    Данная операция называется pullback.
\end{definition}

\begin{proposition}\indent
    \begin{enumerate}
        \item Отображение \(\Phi^*: A_k(W) \ra A_k(V)\) линейно и \(\Phi^*(\omega \wedge \tau) = \Phi^*\omega \wedge \Phi^*\tau\).
        \item Для \(\Psi: W \ra Z\) --- линейного отображения, верно, что \((\Psi\Phi)^* = \Phi^*\Psi^*\)
    \end{enumerate}
\end{proposition}
\begin{proof}\indent
    \begin{enumerate}
        \item Очевидно
        \item 
        \[(\Psi\Phi)^*\omega(v_1, \dots v_k) = \omega(\Psi\Phi v_1, \dots \Psi\Phi v_k) = \omega(\Psi(\Phi v_1), \dots \Psi(\Phi v_k)) =\]
        \[= \Psi^*\omega(\Phi v_1, \dots \Phi v_k) = \Phi^*\Psi^*\omega(v_1, \dots v_k)\]
    \end{enumerate}
\end{proof}

\begin{example}[Правило детерминанта]
    Пусть \(\alpha^1, \dots \alpha^k \in V^*\), тогда
    \[\alpha^1 \wedge \dots \wedge \alpha^k(v_1, \dots v_k) = \sum_{\sigma \in S_k} \epsilon(\sigma)\alpha^1(v_{\sigma(1)})\dots \alpha^k(v_{\sigma(k)}) = \det(\alpha^i(v_j))\]
\end{example}

\subsection{Дифференциальные формы на открытых подмножествах \(\R^m\)}

Будем отождествлять \(T_p\R^m = \R^m\). Более формально будет записать \((p, \R^m)\), т.е. для каждой точки у нас будет свое касательное пространство

\begin{definition}
    Пусть \(U \subset \R^m\) --- открыто, \(k \in \N\). Дифференциальной \(k\)-формой на \(U\) называется функция \(U \ni p \mapsto w_p \in A_k(\R^n)\).
\end{definition}

Дифференциалы координатных функций \(dx_1, \dots dx_n\) образуют базис в \((\R^m)^*\) (двойственнен к стандартному), тогда \(\{dx_{i_1} \wedge \dots dx_{i_k} : 1 \le i_1 \le \dots \le i_k \le m\}\) образуют базис в \(A_k(\R^m)\). Поэтому имеет место представление
\[\omega_p = \sum_{i_1 < \dots < i_k}f_{i_1, \dots i_k}dx_{i_1} \wedge \dots \wedge dx_{i_1} = \sum_{I \in \mathbb{I}_k} f_Idx^I\]

Функции \(f_I: U \ra \R\) называется координатным представлением формы \(\omega\). Пусть \(r \in \N_0 \cup \{\infty\}\). Говорят, что дифференциальная форма \(\omega\) класса \(C^r(U)\), если все координатные функции \(\in C^r(U)\).

В дальнейшем (если не указано иное) будем предполагать, что все рассматриваемые нами формы \(\in C^\infty\), множество \(k\)-форм класса \(C^\infty\) обозначается \(\Omega^k(U)\). Напомним, что \(A_0(\R^m) = \R\), поэтому \(\Omega^0(U) = C^\infty(U)\)

\begin{enumerate}
    \item Пусть \(\omega, \tau \in \Omega^k(U), f \in C^\infty(U)\), тогда \(f\omega \in \Omega^k(U), p\mapsto f(p)\omega_p, \omega + \tau \in \Omega^k(U), p \mapsto \omega_p + \tau_p\).
    \item Пусть \(\omega \in \Omega^k(U), \tau \in \Omega^l(U)\), положим \(\omega \wedge \tau \in \Omega^{k + l}(U), p \mapsto \omega_p \wedge \tau_p\).
\end{enumerate}

Таким образом, \(\Omega^k(U)\) является линейным пространством

\begin{example}
    Пусть \(\omega = \sum_{I \in \mathbb{I}_k}f_Idx^I, \tau = \sum_{I \in \mathbb{I}_k}g_Idx^I\). Тогда:
    \[\omega \wedge \tau = \sum_{I, J}f_Ig_Jdx^I \wedge dx^J\]
\end{example}

Пусть \(f: U \ra V\), \(U\) открыто в \(\R^m\), \(V\) открыто в \(\R^n\), \(f \in C^\infty\). Тогда для \(p \in U: df_p: \R^m \ra \R^n\) --- линейное отображение, поэтмоу если \(\omega_p \in A_k(\R^n)\), то \((df_p)^*: A_k(\R^n) \ra A_k(\R^m)\) и \((df_p)^*\omega_p = \omega_{f(p)}(df_p(v_1), \dots df_p(v_k))\).

\begin{definition}
    Отображение \(p \mapsto (df_p^*\omega)_p\) определяем дифференциальную форму на \(U\), которая обозначается \(f^*\omega\) и называется переносом формы \(\omega\).
\end{definition}

В частности, при \(k = 0\), т.е. для \(g \in C^\infty(V)\), имеем следующее:
\[f^*g = g \circ f\]

\begin{proposition}
    \begin{enumerate}
        \item Перенос линеен и \(f^*(\omega \wedge \tau) = f^*\omega \wedge f^*\tau\).
        \item \(f: \underbrace{U}_{\subset \R^m} \ra \underbrace{V}_{\subset \R^n}, g: V \ra \underbrace{W}_{\subset \R^k}\) --- класса \(C^\infty \Ra (g \circ f)^* = f^* \circ g^*\).
    \end{enumerate}
\end{proposition}
\begin{proof}\indent
    \begin{enumerate}
        \item Очевидно
        \item Композиция \(df_p^*: A_k(\R^b) \ra A_k(\R^m), dg_q^*: A_k(\R^k) \ra A_k(\R^n)\). Так как \(dg_q \circ df_p = d(g \circ f)_p\), то:
        \[f^*(g^*\omega) = (g \circ f)^*\omega \forall \omega in \Omega^k(W)\]
    \end{enumerate}
\end{proof}

Получим координатную запись \(f^*\).

\begin{lemma}(Перенос как замена переменных)
    Пусть \(f: \underbrace{U}_{\subset \R^m} \ra \underbrace{V}_{\subset \R^n}\) класса \(C^\infty, f = (f_1, \dots f_n)\). Если \(\omega \in \Omega^k(V), \omega = \sum_{(i_1, \dots i_k) \in \mathbb{I}_k}a_Idx_{i_1} \wedge \dots \wedge dx_{i_k}\), то \(f^*\omega = \sum_{I \in \mathbb{I}_k}a_I \circ f df_{i_1} \wedge \dots \wedge df_{i_k}\)
\end{lemma}
\begin{proof}
    По определению, \(f^*a_I = a_I \circ f\). \(f(dx_i)(v) = dx_i(df(v)) = d(x_i \circ f)(v) = df_i(v)\). Поэтому, \(f^*\omega = \sum_{I \in \mathbb{I}_k}f^*a_I(f^*dx_{i_1})\wedge \dots \wedge (f^*dx_{i_k}) = \sum_{I \in \mathbb{I}_k}a_I \circ f df_{i_1} \wedge \dots \wedge df_{i_k}\).
\end{proof}

\begin{corollary}
    \(f^*: \Omega^k(V) \ra \Omega^k(U)\) --- то есть сохраняет гладкость.
\end{corollary}
\begin{proof}
    \(df_i \in \Omega^1(U), a_I \circ f \in C^\infty(U) \Ra f^*\omega \in \Omega^k(U)\)
\end{proof}

\begin{example}
    Пусть \(U \subset \R^m\) --- открыто и задана \(m\)-форма, \(\omega = f(x)dx_1 \wedge \dots \wedge dx_m\) и пусть \(g: W \ra U\) --- диффеоморфизм, \(x = g(t)\), тогда
    \[dg_1 \wedge \dots \wedge dg_m = \sum_{\sigma \in S_m}\frac{\partial g_1}{\partial t_{\sigma(1)}}\dots \frac{\partial g_m}{\partial t_{\sigma(m)}}dt_{\sigma(1)} \wedge \dots \wedge dt_{\sigma(m)} = \]
    \[ = \sum_{\sigma \in S_m} \epsilon(\sigma)\frac{\partial g_1}{\partial t_{\sigma(1)}}\dots \frac{\partial g_m}{\partial t_{\sigma(m)}}dt_{1} \wedge \dots \wedge dt_{m} = J_g dt_{1} \wedge \dots \wedge dt_{m}\]
    В итоге:
    \[g^*\omega = f \circ g dg_1 \wedge \dots \wedge dg_m = f \circ g \cdot J_g dt_{1} \wedge \dots \wedge dt_{m}\]
\end{example}

\begin{definition}
    Пусть \(U \subset \R^m\) открыто и \(\omega = \sum_{I \in \mathbb{I}_k}f_Idx^I \in \Omega^k(U)\). Внешним дифференциалом \(\omega\) называется:
    \[d\omega = \sum_{I \in \mathbb{I}_k}df_I \wedge dx^I \in \Omega^{k + 1}(U)\]
\end{definition}

\begin{example}
    Пусть \(\omega = Pdx + Qdy\) в \(\R^2\). Тогда:
    \[d\omega = dP \wedge dx + dQ \wedge dy = \left( \frac{\partial P}{\partial x}dx + \frac{\partial P}{\partial y}dy \right)\wedge dx + \left( \frac{\partial Q}{\partial x}dx + \frac{\partial Q}{\partial y}dy \right)\wedge dy =\]
    \[= \frac{\partial P}{\partial x} dx \wedge dy + \frac{\partial Q}{\partial y}dy \wedge dx = \left( \frac{\partial P}{\partial x} - \frac{\partial Q}{\partial y} \right)dx \wedge dy\]
\end{example}

\begin{problem}
    Покажите, что \(d\omega_p(v_1, \dots v_{k + 1}) = \sum_{j = 1}^{k + 1}(-1)^{j - 1}\left.\frac{d}{dt}\right|_{t = 0}\omega_p + tv_j(v_1, \dots v_{j - 1}, v_{j + 1}, \dots v_{k + 1})\)
\end{problem}
