% !TEX root = ../../../main.tex

\begin{note}
    Если \(g: E \ra \overline{\R}\) измерима, то \(\tilde{g}: E \times \R \ra \overline{\R}: \tilde{g}(x, t) = g(x)\) также измерима.
\end{note}
\begin{proof}
    Вытекает из того, что если \(A = \{x \in E: g(x) < a\}\) измеримо, то \(\{(x, t): \tilde{g}(x, t) < a\} = A \times \R\).
\end{proof}
\begin{problem}
    Проверить, что \(A \times \R\) измеримо и тогда доказать замечание.
\end{problem}

Рассмотрим важнейшее приложение принципа Кавальери:
\begin{corollary}
    Пусть \(f: E \ra [0, 1]\) измерима, то подграфик: \(S_f = \{(x, t): x \in E 0 < t < f(x)\}\) измерим в \(\R^{n + 1}\) и \(\mu_{n + 1}(S_f) = \int_E f d\mu\)
\end{corollary}
\begin{proof}
    По замечанию, \(F(x, t) = t - f(x)\) измерима. Тогда \(S_f = \{(x, t): t > 0\} \cap \{(x, t): F(x, t) < 0\}\) измеримо.
    \[(S_f)_x\left\{\begin{array}{l}
        (0, f(x)), x \in E \\
        \emptyset, x \notin E
    \end{array}\right. \Ra \mu_1((S_f)_x) = f(x)\]
    Тогда по принципу Кавальери, \(\mu_{n + 1}(S_f) = \int_E f d\mu\).
\end{proof}

\begin{problem}
    Доказать, что \(\mu_{n + 1}(S_f) = \int_0^{+\infty} \mu(\{x \in E: f(x) > t\})dt\)
\end{problem}

\begin{problem}
    Пусть \(T: \R^n \ra \R^n\) --- обратимое линейное отображение и \(E \subset \R^n\) измеримо. Докажите, что \(\mu(T(E)) = |\det T|\mu(E)\). Указание: \(\Delta = [0, 1)^n\).
\end{problem}

\subsection{Замена переменных в кратном интеграле}
\begin{definition}
    Пусть \(U, V\) открыты в \(\R^n\). Отображение \(g: U \ra V\) называется \(C^r\)-диффеоморфизмом, если \(g\) --- биекция, \(g \in C^r(U, V), g^{-1} \in C^r(V, U)\).
\end{definition}

\begin{definition}
    Всюду далее за \(Dg(x)\) обозначается матрица Якоби отображения \(g\) в точке \(x\). Определитель \(\mathcal{G}_g(x) = |\det Dg(x)|\) называется Якобианом.
\end{definition}

\begin{definition}
    Если \(g\) --- диффеоморфизм, то \(g\circ g^{-1} = I_U\) --- тождественное отображение \(U\). По правилу дифференцирования композиции для матриц Якоби, \(Dg^{-1}(g(x))Dg(g^{-1}(x)) = E\). Следовательно, \(Dg(x)\) невырождена и \(Dg^{-1}(g(x)) = (Dg(x))^{-1}\)
\end{definition}

\begin{theorem}
    Пусть \(g: U \ra V\) --- диффеоморфизм открытых \(U, V \subset \R^n\). Пусть \(E \subset U\) --- измеримо по Лебегу. Если \(f\) --- неотрицательная измеримая или интегрируемая на \(E\) функция, то
    \[\int_{g(E)} f(y) d\mu(y) \int_E f \circ g(x) |\det Dg(x)|d\mu(x)\]
\end{theorem}
В основе доказательства будет лежать следующее утверждение:
\begin{lemma}
    Пусть \(x_0 \in U, \epsilon > 0\). Положим \(l(Q)\) --- длина ребра куба \(Q\). Тогда найдется такое \(\delta > 0\), что для всягого замкнутого куба \(Q, x \in Q \subset U, l(Q) < \delta\) справедливо равенство:
    \[\frac{\mu(g(Q))}{\mu(Q)} \le |\det \mathcal{G}_g(x_0)| + \epsilon\]
\end{lemma}
\begin{proof}
    Все мы помним, что в \(\R^n\) все нормы эквивалентны. Будем рассматривать норму \(\|x\| = \max_{1 \le i \le n}|x_i| =\|\cdot\|_{\infty}\). Тогда \(C_r(y) = \{x: \|x - y\| \le r\}\) --- замкнутый куб и \(l(C) = 2r\). Сначла рассмотрим случай \(Dg(x_0) = E\) и \(g(x_0) = 0\). По определению диффеоморфизма, имеем:
    \[g(x) = x + \alpha(x) \|x - x_0\|, \alpha(x) \ra 0, x \ra x_0\]
    Выберем \(\eta > 0\) так, что \(() < 1 + \epsilon\) и пусть \(\delta > 0\) такое, что если \(x \in U (\|x - x_0\| < \delta \Ra \|\alpha(x)\| < \eta)\). Пусть \(Q\) --- замкнутый куб, такой, что \(Q, x \in Q \subset U, l(Q) < \delta\). Положим \(s = L(Q)\) Если \(a\) --- центр \(Q\), то \(Q = C_{\frac{s}{2}}(a)\). Пусть \(x \in Q\). Тогда \(\|x - x_0\| < \|x - a\| + \|a - x_0\| \le \frac{s}{2} + \frac{s}{2} < s < \delta \Ra \|\alpha(x)\| \le \eta\). Следовательно, \(\|g(x) - a\| \le \|x - a\| + \|\alpha(x)\|\cdot\|x-x_0\| < \frac{s}{2} + \eta s = \frac{s}{2}(1 + 2\eta) \Ra g(x) \in C_{\frac{s}{2}(1 + 2\eta)}(a) \Ra g(Q) \subset C_{\frac{s}{2}(1 + 2\eta)}(a)\). По монотонности меры, \(\mu(g(Q)) \le s^n(1 + 2\eta)\). Учитывая, что \(\mu(Q) = s^n \Ra \frac{\mu(g(Q))}{\mu(Q)} \le (1 + 2\eta)^n < 1 + \epsilon\).

    Для общего случая, рассмотрим \(h = T^{-1} g - T^{-1} g(x_0)\) --- диффеоморфизм. \(Dh(x_0) = T^{-1}T = E\). Следовательно, \(\exists \delta > 0: l(Q) < \delta \Ra \frac{\mu(h(Q))}{\mu(Q)} \le |\mathcal{G}_g(x_0)| + \frac{\epsilon}{|\det T|} \le 1 + \frac{\epsilon}{|\det T|}\).
    \[\mu(h(Q)) = \mu(T^{-1}g(Q)) = |\det T^{-1}|\mu(g(Q)) = \frac{\mu(g(Q))}{|\det T|} \Ra \frac{\mu(g(Q))}{\mu(Q)} \le |\det T| + \epsilon\]
    Получили желаемое.
\end{proof}

\begin{corollary}
    Пусть \(\{Q_k\}\) --- последовательность замкнутых, вложенных кубов и \(l(Q_k) \ra 0\). Если \(x_0\) --- единственная точка из \(\bigcap_{k = 1}^\infty Q_k\), то \(\limsup_{n \ra \infty} \frac{\mu(g(Q_k))}{\mu(Q_k)} \le |\mathcal{G}_g(x_0)|\)
\end{corollary}

\begin{lemma}
    Если \(G\) --- открытое в \(U\), то \(g(G)\) открыто в \(V\) и \(\mu(g(G)) \le \int_G |\mathcal{G}_g| d \mu (*)\).
\end{lemma}
\begin{proof}
    По свойству счетной аддитивности меры Лебега и счетной аддитивности интеграла, достаточно установить \((*)\) для двоичного куба.
    \[G = \bigsqcup_{k = 1}^\infty Q_k, \text{где \(Q_k\) --- двоичные кубы} \Ra Q_k = \left[\frac{p}{2^q}, \frac{p + 1}{2^q}\right)^n, p \in \Z, q \in \N\]
    Т.к. \(Q = \bigcup_{i = 1}^\infty \left[\frac{p}{2^q}, \frac{p + 1}{2^q} - \frac{1}{2^q\cdot i}\right]^n\) --- \(\sigma\)-компакт, тогда и \(g(Q)\) --- \(\sigma\)-компакт --- измеримое множество.

    Предположим, что \(\exists Q\) --- двоичный куб, такой, что \(\overline{Q} \subset U\), для которого \(\mu(g(Q)) > \int_Q |\mathcal{G}_g|d\mu + \epsilon \mu(Q)\). Деля каждое ребро пополам, получаем разбиение \(Q\) на \(2^n\) двоичных кубов \(\{Q_i\}\). Предположим, что для каждого из \(Q_i\) выполнена лемма. Тогда пользуясь конечной аддитивностью меры и интеграла, для \(Q\) выполнена лемма. Тогда Б.О.О, \(\mu(g(Q_1)) > \int_{Q_1} |\mathcal{G}_g|d\mu + \epsilon \mu(Q_1)\). Положим \(C_1 = \overline{Q}, C_2 = \overline{Q_1}, \dots\). По индукции строим последовательность замкнутых вложенных кубов, \(l(C_k) \ra 0\), такую, что \(\forall k \mu(g(C_k)) > \int_{C_k}|\mathcal{G}_g(x)|d \mu(x) + \epsilon(\mu(C_k))\). Т.к. \(\mathcal{G}_g\) --- непрерывна, то \(\frac{1}{\mu(C_k)}\int_{C_k} |\mathcal{G}_g|d\mu \ra_{k \ra \infty} |\mathcal{G}_g(x)| \Ra \limsup_{k \ra \infty} \frac{\mu(g(C_k))}{\mu(C_k)} \ge |\mathcal{G}_g(x_0)| + \epsilon\).
\end{proof}

\begin{example}
    Вычислить меру \(n\)-мерного шара \(B_R(a)\)
\end{example}
\begin{proof}[Решение]
    По свойству преобразования меры при сдвиге и гомотетии, имеем:
    \[\mu_n(B_R(a)) = \mu_n(B_R(0)) = R^n\mu_n(B_1(0))\]
    Положим \(B = B_1(0), \mu_n(B) = \omega_n\). Рассмотрим сечение \(B_{(x_1, x_2)} = \{y: |y|^2 < 1 - x_1^2 - x_2^2\} \Ra B_{(x_1, x_2)} = \left\{\begin{array}{l}
        B_{\sqrt{1 - x_1^2 - x_2^2}}(0), x_1^2 + x_2^2 < 1 \\
        \emptyset, \text{иначе}
    \end{array}\right.\)
    Следовательно, по принципу Кавальери,
    \[\omega_n = \int_{x_1^2 + x_2^2 < 1}\mu_{n - 2}(B_{\sqrt{1 - x_1^2 - x_2^2}}(0))dx_1dx_2 = \int_{x_1^2 + x_2^2 < 1} \sqrt{R^n}\omega_{n - 2}dx_1dx_2 =\]
    \[= \omega_{n - 2}\int_{x_1^2 + x_2^2 < 1} (1 - x_1^2 - x_2^2)^{\frac{n - 1}{2}}dx_1dx_2 = (*)\]
    Рассмотрим \(\left\{\begin{array}{l}
        x = r\cos \phi \\
        y = r\sin \phi
    \end{array}\right.\). Пусть \(U = [0, 2\pi] \times [0, +\infty), V = \R^2\). Тогда при замене, мы должны домножить на \(\left|\begin{array}{cc}
        \cos \phi & \sin \phi \\
        -r \sin \phi & r\cos \phi
    \end{array}\right|\).
    \[(*) = \omega_{n - 2}\int_0^{2\pi}d \phi \int_0^1 (1 - r^2)^{\frac{n - 2}{2}}rdr = \frac{-2\pi}{2}\omega_{n - 2}\int_0^1(1 - r^2)^{\frac{n - 2}{2}} = \left.-\pi\omega_{n - 2}\frac{(1 - r^2)^{\frac{n}{2}}}{\frac{n}{2}}\right|_0^1 = \frac{2\pi}{n}\omega_{n - 2}\]
    \begin{enumerate}
        \item \(n = 2k\)
        \[\omega_{2k} = \frac{2\pi}{2k}\omega_{2k - 2} = \dots = 2\frac{(2\pi)^{k - 1}}{(2k)!!} = \frac{(2\pi)^k}{(2k)!!}\]
        \item \(n = 2k + 1\)
        \[\omega_{2k + 1} = \frac{2\pi}{2k + 1}\omega_{2k - 1} = \dots = 2\frac{(2\pi)^{k - 1}}{(2k - 1)!!}\omega_1 = 2\frac{(2\pi)^k}{(2k - 1)!!}\]
    \end{enumerate}
\end{proof}

