% !TEX root = ../../../main.tex

\subsection{Дифференциал гладкого отображения}
\begin{definition}
    Пусть \(M\) --- гладкое многообразие, \(f: M \ra \R^l\) --- гладкое отображение. Дифференциал \(df_p: T_pM \ra \R^l\) определяется следующим образом:
    Пусть \(\gamma(-\delta, \delta)\) --- гладкая кривая, \(\gamma(0) = p, \gamma'(0) = v\), тогда 
    \begin{equation}
        df_p(v) = \left.\frac{d}{dt}\right|_{t = 0}f(\gamma(t))
    \end{equation}
\end{definition}

\begin{theorem}
    Пусть \(M\) --- гладкое \(m\)-мерное многообразие в \(\R^n\) и задано гладкое отображение \(f: M \ra \R^l\). Тогда справедливы следующие утверждения:
    \begin{enumerate}
        \item (3.1) не зависит от выбора \(\gamma\)
        \item \(df_p: M \ra \R^l\) линейное
        \item Если \(N\) --- гладкое многообразие в \(\R^l, f(M) \subset N\), то \(df_p(T_pM) \subset T_qN\), где \(q = f(p)\).
    \end{enumerate}
\end{theorem}
\begin{proof}\indent
    \begin{enumerate}
        \item Пусть \(v \in T_pM\). Выебрем гладкую кривую \(\gamma: (-\delta, \delta) \ra M\) так, что \(\gamma(0) = p, \gamma'(0) = v\). Рассмотрим \(F: U \ra \R^l\) --- гладкое продолжение \(f\) в некоторой окрестности \(U\) точки \(p\) в \(\R^n\). Т.к. \(\gamma(t) \in M \cap U\) при малых \(|t|\), то если обозначить через \(DF(p)\) матрицу Якоби отображение \(F\) в точке \(p\), имеем:
        \[DF(p)v = DF(\gamma(0))\gamma'(0) = \left.\frac{d}{dt}\right|_{t = 0}F(\gamma(t)) = \left.\frac{d}{dt}\right|_{t = 0}f(\gamma(t))\]
        Заметим, что \(DF(p)v\) не зависит от \(\gamma\), а \(\left.\frac{d}{dt}\right|_{t = 0}f(\gamma(t))\) --- от \(F\).

        \item Линейность следует из того, что \(df_p(v) = DF(p)v\)
        \item Если \(\gamma: (-\delta, \delta) \ra M\) --- гладкая кривая, то \(\gamma(0) = p, \gamma'(0) = v\), то \(\beta = f \circ \gamma: (-\delta, \delta) \ra N\) --- гладкая кривая, \(\beta(0) = f(p) = q, \beta'(0) = df_p(v)\). Тогда \(df_p(v) \in T_qN\).
    \end{enumerate}
\end{proof}

\begin{corollary}
    Если в пункте 3 дополнительно \(g: N \ra \R^s\) --- гладкое, то справедливо цепное правило:
    \[d(g \circ f)_p = dg_{f(p)} \circ df_p\]
\end{corollary}
\begin{proof}
    Достаточно продифференцировать функцию \(g(f(t))\) в \(t = 0\) по правилу дифференцирования композиции.
\end{proof}

\begin{corollary}
    Пусть \(M\) --- гладкое \(m\)-мерное многообразие, \(N\) --- гладкое \(r\)-мерное многообразие. Если \(f: M \ra N\) --- диффеоморфизм, то \(m = r\).
\end{corollary}
\begin{proof}
    Пусть \(p \in M, q = f(p) \in N\). Обозначим через \(g = f^{-1}: N \ra M\), тогда \(g \circ f = \id_M, f \circ g = \id_N\), откуда:
    \[dg_q \circ df_p = \id: T_pM \ra T_pM\]
    \[df_p \circ dg_q = \id: T_qN \ra T_qN\]
    Следовательно, \(df_p: T_pM \ra T_qN\) --- линейный изоморфизм и \((df_p)^{-1} = dg_p\). Следовательно \(m = \dim T_pM = \dim T_qN = r\).
\end{proof}

\begin{note}
    Пусть \(\phi: V \ra \R^n\) --- параметризация \(M\) в окрестности точки \(p = \phi(a)\), \(v \in T_pM\) и \(d\phi_a(h) = v\), тогда 
    \[df_p(v) = df_p(d\phi_a(h)) = d(f \circ \phi)_a(h)\]
    В частности, в координатах \(df_p\) задается матрицей Якоби координатного представления \(f \circ \phi\).
\end{note}

\begin{definition}
    Пусть \(f: M \ra N\) --- гладкое отображение гладкого многообразия \(M\) в гладкое многообразие \(N\). Точка \(q \in N\) называется регулярным значением \(f\), если \(df_p: T_pM \ra T_qN\) сюръективен \(\forall p \in f^{-1}(q)\).
\end{definition}

\begin{theorem}[О регулярном значении]
    Пусть \(f: M \ra N\) --- гладкое отображение гладкого \(m\)-мерного многообразия \(M \subset \R^s\) в гладкое \(n\)-мерное многообразие \(N\), \(n < m\) и \(q \in N\) --- регулярное значение \(f\). Тогда \(P = f^{-1}(q) = \{p \in M: f(p) = q\}\) является гладким \(m - n\) мерным многообразием в \(\R^s\).
\end{theorem}
\begin{proof}
    Зафиксируем \(p \in f^{-1}(q)\) и рассмотрим параметризации \(\phi: U_0 \ra U\) в окрестности \(p\) в \(M\), \(\psi: V_0 \ra V\) в окрестности \(q\) в \(N\). Уменьшая \(U\), если это необходимо, и пользуясь непрерывностью \(f\) в точке \(p\), можно считать, что \(f(U) \subset V\). Тогда определено отображение \(f_0 = \psi \circ f \circ \phi: U_0 \ra V_0\) --- координатное представление \(f\). Пусть \(b = \psi^{-1}(q)\). Если \(a \in U_0\) такая, что \(f_0(a) = b\), то \(\phi(a) = p \in U \cap P\) и отображения \(d\phi_a: \R^m \ra T_pM, df_p: T_pM \ra T_qN, d\psi^{-1}_q: T_qN \ra \R^n\) сюръективны, а значит сюръективна и их композиция, т.е. \(d(f_0)_a: \R^m \ra \R^n\). Следовательно, \(b\) --- регулярное значение \(f_0\). Множества \(f^{-1}_0(b) = \{x \in U_0: f(\phi(x)) = q\} = \phi^{-1}(U \cap P)\) является \(m - n\) мерным гладким многообразием в \(\R^s\). Уменьшая \(U_0\), если это необходимо, найдем открытые \(W \subset \R^{m - n}\) и параметризацию \(\alpha: W \ra \phi^{-1}(U \cap P)\). Следовательно, \(\phi \circ \alpha: W \ra U \cap P\) --- параметризация \(P\) в окрестности \(p\).
\end{proof}

\begin{problem}
    Докажите, что \(T_pP = \ker df_p\)
\end{problem}

\section{Экстремумы функций многих переменных}
\subsection{Безусловный экстремум}
Пусть \(f: U \ra \R\), где \(U\) --- открыто в \(\R^n\)
\begin{definition}
    Точка \(a \in U\) называется точкой локального максимума \(f\), если \(\exists \delta > 0: \forall x \in \stackrel{\circ}{B_{\delta}}(a) (f(x) \le f(a))\). Аналогично определяются локальный минимум, и локальные строгие максимум и минимум (если соответствующий знак строгий).
\end{definition}

\begin{note}
    Все такие точки называются точками локального экстремума
\end{note}

\begin{theorem}
    Если \(a\) --- точка экстремума \(f\) и существует \(\frac{\partial f}{\partial x_k}(a)\), то \(\frac{\partial f}{\partial x_k}(a) = 0\)
\end{theorem}
\begin{proof}
    Заметим, что \(a_k\) --- экстремум функции \(\phi(t) = f(a_1, a_2, \dots a_{k - 1}, t, a_{k + 1}, \dots a_n)\). По условию, \(\phi(t)\) дифференцируема в точке \(a\), поэтому по теореме Ферма, \(0 = \phi'(a_k) = \frac{\partial f}{\partial x_k}(a) = 0\)
\end{proof}

\begin{corollary}
    Если \(a\) --- точка экстремума функции \(f\) и \(f\) дифференцируема в точке \(a\), то \(df_a = \nabla f(a) = \vec{0}\)
\end{corollary}

\begin{definition}
    Точка \(a \in U\) называется стационарной точкой функции \(f\), если \(f\) дифференцируема в этой точке и \(df_a = 0\).
\end{definition}

\begin{reminder}
    Если \(f \in C^2(U)\), то \(d^2f_a(h) = \sum_{i = 1}^n\sum_{j = 1}^n \frac{\partial^2 f}{\partial x_i \partial x_j}(a)h_ih_j\) --- квадратичная форма в \(\R^n\) относительно компонент вектора \(h\).
\end{reminder}

\begin{definition}
    Матрица \(d^2f_a\) обозначается через \(Hf(a)\), причем
    \[Hf(a) = \left( \frac{\partial^2 f}{\partial x_i \partial x_j}\right)\]
    Называется матрицей Гессе
\end{definition}

\begin{theorem}
    Пусть \(f \in C^2(U)\) и \(a\) --- стационарная точка \(f\). Тогда справедливы следующие утверждения:
    \begin{enumerate}
        \item Если \(d^2f_a(h) > 0 \forall h \ne 0 \Ra a\) --- точка строгого минимума
        \item Если \(d^2f_a(h) < 0 \forall h \ne 0 \Ra a\) --- точка строгого максимума
        \item Если \(\exists h_+, h_- \in \R^n\), такие, что \(d^2f_a(h_+) > 0, d^2f_a(h_-) < 0\), то \(a\) не является точкой экстремума функции \(f\).
    \end{enumerate}
\end{theorem}
\begin{proof}\indent
    \begin{enumerate}
        \item По формуле Тейлора с остаточным членом в форме Пеано, имеем:
        \[f(a + h) = f(a) + \frac{1}{2}d^2f_a(h) + \alpha(h)|h|^2 = f(a) + \frac{1}{2}|h|^2\left( d^2f_a\left( \frac{h}{|h|} \right) + \alpha(h)\right)\]
        Где \(\alpha(h) \ra 0, h \ra 0\). Функция \(d^2f_a\left( h \right)\) --- непрерывная функция относительно \(h\) и \(S = \{h \in \R^n: |h| = 1\}\) --- компакт в \(\R^n\). Тогда по теореме Вейшерштрасса, \(\exists m = \inf_{|h| = 1} d^2f_a(h) > 0\). Выберем \(\delta > 0\) так, что \(|\alpha(h)| \le \frac{m}{4}|h|^2 \forall h: 0 < |h| < \delta\). Тогда \(f(a + h) - f(a) \ge \frac{m}{4} > 0 \forall h \in \stackrel{\circ}{B_{\delta}}(0)\). Это доказывает, что \(a\) --- точка строгого локального минимума.

        \item Доказывается аналогично
        \item По формуле Тейлора, 
        \[f(a + th_+) = f(a) + t^2\left( \frac{1}{2}d^2f_a(h_+) + \alpha(th_+)|h_+|^2 \right)\]
        Заметим, что \(\lim_{t \ra 0} \frac{1}{2}d^2f_a(h_+) + \alpha(th_+)|h_+|^2 > 0\), поэтому \(\exists \delta_1 > 0: f(a + th_+) - f(a) > 0 \forall t: 0 < |t| < \delta_1 \). Аналогично, \(\exists \delta_2 > 0: f(a + th_-) - f(a) < 0 \forall t: 0 < |t| < \delta_2 \). Выражение \(f(x) - f(a)\) не сохраняет знак ни в какой окрестности точки \(a\), т.е. \(a\) не является точкой локального экстремума функции \(f\)
    \end{enumerate}
\end{proof}

\begin{note}
    Если \(d^2f_a\) как квадратичная форма полуопределена, то теорема не позволяет сделать вывод о наличии экстремума в точке \(a\).
\end{note}

\begin{example}
    \(f: \R^2 \ra \R, f(x, y) = x^2 + y^n, n \in \N, n \ge 3\). Тогда \(d^2f_O(h_1, h_2) = 2h_1^2\) --- положительно полуопределенная квадратичная форма. Однако при четных \(n\), \(O\) является точка строгого минимума \(f\), а при нечетных --- не является.
\end{example}

\begin{note}
    Для исследования формы на определенность можно либо привести ее к диагональному виду, либо воспользоваться критерием Сильвестра.
\end{note}

\subsection{Условный экстремум}
Пусть \(U \subset \R^n\) --- открыто и задана функция \(g: U \ra \R^m, g = (g_1, g_2, \dots g_m), 0 < m < n\). Изучим \(f: U \ra \R\) на экстремум на множестве нулей \(M = g^{-1}(0)\).
\begin{definition}
    Точка \(p \in M\) называется точкой локального условного максимума функции \(f\) на множестве \(M\), если \(\exists \delta > 0 \forall x \in \stackrel{\circ}{B}_\delta(p) \cap M (f(x) \le f(p))\). Аналогично определяются точки других типов условного экстремума.
\end{definition}

\begin{theorem}[Лагранж]
    Пусть \(f \in C^1(U, g \in C^1(U, \R^n)), rk\;Dg(p) = m\). Если \(p\) --- точка условного экстремума \(f\) на \(M\), то \(\exists \lambda_1, \lambda_2, \dots \lambda_m \in \R: \nabla f(p) = \sum_{k = 1}^m \lambda_k \nabla g_k(p)\).
\end{theorem}
\begin{proof}
    Т.к. \(rk\;Dg(p) = m\), то матрица Якоби имеет минор порядка \(m\), \(\ne 0\). Учитывая, что \(g \in C^1\), то этот минор будет отличен от \(0\) в некоторой окрестности точки \(p\). Б.О.О. можно считать, что \(rk\;Dg(x) = m \forall x \in U\). Тогда \(M\) является гладким \(m - n\) мерным многообразием в \(\R^n\). Пусть \(v \in T_pM\). Рассмотрим гладкую кривую \(\gamma: (-\delta, \delta) \ra M\), такую, что \(\gamma(0) = p, \gamma'(0) = v\). Функция \(f \circ \gamma\) имеет экстремум в точке \(t = 0\), поэтому
    \[0 = \left.\frac{d}{dt}\right|_{t = 0}f(\gamma(t)) = (\nabla f(p), v)\]
    То есть \(\nabla f(p) \in (T_pM)^{\perp}\). Т.к. \(\nabla g_1(p), dots \nabla g_m(p)\) образуют базис в \((T_pM)^{\perp}\), поэтому такие \(\lambda_1, \lambda_2, \dots \lambda_m\) найдутся
\end{proof}
