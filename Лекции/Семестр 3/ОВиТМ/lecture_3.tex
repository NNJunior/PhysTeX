% !TEX root = ../../../main.tex

\begin{proposition}[Неравенство Маркова]
    Пусть \(\xi \ge 0, a > 0\). Тогда:
    \[P(\xi \ge a) \le \frac{\E \xi}{a}\]
\end{proposition}
\begin{proof}
    Пусть \(I_{\{\xi \in A\}}(\omega) = \left\{\begin{array}{l}
        1, \xi(\omega) \in A \\
        0, \xi(\omega) \notin A
    \end{array}\right.\)
    \[\E\xi = \E(\xi I_{\{\xi \ge a\}} + \xi I_{\{\xi < a\}}) = \E\xi I_{\{\xi \ge a\}} + \E\xi I_{\{\xi < a\}}\ge \E aI_{\{\xi \ge a\}} = a \E I_{\{\xi \ge a\}} = aP(\xi \ge a)\]
    \[\frac{\E \xi}{a} \ge P(\xi \ge a)\]
\end{proof}

\begin{proposition}[Неравенство Чебышева]
    Пусть \(\xi, \E \xi < \infty, \Variance \xi < \infty, a > 0\). Тогда:
    \[P(\{|\xi - \E\xi| > a\}) \le \frac{\Variance \xi}{a^2}\]
\end{proposition}
\begin{proof}
    \[P(\{|\xi - \E\xi| > a\}) = P(\{(\xi - \E\xi)^2 > a^2\}) \le \frac{\Variance \xi}{a^2}\]
\end{proof}

\begin{theorem}[Закон Больших Чисел]
    Пусть \(\{\xi_i\}\) --- некоторые случайные величины, для которых выполнено:
    \begin{enumerate}
        \item \(\E \xi_n = \E \xi_1 \forall n\)
        \item \(\xi_i\) попарно некоррелированные
        \item \(\forall n \Variance \xi_n \le C\).
    \end{enumerate}
    Тогда
    \[\forall \epsilon > 0: \forall \epsilon > 0: P\left\{\left|\frac{S_n}{n} - \E\frac{S_n}{n}\right|\right\} \ra 0\]
\end{theorem}
\begin{proof}
    Положим \(S_n = \sum_{k = 1}^n \xi_k\). Тогда 
    \[\forall \epsilon > 0: P\left\{\left|\frac{S_n}{n} - \underbrace{\E\frac{S_n}{n}}_{\E\xi_1}\right|\right\} > \epsilon \le \frac{\Variance \frac{S_n}{n}}{\epsilon^2} = \frac{\Variance S_n}{n^2\epsilon^2} = \frac{\sum_{k = 1}^n \Variance \xi_k + \sum cov(\xi_i, \xi_j)}{n^2\epsilon^2} \le \frac{C}{n\epsilon^2} \ra 0\]
\end{proof}

\begin{theorem}[Центральная предельная теорема (б/д)]
    Пусть \(\xi_n\) --- последовательность независимых в совокупности одинаково распределенных случайных величин с ограниченными математическими ожиданиями и дисперсиями. Тогда \(\forall a, b \in [-\infty, +\infty], a < b\):
    \[P\left\{a \le \frac{S_n - \E S_n}{\sqrt{\Variance S_n}}\le b\right\} \ra_{n \ra \infty} \int_a^b \frac{1}{\sqrt{2\pi}}e^{-\frac{x^2}{2}}dx\]
\end{theorem}

\section{Теория Меры}
\begin{definition}
    Пусть дано некоторое множество \(\Omega\). Множество \(S \subset 2^\Omega\) называется полукольцом, если:
    \begin{enumerate}
        \item \(A, B \in S \Ra A \cap B \in S\)
        \item \(A, B \in S \Ra A \setminus B = \bigsqcup_{k = 1}^n A_k, A_k \in S\)
    \end{enumerate}
\end{definition}

\begin{example}
    Рассмотрим \(\Omega = [0, 1)\). Тогда \(S = \{[a, b) | [a, b) \subset \Omega\}\)
\end{example}

\begin{definition}
    Полуалгебра --- такое полукольцо, что \(\exists E \in S: \forall A \in S: A \subset E\).
\end{definition}

\begin{definition}
    Пусть дано некоторое множество \(\Omega\). Множество \(R \subset 2^\Omega\) называется кольцом, если:
    \begin{enumerate}
        \item \(A, B \in R \Ra A \cap B \in R\)
        \item \(A, B \in R \Ra A \Delta B \in R\)
    \end{enumerate}
\end{definition}

\begin{definition}
    Алгебра --- кольцо, являющееся полуалгеброй
\end{definition}

\begin{proposition}
    \(\forall \mathcal{X} \subset 2^\Omega \exists\) наименьшее по включению кольцо (алгебра) над \(x\)
\end{proposition}
\begin{proof}
    Положим \(D = \{\text{кольца}\supset \mathcal{X}\}\). Рассмотрим
    \(R = \bigcap_{d \in D} d\) --- тоже кольцо (алгебра) над \(\mathcal{X}\).
    \[A, B \in R \Ra \forall S \in D: A, B \in S \Ra A \cap B \in S, A \Delta B \in S\]
\end{proof}

\begin{proposition}
    Пусть \(S\) --- полукольцо. Тогда \(\forall A, B_1, \dots B_n \in S \exists m, A_1, \dots A_m \in S\), такие, что
    \[A \setminus B_1 \setminus B_2 \setminus \dots \setminus B_n = \bigsqcup_{k = 1}^m A_k, A_k \in S, m \in \N\]
\end{proposition}
\begin{proof}
    Ведем индукцию по \(n\)
    \begin{enumerate}
        \item[] \textbf{База:} \(n = 1\)
        \item[] \textbf{Переход:}
        \[A \setminus B_1 \setminus B_2 \setminus \dots \setminus B_n = \left(\bigsqcup_{k = 1}^m A_k\right)\setminus B_n = \bigsqcup_{k = 1}^m \underbrace{(A_k \setminus B_n)}_{\in S}\]
    \end{enumerate}
\end{proof}

\begin{proposition}
    Пусть \(S\) --- полукольцо. Тогда \(R(S) = \left\{\bigcup_{k = 1}^n A_k | n \in \N, A_k \in S\right\}\). Тогда \(R(S)\) --- минимальное кольцо, содержащее \(S\).
\end{proposition}
\begin{proof}
    Пусть \(\mathcal{R}\) --- наименьшее кольцо, содержащее \(S\). Докажем, что \(R(S) = \mathcal{R}\)
    \begin{enumerate}
        \item [] \(R(S) \subset \mathcal{R}\) --- очевидно, т.к. любое кольцо включает в себя \(R(S)\) как подсистему.
        \item[]\(R(S) \supset \mathcal{R}\) Рассмотрим 
        \[\bigsqcup_{k = 1}^n A_k \cap \bigsqcup_{s = 1}^m B_s = \bigsqcup_{k \le n, s \le m} \underbrace{(A_k \cap B_s)}_{\in S}\]
        Теперь рассмотрим 
        \[\bigsqcup_{k = 1}^n A_k \Delta \bigsqcup_{s = 1}^m B_s = \bigsqcup_{k = 1}^n (A_k \setminus B_1 \setminus B_2 \setminus \dots \setminus B_m) \cup \bigsqcup_{s = 1}^m (B_s \setminus A_1 \setminus A_2 \setminus \dots \setminus A_n)\]
    \end{enumerate}
\end{proof}

\begin{proposition}[Об общих кирпичах]
    \(\forall A_1, A_2, \dots A_k \in S \exists B_1, \dots B_m\) --- попарно непересекающиеся множества, такие, что \(\forall i = 1, \dots n \exists \Gamma_i \subset \{1, 2, \dots m\}: A_i = \bigsqcup_{\gamma \in \Gamma_i}B_\gamma\).
\end{proposition}
\begin{proof}
    Ведем индукцию по \(n\)
    \begin{enumerate}
        \item[] \textbf{База:} \(n = 1\), берем \(m = 1, B_1 = A\).
        \item[] \textbf{Переход:} известно:
        \[A_{n + 1} \setminus B_1 \setminus \dots \setminus B_m = \bigsqcup_j D_j\]
        Каждое из множеств \(B_s \setminus A_{n + 1}\) разобъем по определению, получим множества \(B_{s, i}\). Итого искомые множества: \(B_{s, i}, D_j, A_{n + 1} \cap B_s\).
    \end{enumerate}
\end{proof}

\begin{definition}
    \(\sigma\)-алгебра --- такая алгебра, которая замкнута относительно относительно \(\bigcap^\infty A_i\)
\end{definition}

\begin{definition}
    \(\delta\)-алгебра --- такая алгебра, которая замкнута относительно относительно \(\bigcup^\infty A_i\)
\end{definition}
\begin{note}
    \(\sigma\)-алгебра и \(\delta\)-алгебра --- это одно и то же
\end{note}

\begin{definition}
    \(\sigma\)-кольцо --- такое кольцо, которое замкнуто относительно относительно \(\bigcap^\infty A_i\)
\end{definition}

\begin{definition}
    \(\delta\)-кольцо --- такая кольцо, которое замкнуто относительно относительно \(\bigcup^\infty A_i\)
\end{definition}
\begin{note}
    \(\sigma\)-кольцо и \(\delta\)-кольцо --- это \textbf{НЕ} одно и то же, однако \(\sigma\)-кольцо всегда является \(\delta\)-кольцом
\end{note}
\begin{example}
    Рассмотрим \(X = \{A \subset \N |\;|A| < \infty\}\). Оно является \(\delta\)-кольцом, но не \(\sigma\) (т.к. \(\bigcap_{n \in \N}\{n\} = \N \notin X\)).
\end{example}

\begin{definition}
    \(\mathcal{B}(\R)\) --- борелевская \(\sigma\)-алгебра --- минимальная по включению алгебра, содержащая все открытые множества
\end{definition}
