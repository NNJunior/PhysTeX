% !TEX root = ../../../main.tex

\subsection{Меры Жордана и Лебега}
\begin{definition}
    Пусть \(S\) --- полуалгебра, \(m: S \ra [0, +\infty)\). Тогда внешняей мерой Жордана называется функция: \(\mu_J^*(A) = \inf_{\bigcup^n A_k \supset A} \sum m(A_k), A_k \in S\).
\end{definition}

\begin{example}[Мера Жордана не \(\sigma\)-аддитивна]
    Расмотрим \(\mu_J^*(\Q\cap [0, 1]) = 1\), однако, если рассматривать отрезки длины \(q^n\) для каждой точки с номером \(n\), и устремить \(q \ra 0\), то получим, что \(\sum q^n = 0\).
\end{example}

Приведем внешнюю меру, которая является \(\sigma\)-аддитивной на полуалгебре \(S\).

\begin{definition}
    Пусть \(S\) --- полуалгебра, \(m: S \ra [0, +\infty)\). Тогда внешняей мерой Жордана называется функция: \(\mu_J^*(A) = \inf_{\bigcup^infty A_k \supset A} \sum m(A_k), A_k \in S\).
\end{definition}

\begin{proof}
    Сужения 
\end{proof}
\begin{proof}
    
\end{proof}
