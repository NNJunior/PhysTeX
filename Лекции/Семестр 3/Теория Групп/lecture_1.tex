% !TEX root = ../../../main.tex

\section{Введение}
\subsection{Определения}
\begin{definition}
    Группа --- это множество \(G\) с введенным на нем бинарной операцией \(*\), удовлетворяющее следующим свойствам:
    \begin{enumerate}
        \item \textbf{Ассоциативность:} \((a * b) * c = a * (b * c)\)
        \item \textbf{Наличие нейтрального элемента:} \(\exists e \in G: \forall g \in G g * e = e * g = g\).
        \item \textbf{Наличие обратного элемента:} \(\forall a \in G \exists a^{-1}: a * a^{-1} = a^{-1} * a = e\)
    \end{enumerate}
\end{definition}

\begin{note}
    В группе нейтральный элемент единственнен
\end{note}
\begin{proof}
    От противного, тогда \(e_1 = e_1 * e_2 = e_2\).
\end{proof}

\begin{note}
    В группе обраный элемент единственнен для любого элемента \(G\).
\end{note}
\begin{proof}
    Пусть \(\exists b, c: a * b = b * a = e, a * c = c * a = e\). Рассмотрим \(b = (c * a) * b = c * (a * b) = c\).
\end{proof}

\begin{note}
    Существует более слабое определение группы --- можно не писать \textbf{одну} из коммутативностей в пунктах \(2, 3\).
\end{note}

\begin{proposition}
    В группе выполняется правило левого и правого сокращения, т.е. \(a * b = c * b \Lra a = c\) или \(b * a = b * c \Lra a = c\) (необходимо домножить на \(b^{-1}\) справа или слева).
\end{proposition}

\begin{definition}
    Группа \(G\) называется Абелевой, если операция \(*\) \textbf{коммутативна}, т.е. \(\forall a, b \in G a * b = b * a\).
\end{definition}

\begin{definition}
    Порядок группы --- \(|G| \in \N \cup \{\infty\}\) --- мощность группы (в случае бесконечной, порядок равен \(\infty\)).
\end{definition}

\begin{definition}
    \(a^n = \underbrace{a * a * \dots * a}_{n}, a^0 = e\).
\end{definition}

\begin{definition}
    Порядок элемента группы --- \(\ord a = \min\{n \in \N | a^n = e\}\) (или \(\infty\) в случае пустоты указанного множества).
\end{definition}

\begin{definition}
    Множество \(H\) называется подгруппой \(G\), если \(G \subset H\) и \(H\) является группой относительно операции \(*_G\).
\end{definition}

\begin{proposition}[Критерий подгруппы]
    Непустое подмножество \(H\) в группе \(G\) является подгруппой (\(H \le G\)), если
    \begin{enumerate}
        \item \(H\) замкнуто относительно операции \(*\), т.е. \(a, b \in H \Ra a * b \in H\)
        \item \(H\) замкнуто относительно операции взятия обратного элемента, т.е. \(a \in H \Ra a ^{-1} \in H\)
    \end{enumerate}
\end{proposition}
\begin{proof}
    \item \textbf{Ассоциативность:} следует из ассоциативности группы \(G\) и замкнутости относительно операции \(*\).
    \item \textbf{Наличие нейтрального элемента:} следует из замкнитости относительно взятия обратного и произведения \(a * a^{-1} = e\)
    \item \textbf{Наличие обратного элемента:} следует из замкнутости относительно операции взятия нейтрального элемента.
\end{proof}

\begin{definition}
    Пусть \((G_1, *), (G_2, \cdot)\) --- группы. Гомоморфизмом \(G_1 \ra G_2\) называется всякое отображение \(f: G_1 \ra G_2\), такое, что \(f(a * b) = f(a) \cdot f(b)\).
\end{definition}

\begin{definition}
    Изоморфизм --- гомоморфизм, являющийся биекцией. Если группы изоморфны, пишут \(G \cong H\).
\end{definition}

\begin{theorem}[Кэли]
    Всякая конечная группа, порядок которой нравен \(n\), изоморфна некоторой подгруппе группы \(S_n\).
\end{theorem}

\subsection{Примеры групп}
\begin{example}
    \((\Z_n, +), |Z_n| = n\)
\end{example}

\begin{example}
    \((V, +)\)
\end{example}

\begin{example}
    \(GL_n(F)\) --- группа невырожденных матриц
\end{example}

\begin{example}
    \(S_n\) --- группа перестановок, \(|S_n| = n!\)
\end{example}

\begin{example}
    \(Q\) --- группа кватернионов, \(|Q| = 8, Q = \{\pm 1, \pm i, \pm j, \pm k\}\),
    \[ij = k, jk = i, ki = j, ji = -k, kj = -i, ik = -j\]
\end{example}

\subsection{Примеры подгрупп}
\begin{example}
    \(n\Z \le \Z\)
\end{example}

\begin{example}
    \(W \le V\)
\end{example}

\subsection{Подгруппа, порожденная подмножеством}
Пусть \(M \subset G\). Рассмотрим \(\langle M \rangle = \bigcap_{M \subset H_i \le G} H_i\)

\begin{theorem}[Об описании подгруппы, порожденной множеством]
    \(\langle M \rangle = \{m_1^{\epsilon_1}m_2^{\epsilon_2}\dots m_s^{\epsilon_s} | s \in \Z_{\ge 0}\}\).
\end{theorem}
\begin{proof}
    \begin{enumerate}
        \item[\(\subset\)] Заметим, что полученное множество является подгруппой по критерию подгруппы (\((m_1^{\epsilon_1}\dots m_s^{\epsilon_s})^{-1} = m_s^{-\epsilon_1}\dots m_1^{-\epsilon_1}\)).
        \item[\(\supset\)] Все представленные элементы обязаны лежать в \(\langle M \rangle\), т.к. они лежат в каждой группе, содержащей \(M\).
    \end{enumerate}
\end{proof}

\begin{definition}
    \(G = \langle M \rangle\) --- тогда говорят, что \(G\) порождается множеством \(M\). Тогда элементы из \(M\) называются порождающими элементами.
\end{definition}

\begin{definition}
    Пусть \(a \in G\). Тогда группа \(\langle a \rangle \) называется циклической.
\end{definition}

\begin{theorem}[Об элементе конечного порядка]
    Пусть \(a \in G, \ord a = n\). Тогда \(\langle a \rangle\) конечна и \(|\langle a \rangle| = n\) и \(\langle a \rangle = \{e, a, a^2, \dots a^{n - 1}\}\) 
\end{theorem}

\begin{theorem}
    Все циклические группы одного и того же порядка (в том числе и бесконечного) изоморфны.
\end{theorem}
\begin{proof}
    \begin{enumerate}
        \item \(\ord \in \N\). Тогда эта группа изоморфна \((Z_n, +)\)
        \item \(\ord = \infty\). Тогда эта группа изоморфна \((Z, +)\)
    \end{enumerate}
\end{proof}

\subsection{Описание подгрупп циклических групп}
\begin{theorem}
    Всякая подгруппа циклической группы является циклической
\end{theorem}

\begin{theorem}
    Если \(G = \langle a \rangle \cong C_n\) и \(H_d = \langle a^d \rangle, d | n\), то
    \begin{enumerate}
        \item \(H_d \le G, |H_d| = \frac{n}{d}\)
        \item \(d_1 \ne d_2 \Ra H_{d_1} \ne H_{d_2}\)
        \item У группы \(G\) не существует никаких других подгрупп, кроме \(H_d, d | n\).
    \end{enumerate}
\end{theorem}

\begin{definition}
    Пусть \(A, B \subset G\). \(A \cdot B = \{ab | a \in A, b \in B\}\).
\end{definition}

\begin{note}
    \(A(BC) = (AB)C\)
\end{note}

\begin{definition}
    Если \(H\) --- подгруппа в \(G, x \in G\), то \(xH\) называется левым смежным классом, а \(Hx\) --- правым смежным классом.
\end{definition}

\subsection{Свойства правых и левых смежных классов}
\begin{proposition}
    \(y \in xH \Ra xH = yH\)
\end{proposition}
\begin{proof}
    \(\exists h \in G: yh = x \Ra yH = x(hH) = xH\).
\end{proof}

\begin{corollary}
    Любые два смежных класса либо совпадают, либо не пересекаются.
\end{corollary}

\begin{corollary}
    \(\forall H \le G, G = \bigsqcup x_i H\) для некоторых \(x_i\)
\end{corollary}
\begin{proof}
    \(x \in xH \Ra G = \bigcup_{x \in G} xH\). Оставим в данном объединении только по одному представителю каждого смежного класса. Получили желаемое.
\end{proof}

Аналогичные свойства верны и для правых смежных классов.

\begin{theorem}[Лагранжа]
    Порядок любой подгруппы \(H\) конечной группы \(G\) является делителем порядка группы.
\end{theorem}
\begin{proof}
    Разложим \(G\) по \(H\). Получим, что \(G = \bigsqcup x_i H \Ra |G| = \sum_i |H|\).
\end{proof}

\begin{definition}
    \((G:H) = |G:H| = \frac{|G|}{|H|}\), если \(H \le G\).
\end{definition}

\begin{corollary}
    \(a \in G \Ra \ord a |\;|G|\).
\end{corollary}

\begin{corollary}
    \(|G| = p \Ra G\) --- циклическая
\end{corollary}

\begin{corollary}
    Существует единственная с точностью до изоморфизма группа порядка \(p\).
\end{corollary}

\begin{corollary}[Теорема Эйлера]
    Пусть \((a, n) = 1 \Ra a^{\phi(n)} \equiv_n 1\)
\end{corollary}

\begin{corollary}[Малая Теорема Ферма]
    Пусть \(a^p \equiv_p a\)
\end{corollary}

