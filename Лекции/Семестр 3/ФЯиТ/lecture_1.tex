% !TEX root = ../../../main.tex

\section{Конечные Автоматы}
\subsection{Напоминание}
\begin{definition}
    Алфавит --- конечное непустое множество
\end{definition}

\begin{definition}
    Слово --- кортеж из букв
\end{definition}

\begin{definition}
    Пусть \(L_1, L_2\) --- множества слов. \(L_1L_2 = \{uv | u \in L_1, v \in L_2\}\)
\end{definition}

\begin{definition}
    \(\Sigma^* = \bigcup_{\bigcup_{k = 0}^\infty} \Sigma^k\) --- звезда Клини
\end{definition}

\begin{definition}
    \(\epsilon\) --- пустое слово
\end{definition}

\subsection{Недетерминированный Конечный Автомат}
\begin{definition}
    Недетерминированный Конечный Автомат --- кортеж \((Q, \Sigma, \Delta, q_0, F)\), удовлетворяющий следующим условиям:
    \begin{enumerate}
        \item \(Q\) называется множеством состояний, \(Q\) конечное
        \item \(\Sigma\) --- Алфавит
        \item \(\Delta \subset Q \times \Sigma^* \times Q\) --- переходы, \(\Delta\) конечное
        \item \(q_0 \in Q\) --- стартовое состояние
        \item \(F \subset Q\) --- множество завершающих состояний
    \end{enumerate}
\end{definition}

\begin{definition}
    Конфигурация Автомата --- элемент \(Q \times \Sigma^*\). обозначается \(\langle q, w\rangle\)
\end{definition}

\begin{definition}
    \(\vdash\) --- наименьшее по включению рефлексивное транзитивное отношение над \(Q \times \Sigma^*\), что \(\forall u \in \Sigma^*, (\langle q_1, w \rangle \ra q_2) \in \Delta \Ra \langle q_1, wu\rangle \vdash \langle q_2, u\rangle)\).
\end{definition}

\begin{definition}
    \(L(M) = \{w \in \Sigma^* | \exists q_F \in F: \langle q_0, w \rangle \vdash \langle q_F, \epsilon \rangle\}\).
\end{definition}

\begin{definition}
    Язык \(L\) называется автоматным, если \(\exists M\) --- автомат, такой, что \(L(M) = L\)
\end{definition}

\begin{proposition}
    В определении автомата можно считать, что \(|F| = 1\).
\end{proposition}
\begin{proof}[Неформально]
    Добавим новое состояние \(q_F\). Из всех конечных состоянии сделаем в него переходы по \(\epsilon\). Сделаем \(q_F\) единственным конечным. Тогда наш автомат успешно завершился \(\Lra\) мы перешли в \(q_F\) и слово закончилось, но в него можно было прийти только по \(\epsilon\)-переходу, значит в исходном автомате мы бы остановились в конечном состоянии и на пустом слове.
\end{proof}
\begin{proof}
    Рассмотрим автомат \(M' = (Q \cup \{q_F\}, \Sigma, \Delta \cup F \times \{\epsilon\} \times \{q_F\}, q_0, \{q_F\})\).
    \begin{enumerate}
        \item \(L(M) \subset L(M')\)
        \[w \in L(M) \Ra \exists q \in F: \langle q_0, w \rangle \vdash_M \langle q, \epsilon \rangle\]
        \[\Delta \subset \Delta' \Ra \langle q_0, w \rangle \vdash_{M'} \langle q, \epsilon \rangle \vdash_{M'} \langle q_F, \epsilon \rangle \Ra \langle q_0, w \rangle \vdash_{M'} \langle q_F, \epsilon \rangle\]
        \item \(L(M) \supset L(M')\)
        \[w \im L(M') \Ra \langle q_0, w\rangle \vdash_{M'} \langle q_F, \epsilon\rangle\]
        \[\exists u, q_1 \in F: \langle q_0, w\rangle \vdash_{M'} \langle q_1, u\rangle \vdash_{M', 1} \langle q_F, \epsilon\rangle \Ra u = \epsilon \Ra \langle q_0, w\rangle \vdash_{M'} \langle q_1, \epsilon\rangle\]
    \end{enumerate}
\end{proof}

\begin{proposition}
    В определении автомата можно считать, что \(\forall \langle q_1, w \rangle \ra q_2 \in \Delta: |w| \le 1\).
\end{proposition}
\begin{proof}
    Превращаем \(q_1 \ra_{abc} q_2\) в \(q_1 \ra_a q_3 \ra_b q_4 \ra_c q_2\).
\end{proof}

\begin{theorem}
    В определении автомата можно считать, что \(\forall \langle q_1, w \rangle \ra q_2 \in \Delta: |w| = 1\).
\end{theorem}
\begin{proof}[Неформально]
    Делаем две операции:
    \begin{enumerate}
        \item В цепочку \(q_1 \ra_\epsilon q_2 \ra_\epsilon q_3 \ra_\epsilon \dots \ra_\epsilon q_{n - 1} \ra_a q_n\) удаляем \(\epsilon\)-переходы, вместо них добавим переходы вида \(q_1 \ra_a q_i\).
        
        \item В цепочке \(q_1 \ra_\epsilon q_2 \ra_\epsilon q_3 \ra_\epsilon \dots \ra_\epsilon  q_n\), где \(q_n\) --- конечное состояние делаем все состояния конечными
    \end{enumerate}
    Сначала до упора делаем 1, потом 2, потом удаляем оставшиеся \(\epsilon\)-переходы.
\end{proof}
\begin{proof}
    Положим \(\Delta(q, w) = \{q' \in Q | \langle q, w\rangle \vdash \langle q', \epsilon\rangle\}\), \(M' = \{Q, \Sigma, \Delta', q_0, F'\}\), где \(\Delta' = \{\langle q_1, w\rangle \ra q_2 | \exists q_3 \in \Delta(q_1, \epsilon): \langle q_3, \epsilon\rangle \ra q_2 \in \Delta\}, F' = \{q' | \exists q \in F: \langle q', \epsilon\rangle \vdash \langle q, \epsilon\rangle\}\)

    \begin{enumerate}
        \item \(L(M) \subset L(M')\)
        \[w \in L(M), w = w_1w_2\dots w_k, w_i \in \Sigma\]
        \[\exists q \in F \langle q_0, w\rangle \vdash \langle q, \epsilon\rangle\]
        \[\Ra \begin{array}{lclclcl}
            \langle q_0, w_1w_2\dots w_k\rangle & \vdash_{M} & \langle q_1, w_1w_2\dots w_k\rangle & \vdash_{M, 1} & \langle q_1', w_2\dots w_k\rangle & \vdash_M \\
            
            & \vdash_{M} & \langle q_2, w_2\dots w_k\rangle & \vdash_{M, 1} & \langle q_2', w_3\dots w_k\rangle & \vdash_M \\
            
            & & \vdots & & & \\

            & \vdash_{M} & \langle q_s, w_k\rangle & \vdash_{M, 1} & \langle q_s', \epsilon\rangle & \vdash_M  &\langle q, \epsilon\rangle\\
        \end{array}\]
    \end{enumerate}
\end{proof}
