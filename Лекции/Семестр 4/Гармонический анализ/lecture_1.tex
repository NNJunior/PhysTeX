% !TEX root = ../../../main.tex


\section{Пространство Лебега}
\begin{definition}
    Пусть \(E \subset \R^m\). Будем говорить, что \(f: E \ra \Cm\) измерима (интегрируема) по Лебегу, если \(\Re f, \Im f\) измеримы (интегрируемы) по Лебегу.
\end{definition}

\begin{definition}
    В случае интегрируемости положим \(\int_E f = \int_E \Re f + i \int_E \Im f\). Полученный интеграл линеен и аддитивен по множествам
\end{definition}

\begin{note}
    \[\left\|\int_E f \right\| \le \int_E |f|\]
\end{note}
\begin{proof}
    \[\int_E f = \left\|\int_E f\right\| e^{i \theta} \Ra \int_E fe^{-i \theta} = \left\|\int_E f\right\| = \int_E \Re(e^{-i\theta} f) \le \int_E |e^{i\theta} f| = \int_E |f|\]
\end{proof}

\begin{definition}
    Пусть \(1 \le p \le < \infty\) и \(E\) измеримо. Определим:
    \[L_p(E) = \{f: E \ra \Cm | \int_E |f|^p < \infty\}\]
    В таком случае положим \(\|f\|_p = \left( \int_E |f|^p \right)^{\frac{1}{p}}\)
\end{definition}

Пусть \(f, g \in L_p(E)\). Если \(\lambda \in \Cm\), то \(\lambda f \in L_p(E)\) и ввиду \(|f + g|^p \le (2\max\{|f|, |g|\})^p \le 2^p(|f|^p + |g|^p)\) выполнено \(f + g \in L_p\). Получили, что \(L_p\) является линейным пространством относительно \(+, \lambda\cdot\).



\begin{lemma}
    Пусть \(a, b \ge 0\). Если \(1 < p < \infty, \frac{1}{p} + \frac{1}{q} = 1\), то \(ab \le \frac{a^p}{p} + \frac{b^q}{q}\), причем равенство достигается тогда и только тогда, когда \(a^p = b^q\)
\end{lemma}
\begin{proof}
    Можно считать, что \(ab > 0\). Ввиду выпуклости экспоненты, имеем:
    \[e^{\frac{x}{p} + \frac{y}{q}} \le \frac{1}{p}e^x + \frac{1}{q}e^y\]
    Положим \(x = p\ln a, y = q\ln b\) и получаем желаемое.
\end{proof}

\begin{theorem}[Гельдер]
    Пусть \(1 < p < \infty, \frac{1}{p} + \frac{1}{q} = 1\). Если \(f \in L_p(E), g \in L_q(E)\), то \(fg \in L_1(E)\) и \(\|fg\|_1 \le \|f\|_p\|g\|_q\)
\end{theorem}
\begin{proof}
    Если \(\|f\|_p = 0 \Ra f = 0\) п.в. \(\Ra fg = 0\) п.в. \(\Ra \) утверждение доказано. Аналогично и для случая \(\|g\|_q = 0\). В противном случае, получаем:
    \[\|f\|_p\|g\|_q > 0\]
    По предыдущей лемме, имеем:
    \[\frac{|f(x)|}{\|f\|_p}\frac{|g(x)|}{\|g\|_q} \le \frac{1}{p}\left( \frac{|f(x)|}{\|f\|_p} \right)^p + \frac{1}{q}\left( \frac{|g(x)|}{\|g\|_q} \right)^q\]
    Проинтегрировав, получим:
    \[\frac{1}{\|f\|_p\|g\|_q}\int_E |fg| \le \frac{1}{p}\left( \int_E \frac{|f|^p}{\|f\|_p^p} \right) + \frac{1}{q}\left( \int_E \frac{|g|^q}{\|g\|_q^q} \right) = 1\]
\end{proof}

\begin{note}
    В неравенстве гельдера равенство имеет место тогда и только тогда, когда \(|f|^p = c|g|^q\) п.в. на \(E\) для некоторого \(c > 0\).
\end{note}

\begin{theorem}[Минковский]
    Пусть \(1 \le p < \infty\). Если \(f, g \in L_p(E)\), то \(f + g \in L_p(E)\) и \(\|f + g\|_p \le \|f\|_p + \|g\|_p\).
\end{theorem}
\begin{proof}
    При \(p = 1\) применим \(|f + g| \le |f| + |g|\). Далее при \(p \ge 1\): Применим неравенство Гельдера для \(p, q = \frac{p}{p - 1} \Ra \frac{1}{p} + \frac{1}{q} = 1\). 
    \[\|f + g\|_p^p = \int_E |f + g|^p \le \int_E |f||f + g|^{p - 1} + \int_E |g||f + g|^{p - 1} \le\]
    \[\le \left( \int_E |f|^p \right)^{\frac{1}{p}}\left( \int_E |f + g|^p \right)^{\frac{1}{q}} + \left( \int_E |g|^p \right)^{\frac{1}{p}}\left( \int_E |f + g|^p \right)^{\frac{1}{q}} = \]
    \[ = \|f + g\|_p^{p - 1}(\|f\|_p + \|g\|_p)\]
\end{proof}

\begin{note}
    Равенство в теореме Минковского выполняется тогда и только тогда, когда \(f = cg\) п.в. для некоторой \(c > 0\)
\end{note}

\begin{definition}
    На \(L_p\) введем отношение \(\sim\). Будем говорить, что \(f \sim g \Lra f = g\) п.в. на \(E\).
\end{definition}

\begin{note}
    \(\sim\) --- отношение эквивалентности на \(L_p\), согласованное с операциями сложения и умножения на скаляр. Факторпространство \(L_p(E)/_\sim\) будем также обозначать \(L_p(E)\)
\end{note}

\begin{corollary}
    Пространство \(L_p\) относительно нормы \(\|\cdot\|_p\) является нормированным линейным пространством
\end{corollary}
\begin{proof}\indent
    \begin{enumerate}
        \item \(\|f\|_p \ge 0, \|f\|_p = 0 \Lra f \sim 0\)
        \item \(\|\lambda f\|_p = \lambda \|f\|_p\)
        \item \(\|f + g\|_p \le \|f\|_p + \|g\|_p\)
    \end{enumerate}
\end{proof}

\begin{problem}
    Если \(\mu(E) < \infty, 1 \le p < q < \infty\), то \(L_q(E) \subsetneq L_p(E)\)
\end{problem}

\begin{reminder}
    Полное метрическое пространство --- такое, что любая фундаментальная последовательность сходится
\end{reminder}

\begin{theorem}[Рисса]
    Пространство \(L_p\) банахово (т.е. является полным относительно метрики, порожденной \(p\)-нормой). 
\end{theorem}
\begin{proof}
    Будет позднее
\end{proof}

\begin{reminder}
    Напомним, что \(\supp(f) = \overline{\{x: f(x) \ne 0\}}\). Будем называть функции с компактным носителем финитными.
\end{reminder}

\begin{lemma}
    Пусть \(f \in L_p, \epsilon > 0\). Тогда \(\exists\) простая финитная функция \(\phi\) такая, что \(\|f - \phi\|_p < \epsilon\).
\end{lemma}
\begin{proof}
    Можно считать, что \(f\) вещественнозначная (иначе приближаем \(\Re f, \Im f\)) отдельно. Т.к. \(|f - I_{B_k(0)}|^p \le |f|^p\). Тогда по Теореме Лебега о мажорируемой сходимости, \(\|f - fI_{B_k(0)}\|_p \ra 0, k \ra \infty\). Заменяя функцию \(f\) на \(fI_{B_k(0)}\) для достаточно большого \(k\), заключаем, что она финитная. Пусть сначала \(f \ge 0\). По теореме о приближении, найдется последовательность \(\{\phi_k\}\) --- простых функций, т.ч. \(0 \le \phi_1 \le \dots \), \(\phi_k \ra f\). Т.к. \(|f - \phi_k|^p \le |f|^p\). По теореме Лебега о мажорируемой сходимости \(f - \phi_k \ra 0, k \ra \infty\), причем все \(\phi_k\) финитны, т.к. \(0 \le \phi_k \le f\) на \(E\). Пусть \(f\) теперь произвольного знака \(\Ra f = f^+ - f^-\). Тогда по доказанному найдутся простые финитные функции \(\phi^+, \phi^-\), такие, что \(\|f^+ - \phi^+\|_p < \frac{\epsilon}{2}, \|f^- - \phi^-\|_p < \frac{\epsilon}{2}\). Положим \(\phi = \phi^+ - \phi^-\). Тогда \(\phi\) --- простая финитная функция и по неравенству треугольника:
    \[\|f - \phi\|_p \le \|f^+ - \phi^+\| + \|f^- - \phi^-\| < \epsilon\]
\end{proof}

\begin{theorem}
    Пусть \(f \in L_p, \epsilon > 0\). Тогда \(\exists g \in C_c^\infty(\R^m)\), т.ч. \(\|f - g\|_p < \epsilon\)
\end{theorem}
\begin{proof}
    По предыдущей лемме, любую функцию можно приблизить финитной простой функцией. Всякая простая функция есть линейная комбинация индикаторов. Из этого заключаем, что достаточно доказать теорему для случая \(f = I_A\), где \(A\) --- ограниченное измеримое множество. По свойству регулярности меры Лебега, \(\exists G, H\) --- открытые, такие, что \(G \supset A, H \supset A^c\) и \(\mu(G \setminus A) < \frac{\epsilon}{2}, \mu(H \setminus A^c) < \frac{\epsilon}{2}\). Положим \(k = H^c\) --- замкнутое и ограниченное (т.к. лежит в \(A\)), т.е. компакт, лежащий в \(A\). \(\mu(G \setminus A) \le \mu(G \setminus A) + \mu(H \setminus A^c) < \epsilon\). По теореме о гладком разбиении единицы, \(\exists g \in C^\infty(\R^m)\) с носителем в \(G\), такая, что \(0 \le g \le 1\) и \(g|_K = 1\). Поэтому \(\|I_A - g|_E\|_p^p = \int_E |I_A - g|^p \le \int_{(G \cap E) \setminus K} |I_A - g|^p \le \mu(G \setminus K) < \epsilon\)
\end{proof}
