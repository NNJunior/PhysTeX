% !TEX root = ../../../main.tex

\section{Введение}
\subsection{Предисловие}
Ну чето рассказали там про принцип устойчивости частот, про то, что ля-ля-ля тополя

\subsection{Напоминание с ОВиТМа}

\begin{definition}
    \(\mathcal{F}\) --- алгебра над \(\Omega\), если 
    \begin{enumerate}
        \item \(\Omega \in \mathcal{F}\)
        \item \(A, B \in \mathcal{F} \Ra A \cap B \in \mathcal{F}\)
        \item \(A \in \mathcal{F} \Ra \overline{A} \in \mathcal{F}\)
    \end{enumerate}
\end{definition}

\begin{note}
    Элементы \(\Omega\) называются элементарными событиями.
\end{note}

\begin{note}
    Элементы \(\mathcal{F}\) называются событиями.
\end{note}

\begin{definition}
    \(\mathcal{F}\) --- \(\sigma\)-алгебра над \(\Omega\), если 
    \begin{enumerate}
        \item \(\mathcal{F}\) --- алгебра
        \item \(A_1, \dots \in \mathcal{F} \Ra \bigcup_{i = 1}^\infty A_i \in \mathcal{F}\)
    \end{enumerate}
\end{definition}

\begin{definition}
    Функция \(P: \mathcal{F} \ra [0, 1]\) называется вероятностной мерой, если \(P(\Omega) = 1, P\) --- \(\sigma\)-аддитивна.
\end{definition}

\begin{note}
    \begin{enumerate}
        \item \(P(\emptyset) = 0\)
        \item \(A \subset B \Ra P(A) \le P(B)\) --- \textbf{монотонность меры}
        \item \(P\) конечно аддитивна
        \item Для \(P\) верна формула включений-исключений.
        \item \(P\left( \bigcup_{i = 1}^\infty A_i \right) = \le \sum_{i = 1}^\infty P(A_i)\)
    \end{enumerate}
\end{note}

\begin{theorem}[О непрерывности вероятностной меры]
    Пусть \((\Omega, \mathcal{F})\) таково, что \(\mathcal{F}\) --- алгебра над \(\Omega\), \(P\) --- мера и \(P(\Omega) = 1\). Тогда следующие условия эквивалентны:
    \begin{enumerate}
        \item \(P\) --- \(\sigma\)-аддитивна на \(\mathcal{F}\)
        \item \(P\) непрерывна в нуле, т.е. \(\bigcap A_i = \emptyset \Ra P(A_i) \ra P(A)\)
        \item \(P\) непрерывна сверху, т.е. \(\bigcap A_i = A \Ra P(A_i) \ra P(A)\)
        \item \(P\) непрерывна снизу, т.е. \(\bigcup A_i = A \Ra P(A_i) \ra P(A)\)
    \end{enumerate}
\end{theorem}

\begin{definition}
    Вероятностное пространство --- это измеримое пространство \((\Omega, \mathcal{F}, P)\), т.е. \(\Omega\) --- множество, \(\mathcal{F}\) --- некая \(\sigma\)-алгебра над \(\Omega\), \(P\) --- вероятностная мера на \((\Omega, \mathcal{F})\).
\end{definition}

\begin{definition}
    Система \(\mathcal{M}\) подмножеств \(\Omega\) называется \(\pi\)-системой, если \(\mathcal{M}\) замкнуто относительно счетного пересечения
\end{definition}

\begin{definition}
    Система \(\mathcal{M}\) подмножеств \(\Omega\) называется \(\lambda\)-системой, если
    \begin{enumerate}
        \item \(\Omega \in \mathcal{M}\)
        \item \(A, B \in \mathcal{M}, A \subset B \Ra B \setminus A \in \mathcal{M}\)
        \item \(A_i \in \mathcal{M} \Ra \bigcup A_i \in \mathcal{M}\)
    \end{enumerate}
\end{definition}

\begin{theorem}[Первая теорема о \(\lambda\)-системах]
    Система \(\mathcal{F}\) является \(\sigma\)-алгеброй над \(\Omega\) тогда и только тогда, когда она является \(\lambda\)-системой и \(\pi\)-системой.
\end{theorem}

\begin{proposition}
    Для любой системы \(\mathcal{M}\) подмножеств \(\Omega\) существует минимальная по включению, содержащаяся в \(\mathcal{M}\)
\end{proposition}

\begin{note}
    \(\sigma(\mathcal{M}), \alpha(\mathcal{M}), \pi(\mathcal{M}), \lambda(\mathcal{M})\) --- порожденные (минимальные) \(\sigma\)-алгебра, алгебра, \(\pi\)-система, \(\lambda\)-система соответственно.
\end{note}

\begin{theorem}[Вторая теорема о \(\lambda\)-системах]
    Если \(\mathcal{M}\) --- \(\pi\)-система на \(\Omega\), то \(\sigma(\mathcal{M}) = \lambda(\mathcal{M})\)
\end{theorem}
\begin{proof}
    См. доказательство из курса ОВиТМа
\end{proof}

\begin{example}[Борелевская \(\sigma\)-алгебра]
    \(\mathcal{B}(\R)\) --- \(\sigma\)-алгебра, порожденная всеми открытыми множествами (или, что равносильно, всеми полуинтервалами)
\end{example}

\begin{example}[Борелевская \(\sigma\)-алгебра в \(\R^n\)]
    \(\mathcal{B}(\R^n)\) --- \(\sigma\)-алгебра, порожденная всеми открытыми множествами (или, что равносильно, всеми кубами, где куб --- декартово произведение борелевских множеств).
\end{example}

\begin{example}
    \(\mathcal{B}(\R^\infty)\) --- \(\sigma\)-алгебра, порожденная всеми цилиндрами. Цилиндр --- множество \({x \in \R^\infty : (x_1, \dots x_n) \in B}, B \in \mathcal{B}(\R^n)\)
\end{example}


\begin{example}[Борелевская \(\sigma\)-алгебра в общем случае]
    Пусть \((S, \rho)\) --- метрическое пространство, тогда \(\mathcal{B}(S)\) --- \(\sigma\)-алгебра, порожденная всеми открытыми множествами.
\end{example}

\section{Вероятностная мера на прямой}
Пусть \(P\) --- вероятностная мера на \(\R, \mathcal{B}(\R)\)
\begin{definition}
    Функцией распределения называется функция \(F(x) = P((-\infty, x]), x \in \R\).
\end{definition}

\begin{lemma}[О свойствах функции распределения]
    \indent
    \begin{enumerate}
        \item \(F\) не убывает
        \item \(\lim_{x \ra -\infty} F(x) = 0\), \(\lim_{x \ra +\infty} F(x) = 1\)
        \item \(F\) непрерывна справа
    \end{enumerate}
\end{lemma}
\begin{proof}
    \indent
    \begin{enumerate}
        \item \(x \le y \Ra (-\infty, x] \subset (-\infty, y] \Ra F(x) \le F(y)\)
        \item \(x_n \ra -\infty \Ra (-\infty, x_n] \ra \emptyset \Ra \lim_{n \ra \infty} F(x_n) = 0 = \lim_{x \ra +\infty} F(x)\), \(x_n \ra +\infty \Ra (-\infty, x_n] \ra \R \Ra \lim_{n \ra \infty} F(x_n) = 1 = \lim_{x \ra +\infty} F(x)\)
        \item Если \(x_n \searrow x\), то \((-\infty, x_n] \searrow (-\infty, x]\) и \(F(x_n) \ra F(x)\)
    \end{enumerate}
\end{proof}

\begin{theorem}[О взаимно-однозначном соответствии функции распределения и вероятностной меры]
    Пусть \(F\) удовлетворяет условиям 1-3 из предыдущей теоремы. Тогда существует единственная вероятностная мера \(P\) на \((\R, \mathcal{B}(\R))\), т.ч. \(F\) --- её функция распределения.
\end{theorem}

\subsection{Классификация вероятностных мер}

Далее мы будем отождествлять понятия распределения и вероятностной меры.

\subsubsection{Дискретные распределения}

\begin{definition}
    Вероятностная мера \(P\) на \((R, \mathcal{B}(R))\) называется дискретной, если \(\exists X\) --- не более. чем счетое множество на \(\R\), такое, что \(P(\R \setminus X)\) и \(\forall x \in X P(\{x\}) > 0\).
\end{definition}

\paragraph{Примеры}
\begin{enumerate}
    \item Распределение Бернулли: \(P \sim Bern(p)\), если \(X = \{0, 1\}, P(\{1\}) = p, P(\{0\}) = 1 - p\)
    \item Равномерное распределение: \(X = {1, 2, \dots n}\), \(P(\{i\}) = \frac{1}{n}\)
    \item Биномиальное распределение: \(P \sim Bin(n, p)\), если \(X = \N, P(\{k\}) = C_n^kp^k(1 - p)^{n - k}, P(\{0\})\), моделирует количество успехов среди \(n\) испытаний.
    \item Пуассоновское распределение: \(P \sim Pois(n, p)\), если \(X = \N, P(\{k\}) = \frac{\lambda^k}{k!}e^{-\lambda}, P(\{0\})\), моделирует редкие события
    \item Геометрическое распределение: \(P \sim Geom(p)\), если \(X = \N, P(\{k\}) = p(1 - p)^{k-1}, P(\{0\})\), моделирует первый момент удачи в бесконечной схеме испытаний Бернулли
\end{enumerate}

\subsubsection{Абсолютно непрерывные распределения}

\begin{definition}
    \(F(x)\) называется абсолютно непрерывной, если \(\exists p(t) \ge 0: \forall x \in \R F(x) = \int_{-\infty}^x p(t)dt\). В таком случае мы говорим, что \(p(t)\) является плотностью функции \(F\) или соответствующего распределения (вероятностной меры).
\end{definition}

\begin{note}
    В таком случае \(F'(x) = p(x)\) почти всюду по мере Лебега.
\end{note}

\paragraph{Примеры}
\begin{enumerate}
    \item Равномерное распределение на отрезке \([a, b]\) --- \(U(a, b)\): 
    \[p(x) = \left\{\begin{array}{l}
        \frac{1}{b - a}, a \le x \le b \\
        0, \text{иначе}
    \end{array}\right., F(x) = \left\{\begin{array}{l}
        0, x < a \\
        \frac{x - a}{b - a}, a \le x \le b \\
        1, x > b
    \end{array}\right.\]
    Моделирует случайную точку из отрезка \([a, b]\)
    \item Нормальное распределение --- \(N(a, \sigma^2)\):
    \[p(x) = \frac{1}{\sqrt{2\pi \sigma^2s}}e^{-\frac{(x - a)^2}{2\sigma^2}}\]
    Моделирует измерение с ошибкой

    \item Экспоненциальное распределение --- \(N(a, \sigma^2)\):
    \[p(x) = \lambda e^{\lambda x} I\{x > 0\}\]
    \[F(x) = \left\{\begin{array}{l}
        0, x < 0 \\
        1 - e^{-\lambda x}, x \ge 0
    \end{array}\right.\]
    Моделирует время ожидание

    \item Гамма-распределение: \(\Gamma(\lambda, \alpha), \lambda, \alpha > 0\).
    \[p(x) = \frac{x^{\alpha - 1}\lambda^\alpha}{\Gamma(\alpha)}e^{-\lambda x}I\{x > 0\}\]
    Где:
    \[\Gamma(\alpha) = \int_0^{+\infty} x^{\alpha - 1}e^{-x}dx\]
    Нам в дальнейшем потребуются различные свойства \(\Gamma\)-функции: \(\Gamma(n) = (n - 1)!, \Gamma(\alpha + 1) = \alpha\Gamma(\alpha)\)

    \item Распределение Коши \(K(\sigma), \sigma > 0\)
    \[p(x) = \frac{\sigma}{\pi(x^2 + \sigma^2)}\]
    \[F(x) = \frac{1}{2} + \frac{1}{\pi}\arctan \frac{x}{\sigma}\]
    Модель
\end{enumerate}
