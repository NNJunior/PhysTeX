% !TEX root = ../../../main.tex

\subsubsection{Сингулярные распределения}

\begin{definition}
    Точка \(x\) называется точкой роста фукнции распределения \(F(x)\), если \(\forall \epsilon > 0: F(x + \epsilon) - F(x - \epsilon) > 0\)
\end{definition}

\begin{definition}
    Функция распределения \(F(x)\) наызвается сингулярной, если она непрерывна, и множество точек ее роста имеет \(\mu = 0\).
\end{definition}

\paragraph{Примеры}
\begin{enumerate}
    \item Канторова лестница --- ее точками роста ялвяется канторово множество.
\end{enumerate}

\begin{theorem}[Лебега о разложении]
    Если \(F(x)\) --- функция распределения на прямой, тогда \(\exists !\) разложение вида: \(F(x) = \alpha_1F_1(x) + \alpha_2F_2(x) + \alpha_3F_3(x)\), где \(\alpha_i \ge 0, \alpha_1 + \alpha_2 + \alpha_3 = 1\) и \(F_1\) --- дискретная, \(F_2\) --- абсолютно непрерывная, \(F_3\) --- сингулярная.
\end{theorem}

\section{Вероятностные меры в \(\R^n\)}
Пусть \(P\) --- вероятностная мера на \((\R^n, \mathcal{B}(\R^n))\). 
\begin{definition}
    Функцией распределения \(P\) называется \(F(x_1, \dots x_n) = P((-\infty, x_1] \times \dots \times (-\infty, x_n])\)
\end{definition}

\begin{note}[Обозначения]
    \begin{enumerate}
        \item \(\vec{x} = (x_1, \dots x_n)\)
        \item \(\vec{x} \ge \vec{y} \Lra x_i \ge y_i\)
        \item \((-\infty, \vec{x}] = (-\infty, x_1] \times \dots \times (-\infty, x_n]\)
        \item \(\vec{x}_n \downarrow \vec{x}\) если \(\vec{x} = \lim_{n \ra \infty}\vec{x}_n\) и \(\vec{x}_n \ge \vec{x}_{n + 1}\).
    \end{enumerate}
\end{note}

\begin{definition}
    Для \(i = 1, \dots n, a_i < b_i\) введем:
    \[\Delta_{a_i, b_i}^iF(x_1, \dots x_n) = F(x_1, x_{i - 1}, b_i, x_{i + 1}, \dots x_n) - F(x_1, x_{i - 1}, a_i, x_{i + 1}, \dots x_n)\]
\end{definition}

\begin{lemma}[Свойства многомерных функций распределения]
    \begin{enumerate}
        \item Если \(\vec{x}_n \downarrow \vec{x}\), то \(F(\vec{x}_n) \ra F(\vec{x})\)
        \item \(\lim_{\begin{array}{c}
            x_1 \ra \infty \\
            \vdots \\
            x_n \ra \infty
        \end{array}} F(x_1, \dots x_n) = 1\)
        \item \(\forall i = 1, \dots n: \lim_{x_i \ra -\infty} F(x_1, \dots x_n) = 0\).
        \item Для любых \(a_i < b_i, i = 1, \dots n\):
        \[\Delta_{a_1, b_1}^1 \circ \Delta_{a_2, b_2}^2 \circ \dots \circ \Delta_{a_n, b_n}^n F(x_1, \dots x_n) \ge 0\]
    \end{enumerate}
\end{lemma}
\begin{proof}\indent
    \begin{enumerate}
        \item Если \(\vec{x}_n \downarrow \vec{x}\), то \((-\infty, \vec{x}_n] \downarrow (-\infty, \vec{x}] \Ra\) в силу непрерывности вероятностной меры, получаем \(F(\vec{x}_n) \ra F(\vec{x})\)
        \item Если \(\vec{x}_n \uparrow (+\infty, \dots +\infty)\), то \((-\infty, \vec{x}_n] \uparrow \R^n \Ra\) в силу непрерывности вероятностной меры, получаем \(F(\vec{x}_n) \ra P(\R^n) = 1\)
        \item Если \(x_i \downarrow -\infty\), то \((-\infty, \vec{x}_n] \downarrow \emptyset \Ra\) в силу непрерывности вероятностной меры, получаем \(F(\vec{x}_n) \ra P(\emptyset) = 0\)
        \item Для \(n = 2\):
        \[\Delta^1_{a_1, b_1} \circ \Delta^2_{a_2, b_2} F(x_1, x_2) = F(b_1, b_2) - F(b_1, a_2) - F(a_1, b_2) + F(a_1, a_2) = P((a_1, b_1] \times (a_2, b_2]) \ge 0\]
        В общем случае:
        \[\Delta_{a_i, b_i}^i P(A_1 \times \dots \times A_{i - 1} \times (-\infty, x_i) \times A_{i + 1} \times \dots \times A_n) = \]
        \[= P(A_1 \times \dots \times A_{i - 1} \times (a_i, b_i] \times A_{i + 1} \times \dots \times A_n)\]
        Получаем, что 
        \[\Delta_{a_1, b_1}^1 \circ \Delta_{a_2, b_2}^2 \circ \dots \circ \Delta_{a_n, b_n}^n F(x_1, \dots x_n) = P((a_1, b_1] \times \dots \times (a_n, b_n]) \ge 0\]
    \end{enumerate}
\end{proof}

\begin{theorem}[О взаимно однозначном соответствии]
    Если \(F\) удовлетворяет свойствами 1-3 из леммы, то \(\exists !\) вероятностная мера \(P\) на \((\R^n, \mathcal{B}(\R^n))\), такая, что \(F\) --- ее функция распределения
\end{theorem}
\begin{proof}
    См. ОВиТМ
\end{proof}

\begin{note}
    Свойство 3 нельзя заменить на неубывание по каждой из переменнх.
\end{note}
\begin{proof}
    Рассмотрим \(F(x, y) = \left\{\begin{array}{l}
        1, x + y \ge 0 \\
        0, x + y < 0
    \end{array}\right.\).
    Заметим, что \(F\) удовлетворяет свойствам 1, 2 и не убывает по обеим переменным. При этом, если мы возьем:
    \[\Delta^x_{-1, 1} \circ \Delta^y_{-1, 1} F(x, y) = F(1, 1) - F(-1, 1) - F(1, -1) + F(-1, -1) = 1 - 2 + 0 < 0\]
    Получаем, что \(F\) --- не двумерная функция распределения.
\end{proof}

\subsection{Примеры}
\begin{enumerate}
    \item Пусть \(F_1, F_2, \dots F_n\) --- одномерные функции распределения. Рассмотрим \(F(x_1, \dots x_n) = F_1(x_1)\dots F_n(x_n)\) --- многомерная функция распределения. Свойства 1, 2 очевидны, проверим 3:
    \[\Delta_{a_1, b_1}^1 \circ \Delta_{a_2, b_2}^2 \circ \dots \circ \Delta_{a_n, b_n}^n F(x_1, \dots x_n) = \prod_{k = 1}^n (F_k(b_k) - F_k(a_k)) \ge 0\]

    \item Пусть \(p(t_1, \dots t_n) \ge 0\), т.ч.
    \[\int_{\R^n} p(t_1, \dots t_n) dt_1dt_2\dots dt_n = 1\]
    Тогда:
    \[(*)\;\;F(x_1, \dots x_n) = \int_{-\infty}^{x_1}\int_{-\infty}^{x_2}\dots \int_{-\infty}^{x_n} p(t_1, \dots t_n)dt_1dt_2\dots dt_n\]
    Свойства 1, 2 очевидны, проверим 3:
    \[\Delta_{a_1, b_1}^1 \circ \Delta_{a_2, b_2}^2 \circ \dots \circ \Delta_{a_n, b_n}^n F(x_1, \dots x_n) = \int_{a_1}^{b_1}\int_{a_2}^{b_2}\dots \int_{a_n}^{b_n} p(t_1, \dots t_n)dt_1dt_2\dots dt_n\]
\end{enumerate}

\begin{definition}
    Если имеет место представление \((*)\), то \(p(t_1, \dots t_n)\) называется плотностью функции распределения \(F\).
\end{definition}

\section{Вероятностные меры в \(\R^\infty\)}

\begin{definition}
    Пусть \(P\) --- вероятностная мера на \((\R^{\infty}, \mathcal{B}(\R^\infty))\). Рассмотрим для \(n \in \N\) вероятностную меру \(P_n\) на \((\R^n, \mathcal{B}(\R^n))\), т.ч.
    \[P_n(B) = P(Cyl(n, B)), B \in \mathcal{B}(\R^n)\]
    Тогда можно заметить, что \(P_{n + 1}(B \times \R) = P_n(B)\).
\end{definition}

\begin{definition}
    Свойство выше называется согласованностью для последовательности вероятностных мер \(\{P_n\}\)
\end{definition}

\begin{theorem}[Колмогорова, о мерах в \(\R^\infty\)]
    Пусть \(\{P_n, n \in \N\}\) --- последовательность согласованных вероятностных мер, \(P_n\) --- мера на \(\R^n\). Тогда \(\exists !\) вероятностная мера \(P\) на \((\R^\infty, \mathcal{B}(R^\infty))\), такая, что \(\forall n \in \N: \forall B_n \in \mathcal{B}(\R^n)\):
    \[(*)\;\;\;P_n(B_n) = P(Cyl(n, B_n))\]
\end{theorem}
\begin{proof}
    Зададим меру \(P\) на цилиндрах по правилу \((*)\). Цилиндры образуют алгебру \(\mathcal{A}\). Проверим корректность задания \(P\). Если \(Cyl(n, B_n) = Cyl(n + k, B_{n + k})\), то \(B_{n + k} = B_n \times \R^k\). Тогда в силу согласованности:
    \[P_n(B_n) = P_{n + k}(B_{n + k})\]
    Проверим, что \(P\) --- конечно аддитивна на \(\mathcal{A}\). Если \(\tilde{B_1}, \dots \tilde{B_N} \in \mathcal{A}\) --- непересекаются, то будем считать, что \(\exists n: \tilde{B_i} = Cyl(n, B_i), i = 1, \dots N, B_i \in \mathcal{B}(\R^n)\). Тогда:
    \[P\left( \bigsqcup_{i = 1}^N \tilde{B_i} \right) = P\left( Cyl\left( n, \bigsqcup_{i = 1}^N \tilde{B_i} \right) \right) = P_n\left( \bigsqcup_{i = 1}^N B_i \right) = \sum_{i = 1}^N P_n(B_i) = \sum_{i = 1}^N P(\tilde{B_i})\]

    Проверим, что \(P\) непрерывна в нуле (на \(\mathcal{A}\)). Пусть \(\tilde{B_n} \downarrow \emptyset, \tilde{B_n} \in \mathcal{A}\). Без ограничения общности, считаем, что \(\tilde{B_n} = Cyl(n, B_n)\). 

    От противного. Пусть \(\lim_{n \ra \infty} P(\tilde{B_n}) = \delta > 0\). Тогда для \(\forall n \in \N\) выберем компактные \(A_n \subset \R^n\), такие, что \(A_n \subset B_n\) и \(P(B_n \setminus A_n) \le \frac{\delta}{2^{n + 1}}\). Обозначим \(\tilde{A_n} = Cyl(n, A_n)\). Тогда: \(P(\tilde{B_n} \setminus \tilde{A_n}) = P_n(B_n \setminus A_n) \le \frac{\delta}{2^{n + 1}}\). Введем \(\tilde{C_n} = \bigcap_{i = 1}^n \tilde{A_i}\). Тогда \(\tilde{C_n} \downarrow \emptyset, \tilde{C_n} = Cyl(n, C_n)\), где \(C_n = A_n \cap (A_{n - 1} \times \R) \cap (A_{n - 2} \times \R^2) \cap \dots \cap (A_{1} \times \R^{n - 1})\) --- тоже компакт в \(\R^n\). Далее:
    \[P(\tilde{B_n} \setminus \tilde{C_n}) \le \sum_{i = 1}^n P(\tilde{B_n} \setminus \tilde{A_i}) \le |B_i \subset B_n, i \ge n| \le \sum_{i = 1}^n P(\tilde{B_i} \setminus \tilde{A_i}) \le \frac{\delta}{2} \Ra \lim_{n \ra \infty} P(\tilde{C_n}) \ge \frac{delta}{2} > 0\]
    Возьмем в каждом \(\tilde{C_n}\) по точке \((x_1^{(n)}, x_2^{(n)}, \dots) \in \tilde{C_n}\). Тогда \((x_1^{(n)}, \dots x_n^{(n)}) \in C_n\). Рассмотирм последовательность \(x_1^{(n)}\). Эти все точки лежат в \(C_1\). Выберем подпоследовательность \(x_1^{(n_k)} \ra x_1^0 \in C_1\)
\end{proof}
